\documentclass[8pt]{beamer}
\usetheme{Madrid}
\usepackage{graphicx}
\usepackage{booktabs}
\usepackage{adjustbox}
\usepackage{multicol}
\usepackage{amsmath}

% Color definitions
\definecolor{mlblue}{RGB}{31, 119, 180}
\definecolor{mlorange}{RGB}{255, 127, 14}
\definecolor{mlgreen}{RGB}{44, 160, 44}
\definecolor{mlred}{RGB}{214, 39, 40}
\definecolor{mlpurple}{RGB}{148, 103, 189}
\definecolor{mlgray}{RGB}{127, 127, 127}

% Remove navigation symbols
\setbeamertemplate{navigation symbols}{}

% Clean itemize/enumerate
\setbeamertemplate{itemize items}[circle]
\setbeamertemplate{enumerate items}[default]

% Reduce margins for more content space
\setbeamersize{text margin left=5mm,text margin right=5mm}

% Title information
\title{ML-Augmented Design Thinking}
\subtitle{Integrating Machine Learning into the Design Process}
\author{Prof. Dr. Joerg Osterrieder}
\institute{BSc Course - 12 Week Program}
\date{\today}

\begin{document}

% Title slide
\begin{frame}[t]
\titlepage
\end{frame}

% Table of contents
\begin{frame}[t]{Course Overview}
\tableofcontents
\vfill
\footnotesize
\textbf{Course Methodology:} Blended learning approach combining theoretical foundations with hands-on ML implementation. Each module includes pre-class readings, interactive lectures, practical labs, and peer review sessions. Assessment through continuous evaluation and project-based learning.
\end{frame}

% Section 1: Introduction
\section{Introduction to ML and Design Thinking}

\begin{frame}[t]
\vfill
\centering
\begin{beamercolorbox}[sep=8pt,center]{title}
\usebeamerfont{title}\Large Introduction to ML and Design Thinking\par
\end{beamercolorbox}
\vfill
\end{frame}

\begin{frame}[t]{Design Thinking Process}
\begin{columns}[T]
\begin{column}{0.38\textwidth}
\textbf{Traditional Stages:}
\begin{enumerate}
\item Empathize
\item Define
\item Ideate
\item Prototype
\item Test
\end{enumerate}
\vspace{10pt}
\textbf{Key Question:}\\
How can ML enhance each stage?
\end{column}
\begin{column}{0.58\textwidth}
\includegraphics[width=\textwidth]{charts/ml_impact.pdf}
\end{column}
\end{columns}
\vfill
\footnotesize
\textbf{Integration Points:} ML augments rather than replaces human creativity. Iterative feedback loops between stages. Data-driven validation at each transition. Continuous learning from user interactions.
\end{frame}

% Section 2: Data-Driven Empathy
\section{Data-Driven Empathy}

\begin{frame}[t]
\vfill
\centering
\begin{beamercolorbox}[sep=8pt,center]{title}
\usebeamerfont{title}\Large Data-Driven Empathy\par
\end{beamercolorbox}
\vfill
\end{frame}

\begin{frame}[t]{User Sentiment Analysis}
\centering
\includegraphics[width=0.95\textwidth]{charts/sentiment_distribution.pdf}
\vspace{5pt}
\small Key insight: ML reveals hidden patterns in user feedback
\vfill
\footnotesize
\textbf{NLP Methods:} BERT-based transformer models for context-aware sentiment classification. Aspect-based sentiment analysis to identify specific pain points. Topic modeling with LDA for theme extraction. Real-time processing pipeline handles 10K reviews/minute.
\end{frame}

\begin{frame}[t]{User Clustering and Personas}
\begin{columns}[T]
\begin{column}{0.25\textwidth}
\textbf{Discovered Segments:}
\begin{itemize}
\item Beginners (35\%)
\item Intermediate (45\%)
\item Advanced (20\%)
\end{itemize}
\vspace{10pt}
\textbf{Key Features:}
\begin{itemize}
\item Technical skills
\item Design experience
\item Tool familiarity
\end{itemize}
\end{column}
\begin{column}{0.75\textwidth}
\includegraphics[width=\textwidth]{charts/user_clusters.pdf}
\end{column}
\end{columns}
\end{frame}

% Section 3: ML-Enhanced Ideation
\section{ML-Enhanced Ideation}

\begin{frame}[t]
\vfill
\centering
\begin{beamercolorbox}[sep=8pt,center]{title}
\usebeamerfont{title}\Large ML-Enhanced Ideation\par
\end{beamercolorbox}
\vfill
\end{frame}

\begin{frame}[t]{Generative AI Performance}
\centering
\includegraphics[width=0.9\textwidth]{charts/ideation_comparison.pdf}
\begin{columns}[T]
\begin{column}{0.3\textwidth}
\small\textcolor{mlblue}{Blue: Quantity}
\end{column}
\begin{column}{0.3\textwidth}
\small\textcolor{mlorange}{Orange: Quality}
\end{column}
\begin{column}{0.3\textwidth}
\small\textcolor{mlgreen}{Green: Novelty}
\end{column}
\end{columns}
\vfill
\footnotesize
\textbf{Evaluation Metrics:} Quantity measured by ideas/hour. Quality assessed via expert panel ratings (Cohen's kappa = 0.78). Novelty computed using semantic distance from existing solutions. Baseline: Traditional brainstorming sessions with n=50 participants.
\end{frame}

% New slide with formula at bottom
\begin{frame}[t]{Optimization Objective}
\textbf{Cross-Entropy Loss for Classification}

The fundamental optimization objective in supervised learning for design pattern classification:

\begin{itemize}
\item Minimize empirical risk over training data
\item Balance between model complexity and accuracy
\item Gradient-based optimization using backpropagation
\end{itemize}

\vfill
\begin{equation}
\mathcal{L}(\theta) = -\frac{1}{N} \sum_{i=1}^{N} \sum_{k=1}^{K} y_{ik} \log(\hat{p}_{ik}) + \lambda \|\theta\|_2^2
\end{equation}
\small
where $N$ = samples, $K$ = classes, $y_{ik}$ = true label, $\hat{p}_{ik}$ = predicted probability, $\lambda$ = regularization
\end{frame}

% Section 4: Learning Progress
\section{Student Learning Progress}

\begin{frame}[t]
\vfill
\centering
\begin{beamercolorbox}[sep=8pt,center]{title}
\usebeamerfont{title}\Large Student Learning Progress\par
\end{beamercolorbox}
\vfill
\end{frame}

\begin{frame}[t]{12-Week Learning Journey}
\includegraphics[width=\textwidth]{charts/learning_progress.pdf}
\vspace{5pt}
\centering
\small Steady progression in both theoretical understanding and practical skills
\vfill
\footnotesize
\textbf{Assessment Methodology:} Weekly formative assessments via automated coding challenges. Bi-weekly summative evaluations through project milestones. Peer assessment component (20\%). Self-reflection portfolios. Competency-based progression thresholds.
\end{frame}

\begin{frame}[t]{Skill Development Matrix}
\centering
\includegraphics[width=0.85\textwidth]{charts/skill_correlation.pdf}
\vfill
\footnotesize
\textbf{Interpretation Guide:} Correlation values: 0.7-1.0 = strong positive relationship, 0.4-0.7 = moderate, 0-0.4 = weak. Key insight: ML Theory strongly correlates with Python skills (r=0.82). Design Thinking shows moderate correlation with Data Viz (r=0.72), indicating importance of visual communication.
\end{frame}

% New slide with technical text at bottom
\begin{frame}[t]{Technical Architecture}
\textbf{System Design Specifications}

\begin{columns}[T]
\begin{column}{0.45\textwidth}
\textbf{Data Pipeline:}
\begin{itemize}
\item Ingestion: Real-time streaming
\item Processing: Apache Spark clusters
\item Storage: Distributed NoSQL
\end{itemize}
\end{column}
\begin{column}{0.45\textwidth}
\textbf{Model Infrastructure:}
\begin{itemize}
\item Training: GPU-accelerated
\item Serving: Kubernetes pods
\item Monitoring: Prometheus metrics
\end{itemize}
\end{column}
\end{columns}

\vfill
\footnotesize
\textbf{Technical Requirements:} Python 3.9+, TensorFlow 2.12, CUDA 11.8, Docker 24.0, Kubernetes 1.28. Memory: 32GB RAM minimum for training, 8GB for inference. Processing: NVIDIA A100 40GB or equivalent for optimal performance. Latency: <100ms p95 for inference API. Throughput: 10,000 requests/second sustained load. Storage: 1TB SSD for model artifacts, 10TB for training data. Network: 10Gbps internal bandwidth.
\end{frame}

% Section 5: Module Performance
\section{Module Performance Analysis}

\begin{frame}[t]
\vfill
\centering
\begin{beamercolorbox}[sep=8pt,center]{title}
\usebeamerfont{title}\Large Module Performance Analysis\par
\end{beamercolorbox}
\vfill
\end{frame}

\begin{frame}[t]{Module Completion Rates}
\centering
\includegraphics[width=0.95\textwidth]{charts/module_completion.pdf}
\vspace{10pt}
\small Weekly breakdown shows consistent engagement across all modules
\vfill
\footnotesize
\textbf{Tracking Methodology:} Real-time learning analytics dashboard. Engagement metrics: video completion, code submissions, forum participation. Early warning system flags at-risk students (<70\% completion by week 3). Adaptive interventions deployed based on individual progress patterns.
\end{frame}

\begin{frame}[t]{Performance Distribution}
\begin{columns}[T]
\begin{column}{0.38\textwidth}
\centering
\textbf{Score Distribution}\\
\includegraphics[width=\textwidth]{charts/score_distribution.pdf}
\end{column}
\begin{column}{0.58\textwidth}
\centering
\textbf{Time Investment}\\
\includegraphics[width=\textwidth]{charts/time_allocation.pdf}
\end{column}
\end{columns}
\end{frame}

% Section 6: Competency Assessment
\section{Final Competency Assessment}

\begin{frame}[t]
\vfill
\centering
\begin{beamercolorbox}[sep=8pt,center]{title}
\usebeamerfont{title}\Large Final Competency Assessment\par
\end{beamercolorbox}
\vfill
\end{frame}

\begin{frame}[t]{Comprehensive Skills Evaluation}
\centering
\includegraphics[width=0.75\textwidth]{charts/competency_radar.pdf}
\vspace{10pt}
\small Multi-dimensional assessment across all learning objectives
\vfill
\footnotesize
\textbf{Competency Framework:} Based on Bloom's revised taxonomy adapted for ML-Design integration. Eight core competencies mapped to industry requirements. 360-degree assessment: self, peer, instructor, and automated evaluation. Minimum threshold: 6/10 per competency for certification.
\end{frame}

% Multiple charts comparison
\begin{frame}[t]{Comparative Analysis: Methods}
\begin{columns}[T]
\begin{column}{0.48\textwidth}
\centering
\textbf{Traditional Approach}\\
\includegraphics[width=\textwidth]{charts/traditional_results.pdf}
\end{column}
\begin{column}{0.48\textwidth}
\centering
\textbf{ML-Enhanced Approach}\\
\includegraphics[width=\textwidth]{charts/ml_enhanced_results.pdf}
\end{column}
\end{columns}
\vspace{5pt}
\begin{columns}[T]
\begin{column}{0.48\textwidth}
\centering
\textbf{Hybrid Model}\\
\includegraphics[width=\textwidth]{charts/hybrid_results.pdf}
\end{column}
\begin{column}{0.48\textwidth}
\centering
\textbf{Overall Comparison}\\
\includegraphics[width=\textwidth]{charts/method_comparison.pdf}
\end{column}
\end{columns}
\end{frame}

% Key metrics with table
\begin{frame}[t]{Performance Metrics Summary}
\small
\begin{table}
\centering
\begin{tabular}{lcccc}
\toprule
Module & Completion & Avg Score & Satisfaction & ML Usage \\
\midrule
Introduction & 100\% & 85\% & 4.2/5 & 60\% \\
Empathy & 98\% & 82\% & 4.5/5 & 75\% \\
Define & 95\% & 78\% & 4.1/5 & 70\% \\
Ideate & 96\% & 88\% & 4.6/5 & 90\% \\
Prototype & 92\% & 75\% & 4.3/5 & 85\% \\
Test & 90\% & 80\% & 4.4/5 & 80\% \\
\bottomrule
\end{tabular}
\end{table}
\vspace{10pt}
\includegraphics[width=0.8\textwidth]{charts/metrics_trend.pdf}
\end{frame}

% Conclusion
\section{Conclusions and Next Steps}

\begin{frame}[t]{Key Takeaways}
\begin{columns}[T]
\begin{column}{0.4\textwidth}
\textbf{Successes:}
\begin{itemize}
\item 92\% average completion rate
\item Strong skill correlation
\item Effective ML integration
\item High student satisfaction
\end{itemize}
\vspace{10pt}
\textbf{Areas for Improvement:}
\begin{itemize}
\item More hands-on practice
\item Industry partnerships
\item Advanced ML topics
\end{itemize}
\end{column}
\begin{column}{0.6\textwidth}
\includegraphics[width=\textwidth]{charts/success_factors.pdf}
\end{column}
\end{columns}
\end{frame}

\begin{frame}[t]{Thank You}
\centering
\Large Questions and Discussion\\
\vspace{20pt}
\normalsize
\textbf{Contact:}\\
prof.osterrieder@university.edu\\
\vspace{10pt}
\textbf{Course Materials:}\\
github.com/ml-design-thinking\\
\vspace{10pt}
\textbf{Next Cohort:}\\
Starting Spring 2025
\end{frame}

\end{document}