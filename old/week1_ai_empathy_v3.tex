\documentclass[8pt,aspectratio=169]{beamer}
\usepackage[utf8]{inputenc}
\usepackage{amsmath,amssymb,amsthm}
\usepackage{graphicx}
\usepackage{tcolorbox}
\usepackage{booktabs}
\usepackage{multicol}
\usepackage{adjustbox}
\usepackage{colortbl}
\usepackage{hyperref}

% Theme
\usetheme{Madrid}
\usecolortheme{seahorse}
\setbeamertemplate{navigation symbols}{}

% Colors
\definecolor{myblue}{RGB}{0,102,204}
\definecolor{mygreen}{RGB}{0,153,76}
\definecolor{myred}{RGB}{204,0,0}
\definecolor{lightblue}{RGB}{173,216,230}
\definecolor{darkgray}{RGB}{64,64,64}

% Commands
\newcommand{\highlight}[1]{\textcolor{myblue}{\textbf{#1}}}
\newcommand{\concept}[1]{\textcolor{mygreen}{\texttt{#1}}}
\newcommand{\emphred}[1]{\textcolor{myred}{\textbf{#1}}}
\newcommand{\term}[1]{\textbf{#1}\footnote{See glossary}}

% Footnote size
\renewcommand{\footnotesize}{\tiny}

% Hyperlink setup
\hypersetup{
    colorlinks=true,
    linkcolor=myblue,
    urlcolor=myblue,
    citecolor=mygreen
}

\title{Week 1: AI as the Empathy Engine}
\subtitle{How ML/AI/GenAI Drives Understanding at Scale}
\author{ML/AI/GenAI-Driven Design Thinking}
\date{}

\begin{document}

% Title Slide
\begin{frame}
\titlepage
\end{frame}

% NEW: ML Pipeline Overview
\begin{frame}[t]{The Machine Learning Pipeline}
\Large\textbf{How AI Systems Learn and Generate Insights}
\vspace{0.5em}

\normalsize
\begin{center}
\begin{tabular}{|c|c|c|c|}
\hline
\cellcolor{lightblue}\textbf{1. Data} & \cellcolor{lightblue}\textbf{2. Training} & \cellcolor{lightblue}\textbf{3. Model} & \cellcolor{lightblue}\textbf{4. Inference} \\
\hline
\begin{tabular}{@{}c@{}}
Collect \\
Clean \\
Prepare
\end{tabular} & 
\begin{tabular}{@{}c@{}}
Algorithm \\
Optimization \\
Validation
\end{tabular} & 
\begin{tabular}{@{}c@{}}
Parameters \\
Weights \\
Structure
\end{tabular} & 
\begin{tabular}{@{}c@{}}
Predictions \\
Insights \\
Actions
\end{tabular} \\
\hline
\end{tabular}
\end{center}

\vspace{0.8em}
\textbf{Key Process Steps:}
\begin{itemize}
\item \textbf{Input}: Raw user data (text, behavior, feedback)
\item \textbf{Processing}: Feature extraction, pattern recognition
\item \textbf{Learning}: Model training on historical data
\item \textbf{Output}: Actionable insights for design decisions
\end{itemize}

\vspace{0.5em}
\begin{tcolorbox}[colback=lightblue!20]
\textbf{Today's Focus}: How this pipeline transforms empathy research
\end{tcolorbox}
\end{frame}

% NEW: 5-Stage Design Innovation Process
\begin{frame}[t]{The 5-Stage Design Innovation Process}
\Large\textbf{Traditional Framework + AI Enhancement}
\vspace{0.5em}

\normalsize
\begin{center}
\begin{tabular}{|l|l|l|}
\hline
\cellcolor{mygreen}\textcolor{white}{\textbf{Stage}} & \cellcolor{mygreen}\textcolor{white}{\textbf{Traditional}} & \cellcolor{mygreen}\textcolor{white}{\textbf{AI-Enhanced}} \\
\hline
\textbf{1. Empathize} & User interviews (n=20) & Analyze millions of interactions \\
\hline
\textbf{2. Define} & Manual synthesis & Pattern recognition algorithms \\
\hline
\textbf{3. Ideate} & Brainstorming sessions & GenAI-powered ideation \\
\hline
\textbf{4. Prototype} & Physical/digital mockups & Rapid AI simulations \\
\hline
\textbf{5. Test} & User testing (n=10) & A/B testing at scale \\
\hline
\end{tabular}
\end{center}

\vspace{0.8em}
\textbf{Week 1 Focus: Empathize Stage}
\begin{itemize}
\item Transform from qualitative $\rightarrow$ quantitative
\item Scale from dozens $\rightarrow$ millions
\item Speed from weeks $\rightarrow$ hours
\item Depth from surface $\rightarrow$ hidden patterns
\end{itemize}
\end{frame}

% Course Overview
\begin{frame}[t]{Course Roadmap}
\begin{center}
\textbf{Where We Are in the 12-Week Journey}
\vspace{0.5em}

\begin{tabular}{|c|c|c|c|}
\hline
\cellcolor{myblue}\textcolor{white}{\textbf{Week 1}} & Week 2 & Week 3 & Week 4 \\
\textbf{Empathy} & Personas & Problems & Ideation \\
\hline
Week 5 & Week 6 & Week 7 & Week 8 \\
Prototyping & Testing & Optimization & Personalization \\
\hline
Week 9 & Week 10 & Week 11 & Week 12 \\
Ethics & Systems & Evolution & Future \\
\hline
\end{tabular}
\end{center}

\vspace{0.5em}
\textbf{Today's Focus}: How AI transforms understanding users from dozens to millions
\end{frame}

% Learning Objectives
\begin{frame}[t]{Today's Learning Objectives}
\Large
\textbf{By the end of today, you will understand:}
\vspace{0.5em}

\normalsize
\begin{enumerate}
\setlength{\itemsep}{0.8em}
\item How AI discovers \highlight{hidden patterns} in user data
\item The power of \highlight{scale} - from 10 to 1,000,000 users
\item \highlight{NLP} techniques that process text automatically
\item How \highlight{GenAI} creates user narratives
\item The \highlight{speed} advantage - weeks to hours
\end{enumerate}

\vspace{1em}
\begin{tcolorbox}[colback=lightblue!20]
\textbf{Key Transformation}: Manual empathy $\rightarrow$ Automated understanding
\end{tcolorbox}
\end{frame}

% Section 1: Introduction
\begin{frame}[c]
\begin{center}
\Huge\textbf{Section 1}\\
\vspace{0.5em}
\Large\textcolor{myblue}{The Paradigm Shift}\\
\vspace{0.3em}
\normalsize From Manual to Machine Understanding
\end{center}
\end{frame}

% The Old Way
\begin{frame}[t]{The Traditional Approach}
\textbf{How We Used to Understand Users:}
\vspace{0.5em}

\begin{itemize}
\setlength{\itemsep}{0.5em}
\item \textbf{In-person interviews}: 20-30 users maximum
\item \textbf{Focus groups}: 8-12 participants
\item \textbf{Surveys}: Low response rates (5-10\%)
\item \textbf{Observation}: Time-intensive shadowing
\item \textbf{Analysis}: Manual coding and themes
\end{itemize}

\vspace{0.5em}
\textbf{Limitations:}
\begin{itemize}
\item \emphred{Small sample sizes} - Statistical uncertainty
\item \emphred{Time consuming} - Weeks of effort
\item \emphred{Expensive} - High cost per insight
\item \emphred{Bias prone} - Interviewer influence
\end{itemize}

\begin{tcolorbox}[colback=lightblue!20]
\textbf{Result}: Good depth, limited breadth
\end{tcolorbox}
\end{frame}

% The New Way - Text-based instead of visual
\begin{frame}[t]{The AI-Powered Revolution: Scale Comparison}
\textbf{Understanding Users at Different Scales:}
\vspace{0.5em}

\begin{center}
\begin{tabular}{|l|r|l|}
\hline
\cellcolor{lightblue}\textbf{Method} & \cellcolor{lightblue}\textbf{Users} & \cellcolor{lightblue}\textbf{Time} \\
\hline
Traditional Interview & 20 & 2 weeks \\
Focus Groups & 50 & 1 week \\
Online Survey & 500 & 3 days \\
\hline
\cellcolor{mygreen}\textcolor{white}{\textbf{AI Analysis}} & \cellcolor{mygreen}\textcolor{white}{\textbf{1,000,000+}} & \cellcolor{mygreen}\textcolor{white}{\textbf{Hours}} \\
\hline
\end{tabular}
\end{center}

\vspace{0.5em}
\textbf{What AI Enables:}
\begin{itemize}
\item \highlight{Massive scale}: Analyze every customer interaction
\item \highlight{Real-time}: Continuous learning and updating
\item \highlight{Unbiased}: No interviewer effect
\item \highlight{Comprehensive}: Find patterns humans miss
\item \highlight{Cost-effective}: Pennies per user analyzed
\end{itemize}

\begin{tcolorbox}[colback=lightblue!20]
\textbf{1 million users} = \textbf{50,000} traditional studies
\end{tcolorbox}
\end{frame}

% Neural Network Learning - Slide 1
\begin{frame}[t]{The Power of Neural Networks: Architecture}
\begin{columns}[T]
\column{0.5\textwidth}
\textbf{How Neural Networks Work:}
\vspace{0.5em}
\begin{itemize}
\setlength{\itemsep}{0.5em}
\item \textbf{Input Layer}: Raw data features
\item \textbf{Hidden Layers}: Pattern extraction
\item \textbf{Output Layer}: Predictions
\item \textbf{Connections}: Weighted links
\end{itemize}

\vspace{0.5em}
\textbf{Key Advantages:}
\begin{itemize}
\item Non-linear pattern recognition
\item Automatic feature learning
\item Scalable to millions of parameters
\end{itemize}

\column{0.5\textwidth}
\includegraphics[width=\textwidth]{charts/neural_network_learning.pdf}
\end{columns}

\vspace{0.3em}
\begin{tcolorbox}[colback=lightblue!20]
\textbf{Neurons mimic the brain}: Each node processes signals and passes them forward
\end{tcolorbox}
\end{frame}

% Neural Network Learning - Slide 2
\begin{frame}[t]{The Power of Neural Networks: Learning Process}
\textbf{How Networks Learn from Data:}
\vspace{0.5em}

\begin{enumerate}
\setlength{\itemsep}{0.5em}
\item \textbf{Forward Pass}: Input flows through network, produces output
\item \textbf{Error Calculation}: Compare output to truth, measure loss
\item \textbf{Backpropagation}: Send error backwards, adjust weights
\item \textbf{Iteration}: Repeat thousands of times until convergence
\end{enumerate}

\vspace{0.5em}
\textbf{Why Deep Learning Wins:}
\begin{columns}[T]
\column{0.48\textwidth}
\textcolor{myred}{\textbf{Traditional ML:}}
\begin{itemize}
\item Manual feature engineering
\item Linear relationships only
\item Limited complexity
\item Plateaus quickly
\end{itemize}

\column{0.48\textwidth}
\textcolor{mygreen}{\textbf{Deep Learning:}}
\begin{itemize}
\item Automatic feature discovery
\item Complex non-linear patterns
\item Unlimited depth
\item Continuous improvement
\end{itemize}
\end{columns}

\vspace{0.3em}
\begin{tcolorbox}[colback=lightblue!20]
\textbf{Result}: 10-50\% accuracy gains on complex pattern recognition tasks
\end{tcolorbox}
\end{frame}

% Section 1 Summary
\begin{frame}[t]{Section 1 Summary}
\Large\textbf{Key Takeaways:}
\vspace{0.5em}

\normalsize
\begin{enumerate}
\setlength{\itemsep}{0.8em}
\item Traditional methods: \highlight{Deep but narrow}
\item AI methods: \highlight{Wide and deep}
\item Speed improvement: \highlight{100x faster}
\item Scale improvement: \highlight{10,000x more users}
\item Cost reduction: \highlight{75\% savings}
\end{enumerate}

\vspace{0.5em}
\begin{center}
\textbf{Next: How pattern recognition works at scale}
\end{center}
\end{frame}

% Section 2: Pattern Recognition
\begin{frame}[c]
\begin{center}
\Huge\textbf{Section 2}\\
\vspace{0.5em}
\Large\textcolor{myblue}{Pattern Recognition at Scale}\\
\vspace{0.3em}
\normalsize Discovering What Humans Can't See
\end{center}
\end{frame}

% What is Pattern Recognition
\begin{frame}[t]{Understanding Pattern Recognition}
\textbf{What is Pattern Recognition?}
\vspace{0.5em}

Finding regularities in data automatically:
\begin{itemize}
\item \textbf{Clustering}: Groups of similar users
\item \textbf{Trends}: Changes over time
\item \textbf{Correlations}: Related behaviors
\item \textbf{Anomalies}: Unusual patterns
\end{itemize}

\vspace{0.5em}
\textbf{Human vs Machine Capabilities:}
\begin{center}
\begin{tabular}{|l|c|c|}
\hline
\textbf{Pattern Type} & \textbf{Human} & \textbf{Machine} \\
\hline
Simple linear & Good & Excellent \\
Complex non-linear & Poor & Excellent \\
High-dimensional & Impossible & Excellent \\
Hidden correlations & Rare & Common \\
\hline
\end{tabular}
\end{center}

\begin{tcolorbox}[colback=lightblue!20]
Machines find patterns in \textbf{milliseconds} that humans might \textbf{never discover}
\end{tcolorbox}
\end{frame}

% Hidden Structure - Slide 1: The Challenge
\begin{frame}[t]{Discovering Hidden Structure: The Challenge}
\textbf{Why Humans Can't See All Patterns:}
\vspace{0.5em}

\begin{itemize}
\setlength{\itemsep}{0.8em}
\item \textbf{Dimensionality Curse}: 
  \begin{itemize}
  \item Humans visualize max 3 dimensions
  \item Real data has 30-1000+ dimensions
  \item Each dimension = one feature/attribute
  \end{itemize}
  
\item \textbf{Non-Linear Relationships}:
  \begin{itemize}
  \item Simple correlations: Easy to spot
  \item Complex interactions: Nearly impossible
  \item Example: Feature A × sin(Feature B) + Feature C²
  \end{itemize}
  
\item \textbf{Scale Limitations}:
  \begin{itemize}
  \item Manual analysis: 100s of data points
  \item ML analysis: Millions of data points
  \item Pattern significance emerges at scale
  \end{itemize}
\end{itemize}

\vspace{0.5em}
\begin{tcolorbox}[colback=lightblue!20]
\textbf{The Hidden 97\%}: Studies show humans miss 97\% of complex patterns in high-dimensional data
\end{tcolorbox}
\end{frame}

% Hidden Structure - Slide 2: t-SNE Visualization
\begin{frame}[t]{Discovering Hidden Structure: t-SNE Magic}
\begin{center}
\includegraphics[width=0.85\textwidth]{charts/tsne_visualization.pdf}
\end{center}

\vspace{-0.3em}
\textbf{t-SNE (t-distributed Stochastic Neighbor Embedding):}
\begin{itemize}
\item Reduces 50 dimensions → 2 dimensions while preserving structure
\item Reveals 8 distinct clusters invisible in raw data
\item Each cluster = different user behavior pattern
\end{itemize}
\end{frame}

% Hidden Structure - Slide 3: Information Discovery
\begin{frame}[t]{Discovering Hidden Structure: Information Metrics}
\textbf{Quantifying Pattern Discovery:}
\vspace{0.5em}

\begin{columns}[T]
\column{0.5\textwidth}
\textbf{Mutual Information Analysis:}
\begin{itemize}
\item Total information: 14.4 bits
\item Top 5 features: 3.8 bits
\item Hidden correlations: 10.6 bits
\end{itemize}

\vspace{0.5em}
\textbf{Pattern Detection Rates:}
\begin{itemize}
\item Visual inspection: 2/50 patterns
\item Statistical tests: 8/50 patterns
\item ML algorithms: 47/50 patterns
\end{itemize}

\column{0.5\textwidth}
\textbf{Discovery Speed:}
\begin{itemize}
\item Manual: 120 minutes → 7 bits
\item ML: 50 minutes → 13 bits (90\%)
\item Deep Learning: 2 minutes → 14 bits
\end{itemize}

\vspace{0.5em}
\textbf{Business Impact:}
\begin{itemize}
\item Each bit = actionable insight
\item 14.4 bits = 14 major findings
\item Worth \$100K+ in user understanding
\end{itemize}
\end{columns}

\vspace{0.5em}
\begin{tcolorbox}[colback=lightblue!20]
\textbf{Key Insight}: ML discovers 23.5× more patterns, 60× faster than manual analysis
\end{tcolorbox}
\end{frame}

% User Segmentation Example - Slide 1
\begin{frame}[t]{Example: AI-Powered User Segmentation Discovery}
\begin{center}
\includegraphics[width=0.8\textwidth]{charts/clustering_results.pdf}
\end{center}

\vspace{-0.3em}
\textbf{Real Results from 10,000 Users × 30 Behavioral Features:}
\begin{itemize}
\item Manual observation: Found 3 obvious segments
\item K-means clustering: Discovered 12 distinct segments
\item Quality improvement: 591\% better separation (silhouette score)
\end{itemize}
\end{frame}

% User Segmentation Example - Slide 2
\begin{frame}[t]{Example: Segment Insights and Business Value}
\textbf{What the 12 Discovered Segments Reveal:}
\vspace{0.5em}

\begin{columns}[T]
\column{0.48\textwidth}
\textbf{Segment Examples:}
\begin{itemize}
\setlength{\itemsep}{0.3em}
\item \textbf{Power Users} (8\%): High engagement, all features
\item \textbf{Mobile-Only} (15\%): Never use desktop
\item \textbf{Weekend Warriors} (12\%): Sat/Sun only
\item \textbf{Quick Checkers} (20\%): <30 sec sessions
\item \textbf{Data Explorers} (7\%): Export heavy users
\item \textbf{Social Sharers} (10\%): High viral coefficient
\end{itemize}

\column{0.48\textwidth}
\textbf{Business Actions:}
\begin{itemize}
\setlength{\itemsep}{0.3em}
\item Personalized onboarding paths
\item Segment-specific features
\item Targeted pricing strategies
\item Custom retention programs
\item Predictive churn models
\item Cross-sell opportunities
\end{itemize}
\end{columns}

\vspace{0.5em}
\textbf{Value Created:}
\begin{itemize}
\item 34\% increase in conversion (segment-specific messaging)
\item 28\% reduction in churn (targeted retention)
\item 45\% higher LTV (personalized upsells)
\end{itemize}

\begin{tcolorbox}[colback=lightblue!20]
\textbf{ROI}: Each micro-segment insight worth \$50K-200K in annual revenue
\end{tcolorbox}
\end{frame}

% Hidden Insights
\begin{frame}[t]{Discovering Hidden Insights}
\textbf{What AI Reveals That Humans Miss:}
\vspace{0.5em}

\begin{enumerate}
\setlength{\itemsep}{0.5em}
\item \textbf{Micro-segments}: Groups of 50-100 users with unique needs
\item \textbf{Temporal patterns}: Usage spikes at 3:17 AM
\item \textbf{Cross-correlations}: Feature A users love Feature Z
\item \textbf{Sentiment shifts}: Gradual opinion changes
\item \textbf{Predictive signals}: Early warning signs
\end{enumerate}

\vspace{0.5em}
\textbf{Real Case Study:}
\begin{tcolorbox}[colback=lightblue!20]
E-commerce site discovered 127 micro-personas vs 5 manual ones\\
Result: 34\% increase in conversion rate
\end{tcolorbox}
\end{frame}

% Section 2 Summary
\begin{frame}[t]{Section 2 Summary}
\Large\textbf{Pattern Recognition Enables:}
\vspace{0.5em}

\normalsize
\begin{itemize}
\setlength{\itemsep}{0.8em}
\item Finding \highlight{invisible connections}
\item Discovering \highlight{micro-segments}
\item Detecting \highlight{weak signals}
\item Predicting \highlight{future behaviors}
\item Revealing \highlight{counter-intuitive insights}
\end{itemize}

\vspace{0.5em}
\begin{center}
\textbf{Next: Transforming raw data into actionable insights}
\end{center}
\end{frame}

% Section 3: From Data to Insights
\begin{frame}[c]
\begin{center}
\Huge\textbf{Section 3}\\
\vspace{0.5em}
\Large\textcolor{myblue}{From Data to Insights}\\
\vspace{0.3em}
\normalsize The NLP Processing Pipeline
\end{center}
\end{frame}

% NLP Pipeline - Text-based
\begin{frame}[t]{The NLP Processing Pipeline}
\textbf{How AI Processes Text Data:}
\vspace{0.5em}

\begin{enumerate}
\setlength{\itemsep}{0.5em}
\item \textbf{Data Collection}
   \begin{itemize}
   \item Reviews, feedback, support tickets
   \item Social media, forums, surveys
   \end{itemize}
   
\item \textbf{Preprocessing}
   \begin{itemize}
   \item Tokenization: Split into words/phrases
   \item Cleaning: Remove noise, normalize text
   \end{itemize}
   
\item \textbf{Analysis}
   \begin{itemize}
   \item Sentiment: Positive/negative/neutral
   \item Topics: Main themes and categories
   \item Entities: People, products, features
   \end{itemize}
   
\item \textbf{Insights}
   \begin{itemize}
   \item Trends, patterns, recommendations
   \item Actionable design decisions
   \end{itemize}
\end{enumerate}

\begin{tcolorbox}[colback=lightblue!20]
\textbf{10,000 reviews} $\rightarrow$ \textbf{50 insights} in \textbf{minutes}
\end{tcolorbox}
\end{frame}

% Emotion Analysis Results
\begin{frame}[t]{Understanding Human Emotions at Scale}
\begin{center}
\includegraphics[width=0.9\textwidth]{charts/emotion_wheel.pdf}
\end{center}

\vspace{-0.3em}
\textbf{AI understands complex human emotions at unprecedented scale}
\end{frame}

% Topic Modeling Results
\begin{frame}[t]{Topic Discovery via LDA}
\begin{center}
\includegraphics[width=0.9\textwidth]{charts/topic_modeling.pdf}
\end{center}

\vspace{-0.3em}
\textbf{8 major themes automatically extracted from 5000 documents}
\end{frame}

% From Numbers to Stories
\begin{frame}[t]{From Numbers to Narratives}
\textbf{How \term{GenAI} Creates User Stories:}
\vspace{0.5em}

\textbf{Input}: 10,000 data points about User Segment A

\textbf{Output}: Generated user narrative:
\begin{tcolorbox}[colback=lightblue!20]
\textit{``Sarah, 34, values efficiency above all. She uses the app during her commute (7:15-7:45 AM) and lunch break. Frustrated by multi-step processes. Loves quick actions and keyboard shortcuts. Would pay for time-saving features.''}
\end{tcolorbox}

\textbf{Benefits:}
\begin{itemize}
\item Makes data \highlight{relatable}
\item Creates \highlight{empathy}
\item Guides \highlight{design decisions}
\item Communicates \highlight{insights clearly}
\end{itemize}
\end{frame}

% Section 3 Summary
\begin{frame}[t]{Section 3 Summary}
\Large\textbf{Data to Insights Pipeline:}
\vspace{0.5em}

\normalsize
\begin{enumerate}
\setlength{\itemsep}{0.8em}
\item Raw text $\rightarrow$ \highlight{Structured data}
\item Sentiment $\rightarrow$ \highlight{Emotional understanding}
\item Topics $\rightarrow$ \highlight{Main concerns}
\item Patterns $\rightarrow$ \highlight{User behaviors}
\item Numbers $\rightarrow$ \highlight{Human stories}
\end{enumerate}

\vspace{0.5em}
\begin{center}
\textbf{Next: AI as a creative partner in design}
\end{center}
\end{frame}

% Section 4: AI as Creative Partner
\begin{frame}[c]
\begin{center}
\Huge\textbf{Section 4}\\
\vspace{0.5em}
\Large\textcolor{myblue}{AI as Creative Partner}\\
\vspace{0.3em}
\normalsize Beyond Analysis to Generation
\end{center}
\end{frame}

% Generative Capabilities
\begin{frame}[t]{Generative AI in Design Thinking}
\textbf{What Can GenAI Create?}
\vspace{0.5em}

\begin{itemize}
\setlength{\itemsep}{0.5em}
\item \textbf{User Personas}: Data-driven profiles
\item \textbf{Journey Maps}: Automated path analysis
\item \textbf{Problem Statements}: Synthesized challenges
\item \textbf{Solution Ideas}: Creative concepts
\item \textbf{Prototypes}: Quick mockups and flows
\end{itemize}

\vspace{0.5em}
\textbf{The Creative Loop:}
\begin{enumerate}
\item Analyze user data
\item Generate hypotheses
\item Create solutions
\item Simulate outcomes
\item Iterate rapidly
\end{enumerate}

\begin{tcolorbox}[colback=lightblue!20]
GenAI doesn't replace creativity - it \textbf{amplifies} it
\end{tcolorbox}
\end{frame}

% Deep Learning Advantage
\begin{frame}[t]{The Deep Learning Revolution}
\begin{center}
\includegraphics[width=0.9\textwidth]{charts/deep_learning_advantage.pdf}
\end{center}

\vspace{-0.3em}
\textbf{Deep learning continuously improves with more data}
\end{frame}

% Hypothesis Generation
\begin{frame}[t]{AI-Generated Hypotheses}
\textbf{From Patterns to Testable Ideas:}
\vspace{0.5em}

\textbf{Pattern Found}: Users abandon cart at shipping

\textbf{AI Hypotheses}:
\begin{enumerate}
\item Price sensitivity at \$8.99 threshold
\item International users see high shipping
\item Mobile users can't find shipping info
\item Premium users expect free shipping
\end{enumerate}

\textbf{AI Suggests Tests}:
\begin{itemize}
\item A/B test free shipping threshold
\item Geo-targeted shipping messages
\item Mobile UI shipping visibility
\item Premium tier shipping benefits
\end{itemize}

\begin{tcolorbox}[colback=lightblue!20]
Each hypothesis backed by \textbf{data from thousands} of users
\end{tcolorbox}
\end{frame}

% Section 4 Summary
\begin{frame}[t]{Section 4 Summary}
\Large\textbf{AI as Creative Partner:}
\vspace{0.5em}

\normalsize
\begin{itemize}
\setlength{\itemsep}{0.8em}
\item Generates \highlight{data-driven personas}
\item Creates \highlight{testable hypotheses}
\item Suggests \highlight{solution concepts}
\item Simulates \highlight{user reactions}
\item Accelerates \highlight{iteration cycles}
\end{itemize}

\vspace{0.5em}
\begin{center}
\textbf{Next: Implementation and ethical considerations}
\end{center}
\end{frame}

% Section 5: Implementation and Ethics
\begin{frame}[c]
\begin{center}
\Huge\textbf{Section 5}\\
\vspace{0.5em}
\Large\textcolor{myblue}{Implementation \& Ethics}\\
\vspace{0.3em}
\normalsize Responsible AI-Driven Empathy
\end{center}
\end{frame}

% Getting Started
\begin{frame}[t]{Getting Started with AI Empathy}
\textbf{Your 90-Day Implementation Roadmap:}
\vspace{0.5em}

\begin{enumerate}
\setlength{\itemsep}{0.5em}
\item \textbf{Days 1-7: Start Small \& Quick Win}
   \begin{itemize}
   \item Pick ONE data source (e.g., app reviews)
   \item Run sentiment analysis (2 hours setup)
   \item Share 3 surprising insights with team
   \end{itemize}
   
\item \textbf{Days 8-30: Tool Selection \& Setup}
   \begin{itemize}
   \item Cloud: Start with Google Colab (free)
   \item Models: Hugging Face pre-trained (BERT)
   \item Stack: Python + scikit-learn + pandas
   \end{itemize}
   
\item \textbf{Days 31-60: Team Enablement}
   \begin{itemize}
   \item 2-day ML workshop for team
   \item Hire/partner with 1 data scientist
   \item Create first automated dashboard
   \end{itemize}
   
\item \textbf{Days 61-90: Scale \& Optimize}
   \begin{itemize}
   \item Connect 3+ data sources
   \item Automate daily insights email
   \item Launch first ML-driven feature
   \end{itemize}
\end{enumerate}

\textbf{Budget Estimate:} \$5K (tools) + \$10K (training) + \$15K (consultant) = \$30K

\begin{tcolorbox}[colback=lightblue!20]
\textbf{Success Metric}: 10 actionable insights/week by Day 90
\end{tcolorbox}
\end{frame}

% ML Performance Over Time
\begin{frame}[t]{ML Model Performance Evolution}
\begin{center}
\includegraphics[width=0.9\textwidth]{charts/ml_accuracy_timeline.pdf}
\end{center}

\vspace{-0.3em}
\textbf{Models improve continuously with more data and iterations}
\end{frame}

% Cost-Benefit Analysis
\begin{frame}[t]{ROI of AI-Driven Empathy}
\textbf{The Economics of ML-Powered User Understanding:}
\vspace{0.3em}

\begin{itemize}
\item \textbf{Setup Investment}: \$50K initial + \$2K/month operations
\item \textbf{Break-even}: Month 6 (vs traditional research costs)
\item \textbf{Year 1 ROI}: 180\% | Year 2 ROI: 340\%
\end{itemize}

\vspace{0.3em}
\begin{center}
\includegraphics[width=0.85\textwidth]{charts/cost_analysis.pdf}
\end{center}

\vspace{-0.3em}
\textbf{Key Metrics Achieved:}
\begin{columns}[T]
\column{0.33\textwidth}
\centering
\textbf{84× faster}\\
2 weeks → 4 hours
\column{0.33\textwidth}
\centering
\textbf{5000× scale}\\
20 → 100K users
\column{0.33\textwidth}
\centering
\textbf{500× cheaper}\\
\$1500 → \$3/insight
\end{columns}
\end{frame}

% Ethical Considerations
\begin{frame}[t]{Ethical AI Implementation}
\textbf{Building Trust Through Responsible AI:}
\vspace{0.3em}

\textit{With great data comes great responsibility - use ML ethically}
\vspace{0.5em}

\begin{itemize}
\setlength{\itemsep}{0.5em}
\item \textbf{Privacy First}
   \begin{itemize}
   \item Anonymize user data using differential privacy
   \item Follow GDPR/CCPA regulations strictly
   \item Data minimization: Collect only what's needed
   \end{itemize}
   
\item \textbf{Bias Detection and Mitigation}
   \begin{itemize}
   \item Run fairness audits on all models
   \item Check for demographic skews monthly
   \item Validate with diverse user groups (n>1000)
   \end{itemize}
   
\item \textbf{Radical Transparency}
   \begin{itemize}
   \item Explain AI decisions in plain language
   \item Show confidence levels (73\% certain)
   \item Publish model cards and limitations
   \end{itemize}
   
\item \textbf{Human-AI Partnership}
   \begin{itemize}
   \item Keep humans in critical decision loops
   \item Validate AI insights with user interviews
   \item Override capability for edge cases
   \end{itemize}
\end{itemize}

\begin{tcolorbox}[colback=lightblue!20]
\emphred{Golden Rule}: If you wouldn't want it done to your data, don't do it to theirs
\end{tcolorbox}
\end{frame}

% Best Practices
\begin{frame}[t]{Best Practices for AI Empathy}
\textbf{Learn from Those Who've Succeeded (and Failed):}
\vspace{0.5em}

\begin{columns}[T]
\column{0.48\textwidth}
\textcolor{mygreen}{\textbf{DO - Success Stories:}}
\begin{itemize}
\item \textbf{Validate}: Spotify tests every insight
\item \textbf{Combine}: Netflix uses ML + focus groups
\item \textbf{Update}: Amazon retrains daily
\item \textbf{Document}: Google publishes papers
\item \textbf{Test}: Microsoft A/B tests everything
\end{itemize}

\column{0.48\textwidth}
\textcolor{myred}{\textbf{DON'T - Cautionary Tales:}}
\begin{itemize}
\item \textbf{Trust blindly}: Target pregnancy prediction
\item \textbf{Ignore minorities}: Face recognition bias
\item \textbf{Skip validation}: Chatbot disasters
\item \textbf{Assume causation}: Ice cream and crime
\item \textbf{Forget context}: Cultural insensitivity
\end{itemize}
\end{columns}

\vspace{0.5em}
\textbf{Real Examples of What Works:}
\begin{itemize}
\item Airbnb: ML found hosts prefer Sunday check-ins (27\% higher acceptance)
\item Duolingo: AI discovered 3:00 PM reminders get 43\% better engagement
\item Pinterest: Algorithm identified ``DIY Wedding'' micro-trend 3 months early
\end{itemize}

\begin{tcolorbox}[colback=lightblue!20]
\textbf{Success Formula}: Start small + Measure everything + Iterate fast = Win
\end{tcolorbox}
\end{frame}

% Section 5 Summary
\begin{frame}[t]{Section 5 Summary}
\Large\textbf{Implementation Success Factors:}
\vspace{0.5em}

\normalsize
\begin{enumerate}
\setlength{\itemsep}{0.8em}
\item Start small, \highlight{scale gradually}
\item Maintain \highlight{ethical standards}
\item Keep \highlight{humans in loop}
\item Validate \highlight{continuously}
\item Measure \highlight{ROI clearly}
\end{enumerate}

\vspace{0.5em}
\begin{center}
\textbf{Ready to transform your design process!}
\end{center}
\end{frame}

% Key Formulas
\begin{frame}[t]{Key Formulas to Remember}
\textbf{Essential Mathematical Concepts:}
\vspace{0.5em}

\begin{itemize}
\setlength{\itemsep}{0.5em}
\item \textbf{Clustering Distance}: $d = \sqrt{\sum_{i=1}^{n}(x_i - y_i)^2}$
   \begin{itemize}
   \item Measures similarity between users
   \end{itemize}
   
\item \textbf{Sentiment Score}: $S = \frac{\text{Positive} - \text{Negative}}{\text{Total}}$
   \begin{itemize}
   \item Quantifies overall feeling
   \end{itemize}
   
\item \textbf{Topic Probability}: $P(\text{topic}|\text{document})$
   \begin{itemize}
   \item How likely document belongs to topic
   \end{itemize}
   
\item \textbf{Accuracy}: $\frac{\text{Correct Predictions}}{\text{Total Predictions}} \times 100$
   \begin{itemize}
   \item Model performance metric
   \end{itemize}
\end{itemize}

\begin{tcolorbox}[colback=lightblue!20]
Don't memorize - \textbf{understand the concept}
\end{tcolorbox}
\end{frame}

% Wrap-up
\begin{frame}[t]{Today's Journey: From Manual to Machine}
\Large\textbf{What We Learned:}
\vspace{0.5em}

\normalsize
\begin{enumerate}
\setlength{\itemsep}{0.5em}
\item \highlight{Scale}: 20 users $\rightarrow$ 1,000,000 users
\item \highlight{Speed}: 2 weeks $\rightarrow$ 6 hours
\item \highlight{Depth}: Surface $\rightarrow$ Hidden patterns
\item \highlight{Cost}: \$30,000 $\rightarrow$ \$7,500
\item \highlight{Insights}: 5 personas $\rightarrow$ 127 micro-segments
\end{enumerate}

\vspace{0.5em}
\textbf{The Transformation:}
\begin{tcolorbox}[colback=lightblue!20]
\textbf{Before}: ``We think users want X''\\
\textbf{After}: ``Data shows 73\% of Segment A needs Y''
\end{tcolorbox}

\vspace{0.5em}
\textbf{Next Week}: Building AI-Driven Personas
\end{frame}

% References with Links
\begin{frame}[t]{References and Resources}
\textbf{Academic Papers:}
\begin{itemize}
\small
\item BERT: \href{https://arxiv.org/abs/1810.04805}{arxiv.org/abs/1810.04805}
\item Attention Is All You Need: \href{https://arxiv.org/abs/1706.03762}{arxiv.org/abs/1706.03762}
\item LDA Original Paper: \href{https://www.jmlr.org/papers/v3/blei03a.html}{jmlr.org/papers/v3/blei03a.html}
\end{itemize}

\textbf{Courses \& Tutorials:}
\begin{itemize}
\small
\item Andrew Ng's ML Course: \href{https://www.coursera.org/learn/machine-learning}{coursera.org/learn/machine-learning}
\item Fast.ai Practical Deep Learning: \href{https://www.fast.ai/}{fast.ai}
\item Google ML Crash Course: \href{https://developers.google.com/machine-learning}{developers.google.com/machine-learning}
\end{itemize}

\textbf{Tools \& Platforms:}
\begin{itemize}
\small
\item Hugging Face Models: \href{https://huggingface.co/}{huggingface.co}
\item Google What-If Tool: \href{https://pair-code.github.io/what-if-tool/}{pair-code.github.io/what-if-tool}
\item Kaggle Datasets: \href{https://www.kaggle.com/}{kaggle.com}
\end{itemize}

\textbf{Design Thinking:}
\begin{itemize}
\small
\item IDEO Design Thinking: \href{https://www.ideo.com/post/design-thinking}{ideo.com/post/design-thinking}
\item Stanford d.school: \href{https://dschool.stanford.edu/}{dschool.stanford.edu}
\end{itemize}
\end{frame}

% Final Call to Action
\begin{frame}[c]
\begin{center}
\Huge\textbf{Your Turn!}\\
\vspace{1em}
\Large Start with one dataset.\\
\vspace{0.5em}
Find one pattern.\\
\vspace{0.5em}
Generate one insight.\\
\vspace{1em}
\normalsize\textcolor{myblue}{\textbf{ML/AI/GenAI is transforming design innovation from intuition to intelligence.}}
\end{center}
\end{frame}

\end{document}