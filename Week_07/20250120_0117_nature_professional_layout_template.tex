\documentclass[8pt,aspectratio=169]{beamer}
\usetheme{Madrid}
\usepackage[utf8]{inputenc}
\usepackage[english]{babel}
\usepackage{amsmath,amssymb,amsthm}
\usepackage{graphicx}
\usepackage{tikz}
\graphicspath{{figures/}}

% Nature Professional Color Palette
\definecolor{ForestGreen}{RGB}{20,83,45}      % #14532D - Primary dark forest green
\definecolor{Teal}{RGB}{13,148,136}           % #0D9488 - Secondary natural teal
\definecolor{Amber}{RGB}{245,158,11}          % #F59E0B - Accent warm amber
\definecolor{Slate}{RGB}{71,85,105}           % #475569 - Support slate gray
\definecolor{MintCream}{RGB}{240,253,244}     % #F0FDF4 - Background mint cream
\definecolor{LightGreen}{RGB}{134,239,172}    % #86EFAC - Light green
\definecolor{DarkTeal}{RGB}{15,118,110}       % #0F766E - Dark teal variant
\definecolor{LightAmber}{RGB}{252,211,77}     % #FCD34D - Light amber

% Apply Nature Professional colors to beamer
\setbeamercolor{structure}{fg=ForestGreen}
\setbeamercolor{normal text}{fg=ForestGreen,bg=white}
\setbeamercolor{frametitle}{fg=ForestGreen,bg=LightGreen!20}
\setbeamercolor{background canvas}{bg=MintCream}

% Navigation and decoration
\setbeamertemplate{navigation symbols}{}
\setbeamertemplate{footline}[frame number]
\setbeamerfont{frametitle}{series=\bfseries,size=\large}

% Blocks with nature theme
\setbeamertemplate{blocks}[default]
\setbeamercolor{block title}{fg=white,bg=Teal}
\setbeamercolor{block body}{fg=ForestGreen,bg=MintCream}
\setbeamercolor{block title alerted}{fg=white,bg=Amber}
\setbeamercolor{block body alerted}{fg=ForestGreen,bg=LightAmber!20}
\setbeamercolor{block title example}{fg=white,bg=ForestGreen}
\setbeamercolor{block body example}{fg=ForestGreen,bg=LightGreen!20}

% Items with nature accent colors
\setbeamercolor{item}{fg=Amber}
\setbeamercolor{subitem}{fg=Teal}
\setbeamercolor{subsubitem}{fg=DarkTeal}

\title{\textcolor{ForestGreen}{Presentation Template}}
\subtitle{\textcolor{Teal}{Nature Professional Layout Examples}}
\author{\textcolor{Slate}{Author Name}}
\date{}

\begin{document}

% ==================== LAYOUT 1: PLAIN TITLE ====================
\begin{frame}[plain]
\vspace{2cm}
\begin{center}
{\Huge\textcolor{ForestGreen}{Main Title}}\\[0.5cm]
{\Large\textcolor{Teal}{Subtitle or Description}}\\[2cm]
{\normalsize\textcolor{Amber}{Additional Information}}
\end{center}
\end{frame}

% ==================== LAYOUT 2: STANDARD TITLE ====================
\begin{frame}[plain]
\titlepage
\end{frame}

% ==================== LAYOUT 3: TWO COLUMNS TEXT ====================
\begin{frame}{Two Column Layout - Text}
\begin{columns}[T]
\column{0.48\textwidth}
\textcolor{ForestGreen}{\textbf{Left Column Header}}

Main content for the left side. This is where your primary information goes.

\textcolor{Amber}{Key points:}
\begin{itemize}
\item First point
\item Second point
\item Third point with more text
\item Fourth point
\end{itemize}

Additional paragraph text can go here to provide more context or explanation.

\column{0.48\textwidth}
\textcolor{ForestGreen}{\textbf{Right Column Header}}

Supporting content or contrasting information for the right side.

\textcolor{Teal}{Related items:}
\begin{itemize}
\item Supporting point one
\item Supporting point two
\item Supporting point three
\end{itemize}

More descriptive text that complements the left column content.
\end{columns}

\vspace{\fill}
\small\textcolor{Slate}{Bottom annotation: Additional notes, references, or key takeaways}
\end{frame}

% ==================== LAYOUT 4: TWO COLUMNS WITH MATH ====================
\begin{frame}{Two Column Layout - Mathematics}
\begin{columns}[T]
\column{0.48\textwidth}
\textcolor{ForestGreen}{\textbf{Definition}}

A mathematical concept defined:
$$\textcolor{Teal}{f(x) = ax^2 + bx + c}$$

\textcolor{Amber}{Properties:}
\begin{itemize}
\item Property one: $a \neq 0$
\item Property two: Vertex at $x = -\frac{b}{2a}$
\item Property three: Discriminant $\Delta = b^2 - 4ac$
\end{itemize}

\column{0.48\textwidth}
\textcolor{ForestGreen}{\textbf{Example}}

Specific instance:
$$\textcolor{DarkTeal}{f(x) = 2x^2 + 3x + 1}$$

\textcolor{Teal}{Calculation:}
\begin{align*}
f'(x) &= 4x + 3 \\
f'(0) &= 3 \\
f''(x) &= 4
\end{align*}

Result: Minimum at $x = -\frac{3}{4}$
\end{columns}

\vspace{\fill}
\small\textcolor{Slate}{Mathematical concepts are best understood through both theory and examples}
\end{frame}

% ==================== LAYOUT 5: COLUMNS WITH LIST VARIATIONS ====================
\begin{frame}{List Variations}
\begin{columns}[T]
\column{0.48\textwidth}
\textcolor{ForestGreen}{\textbf{Enumerated List}}
\begin{enumerate}
\item First step in process
\item Second step with details
\item Third step
\begin{itemize}
\item Sub-point A
\item Sub-point B
\end{itemize}
\item Final step
\end{enumerate}

\vspace{0.5em}
\textcolor{Teal}{\textbf{Bullet Points}}
\begin{itemize}
\item Main concept
\item Supporting idea
\item Additional thought
\end{itemize}

\column{0.48\textwidth}
\textcolor{ForestGreen}{\textbf{Mixed Content}}

Paragraph text introducing a concept.

\textcolor{Amber}{Key formulas:}
\begin{itemize}
\item Linear: $y = mx + b$
\item Quadratic: $y = ax^2 + bx + c$
\item Exponential: $y = ae^{bx}$
\end{itemize}

Concluding remarks about the formulas and their applications in real-world scenarios.
\end{columns}
\end{frame}

% ==================== LAYOUT 6: THREE-WAY SPLIT ====================
\begin{frame}{Three Column Layout}
\begin{columns}[T]
\column{0.31\textwidth}
\textcolor{ForestGreen}{\textbf{Category A}}

Content for first category:
\begin{itemize}
\item Item 1
\item Item 2
\item Item 3
\end{itemize}

Additional notes about this category.

\column{0.31\textwidth}
\textcolor{Teal}{\textbf{Category B}}

Content for second category:
\begin{itemize}
\item Item 1
\item Item 2
\item Item 3
\end{itemize}

Additional notes about this category.

\column{0.31\textwidth}
\textcolor{Amber}{\textbf{Category C}}

Content for third category:
\begin{itemize}
\item Item 1
\item Item 2
\item Item 3
\end{itemize}

Additional notes about this category.
\end{columns}

\vspace{\fill}
\small\textcolor{Slate}{Three columns work well for comparisons or related concepts}
\end{frame}

% ==================== LAYOUT 7: FULL WIDTH WITH IMAGE SPACE ====================
\begin{frame}{Full Width Content with Image}
\textcolor{ForestGreen}{\textbf{Main Topic Introduction}}

This layout provides space for a full-width explanation followed by an image or chart.

\textcolor{Teal}{Key concepts to understand:}
\begin{itemize}
\item Concept one with brief explanation
\item Concept two with additional details
\item Concept three relating to the visual below
\end{itemize}

\vspace{0.5em}
\begin{center}
% Space for image/chart
\framebox[0.9\textwidth][c]{
\vspace{3cm}
\textcolor{Slate}{[Image/Chart Placeholder]}
\vspace{3cm}
}
\end{center}

\vspace{\fill}
\small\textcolor{Slate}{Visuals should complement and enhance the textual content}
\end{frame}

% ==================== LAYOUT 8: COLUMNS WITH IMAGE ====================
\begin{frame}{Mixed Media Layout}
\begin{columns}[T]
\column{0.48\textwidth}
\textcolor{ForestGreen}{\textbf{Text Content}}

Explanation of concept with supporting details.

\textcolor{Amber}{Important points:}
\begin{itemize}
\item First observation
\item Second observation
\item Third observation
\item Conclusion
\end{itemize}

Formula if needed:
$$\textcolor{Teal}{E = mc^2}$$

\column{0.48\textwidth}
\begin{center}
% Space for image
\framebox[0.9\columnwidth][c]{
\vspace{4cm}
\textcolor{Slate}{[Visual Element]}
\vspace{4cm}
}
\end{center}
\end{columns}

\vspace{\fill}
\small\textcolor{Slate}{Combine text and visuals for maximum impact}
\end{frame}

% ==================== LAYOUT 9: DEFINITION-EXAMPLE PATTERN ====================
\begin{frame}{Definition and Examples}
\begin{columns}[T]
\column{0.48\textwidth}
\textcolor{ForestGreen}{\textbf{Definition}}

Formal statement of concept or theorem.

\vspace{0.5em}
\textcolor{Teal}{\textbf{Properties}}
\begin{itemize}
\item Property 1
\item Property 2
\item Property 3
\end{itemize}

\vspace{0.5em}
\textcolor{DarkTeal}{\textbf{Conditions}}
\begin{itemize}
\item Must satisfy A
\item Must satisfy B
\end{itemize}

\column{0.48\textwidth}
\textcolor{Amber}{\textbf{Example 1}}

Concrete instance demonstrating the concept.

Details:
\begin{itemize}
\item Specific value: 42
\item Result: Valid
\end{itemize}

\vspace{0.5em}
\textcolor{Amber}{\textbf{Example 2}}

Another instance showing different aspect.

Details:
\begin{itemize}
\item Specific value: -5
\item Result: Invalid
\end{itemize}
\end{columns}

\vspace{\fill}
\small\textcolor{Slate}{Definitions paired with examples aid understanding}
\end{frame}

% ==================== LAYOUT 10: COMPARISON TABLE ====================
\begin{frame}{Comparison Layout}
\begin{columns}[T]
\column{0.48\textwidth}
\textcolor{ForestGreen}{\textbf{Method A}}
\begin{itemize}
\item Advantage 1
\item Advantage 2
\item Advantage 3
\end{itemize}

\textcolor{Amber}{\textbf{Disadvantages}}
\begin{itemize}
\item Limitation 1
\item Limitation 2
\end{itemize}

\textcolor{Teal}{\textbf{Best for:}} Scenario type X

\column{0.48\textwidth}
\textcolor{ForestGreen}{\textbf{Method B}}
\begin{itemize}
\item Advantage 1
\item Advantage 2
\item Advantage 3
\end{itemize}

\textcolor{Amber}{\textbf{Disadvantages}}
\begin{itemize}
\item Limitation 1
\item Limitation 2
\end{itemize}

\textcolor{Teal}{\textbf{Best for:}} Scenario type Y
\end{columns}

\vspace{\fill}
\small\textcolor{Slate}{Direct comparisons help in decision making}
\end{frame}

% ==================== LAYOUT 11: PROGRESSIVE REVEAL ====================
\begin{frame}{Step-by-Step Process}
\begin{columns}[T]
\column{0.48\textwidth}
\textcolor{ForestGreen}{\textbf{Initial State}}

Description of starting point:
\begin{itemize}
\item \textcolor{Teal}{Given:} Input data
\item \textcolor{Teal}{Goal:} Desired output
\item \textcolor{Amber}{Constraint:} Time limit
\end{itemize}

\vspace{0.5em}
\textcolor{ForestGreen}{\textbf{Step 1: Preparation}}

Actions taken in first step.

\vspace{0.5em}
\textcolor{ForestGreen}{\textbf{Step 2: Execution}}

Main processing occurs here.

\column{0.48\textwidth}
\textcolor{DarkTeal}{\textbf{Step 3: Refinement}}

Optimization and adjustments.

\vspace{0.5em}
\textcolor{DarkTeal}{\textbf{Step 4: Validation}}

Check results against criteria.

\vspace{0.5em}
\textcolor{Amber}{\textbf{Final State}}

Description of outcome:
\begin{itemize}
\item Result: Success
\item Time: 2.3 seconds
\item Accuracy: 99.5\%
\end{itemize}
\end{columns}

\vspace{\fill}
\small\textcolor{Slate}{Step-by-step breakdowns clarify complex processes}
\end{frame}

% ==================== LAYOUT 12: FORMULA COLLECTION ====================
\begin{frame}{Formula Reference}
\begin{columns}[T]
\column{0.31\textwidth}
\textcolor{ForestGreen}{\textbf{Category 1}}

Basic formulas:
$$\textcolor{Teal}{a + b = c}$$
$$\textcolor{Teal}{x^2 + y^2 = r^2}$$
$$\textcolor{Teal}{F = ma}$$

\column{0.31\textwidth}
\textcolor{DarkTeal}{\textbf{Category 2}}

Intermediate formulas:
$$\textcolor{ForestGreen}{\int_a^b f(x)\,dx}$$
$$\textcolor{ForestGreen}{\sum_{i=1}^n i = \frac{n(n+1)}{2}}$$
$$\textcolor{ForestGreen}{e^{i\pi} + 1 = 0}$$

\column{0.31\textwidth}
\textcolor{Amber}{\textbf{Category 3}}

Advanced formulas:
$$\textcolor{DarkTeal}{\nabla \times \vec{F} = 0}$$
$$\textcolor{DarkTeal}{\frac{\partial u}{\partial t} = k\nabla^2 u}$$
$$\textcolor{DarkTeal}{E = \hbar\omega}$$
\end{columns}

\vspace{\fill}
\small\textcolor{Slate}{Quick reference formulas organized by category}
\end{frame}

% ==================== LAYOUT 13: SUMMARY STYLE ====================
\begin{frame}{Summary Layout}
\begin{columns}[T]
\column{0.48\textwidth}
\textcolor{ForestGreen}{\textbf{Key Concepts}}
\begin{itemize}
\item Main idea 1
\item Main idea 2
\item Main idea 3
\item Main idea 4
\end{itemize}

\vspace{0.5em}
\textcolor{Teal}{\textbf{Methods Covered}}
\begin{itemize}
\item Technique A
\item Technique B
\item Technique C
\end{itemize}

\column{0.48\textwidth}
\textcolor{Amber}{\textbf{Applications}}
\begin{itemize}
\item Real-world use 1
\item Real-world use 2
\item Real-world use 3
\end{itemize}

\vspace{0.5em}
\textcolor{DarkTeal}{\textbf{Next Steps}}
\begin{itemize}
\item Further reading
\item Practice problems
\item Advanced topics
\end{itemize}
\end{columns}

\vspace{\fill}
\small\textcolor{Slate}{Summaries consolidate learning and provide direction}
\end{frame}

% ==================== LAYOUT 14: Q&A STYLE ====================
\begin{frame}{Question and Answer Format}
\begin{columns}[T]
\column{0.48\textwidth}
\textcolor{ForestGreen}{\textbf{Common Questions}}

\textcolor{Teal}{\textit{Q1: What is the main purpose?}}

Answer explaining the primary goal and its importance.

\vspace{0.5em}
\textcolor{Teal}{\textit{Q2: How does it work?}}

Brief explanation of the mechanism or process.

\column{0.48\textwidth}
\textcolor{Amber}{\textit{Q3: When should it be used?}}

Scenarios and conditions for application.

\vspace{0.5em}
\textcolor{Amber}{\textit{Q4: What are the limitations?}}

Known constraints and boundaries.
\end{columns}

\vspace{\fill}
\small\textcolor{Slate}{Anticipating questions improves comprehension}
\end{frame}

% ==================== LAYOUT 15: CLOSING SLIDE ====================
\begin{frame}[plain]
\vspace{3cm}
\begin{center}
{\Large\textcolor{ForestGreen}{Thank you}}\\[2cm]
{\normalsize\textcolor{Teal}{Questions?}}\\[1cm]
{\small\textcolor{Amber}{contact@example.com}}
\end{center}
\end{frame}

% ==================== LAYOUT 16: OVERVIEW WITH SECTIONS ====================
\begin{frame}{Course Overview}
\begin{columns}[T]
\column{0.48\textwidth}
\textcolor{ForestGreen}{\textbf{Part 1: Foundations}}
\begin{itemize}
\item Topic 1.1
\item Topic 1.2
\item Topic 1.3
\item Topic 1.4
\end{itemize}

\vspace{0.5em}
\textcolor{Teal}{\textbf{Part 2: Intermediate}}
\begin{itemize}
\item Topic 2.1
\item Topic 2.2
\item Topic 2.3
\end{itemize}

\column{0.48\textwidth}
\textcolor{DarkTeal}{\textbf{Part 3: Advanced}}
\begin{itemize}
\item Topic 3.1
\item Topic 3.2
\item Topic 3.3
\end{itemize}

\vspace{0.5em}
\textcolor{Amber}{\textbf{Part 4: Applications}}
\begin{itemize}
\item Application A
\item Application B
\item Application C
\item Case Studies
\end{itemize}
\end{columns}

\vspace{\fill}
\small\textcolor{Slate}{Structured overview helps learners navigate content}
\end{frame}

% ==================== LAYOUT 17: CODE AND OUTPUT ====================
\begin{frame}[fragile]{Code Example Layout}
\begin{columns}[T]
\column{0.48\textwidth}
\textcolor{ForestGreen}{\textbf{Input Code}}

\begin{verbatim}
def function(x):
    if x > 0:
        return x * 2
    else:
        return -x

result = function(5)
print(result)
\end{verbatim}

\vspace{0.5em}
\textcolor{Teal}{\textbf{Explanation}}

This function doubles positive numbers and negates negative numbers.

\column{0.48\textwidth}
\textcolor{Amber}{\textbf{Output}}

\texttt{10}

\vspace{0.5em}
\textcolor{DarkTeal}{\textbf{Trace Through}}
\begin{enumerate}
\item Input: $x = 5$
\item Check: $5 > 0$ (True)
\item Execute: $5 \times 2 = 10$
\item Return: $10$
\end{enumerate}

\textcolor{ForestGreen}{\textbf{Other Examples}}
\begin{itemize}
\item $f(3) = 6$
\item $f(-4) = 4$
\item $f(0) = 0$
\end{itemize}
\end{columns}

\vspace{\fill}
\small\textcolor{Slate}{Code examples benefit from step-by-step explanation}
\end{frame}

% ==================== LAYOUT 18: PROS AND CONS ====================
\begin{frame}{Advantages and Disadvantages}
\begin{columns}[T]
\column{0.48\textwidth}
\textcolor{ForestGreen}{\textbf{Advantages}}
\begin{itemize}
\item[\textcolor{Teal}{+}] Benefit one with explanation
\item[\textcolor{Teal}{+}] Benefit two
\item[\textcolor{Teal}{+}] Benefit three
\item[\textcolor{Teal}{+}] Benefit four with additional context
\item[\textcolor{Teal}{+}] Benefit five
\end{itemize}

\column{0.48\textwidth}
\textcolor{ForestGreen}{\textbf{Disadvantages}}
\begin{itemize}
\item[\textcolor{Amber}{-}] Drawback one
\item[\textcolor{Amber}{-}] Drawback two with details
\item[\textcolor{Amber}{-}] Drawback three
\item[\textcolor{Amber}{-}] Drawback four
\end{itemize}

\vspace{0.5em}
\textcolor{DarkTeal}{\textbf{Verdict}}

Best suited for situations where benefits outweigh drawbacks.
\end{columns}

\vspace{\fill}
\small\textcolor{Slate}{Balanced analysis helps informed decision-making}
\end{frame}

% ==================== LAYOUT 19: TIMELINE ====================
\begin{frame}{Timeline Layout}
\begin{columns}[T]
\column{0.48\textwidth}
\textcolor{ForestGreen}{\textbf{Phase 1: Initial Development}}
\begin{itemize}
\item \textcolor{Teal}{Week 1-2:} Planning
\item \textcolor{Teal}{Week 3-4:} Design
\item \textcolor{Teal}{Week 5-6:} Prototype
\end{itemize}

\vspace{0.5em}
\textcolor{ForestGreen}{\textbf{Phase 2: Implementation}}
\begin{itemize}
\item \textcolor{Teal}{Week 7-10:} Core features
\item \textcolor{Teal}{Week 11-12:} Testing
\item \textcolor{Teal}{Week 13-14:} Refinement
\end{itemize}

\column{0.48\textwidth}
\textcolor{Amber}{\textbf{Phase 3: Deployment}}
\begin{itemize}
\item \textcolor{DarkTeal}{Week 15:} Beta release
\item \textcolor{DarkTeal}{Week 16-17:} Feedback
\item \textcolor{DarkTeal}{Week 18:} Final release
\end{itemize}

\vspace{0.5em}
\textcolor{Amber}{\textbf{Phase 4: Maintenance}}
\begin{itemize}
\item \textcolor{DarkTeal}{Ongoing:} Updates
\item \textcolor{DarkTeal}{Monthly:} Reviews
\item \textcolor{DarkTeal}{Quarterly:} Major updates
\end{itemize}
\end{columns}

\vspace{\fill}
\small\textcolor{Slate}{Clear timelines set expectations and track progress}
\end{frame}

% ==================== LAYOUT 20: REFERENCES ====================
\begin{frame}{References and Resources}
\begin{columns}[T]
\column{0.48\textwidth}
\textcolor{ForestGreen}{\textbf{Primary Sources}}
\begin{itemize}
\item Author (2024): \textit{Main Title}
\item Researcher (2023): \textit{Key Paper}
\item Expert (2023): \textit{Foundational Work}
\end{itemize}

\vspace{0.5em}
\textcolor{Teal}{\textbf{Books}}
\begin{itemize}
\item Comprehensive Guide
\item Practical Handbook
\item Theory and Practice
\end{itemize}

\column{0.48\textwidth}
\textcolor{Amber}{\textbf{Online Resources}}
\begin{itemize}
\item Official documentation
\item Video tutorials
\item Interactive examples
\item Community forums
\end{itemize}

\vspace{0.5em}
\textcolor{DarkTeal}{\textbf{Tools}}
\begin{itemize}
\item Software package A
\item Library B
\item Framework C
\end{itemize}
\end{columns}

\vspace{\fill}
\small\textcolor{Slate}{Curated resources accelerate learning}
\end{frame}

% ==================== LAYOUT 21: NATURE-THEMED SPECIAL ====================
\begin{frame}{Nature Professional Special Features}
\begin{columns}[T]
\column{0.48\textwidth}
\textcolor{ForestGreen}{\textbf{Color Psychology}}

The Nature Professional palette creates:
\begin{itemize}
\item \textcolor{Teal}{Calming effect} from natural greens
\item \textcolor{Amber}{Focus points} with amber accents
\item \textcolor{ForestGreen}{Professional tone} with forest green
\item \textcolor{Slate}{Subtle support} with slate elements
\end{itemize}

\vspace{0.5em}
\begin{block}{Environmental Connection}
Natural colors reduce eye strain and improve information retention.
\end{block}

\column{0.48\textwidth}
\textcolor{Teal}{\textbf{Usage Guidelines}}

\begin{itemize}
\item \textcolor{ForestGreen}{Primary text} - Main content
\item \textcolor{Teal}{Secondary elements} - Supporting info
\item \textcolor{Amber}{Highlights} - Key points
\item \textcolor{DarkTeal}{Variations} - Depth
\item \textcolor{Slate}{Annotations} - Meta info
\end{itemize}

\vspace{0.5em}
\begin{alertblock}{Best Practice}
Use color consistently to create visual hierarchy and guide attention.
\end{alertblock}
\end{columns}

\vspace{\fill}
\small\textcolor{Slate}{Nature Professional theme: Where professionalism meets natural harmony}
\end{frame}

\end{document}