% Part 3: Implementation - Building Responsible Systems

\section{Implementation: Tools and Techniques}

% Slide 1: Fairness Toolkit Landscape
\begin{frame}{Fairness Toolkits Comparison}
\begin{columns}[T]
\column{0.55\textwidth}
\includegraphics[width=\textwidth]{charts/fairness_toolkit_comparison.pdf}

\column{0.43\textwidth}
\textcolor{ForestGreen}{\Large Top Toolkits}\\[0.5cm]

\textcolor{Amber}{\textbf{IBM AIF360}}
\begin{itemize}
\item 70+ metrics
\item 10+ algorithms
\item Python \& R
\end{itemize}

\textcolor{Teal}{\textbf{Fairlearn}}
\begin{itemize}
\item Microsoft-backed
\item Sklearn integration
\item Strong visualization
\end{itemize}

\textcolor{DarkTeal}{\textbf{Google What-If}}
\begin{itemize}
\item Interactive exploration
\item TensorBoard integration
\item Visual debugging
\end{itemize}
\end{columns}

\vspace{\fill}
\footnotesize\textcolor{Slate}{Choose based on your stack and needs}
\end{frame}

% Slide 2: IBM AIF360 Deep Dive
\begin{frame}{IBM AIF360: Comprehensive Fairness}
\begin{columns}[T]
\column{0.48\textwidth}
\textcolor{ForestGreen}{\textbf{Key Features}}

\textcolor{Amber}{Metrics:}
\begin{itemize}
\item 70+ fairness metrics
\item Group \& individual fairness
\item Intersectional analysis
\item Temporal fairness
\end{itemize}

\textcolor{Teal}{Algorithms:}
\begin{itemize}
\item Pre-processing (4 methods)
\item In-processing (3 methods)
\item Post-processing (3 methods)
\end{itemize}

\textcolor{DarkTeal}{Datasets:}
\begin{itemize}
\item 10 benchmark datasets
\item Preprocessed \& ready
\item Academic standard
\end{itemize}

\column{0.48\textwidth}
\textcolor{ForestGreen}{\textbf{Example Usage}}

\small
\texttt{from aif360.datasets import}\\
\texttt{~~AdultDataset}\\
\texttt{from aif360.metrics import}\\
\texttt{~~BinaryLabelDatasetMetric}\\

\texttt{dataset = AdultDataset()}\\
\texttt{metric = BinaryLabelDatasetMetric(}\\
\texttt{~~dataset,}\\
\texttt{~~unprivileged\_groups=[}\\
\texttt{~~~~\{'sex': 0\}],}\\
\texttt{~~privileged\_groups=[}\\
\texttt{~~~~\{'sex': 1\}]}\\
\texttt{)}\\

\texttt{print(metric.mean\_difference())}\\
\texttt{print(metric.disparate\_impact())}
\end{columns}

\vspace{\fill}
\footnotesize\textcolor{Slate}{AIF360: Best for comprehensive fairness assessment}
\end{frame}

% Slide 3: Fairlearn Practical Example
\begin{frame}{Fairlearn: Easy sklearn Integration}
\begin{columns}[T]
\column{0.48\textwidth}
\textcolor{ForestGreen}{\textbf{Core Concepts}}

\textcolor{Amber}{Assessment}
\begin{itemize}
\item MetricFrame for group metrics
\item Disaggregated analysis
\item Interactive dashboards
\end{itemize}

\textcolor{Teal}{Mitigation}
\begin{itemize}
\item GridSearch: Threshold optimization
\item ExponentiatedGradient: In-processing
\item ThresholdOptimizer: Post-processing
\end{itemize}

\textcolor{DarkTeal}{Constraints}
\begin{itemize}
\item Demographic parity
\item Equalized odds
\item True positive rate parity
\item Bounded group loss
\end{itemize}

\column{0.48\textwidth}
\textcolor{ForestGreen}{\textbf{Example Code}}

\small
\texttt{from fairlearn.reductions import}\\
\texttt{~~ExponentiatedGradient}\\
\texttt{from fairlearn.reductions import}\\
\texttt{~~DemographicParity}\\

\texttt{constraint = DemographicParity()}\\

\texttt{mitigator = ExponentiatedGradient(}\\
\texttt{~~estimator,}\\
\texttt{~~constraints=constraint}\\
\texttt{)}\\

\texttt{mitigator.fit(X, y,}\\
\texttt{~~sensitive\_features=A)}\\

\texttt{y\_pred = mitigator.predict(X\_test)}
\end{columns}

\vspace{\fill}
\footnotesize\textcolor{Slate}{Fairlearn: Best for quick sklearn integration}
\end{frame}

% Slide 4: Explainability - SHAP
\begin{frame}{SHAP: Explaining Model Decisions}
\begin{columns}[T]
\column{0.55\textwidth}
\includegraphics[width=\textwidth]{charts/shap_explanation_example.pdf}

\column{0.43\textwidth}
\textcolor{ForestGreen}{\Large SHapley Additive exPlanations}\\[0.5cm]

\textcolor{Amber}{\textbf{Theory:}}
\begin{itemize}
\item Game-theoretic foundation
\item Shapley values
\item Additive feature attribution
\item Mathematically rigorous
\end{itemize}

\textcolor{Teal}{\textbf{Practical:}}
\begin{itemize}
\item Model-agnostic
\item Fast approximations
\item Beautiful visualizations
\item Local \& global explanations
\end{itemize}
\end{columns}

\vspace{\fill}
\footnotesize\textcolor{Slate}{SHAP values: The gold standard for feature importance}
\end{frame}

% Slide 5: Explainability - LIME
\begin{frame}{LIME: Local Interpretable Model-agnostic Explanations}
\begin{columns}[T]
\column{0.48\textwidth}
\textcolor{ForestGreen}{\textbf{How LIME Works}}

\begin{enumerate}
\item Perturb input locally
\item Get model predictions
\item Fit simple linear model
\item Explain with coefficients
\end{enumerate}

\vspace{0.3cm}
\textcolor{Amber}{\textbf{Key Idea:}}\\
Complex models are locally linear

\vspace{0.3cm}
\textcolor{Teal}{\textbf{Advantages:}}
\begin{itemize}
\item Truly model-agnostic
\item Works on any data type
\item Human-interpretable
\item Fast computation
\end{itemize}

\column{0.48\textwidth}
\textcolor{ForestGreen}{\textbf{Example Usage}}

\small
\texttt{from lime.lime\_tabular import}\\
\texttt{~~LimeTabularExplainer}\\

\texttt{explainer = LimeTabularExplainer(}\\
\texttt{~~X\_train,}\\
\texttt{~~feature\_names=features,}\\
\texttt{~~class\_names=classes,}\\
\texttt{~~mode='classification'}\\
\texttt{)}\\

\texttt{explanation = explainer.explain\_instance(}\\
\texttt{~~X\_test[i],}\\
\texttt{~~model.predict\_proba}\\
\texttt{)}\\

\texttt{explanation.show\_in\_notebook()}
\end{columns}

\vspace{\fill}
\footnotesize\textcolor{Slate}{LIME: Best for quick local explanations}
\end{frame}

% Slide 6: Model Cards
\begin{frame}{Model Cards: Documentation Standard}
\begin{columns}[T]
\column{0.55\textwidth}
\includegraphics[width=\textwidth]{charts/model_card_template.pdf}

\column{0.43\textwidth}
\textcolor{ForestGreen}{\Large Essential Documentation}\\[0.5cm]

\textcolor{Amber}{\textbf{Required Sections:}}
\begin{itemize}
\item Model details
\item Intended use
\item Factors (demographics)
\item Metrics
\item Training data
\item Evaluation data
\item Ethical considerations
\item Caveats \& recommendations
\end{itemize}

\textcolor{Teal}{\textbf{Benefits:}}
\begin{itemize}
\item Transparency
\item Accountability
\item Risk communication
\end{itemize}
\end{columns}

\vspace{\fill}
\footnotesize\textcolor{Slate}{Model cards should be mandatory for deployed systems}
\end{frame}

% Slide 7: Datasheets for Datasets
\begin{frame}{Datasheets: Documenting Training Data}
\begin{columns}[T]
\column{0.48\textwidth}
\textcolor{ForestGreen}{\textbf{Why Datasheets Matter}}

\textcolor{Amber}{Problems with undocumented data:}
\begin{itemize}
\item Unknown biases
\item Unclear provenance
\item Misuse in new contexts
\item Privacy violations
\item Reproduction failures
\end{itemize}

\vspace{0.3cm}
\textcolor{Teal}{\textbf{Inspiration:}}\\
Electronics datasheets -- standardized, comprehensive, essential

\column{0.48\textwidth}
\textcolor{ForestGreen}{\textbf{Core Questions}}

\textcolor{DarkTeal}{Motivation}
\begin{itemize}
\item Why was dataset created?
\item Who funded it?
\end{itemize}

\textcolor{Amber}{Composition}
\begin{itemize}
\item What do instances represent?
\item How many instances?
\item Missing data?
\end{itemize}

\textcolor{Teal}{Collection}
\begin{itemize}
\item How was data acquired?
\item Who was involved?
\item Ethical review?
\end{itemize}

\textcolor{DarkTeal}{Uses}
\begin{itemize}
\item Intended tasks?
\item What to avoid?
\end{itemize}
\end{columns}

\vspace{\fill}
\footnotesize\textcolor{Slate}{Dataset documentation prevents downstream harms}
\end{frame}

% Slide 8: Privacy-Preserving ML
\begin{frame}{Privacy Techniques}
\begin{columns}[T]
\column{0.55\textwidth}
\includegraphics[width=\textwidth]{charts/privacy_techniques.pdf}

\column{0.43\textwidth}
\textcolor{ForestGreen}{\Large Three Approaches}\\[0.5cm]

\textcolor{Amber}{\textbf{Differential Privacy}}
\begin{itemize}
\item Add calibrated noise
\item Formal privacy guarantee
\item Accuracy trade-off
\end{itemize}

\textcolor{Teal}{\textbf{Federated Learning}}
\begin{itemize}
\item Train locally
\item Share only updates
\item Keep data distributed
\end{itemize}

\textcolor{DarkTeal}{\textbf{Secure Multi-party Computation}}
\begin{itemize}
\item Cryptographic protocols
\item Compute on encrypted data
\item No raw data exposure
\end{itemize}
\end{columns}

\vspace{\fill}
\footnotesize\textcolor{Slate}{Privacy and utility can coexist with right techniques}
\end{frame}

% Slide 9: Differential Privacy
\begin{frame}{Differential Privacy: Formal Guarantees}
\begin{columns}[T]
\column{0.48\textwidth}
\textcolor{ForestGreen}{\textbf{Mathematical Definition}}

A mechanism $M$ is $(\epsilon, \delta)$-differentially private if:

$$\textcolor{Teal}{P(M(D) \in S) \leq e^\epsilon P(M(D') \in S) + \delta}$$

For all datasets $D, D'$ differing in one record

\vspace{0.3cm}
\textcolor{Amber}{\textbf{Intuition:}}
\begin{itemize}
\item Adding/removing one person
\item Changes output minimally
\item Individual contribution hidden
\item Privacy budget $\epsilon$
\end{itemize}

\column{0.48\textwidth}
\textcolor{ForestGreen}{\textbf{Practical Implementation}}

\small
\texttt{from diffprivlib.models import}\\
\texttt{~~LogisticRegression}\\

\texttt{clf = LogisticRegression(}\\
\texttt{~~epsilon=1.0,}\\
\texttt{~~data\_norm=5.0}\\
\texttt{)}\\

\texttt{clf.fit(X, y)}\\

\vspace{0.3cm}
\textcolor{Teal}{\textbf{Trade-off:}}
\begin{itemize}
\item Lower $\epsilon$ = More privacy
\item Lower $\epsilon$ = Less accuracy
\item Typical: $\epsilon \in [0.1, 10]$
\end{itemize}
\end{columns}

\vspace{\fill}
\footnotesize\textcolor{Slate}{Differential privacy: Industry standard for privacy-preserving analytics}
\end{frame}

% Slide 10: Implementation Summary
\begin{frame}{Implementation Summary: Tools \& Techniques}
\begin{columns}[T]
\column{0.48\textwidth}
\textcolor{ForestGreen}{\textbf{Key Takeaways}}

\begin{enumerate}
\item Fairness toolkits exist and work
\item AIF360: Comprehensive
\item Fairlearn: Easy integration
\item SHAP \& LIME: Explainability
\item Model cards: Documentation
\item Privacy techniques available
\end{enumerate}

\vspace{0.3cm}
\textcolor{Amber}{\textbf{Action Item:}}\\
Install and try one toolkit this week

\column{0.48\textwidth}
\textcolor{Teal}{\textbf{Next: Design}}

Human-centered perspective:
\begin{itemize}
\item Inclusive design
\item Accessibility
\item User consent
\item Environmental impact
\item Real case studies
\end{itemize}

\vspace{0.3cm}
\textcolor{DarkTeal}{\textbf{Remember:}}\\
Tools are necessary but not sufficient -- judgment required
\end{columns}

\vspace{\fill}
\begin{center}
\Large\textcolor{ForestGreen}{Implementation Makes Ethics Real}
\end{center}
\end{frame}