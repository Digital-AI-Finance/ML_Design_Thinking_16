% Part 2: Techniques - Experiment Design & Analysis
\section{Techniques: Rigorous Experiment Design}

% Slide 1: Hypothesis Formulation
\begin{frame}{Hypothesis Formulation: The Starting Point}
\begin{columns}[T]
\column{0.48\textwidth}
\textcolor{mlblue}{\textbf{Anatomy of a Good Hypothesis}}

\small
\textbf{SMART Framework:}
\begin{itemize}
\item \textbf{S}pecific: Clearly defined change
\item \textbf{M}easurable: Quantifiable outcome
\item \textbf{A}ctionable: You can implement it
\item \textbf{R}elevant: Aligns with business goals
\item \textbf{T}ime-bound: Experiment duration set
\end{itemize}

\vspace{0.3cm}
\textcolor{mlgreen}{\textbf{Good Example}}

\small
\textbf{Hypothesis:} Switching from content-based to collaborative filtering recommendations will increase click-through rate by at least 1 percentage point (from 5\% to 6\%) over a 2-week test period with 100,000 users.

\vspace{0.2cm}
\textbf{Why good:}
\begin{itemize}
\item Specific change (CF algorithm)
\item Measurable outcome (CTR, 1 pp)
\item Actionable (can deploy CF)
\item Relevant (engagement goal)
\item Time-bound (2 weeks, 100K users)
\end{itemize}

\column{0.48\textwidth}
\textcolor{mlred}{\textbf{Bad Examples}}

\small
\textbf{Vague:} ``New algorithm will be better''
\begin{itemize}
\item Not specific (which algorithm?)
\item Not measurable (better how?)
\item Not time-bound
\end{itemize}

\vspace{0.2cm}
\textbf{Unmeasurable:} ``Users will like recommendations more''
\begin{itemize}
\item ``Like'' not quantified
\item No success criterion
\end{itemize}

\vspace{0.2cm}
\textbf{Too ambitious:} ``Will 10$	imes$ revenue''
\begin{itemize}
\item Unrealistic expectation
\item Sets up for disappointment
\end{itemize}

\vspace{0.3cm}
\textcolor{mlorange}{\textbf{Null vs Alternative Hypothesis}}

\small
\textbf{Null} (H$_0$): No difference between control and treatment\\
$CTR_{treatment} = CTR_{control}$

\vspace{0.2cm}
\textbf{Alternative} (H$_1$): Treatment is better\\
$CTR_{treatment} > CTR_{control}$ (one-tailed)\\
$CTR_{treatment} \neq CTR_{control}$ (two-tailed)
\end{columns}

\vspace{\fill}
\footnotesize\textcolor{mlgray}{Pre-register hypothesis before experiment to avoid p-hacking}
\end{frame}

% Slide 2: Sample Size Calculation
\begin{frame}{Sample Size Calculation: How Many Users?}
\begin{columns}[T]
\column{0.55\textwidth}
\includegraphics[width=\textwidth]{charts/statistical_power_curves.pdf}

\column{0.43\textwidth}
\textcolor{mlblue}{\textbf{Key Parameters}}

\small
\textbf{1. Significance Level} ($\alpha$)
\begin{itemize}
\item Type I error rate (false positive)
\item Standard: 0.05 (5\%)
\item Probability of detecting difference when none exists
\end{itemize}

\vspace{0.2cm}
\textbf{2. Statistical Power} (1 - $\beta$)
\begin{itemize}
\item Type II error rate (false negative)
\item Standard: 0.80 (80\%)
\item Probability of detecting true difference
\end{itemize}

\vspace{0.2cm}
\textbf{3. Minimum Detectable Effect (MDE)}
\begin{itemize}
\item Smallest difference you care about
\item Example: 5\% $

\rightarrow$ 6\% CTR (1 pp absolute, 20\% relative)
\item Smaller MDE needs larger sample
\end{itemize}

\vspace{0.2cm}
\textbf{4. Baseline Metric}
\begin{itemize}
\item Current performance level
\item Variance affects sample size
\end{itemize}
\end{columns}

\vspace{\fill}
\footnotesize\textcolor{mlgray}{Formula: $n = \frac{(z_{\alpha/2} + z_{\beta})^2 \cdot 2\sigma^2}{\delta^2}$ where $\delta$ = effect size}
\end{frame}

% Slide 3: Randomization Strategies
\begin{frame}{Randomization Strategies: More Than Just Coin Flips}
\begin{columns}[T]
\column{0.48\textwidth}
\textcolor{mlblue}{\textbf{Simple Randomization}}

\small
\textbf{Method:} Coin flip per user (50/50)

\vspace{0.2cm}
\textbf{Pros:}
\begin{itemize}
\item Easy to implement
\item Unbiased in expectation
\item Good for large samples
\end{itemize}

\vspace{0.2cm}
\textbf{Cons:}
\begin{itemize}
\item Group sizes may differ
\item Imbalanced covariates possible
\end{itemize}

\vspace{0.3cm}
\textcolor{mlgreen}{\textbf{Stratified Randomization}}

\small
\textbf{Method:} Balance within strata (device, region)

\vspace{0.2cm}
\textbf{Example:}
\begin{itemize}
\item Split iOS users 50/50
\item Split Android users 50/50
\item Ensures device balance
\end{itemize}

\vspace{0.2cm}
\textbf{Benefits:}
\begin{itemize}
\item Reduces variance
\item Controls confounders
\item More statistical power
\end{itemize}

\column{0.48\textwidth}
\textcolor{mlorange}{\textbf{Blocked Randomization}}

\small
\textbf{Method:} Randomize within fixed blocks

\vspace{0.2cm}
\textbf{Example:}
\begin{itemize}
\item Block 1: AABBA (2A, 2B)
\item Block 2: BAABB (2A, 2B)
\item Guarantees equal sizes
\end{itemize}

\vspace{0.2cm}
\textbf{Use case:} Sequential enrollment

\vspace{0.3cm}
\textcolor{mlpurple}{\textbf{Clustered Randomization}}

\small
\textbf{Method:} Randomize groups, not individuals

\vspace{0.2cm}
\textbf{Example:}
\begin{itemize}
\item User sessions (not individual actions)
\item Geographic regions (not cities)
\item Social networks (friends together)
\end{itemize}

\vspace{0.2cm}
\textbf{Why:}
\begin{itemize}
\item Prevent contamination
\item Network effects
\item Administrative ease
\end{itemize}

\vspace{0.2cm}
\textbf{Drawback:} Larger sample size needed
\end{columns}

\vspace{\fill}
\footnotesize\textcolor{mlgray}{Choice depends on experimental context and confounding structure}
\end{frame}

% Slide 4: A/B Testing Math
\begin{frame}{A/B Testing Math: Statistical Tests}
\begin{columns}[T]
\column{0.48\textwidth}
\textcolor{mlblue}{\textbf{Z-Test for Proportions}}

\small
\textbf{When:} Binary outcome (click/no click)

\vspace{0.2cm}
\textbf{Formula:}
$$z = \frac{p_B - p_A}{\sqrt{p(1-p)(\frac{1}{n_A} + \frac{1}{n_B})}}$$

where $p = \frac{n_A p_A + n_B p_B}{n_A + n_B}$ (pooled proportion)

\vspace{0.2cm}
\textbf{Example:}
\begin{itemize}
\item Control: 500/10,000 clicked (5\%)
\item Treatment: 620/10,000 clicked (6.2\%)
\item z = 4.72, p $<$ 0.001
\item Conclusion: Significant improvement
\end{itemize}

\vspace{0.3cm}
\textcolor{mlgreen}{\textbf{T-Test for Means}}

\small
\textbf{When:} Continuous outcome (revenue, time)

\vspace{0.2cm}
\textbf{Formula:}
$$t = \frac{\bar{x}_B - \bar{x}_A}{\sqrt{\frac{s_A^2}{n_A} + \frac{s_B^2}{n_B}}}$$

\vspace{0.2cm}
\textbf{Use Welch's t-test:} Unequal variances OK

\column{0.48\textwidth}
\textcolor{mlorange}{\textbf{Confidence Intervals}}

\small
\textbf{95\% CI for difference:}
$$(\bar{x}_B - \bar{x}_A) \pm 1.96 \cdot SE$$

\vspace{0.2cm}
\textbf{Interpretation:}
\begin{itemize}
\item If CI excludes 0 $

\rightarrow$ Significant
\item CI shows practical range
\item More informative than p-value alone
\end{itemize}

\vspace{0.2cm}
\textbf{Example:}
\begin{itemize}
\item Difference: 1.2 percentage points
\item 95\% CI: [0.8, 1.6]
\item Excludes 0 $

\rightarrow$ Significant
\item Likely between 0.8-1.6pp lift
\end{itemize}

\vspace{0.3cm}
\textcolor{mlpurple}{\textbf{Effect Size}}

\small
\textbf{Cohen's d:}
$$d = \frac{\bar{x}_B - \bar{x}_A}{s_{pooled}}$$

\vspace{0.2cm}
\textbf{Interpretation:}
\begin{itemize}
\item 0.2: Small
\item 0.5: Medium
\item 0.8: Large
\end{itemize}

\vspace{0.2cm}
\textbf{Why it matters:} Statistical significance $\neq$ practical significance
\end{columns}

\vspace{\fill}
\footnotesize\textcolor{mlgray}{Always report effect size and CI, not just p-value}
\end{frame}

% Slide 5: Statistical Significance
\begin{frame}{Statistical Significance: Interpreting P-Values}
\begin{columns}[T]
\column{0.55\textwidth}
\includegraphics[width=\textwidth]{charts/type_i_ii_errors.pdf}

\column{0.43\textwidth}
\textcolor{mlblue}{\textbf{What is a P-Value?}}

\small
\textbf{Definition:} Probability of observing data this extreme if null hypothesis were true.

\vspace{0.2cm}
\textbf{NOT:}
\begin{itemize}
\item Probability null is true
\item Probability you're wrong
\item Size of effect
\end{itemize}

\vspace{0.2cm}
\textbf{Example:}
\begin{itemize}
\item p = 0.03
\item If no real difference exists, 3\% chance of seeing this result by luck
\item Since 3\% $<$ 5\%, reject null
\end{itemize}

\vspace{0.3cm}
\textcolor{mlgreen}{\textbf{Type I vs Type II Errors}}

\small
\textbf{Type I (False Positive):}
\begin{itemize}
\item Declare winner when none exists
\item Rate: $\alpha$ (typically 5\%)
\item ``Ship a loser''
\end{itemize}

\vspace{0.2cm}
\textbf{Type II (False Negative):}
\begin{itemize}
\item Miss real improvement
\item Rate: $\beta$ (typically 20\%)
\item ``Kill a winner''
\end{itemize}
\end{columns}

\vspace{\fill}
\footnotesize\textcolor{mlgray}{Trade-off: Lowering $\alpha$ increases $\beta$ (requires larger sample)}
\end{frame}

% Slide 6: Bayesian A/B Testing
\begin{frame}{Bayesian A/B Testing: Probability of Being Best}
\begin{columns}[T]
\column{0.55\textwidth}
\includegraphics[width=\textwidth]{charts/bayesian_ab_posterior.pdf}

\column{0.43\textwidth}
\textcolor{mlblue}{\textbf{Bayesian Approach}}

\small
\textbf{Instead of p-values:}
\begin{itemize}
\item Calculate P(B $>$ A)
\item Direct probability interpretation
\item Incorporate prior knowledge
\item No arbitrary 0.05 threshold
\end{itemize}

\vspace{0.2cm}
\textbf{Example:}
\begin{itemize}
\item P(B $>$ A) = 0.97
\item 97\% sure treatment is better
\item Can ship with 95\% confidence
\end{itemize}

\vspace{0.3cm}
\textcolor{mlgreen}{\textbf{Advantages}}

\small
\begin{itemize}
\item Intuitive interpretation
\item Earlier stopping (faster decisions)
\item Handles small samples better
\item Can update continuously
\item No ``peeking problem''
\end{itemize}

\vspace{0.3cm}
\textcolor{mlorange}{\textbf{When to Use}}

\small
\begin{itemize}
\item Need fast decisions
\item Low traffic (small samples)
\item Prior knowledge available
\item Continuous monitoring
\end{itemize}
\end{columns}

\vspace{\fill}
\footnotesize\textcolor{mlgray}{Bayesian methods increasingly popular at Google, Netflix, Etsy}
\end{frame}

% Slide 6b: Understanding Confidence Intervals
\begin{frame}{Understanding Confidence Intervals Through Repetition}
\begin{columns}[T]
\column{0.55\textwidth}
\includegraphics[width=\textwidth]{charts/confidence_intervals.pdf}

\column{0.43\textwidth}
\textcolor{mlblue}{\textbf{What 95\% CI Means}}

\small
\textbf{Definition:} If we repeated the experiment 100 times, 95 of the intervals would contain the true value.

\vspace{0.2cm}
\textbf{Key Insights:}
\begin{itemize}
\item Each experiment produces a different interval
\item True value is fixed (but unknown)
\item Some intervals miss (by design!)
\item Wider intervals = more uncertainty
\end{itemize}

\vspace{0.3cm}
\textcolor{mlgreen}{\textbf{Common Misconceptions}}

\small
\textbf{Wrong:} ``95\% chance true value is in this interval''
\begin{itemize}
\item True value either is or isn't in it
\item It's about the procedure, not this interval
\end{itemize}

\vspace{0.2cm}
\textbf{Right:} ``95\% of intervals constructed this way contain the true value''

\vspace{0.3cm}
\textcolor{mlorange}{\textbf{Practical Use}}

\small
\begin{itemize}
\item If CI includes 0: Not significant
\item If CI excludes 0: Significant
\item Width shows precision of estimate
\end{itemize}
\end{columns}

\vspace{\fill}
\footnotesize\textcolor{mlgray}{Visualization: 30 repeated experiments, each with 95\% CI}
\end{frame}

% Slide 7: Sequential Testing
\begin{frame}{Sequential Testing: Early Stopping Without Peeking}
\begin{columns}[T]
\column{0.55\textwidth}
\includegraphics[width=\textwidth]{charts/sequential_testing_boundaries.pdf}

\column{0.43\textwidth}
\textcolor{mlblue}{\textbf{The Peeking Problem}}

\small
\textbf{Bad practice:}
\begin{itemize}
\item Check results daily
\item Stop when p $<$ 0.05
\item False positive rate $>$ 50\%!
\end{itemize}

\vspace{0.2cm}
\textbf{Why bad:}
\begin{itemize}
\item Multiple testing inflates $\alpha$
\item Random fluctuations look significant
\item Results not reproducible
\end{itemize}

\vspace{0.3cm}
\textcolor{mlgreen}{\textbf{Sequential Testing Solution}}

\small
\textbf{Method:} Alpha spending functions
\begin{itemize}
\item Plan interim analyses (e.g., 25\%, 50\%, 75\%, 100\%)
\item Adjust $\alpha$ at each look
\item Control overall Type I error
\end{itemize}

\vspace{0.2cm}
\textbf{O'Brien-Fleming:}
\begin{itemize}
\item Conservative early, liberal late
\item First look: $\alpha$ = 0.0005
\item Last look: $\alpha$ = 0.0455
\item Total $\alpha$ still 0.05
\end{itemize}

\vspace{0.3cm}
\textbf{Benefit:} Stop early if clear winner/loser
\end{columns}

\vspace{\fill}
\footnotesize\textcolor{mlgray}{Sequential testing enables early stopping while maintaining statistical validity}
\end{frame}

% Slide 8: Multi-Armed Bandits
\begin{frame}{Multi-Armed Bandits: Exploration vs Exploitation}
\begin{columns}[T]
\column{0.55\textwidth}
\includegraphics[width=\textwidth]{charts/thompson_sampling_demo.pdf}

\column{0.43\textwidth}
\textcolor{mlblue}{\textbf{The Bandit Problem}}

\small
\textbf{Scenario:} Slot machines (arms) with unknown payouts. Goal: Maximize total reward.

\vspace{0.2cm}
\textbf{Trade-off:}
\begin{itemize}
\item \textbf{Explore:} Try arms to learn
\item \textbf{Exploit:} Use best-known arm
\end{itemize}

\vspace{0.3cm}
\textcolor{mlgreen}{\textbf{Thompson Sampling}}

\small
\textbf{Algorithm:}
\begin{enumerate}
\item Maintain Beta($\alpha$, $\beta$) for each arm
\item Sample from each distribution
\item Pick arm with highest sample
\item Update winner's distribution
\end{enumerate}

\vspace{0.2cm}
\textbf{Why it works:}
\begin{itemize}
\item Automatically balances exploration/exploitation
\item Converges to best arm
\item Minimizes regret
\end{itemize}

\vspace{0.3cm}
\textcolor{mlorange}{\textbf{A/B Test vs Bandit}}

\small
\textbf{A/B Test:}
\begin{itemize}
\item Fixed allocation (50/50)
\item Learn at end
\item Higher regret during test
\end{itemize}

\vspace{0.2cm}
\textbf{Bandit:}
\begin{itemize}
\item Adaptive allocation
\item Learn continuously
\item Lower regret (shift traffic to winner)
\end{itemize}
\end{columns}

\vspace{\fill}
\footnotesize\textcolor{mlgray}{Use bandits when minimizing regret $>$ statistical rigor}
\end{frame}

% Slide 8b: Bandit Performance Analysis
\begin{frame}{Bandit Performance: Exploration vs Regret}
\begin{columns}[T]
\column{0.55\textwidth}
\includegraphics[width=\textwidth]{charts/multi_armed_bandit_exploration.pdf}

\column{0.43\textwidth}
\textcolor{mlblue}{\textbf{How Bandits Learn}}

\small
\textbf{Left Panel: Cumulative Pulls}
\begin{itemize}
\item Initially: Explore all arms
\item Middle: Identify good arms
\item End: Exploit best arm
\item Bad arms stop getting traffic
\end{itemize}

\vspace{0.3cm}
\textcolor{mlgreen}{\textbf{Right Panel: Regret}}

\small
\textbf{Regret:} Total reward lost by not always picking best arm

\vspace{0.2cm}
\textbf{Key Insight:}
\begin{itemize}
\item Regret grows logarithmically
\item Much slower than linear
\item Thompson Sampling is near-optimal
\item A/B test has linear regret
\end{itemize}

\vspace{0.3cm}
\textcolor{mlorange}{\textbf{When to Use Bandits}}

\small
\begin{itemize}
\item High traffic (fast learning)
\item Multiple variants ($>$ 3)
\item Cost of regret is high
\item Can tolerate adaptive allocation
\end{itemize}

\vspace{0.2cm}
\textbf{When A/B is better:}
\begin{itemize}
\item Need clean statistical test
\item Network effects present
\item Regulatory requirements
\end{itemize}
\end{columns}

\vspace{\fill}
\footnotesize\textcolor{mlgray}{Bandits reduce regret by 30-50\% vs fixed A/B tests}
\end{frame}

% Slide 9: Causal Inference Basics
\begin{frame}{Causal Inference: Correlation is Not Causation}
\begin{columns}[T]
\column{0.55\textwidth}
\includegraphics[width=\textwidth]{charts/causal_inference_dag.pdf}

\column{0.43\textwidth}
\textcolor{mlblue}{\textbf{The Fundamental Problem}}

\small
\textbf{Correlation:} X and Y move together

\textbf{Causation:} X causes Y

\vspace{0.2cm}
\textbf{Why different:}
\begin{itemize}
\item Confounders (Z affects both X and Y)
\item Reverse causation (Y causes X)
\item Spurious correlation (coincidence)
\end{itemize}

\vspace{0.3cm}
\textcolor{mlgreen}{\textbf{Example: Ice Cream \& Drowning}}

\small
\begin{itemize}
\item Correlation: High
\item Causation: None
\item Confounder: Hot weather
\item Weather causes both ice cream sales and swimming (drowning risk)
\end{itemize}

\vspace{0.3cm}
\textcolor{mlorange}{\textbf{Randomization Solves This}}

\small
\textbf{How:}
\begin{itemize}
\item Random assignment breaks confounding
\item Only difference is treatment
\item Can infer causation
\end{itemize}

\vspace{0.2cm}
\textbf{Without randomization:}
\begin{itemize}
\item Must control for confounders
\item Assume no unmeasured confounding
\item Causal claims weaker
\end{itemize}
\end{columns}

\vspace{\fill}
\footnotesize\textcolor{mlgray}{RCTs (A/B tests) are gold standard for causal inference}
\end{frame}

% Slide 10: Multi-Objective Optimization
\begin{frame}{Multi-Objective Optimization: Trade-Offs}
\begin{columns}[T]
\column{0.48\textwidth}
\textcolor{mlblue}{\textbf{The Real-World Problem}}

\small
\textbf{Single metric is rare:}
\begin{itemize}
\item CTR vs revenue
\item Short-term vs long-term
\item Engagement vs satisfaction
\item Speed vs accuracy
\end{itemize}

\vspace{0.2cm}
\textbf{Example:} Recommendation algorithm
\begin{itemize}
\item Control: Higher revenue, lower engagement
\item Treatment: Lower revenue, higher engagement
\item Which is better?
\end{itemize}

\vspace{0.3cm}
\textcolor{mlgreen}{\textbf{Solution 1: Primary + Guardrails}}

\small
\textbf{Method:}
\begin{itemize}
\item Primary: Metric you optimize (revenue)
\item Guardrails: Metrics that must not degrade
\item Example: Maximize revenue, but engagement must not drop $>$ 2\%
\end{itemize}

\vspace{0.2cm}
\textbf{Decision rule:}
\begin{itemize}
\item Ship if primary improves AND guardrails met
\item Else, iterate or kill
\end{itemize}

\column{0.48\textwidth}
\textcolor{mlorange}{\textbf{Solution 2: Weighted Utility}}

\small
\textbf{Method:} Combine metrics into single score

$$U = w_1 \cdot revenue + w_2 \cdot engagement$$

\vspace{0.2cm}
\textbf{Example weights:}
\begin{itemize}
\item Revenue: 0.7 (70\% weight)
\item Engagement: 0.3 (30\% weight)
\end{itemize}

\vspace{0.2cm}
\textbf{Challenge:} Choosing weights

\vspace{0.3cm}
\textcolor{mlpurple}{\textbf{Solution 3: Pareto Frontier}}

\small
\textbf{Method:} Find non-dominated solutions
\begin{itemize}
\item Solution A dominates B if better on all metrics
\item Pareto frontier: Set of non-dominated solutions
\item Business decides among frontier
\end{itemize}

\vspace{0.2cm}
\textbf{When to use:}
\begin{itemize}
\item Cannot agree on weights
\item Want to see trade-off space
\item Explore multiple strategies
\end{itemize}
\end{columns}

\vspace{\fill}
\footnotesize\textcolor{mlgray}{Real experiments almost always involve trade-offs—plan for it}
\end{frame}

% Slide 11: Technique Comparison
\begin{frame}{Technique Comparison: When to Use Each}
\begin{columns}[T]
\column{0.32\textwidth}
\textcolor{mlblue}{\textbf{Classical A/B Test}}

\small
\textbf{When:}
\begin{itemize}
\item High traffic
\item Need causal rigor
\item Low risk
\item Regulatory requirements
\item Can wait for full sample
\end{itemize}

\vspace{0.2cm}
\textbf{Pros:}
\begin{itemize}
\item Rigorous
\item Well understood
\item Easy to interpret
\item Reproducible
\end{itemize}

\vspace{0.2cm}
\textbf{Cons:}
\begin{itemize}
\item Slow (fixed duration)
\item Higher regret
\item Can't peek
\end{itemize}

\vspace{0.2cm}
\textbf{Example:}\\
Major algorithm change, need 95\% confidence, 2-week test

\column{0.32\textwidth}
\textcolor{mlgreen}{\textbf{Bayesian A/B Test}}

\small
\textbf{When:}
\begin{itemize}
\item Need faster decisions
\item Low traffic
\item Can update continuously
\item Prior knowledge exists
\item Stakeholders prefer probabilities
\end{itemize}

\vspace{0.2cm}
\textbf{Pros:}
\begin{itemize}
\item Intuitive (P(B $>$ A))
\item Earlier stopping
\item Handles small samples
\item No peeking problem
\end{itemize}

\vspace{0.2cm}
\textbf{Cons:}
\begin{itemize}
\item Prior selection subjective
\item Less familiar to stakeholders
\item Requires Bayesian tooling
\end{itemize}

\vspace{0.2cm}
\textbf{Example:}\\
Startup with 1K daily users, need quick wins

\column{0.32\textwidth}
\textcolor{mlorange}{\textbf{Multi-Armed Bandit}}

\small
\textbf{When:}
\begin{itemize}
\item Minimize regret
\item Continuous optimization
\item Many variants (A/B/C/D/E)
\item Cost of losing high
\item Real-time personalization
\end{itemize}

\vspace{0.2cm}
\textbf{Pros:}
\begin{itemize}
\item Lowest regret
\item Adaptive allocation
\item Scales to many arms
\item Always optimizing
\end{itemize}

\vspace{0.2cm}
\textbf{Cons:}
\begin{itemize}
\item Less statistical rigor
\item Harder to interpret
\item Complex implementation
\end{itemize}

\vspace{0.2cm}
\textbf{Example:}\\
Ad creative testing (50 variants), email subject lines
\end{columns}

\vspace{\fill}
\footnotesize\textcolor{mlgray}{Choice depends on traffic, risk tolerance, and business goals}
\end{frame}