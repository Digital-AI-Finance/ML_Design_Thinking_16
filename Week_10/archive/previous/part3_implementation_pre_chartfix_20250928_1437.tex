% Part 3: Implementation - Production Experiment Systems
\section{Implementation: Building A/B Test Infrastructure}

% Slide 1: Experiment Infrastructure Overview
\begin{frame}{Production Experiment Infrastructure}
\begin{columns}[T]
\column{0.55\textwidth}
\includegraphics[width=\textwidth]{charts/canary_deployment_timeline.pdf}

\column{0.43\textwidth}
\textcolor{mlblue}{\textbf{Core Components}}

\small
\textbf{1. Traffic Splitting}
\begin{itemize}
\item Assign users to control/treatment
\item Consistent hashing (same user, same group)
\item 50/50, 90/10, or custom splits
\end{itemize}

\vspace{0.2cm}
\textbf{2. Feature Flags}
\begin{itemize}
\item Toggle experiments on/off
\item Gradual rollouts (1\% $\rightarrow$ 5\% $\rightarrow$ 25\% $\rightarrow$ 100\%)
\item Instant kill switch
\end{itemize}

\vspace{0.2cm}
\textbf{3. Experiment Tracking}
\begin{itemize}
\item Log user assignment
\item Track metrics per group
\item Store for analysis
\end{itemize}

\vspace{0.2cm}
\textbf{4. Analysis Pipeline}
\begin{itemize}
\item Automated statistical tests
\item Real-time dashboards
\item Alert on guardrail violations
\end{itemize}
\end{columns}

\vspace{\fill}
\footnotesize\textcolor{mlgray}{Modern stacks: LaunchDarkly, Optimizely, GrowthBook, or custom}
\end{frame}

% Slide 2: Python A/B Test Implementation
\begin{frame}[fragile]{Python: Classical A/B Test}
\begin{columns}[T]
\column{0.55\textwidth}
\small
\textbf{Classical A/B Test Implementation:}

\begin{itemize}
\item Import required libraries: \texttt{scipy.stats}, \texttt{statsmodels}
\item Define sample data: control and treatment groups
\item Calculate proportions for each group
\item Perform Z-test for proportions using \texttt{proportions\_ztest}
\item Compare p-value to significance threshold (0.05)
\item Calculate and display key metrics:
  \begin{itemize}
  \item Control rate, treatment rate
  \item Lift percentage
  \item Z-statistic and p-value
  \item Significance decision
  \end{itemize}
\end{itemize}

\column{0.43\textwidth}
\textcolor{mlblue}{\textbf{Output Example}}

\small
\textbf{Sample Output:}

\begin{itemize}
\item Control: 0.050 (5.0\% conversion)
\item Treatment: 0.062 (6.2\% conversion)
\item Lift: 24.0\% improvement
\item Z-statistic: 4.72
\item P-value: 0.0001
\item Result: Significant improvement!
\end{itemize}

\vspace{0.3cm}
\textcolor{mlgreen}{\textbf{Confidence Interval}}

\small
\textbf{Confidence Interval Calculation:}

\begin{itemize}
\item Import \texttt{confint\_proportions\_2indep} from statsmodels
\item Calculate 95\% confidence interval using Wald method
\item Parameters: treatment successes/total, control successes/total
\item Output: lower and upper bounds of difference
\end{itemize}

\vspace{0.2cm}
\textcolor{mlorange}{\textbf{Interpretation}}

\small
If CI excludes 0, treatment significantly better than control
\end{columns}

\vspace{\fill}
\footnotesize\textcolor{mlgray}{scipy and statsmodels provide complete A/B test toolkit}
\end{frame}

% Slide 3: Bayesian A/B Testing Code
\begin{frame}[fragile]{Python: Bayesian A/B Testing with PyMC3}
\begin{columns}[T]
\column{0.55\textwidth}
\small
\textbf{Bayesian A/B Test with PyMC3:}

\begin{itemize}
\item Import PyMC3 and numpy libraries
\item Define observed data: clicks and trials for groups A and B
\item Set up Bayesian model:
  \begin{itemize}
  \item Beta priors for conversion rates (non-informative)
  \item Binomial likelihoods for observed data
  \item Deterministic variable for difference (delta)
  \end{itemize}
\item Sample from posterior distribution (2000 samples)
\item Calculate probability that B outperforms A
\item Direct probability statements without p-value confusion
\end{itemize}

\column{0.43\textwidth}
\textcolor{mlblue}{\textbf{Output}}

\small
\textbf{Bayesian Results:}

\begin{itemize}
\item P(B $>$ A) = 0.997 (99.7\% confidence)
\item Posterior mean delta: 0.012
\item 95\% Credible Interval: [0.008, 0.016]
\end{itemize}

\vspace{0.3cm}
\textcolor{mlgreen}{\textbf{Interpretation}}

\small
\begin{itemize}
\item 99.7\% probability B is better
\item Expected lift: 1.2 pp
\item 95\% sure lift between 0.8-1.6pp
\item High confidence to ship
\end{itemize}

\vspace{0.3cm}
\textcolor{mlorange}{\textbf{Advantages}}

\small
\begin{itemize}
\item Direct probability statements
\item No p-value confusion
\item Can stop early if P(B $>$ A) $>$ 0.95
\item Incorporates prior knowledge
\end{itemize}
\end{columns}

\vspace{\fill}
\footnotesize\textcolor{mlgray}{Bayesian approach increasingly adopted for faster iteration}
\end{frame}

% Slide 4: Multi-Armed Bandit Implementation
\begin{frame}[fragile]{Python: Thompson Sampling Bandit}
\begin{columns}[T]
\column{0.55\textwidth}
\small
\textbf{Thompson Sampling Bandit Implementation:}

\begin{itemize}
\item Class initialization with number of arms
\item Maintain success/failure counts per arm (Beta parameters)
\item Selection algorithm:
  \begin{itemize}
  \item Sample from Beta distribution for each arm
  \item Select arm with highest sample value
  \end{itemize}
\item Update mechanism:
  \begin{itemize}
  \item Increment success count for reward = 1
  \item Increment failure count for reward = 0
  \end{itemize}
\item Main loop: select arm, observe reward, update beliefs
\item Converges to optimal arm while balancing exploration
\end{itemize}

\column{0.43\textwidth}
\textcolor{mlblue}{\textbf{How It Works}}

\small
\textbf{1. Initialize:}
\begin{itemize}
\item Each arm: Beta(1, 1) = uniform prior
\end{itemize}

\vspace{0.2cm}
\textbf{2. Select arm:}
\begin{itemize}
\item Sample from each Beta distribution
\item Pick arm with highest sample
\end{itemize}

\vspace{0.2cm}
\textbf{3. Observe reward:}
\begin{itemize}
\item If success: $\alpha$ + 1
\item If failure: $\beta$ + 1
\end{itemize}

\vspace{0.2cm}
\textbf{4. Repeat}

\vspace{0.3cm}
\textcolor{mlgreen}{\textbf{Convergence}}

\small
\begin{itemize}
\item Best arm gets pulled more often
\item Exploration decreases over time
\item Minimizes cumulative regret
\end{itemize}

\vspace{0.3cm}
\textbf{Use case:} Ad creative testing (50 variants, continuous optimization)
\end{columns}

\vspace{\fill}
\footnotesize\textcolor{mlgray}{Thompson sampling: Simple, effective, Bayesian bandit algorithm}
\end{frame}

% Slide 5: Stratified Randomization Implementation
\begin{frame}[fragile]{Python: Stratified Randomization with sklearn}
\begin{columns}[T]
\column{0.55\textwidth}
\small
\textbf{Stratified Randomization Implementation:}

\begin{itemize}
\item Import \texttt{StratifiedShuffleSplit} from sklearn
\item Create user dataframe with stratification variables
\item Set up stratified split:
  \begin{itemize}
  \item 50/50 split between control and treatment
  \item Stratify by device type (iOS/Android)
  \item Use fixed random state for reproducibility
  \end{itemize}
\item Apply split to create balanced groups
\item Verify stratification worked:
  \begin{itemize}
  \item Check device distribution in both groups
  \item Should be identical proportions
  \end{itemize}
\item Can extend to multiple stratification variables
\end{itemize}

\column{0.43\textwidth}
\textcolor{mlblue}{\textbf{Output}}

\small
\textbf{Verification Results:}

\begin{itemize}
\item \textbf{Control device distribution:}
  \begin{itemize}
  \item iOS: 0.50 (50\%)
  \item Android: 0.50 (50\%)
  \end{itemize}
\item \textbf{Treatment device distribution:}
  \begin{itemize}
  \item iOS: 0.50 (50\%)
  \item Android: 0.50 (50\%)
  \end{itemize}
\end{itemize}

\vspace{0.3cm}
\textcolor{mlgreen}{\textbf{Why This Matters}}

\small
\textbf{Without stratification:}
\begin{itemize}
\item Control might be 55\% iOS
\item Treatment might be 45\% iOS
\item Device becomes confounder
\item Results biased
\end{itemize}

\vspace{0.3cm}
\textbf{With stratification:}
\begin{itemize}
\item Both groups exactly 50\% iOS
\item Device balanced
\item Removes confounding
\item Higher statistical power
\end{itemize}

\vspace{0.3cm}
\textbf{Can stratify on multiple variables:} Device, region, user tenure
\end{columns}

\vspace{\fill}
\footnotesize\textcolor{mlgray}{Stratification reduces variance and increases power}
\end{frame}

% Slide 6: Sequential Testing Implementation
\begin{frame}[fragile]{Python: Sequential Testing with Early Stopping}
\begin{columns}[T]
\column{0.55\textwidth}
\small
\textbf{Sequential Testing with O'Brien-Fleming:}

\begin{itemize}
\item Import numpy and scipy.stats.norm
\item Define O'Brien-Fleming boundary function:
  \begin{itemize}
  \item Input: current analysis number (k), total analyses (K)
  \item Calculate alpha spending based on information fraction
  \item Conservative early, liberal late
  \end{itemize}
\item Plan interim analyses (e.g., 25\%, 50\%, 75\%, 100\%)
\item For each analysis:
  \begin{itemize}
  \item Calculate current p-value
  \item Compare to adjusted alpha threshold
  \item Stop if significant, continue otherwise
  \end{itemize}
\item Maintains overall Type I error rate at 0.05
\item Enables early stopping without ``peeking'' penalty
\end{itemize}

\column{0.43\textwidth}
\textcolor{mlblue}{\textbf{Alpha Spending}}

\small
\textbf{Alpha Spending Schedule:}

\begin{itemize}
\item Look 1 (25\%): alpha = 0.0005
\item Look 2 (50\%): alpha = 0.0139
\item Look 3 (75\%): alpha = 0.0303
\item Look 4 (100\%): alpha = 0.0455
\end{itemize}

\vspace{0.3cm}
\textcolor{mlgreen}{\textbf{Interpretation}}

\small
\begin{itemize}
\item Very conservative early (0.0005)
\item More liberal late (0.0455)
\item Total $\alpha$ still 0.05
\item Can stop early if clear winner
\end{itemize}

\vspace{0.3cm}
\textcolor{mlorange}{\textbf{Benefits}}

\small
\begin{itemize}
\item Valid p-values at each look
\item Can stop for futility (clear loser)
\item Faster decisions on average
\item Maintains Type I error control
\end{itemize}

\vspace{0.3cm}
\textbf{Cost:} Slightly larger sample needed vs fixed design
\end{columns}

\vspace{\fill}
\footnotesize\textcolor{mlgray}{Sequential testing enables early stopping without ``peeking'' penalty}
\end{frame}

% Slide 7: Experiment Monitoring Dashboard
\begin{frame}{Real-Time Experiment Monitoring}
\begin{columns}[T]
\column{0.55\textwidth}
\includegraphics[width=\textwidth]{charts/experiment_velocity_dashboard.pdf}

\column{0.43\textwidth}
\textcolor{mlblue}{\textbf{Key Metrics to Monitor}}

\small
\textbf{Primary Metric:}
\begin{itemize}
\item Current estimate
\item Confidence interval
\item P-value or P(B $>$ A)
\item Sample size achieved
\end{itemize}

\vspace{0.2cm}
\textbf{Guardrail Metrics:}
\begin{itemize}
\item Revenue (must not drop $>$ 2\%)
\item Latency (p95 $<$ 200ms)
\item Error rate ($<$ 0.1\%)
\item Automatic alerts if violated
\end{itemize}

\vspace{0.2cm}
\textbf{Diagnostic Metrics:}
\begin{itemize}
\item Sample ratio mismatch
\item Traffic allocation balance
\item Novelty effects
\item Time-of-day patterns
\end{itemize}

\vspace{0.3cm}
\textcolor{mlgreen}{\textbf{Automated Actions}}

\small
\begin{itemize}
\item Stop if guardrail violated
\item Alert if sample imbalanced
\item Flag if novelty detected
\end{itemize}
\end{columns}

\vspace{\fill}
\footnotesize\textcolor{mlgray}{Continuous monitoring prevents disasters and enables fast decisions}
\end{frame}

% Slide 8: Common Pitfalls
\begin{frame}{Common A/B Testing Pitfalls}
\begin{columns}[T]
\column{0.32\textwidth}
\textcolor{mlred}{\textbf{Peeking Problem}}

\small
\textbf{What:} Check results daily, stop when p $<$ 0.05

\vspace{0.2cm}
\textbf{Why bad:}
\begin{itemize}
\item Inflates false positive rate
\item Random fluctuations look significant
\item Not reproducible
\end{itemize}

\vspace{0.2cm}
\textbf{Fix:}
\begin{itemize}
\item Wait for planned sample
\item Use sequential testing
\item Or Bayesian methods
\end{itemize}

\vspace{0.3cm}
\textcolor{mlorange}{\textbf{Multiple Testing}}

\small
\textbf{What:} Test 20 variants, report winner

\vspace{0.2cm}
\textbf{Why bad:}
\begin{itemize}
\item 1 false positive expected
\item Winner likely spurious
\end{itemize}

\vspace{0.2cm}
\textbf{Fix:}
\begin{itemize}
\item Bonferroni correction
\item Control family-wise error rate
\item Or use bandits
\end{itemize}

\column{0.32\textwidth}
\textcolor{mlblue}{\textbf{Simpson's Paradox}}

\small
\textbf{What:} Treatment wins overall, loses in every segment

\vspace{0.2cm}
\textbf{Example:}
\begin{itemize}
\item iOS: Control 10\%, Treatment 9\%
\item Android: Control 8\%, Treatment 7\%
\item Overall: Treatment wins (imbalance in device mix)
\end{itemize}

\vspace{0.2cm}
\textbf{Fix:}
\begin{itemize}
\item Stratified randomization
\item Analyze by segment
\item Check for confounders
\end{itemize}

\vspace{0.3cm}
\textcolor{mlgreen}{\textbf{Novelty Effects}}

\small
\textbf{What:} Users react to change itself, not treatment

\vspace{0.2cm}
\textbf{Signs:}
\begin{itemize}
\item Week 1: Treatment wins
\item Week 2: Effect shrinks
\item Week 3: No difference
\end{itemize}

\vspace{0.2cm}
\textbf{Fix:}
\begin{itemize}
\item Run longer (2-4 weeks)
\item Analyze by user cohort
\item Exclude early days
\end{itemize}

\column{0.32\textwidth}
\textcolor{mlpurple}{\textbf{Sample Ratio Mismatch}}

\small
\textbf{What:} 50/50 split becomes 52/48

\vspace{0.2cm}
\textbf{Causes:}
\begin{itemize}
\item Bot traffic
\item Implementation bugs
\item Sampling bias
\end{itemize}

\vspace{0.2cm}
\textbf{Why bad:}
\begin{itemize}
\item Groups not comparable
\item Results invalid
\end{itemize}

\vspace{0.2cm}
\textbf{Fix:}
\begin{itemize}
\item Monitor ratio daily
\item Investigate deviations
\item Filter bots
\end{itemize}

\vspace{0.3cm}
\textcolor{mlorange}{\textbf{Ignoring Variance}}

\small
\textbf{What:} Focus only on means, ignore spread

\vspace{0.2cm}
\textbf{Why bad:}
\begin{itemize}
\item High variance = unreliable
\item May hurt some users badly
\end{itemize}

\vspace{0.2cm}
\textbf{Fix:}
\begin{itemize}
\item Report quantiles (p10, p50, p90)
\item Check for outliers
\item Analyze subgroups
\end{itemize}
\end{columns}

\vspace{\fill}
\footnotesize\textcolor{mlgray}{Awareness of pitfalls is half the battle—set up checks proactively}
\end{frame}

% Slide 9: Production Deployment Patterns
\begin{frame}{Production Deployment Patterns}
\begin{columns}[T]
\column{0.32\textwidth}
\textcolor{mlblue}{\textbf{Canary Release}}

\small
\textbf{Process:}
\begin{enumerate}
\item Deploy to 1\% users
\item Monitor for 24 hours
\item If stable, 5\%
\item Then 25\%, 50\%, 100\%
\end{enumerate}

\vspace{0.2cm}
\textbf{When:}
\begin{itemize}
\item High-risk changes
\item New algorithms
\item Large refactors
\end{itemize}

\vspace{0.2cm}
\textbf{Benefits:}
\begin{itemize}
\item Early error detection
\item Minimal blast radius
\item Easy rollback
\end{itemize}

\vspace{0.2cm}
\textbf{Guardrails:}
\begin{itemize}
\item Error rate $<$ 0.5\%
\item Latency $<$ baseline + 20\%
\item Revenue $>$ baseline - 5\%
\end{itemize}

\column{0.32\textwidth}
\textcolor{mlgreen}{\textbf{Blue-Green Deploy}}

\small
\textbf{Process:}
\begin{enumerate}
\item Green: Current (100\%)
\item Blue: New (0\%)
\item Gradually shift traffic
\item Blue becomes 100\%
\end{enumerate}

\vspace{0.2cm}
\textbf{When:}
\begin{itemize}
\item Infrastructure changes
\item Database migrations
\item Zero-downtime deploys
\end{itemize}

\vspace{0.2cm}
\textbf{Benefits:}
\begin{itemize}
\item Instant rollback (flip traffic)
\item Both versions running
\item Compare metrics live
\end{itemize}

\vspace{0.2cm}
\textbf{Challenges:}
\begin{itemize}
\item 2$	imes$ infrastructure cost
\item Data consistency
\item State management
\end{itemize}

\column{0.32\textwidth}
\textcolor{mlorange}{\textbf{Shadow Mode}}

\small
\textbf{Process:}
\begin{enumerate}
\item Send traffic to both models
\item Serve old model to users
\item Log new model predictions
\item Compare offline
\end{enumerate}

\vspace{0.2cm}
\textbf{When:}
\begin{itemize}
\item Validating new model
\item Zero risk testing
\item Performance benchmarking
\end{itemize}

\vspace{0.2cm}
\textbf{Benefits:}
\begin{itemize}
\item No user impact
\item Real traffic testing
\item Can run indefinitely
\end{itemize}

\vspace{0.2cm}
\textbf{Use case:}
\begin{itemize}
\item Test before A/B test
\item Validate offline metrics
\item Check for bugs
\end{itemize}

\vspace{0.2cm}
\textbf{Limitation:} Can't measure business metrics (users don't see it)
\end{columns}

\vspace{\fill}
\footnotesize\textcolor{mlgray}{Modern deployments combine all three: Shadow $\rightarrow$ Canary $\rightarrow$ Blue-Green}
\end{frame}

% Slide 10: Implementation Summary
\begin{frame}{Implementation Summary \& Best Practices}
\begin{columns}[T]
\column{0.48\textwidth}
\textcolor{mlblue}{\textbf{Implementation Checklist}}

\small
\textbf{Before Experiment:}
\begin{itemize}
\item Hypothesis pre-registered
\item Sample size calculated
\item Randomization strategy chosen
\item Guardrails defined
\item Monitoring dashboard ready
\item Kill switch tested
\end{itemize}

\vspace{0.2cm}
\textbf{During Experiment:}
\begin{itemize}
\item Monitor metrics daily
\item Check sample ratio balance
\item Watch for guardrail violations
\item No peeking at p-values (unless sequential)
\item Document any issues
\end{itemize}

\vspace{0.2cm}
\textbf{After Experiment:}
\begin{itemize}
\item Run statistical tests
\item Calculate confidence intervals
\item Check subgroup analyses
\item Review guardrails
\item Make ship/iterate/kill decision
\item Document learnings
\end{itemize}

\column{0.48\textwidth}
\textcolor{mlgreen}{\textbf{Python Tools Ecosystem}}

\small
\textbf{Statistical Testing:}
\begin{itemize}
\item scipy: t-tests, z-tests
\item statsmodels: Advanced tests, power analysis
\item PyMC3: Bayesian methods
\end{itemize}

\vspace{0.2cm}
\textbf{Experimentation Platforms:}
\begin{itemize}
\item GrowthBook: Open-source, feature flags
\item Optimizely: Enterprise A/B testing
\item LaunchDarkly: Feature management
\item Firebase: Mobile experiments
\end{itemize}

\vspace{0.2cm}
\textbf{Visualization:}
\begin{itemize}
\item matplotlib/seaborn: Static plots
\item Plotly: Interactive dashboards
\item Streamlit: Quick apps
\end{itemize}

\vspace{0.3cm}
\textcolor{mlorange}{\textbf{Key Takeaways}}

\small
\begin{itemize}
\item Automate as much as possible
\item Monitor continuously
\item Plan for early stopping
\item Always have guardrails
\item Document everything
\end{itemize}
\end{columns}

\vspace{\fill}
\footnotesize\textcolor{mlgray}{Part 4: Building experimentation culture and communicating results}
\end{frame}