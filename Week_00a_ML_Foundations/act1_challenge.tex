% Act 1: The Challenge - Traditional Programming Hits the Wall
\section{Act 1: The Challenge}

% Section divider
\begin{frame}[plain]
\vfill
\centering
\begin{beamercolorbox}[sep=16pt,center]{title}
\usebeamerfont{title}\Large Act 1: The Challenge\par
\vspace{0.5em}
\large Traditional Programming Hits the Wall\par
\end{beamercolorbox}
\vfill
\end{frame}

% Slide 1: You Want Programs That Improve
\begin{frame}{You Want Programs That Improve}
\twocolslide{
\Large\textbf{The Spam Email Problem}
\normalsize
\vspace{0.5em}

You receive 100 emails per day:
\begin{itemize}
\item 30 are spam (scams, ads, phishing)
\item 70 are legitimate (work, friends, bills)
\item You spend 15 minutes per day deleting spam
\item That is 90 hours per year of wasted time
\end{itemize}

\vspace{0.5em}
\textbf{What you want:}
\begin{itemize}
\item A program that automatically detects spam
\item Gets better over time as spammers adapt
\item Learns your personal preferences
\item Costs you 0 minutes per day
\end{itemize}

\vspace{0.3em}
\keypoint{The program needs to LEARN from experience, not follow fixed rules}
}{
\Large\textbf{Why Traditional Programming Fails}
\normalsize
\vspace{0.5em}

\textbf{Attempt 1:} Write rules manually

\texttt{if email.contains("FREE MONEY") then spam}

\textbf{Problem:} Spammer writes "FR33 M0NEY"

\vspace{0.3em}
\textbf{Attempt 2:} Add more rules

\texttt{if matches("FR*E* M*NEY") then spam}

\textbf{Problem:} Now legitimate email "We offer free money market accounts" is blocked

\vspace{0.3em}
\textbf{Attempt 3:} Add exceptions to rules

After 500 rules and 200 exceptions, system is:
\begin{itemize}
\item Too complex to maintain
\item Full of contradictions
\item Still 40\% error rate
\item Breaks when spammers adapt
\end{itemize}

\bottomnote{This is the fundamental limitation: You cannot encode infinite patterns with finite rules}
}
\end{frame}

% Slide 2: The Rule Explosion Problem (with chart)
\begin{frame}{The Rule Explosion Problem}
\twocolslide{
\Large\textbf{Measured Failure Pattern}
\normalsize
\vspace{0.5em}

\textbf{Microsoft spam filter project (1998):}

\begin{center}
\begin{tabular}{lcc}
\hline
\textbf{Rules Written} & \textbf{Accuracy} & \textbf{Maintenance Hours/Week} \\
\hline
50 rules & 70\% & 2 hours \\
100 rules & 75\% & 5 hours \\
200 rules & 78\% & 12 hours \\
500 rules & 80\% & 35 hours \\
1000 rules & 81\% & 80+ hours \\
\hline
\end{tabular}
\end{center}

\vspace{0.3em}
\keypoint{Accuracy plateaus while complexity explodes}

\textbf{Root cause diagnosis:}
\begin{itemize}
\item Rules interact in unexpected ways
\item Exceptions need exceptions
\item New spam types require complete rewrite
\item Human experts cannot scale
\end{itemize}
}{
\centering
\includegraphics[width=0.85\textwidth]{charts/rule_complexity_explosion.pdf}

\vspace{0.5em}
\textbf{The Critical Insight:}

You need a program that:
\begin{itemize}
\item Discovers patterns automatically
\item Updates itself when data changes
\item Improves with more examples
\item Handles complexity you cannot encode
\end{itemize}

\vspace{0.3em}
This is what learning means.
}
\end{frame}

% Slide 3: What Does Learning Mean? Build from Scratch
\begin{frame}{What Does Learning Mean? Build from Scratch}
\twocolslide{
\Large\textbf{Start with Human Experience}
\normalsize
\vspace{0.5em}

\textbf{How do YOU learn to recognize spam?}

You read 100 emails and notice patterns:
\begin{itemize}
\item Spam uses ALL CAPS frequently
\item Spam has suspicious links (bit.ly/xxxx)
\item Spam comes from unknown senders
\item Spam has bad grammar
\end{itemize}

After 1000 emails, you get very good at this.

\vspace{0.5em}
\textbf{Three key elements:}
\begin{enumerate}
\item \textbf{Experience:} You saw labeled examples
\item \textbf{Task:} Classify spam vs not spam
\item \textbf{Performance:} You improved (60\% → 95\% accuracy)
\end{enumerate}

\keypoint{Learning is improvement through experience}
}{
\Large\textbf{Tom Mitchell's Formal Definition (1997)}
\normalsize
\vspace{0.5em}

A program learns from \textbf{Experience} $E$ at \textbf{Task} $T$ measured by \textbf{Performance} $P$ if its performance at $T$ improves with $E$.

\vspace{0.5em}
\textbf{Concrete spam filter example:}

\begin{itemize}
\item[$E$:] 10,000 labeled emails (spam/not spam)
\item[$T$:] Classify new incoming email
\item[$P$:] Accuracy percentage
\end{itemize}

\textbf{Before training:} Random guessing = 50\% accuracy

\textbf{After 1,000 examples:} 80\% accuracy

\textbf{After 10,000 examples:} 95\% accuracy

\vspace{0.3em}
\keypoint{Performance improved from 50\% to 95\% through experience}

\bottomnote{This definition covers all learning: supervised, unsupervised, reinforcement}
}
\end{frame}

% Slide 4: Three Ways to Learn (with worked examples)
\begin{frame}{Three Ways to Learn: The Complete Taxonomy}
\begin{columns}[T]
\begin{column}{0.32\textwidth}
\centering
\textcolor{mlblue}{\Large\textbf{Supervised Learning}}
\normalsize
\vspace{0.3em}

\textbf{You have labeled examples}

\textbf{Spam filter example:}
\begin{itemize}
\item Email 1: "FREE MONEY" → Spam
\item Email 2: "Meeting at 3pm" → Not spam
\item Email 3: "Click here now!" → Spam
\end{itemize}

Given 10,000 labeled emails, learn function:
\formula{f(\text{email}) \rightarrow \{\text{spam, not spam}\}}

\textbf{Test:} New email "WIN BIG NOW"
\textbf{Prediction:} Spam (98\% confidence)

\vspace{0.3em}
\textbf{Applications:}
\begin{itemize}
\item Image classification
\item Speech recognition
\item Medical diagnosis
\item Price prediction
\end{itemize}
\end{column}

\begin{column}{0.32\textwidth}
\centering
\textcolor{mlgreen}{\Large\textbf{Unsupervised Learning}}
\normalsize
\vspace{0.3em}

\textbf{You have NO labels}

\textbf{Customer segmentation:}
\begin{itemize}
\item Customer A: Buys luxury items, visits weekly
\item Customer B: Buys basics, visits monthly
\item Customer C: Buys luxury items, visits weekly
\end{itemize}

Algorithm discovers patterns:
\begin{itemize}
\item Group 1: Premium customers (A, C)
\item Group 2: Budget customers (B)
\end{itemize}

\textbf{You did not tell it these groups exist}

\vspace{0.3em}
\textbf{Applications:}
\begin{itemize}
\item Customer segmentation
\item Anomaly detection
\item Data compression
\item Topic discovery
\end{itemize}
\end{column}

\begin{column}{0.32\textwidth}
\centering
\textcolor{mlorange}{\Large\textbf{Reinforcement Learning}}
\normalsize
\vspace{0.3em}

\textbf{You learn through trial and reward}

\textbf{Game playing example:}
\begin{itemize}
\item Action: Move chess piece
\item Result: Win (+1) or Lose (-1)
\item No one tells you if each move is good
\item You learn which moves lead to wins
\end{itemize}

After 1 million games:
\begin{itemize}
\item Win rate: 20\% → 95\%
\item Discovers winning strategies
\item Gets superhuman performance
\end{itemize}

\vspace{0.3em}
\textbf{Applications:}
\begin{itemize}
\item Game AI (AlphaGo)
\item Robotics
\item Self-driving cars
\item Resource optimization
\end{itemize}
\end{column}
\end{columns}

\vspace{0.5em}
\bottomnote{This course focuses on supervised and unsupervised learning for innovation}
\end{frame}

% Slide 5: Quantifying the Challenge
\begin{frame}{Quantifying the Challenge: How Much Learning is Needed?}
\twocolslide{
\Large\textbf{The Data Requirements}
\normalsize
\vspace{0.5em}

\textbf{Measured learning curves for spam filter:}

\begin{center}
\begin{tabular}{lcc}
\hline
\textbf{Training Examples} & \textbf{Accuracy} & \textbf{Hours to Train} \\
\hline
100 emails & 65\% & 1 minute \\
1,000 emails & 82\% & 5 minutes \\
10,000 emails & 95\% & 30 minutes \\
100,000 emails & 98\% & 4 hours \\
1,000,000 emails & 98.5\% & 2 days \\
\hline
\end{tabular}
\end{center}

\vspace{0.3em}
\keypoint{Performance improves with data, but shows diminishing returns}

\textbf{Critical questions:}
\begin{itemize}
\item How much data is enough?
\item When do we stop improving?
\item What limits performance?
\end{itemize}
}{
\centering
\includegraphics[width=0.85\textwidth]{charts/learning_curve_spam.pdf}

\vspace{0.5em}
\textbf{The Bias-Variance Tension:}

\textbf{Too simple model:}
\begin{itemize}
\item Underfits the data
\item High training error (high bias)
\item Cannot capture complex patterns
\end{itemize}

\textbf{Too complex model:}
\begin{itemize}
\item Overfits the data
\item Low training error, high test error
\item Memorizes noise (high variance)
\end{itemize}

\textbf{The Challenge:} Find the right balance

\bottomnote{Act 2 will explore this tension through linear models}
}
\end{frame}