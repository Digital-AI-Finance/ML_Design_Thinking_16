% Week 8: Structured Output & Reliable AI Systems
% Machine Learning for Smarter Innovation - BSc Course
% REWRITE V2: Compact, Pedagogical, No Unverified Numbers
% Date: 2025-10-03
\documentclass[8pt,aspectratio=169]{beamer}
\usetheme{Madrid}
\usepackage{graphicx}
\usepackage{booktabs}
\usepackage{adjustbox}
\usepackage{multicol}
\usepackage{amsmath}
\usepackage{amssymb}
\usepackage{tcolorbox}
\usepackage{xcolor}
\usepackage{listings}

% Template Beamer Final Color Definitions (mllavender palette)
\definecolor{mlblue}{RGB}{0,102,204}
\definecolor{mlpurple}{RGB}{51,51,178}
\definecolor{mllavender}{RGB}{173,173,224}
\definecolor{mllavender2}{RGB}{193,193,232}
\definecolor{mllavender3}{RGB}{204,204,235}
\definecolor{mllavender4}{RGB}{214,214,239}
\definecolor{mlorange}{RGB}{255, 127, 14}
\definecolor{mlgreen}{RGB}{44, 160, 44}
\definecolor{mlred}{RGB}{214, 39, 40}
\definecolor{mlgray}{RGB}{127, 127, 127}

% Additional colors for template compatibility
\definecolor{lightgray}{RGB}{240, 240, 240}
\definecolor{midgray}{RGB}{180, 180, 180}
\definecolor{gray}{RGB}{127, 127, 127}

% Apply template theme (mllavender palette)
\setbeamercolor{palette primary}{bg=mllavender3,fg=mlpurple}
\setbeamercolor{palette secondary}{bg=mllavender2,fg=mlpurple}
\setbeamercolor{palette tertiary}{bg=mllavender,fg=white}
\setbeamercolor{palette quaternary}{bg=mlpurple,fg=white}

\setbeamercolor{structure}{fg=mlpurple}
\setbeamercolor{section in toc}{fg=mlpurple}
\setbeamercolor{subsection in toc}{fg=mlblue}
\setbeamercolor{title}{fg=mlpurple}
\setbeamercolor{frametitle}{fg=mlpurple,bg=mllavender3}
\setbeamercolor{block title}{bg=mllavender2,fg=mlpurple}
\setbeamercolor{block body}{bg=mllavender4,fg=black}

% Items with template accent colors
\setbeamercolor{item}{fg=mlorange}
\setbeamercolor{subitem}{fg=mlblue}
\setbeamercolor{subsubitem}{fg=mlpurple}

% Remove navigation symbols
\setbeamertemplate{navigation symbols}{}

% Clean itemize/enumerate
\setbeamertemplate{itemize items}[circle]
\setbeamertemplate{enumerate items}[default]

% Reduce margins for more content space
\setbeamersize{text margin left=5mm,text margin right=5mm}

% Custom footer
\setbeamertemplate{footline}{
  \leavevmode%
  \hbox{%
  \begin{beamercolorbox}[wd=.25\paperwidth,ht=2.25ex,dp=1ex,center]{author in head/foot}%
    \usebeamerfont{author in head/foot}Week 8
  \end{beamercolorbox}%
  \begin{beamercolorbox}[wd=.5\paperwidth,ht=2.25ex,dp=1ex,center]{title in head/foot}%
    \usebeamerfont{title in head/foot}Structured Outputs \& Reliable AI
  \end{beamercolorbox}%
  \begin{beamercolorbox}[wd=.25\paperwidth,ht=2.25ex,dp=1ex,right]{date in head/foot}%
    \usebeamerfont{date in head/foot}\insertframenumber{} / \inserttotalframenumber\hspace*{2ex}
  \end{beamercolorbox}}%
  \vskip0pt%
}

% Command for bottom annotation (template_beamer_final style)
\newcommand{\bottomnote}[1]{%
\vfill
\vspace{-2mm}
\textcolor{mllavender2}{\rule{\textwidth}{0.4pt}}
\vspace{1mm}
\footnotesize
\textbf{#1}%
}

% Graphics path
\graphicspath{{charts/}}

% Listings configuration for code
\lstset{
  basicstyle=\ttfamily\footnotesize,
  breaklines=true,
  frame=single,
  backgroundcolor=\color{mllavender4}
}

% Title information
\title{\Large Machine Learning for Smarter Innovation}
\subtitle{Week 8: Structured Outputs \& Reliable AI\\
When AI Needs Contracts, Not Suggestions}
\author{ML and Design Thinking Course}
\date{BSc Level - October 2025}

\begin{document}

% Title slide with plain style
\begin{frame}[plain]
\vspace{2cm}
\begin{center}
{\Huge \textcolor{mlpurple}{\textbf{Structured Outputs}}}\\[0.2cm]
{\Huge \textcolor{mlpurple}{\textbf{\&}}}\\[0.2cm]
{\Huge \textcolor{mlpurple}{\textbf{Reliable AI}}}\\[0.8cm]
{\Large When AI Needs Contracts, Not Suggestions}\\[1.5cm]
{\large Week 8: Machine Learning for Smarter Innovation}\\[0.3cm]
{\normalsize From Unpredictable Chaos to Production-Ready Systems}\\[0.5cm]
\end{center}
\end{frame}

% Overview slide
\begin{frame}{Today's Journey: Problem -> Method -> Solution}
\begin{columns}[T]
\column{0.48\textwidth}
\textcolor{mlred}{\textbf{Act 1: The Challenge}}
\begin{itemize}
\item The unpredictability problem
\item Why production systems need structure
\item Integration challenges
\item Current state
\end{itemize}

\vspace{0.5cm}
\textcolor{mlorange}{\textbf{Act 2: Naive Approach}}
\begin{itemize}
\item Better prompts seem obvious
\item How prompt engineering works
\item Initial success (builds hope)
\item Failure pattern emerges
\end{itemize}

\column{0.48\textwidth}
\textcolor{mlgreen}{\textbf{Act 3: The Breakthrough}}
\begin{itemize}
\item Human introspection
\item Structure-first hypothesis
\item The 3-layer architecture
\item Real implementation
\item Qualitative improvement
\end{itemize}

\vspace{0.5cm}
\textcolor{mlblue}{\textbf{Act 4: Synthesis}}
\begin{itemize}
\item Production architecture
\item Universal principles
\item Modern applications
\item Workshop preview
\end{itemize}
\end{columns}

\vspace{\fill}
\small\textcolor{mlgray}{From unpredictable outputs to production-ready AI systems}
\end{frame}

% Act 1: The Challenge
% Act 1: The Challenge (5 slides)

\begin{frame}{1. Real Estate Price Prediction}
\begin{columns}
\begin{column}{0.48\textwidth}
\textbf{The Business Problem}
\begin{itemize}
\item Predict house prices from features
\item Features: size, bedrooms, location, age
\item Target: price in thousands
\item Training data: 10,000 historical sales
\end{itemize}

\vspace{0.3cm}
\textbf{Sample Data Points}
\begin{tabular}{|c|c|c|c|}
\hline
Size & Beds & Age & Price \\
\hline
1200 & 2 & 5 & 250k \\
2500 & 4 & 10 & 450k \\
1800 & 3 & 2 & 380k \\
\hline
\end{tabular}
\end{column}
\begin{column}{0.48\textwidth}
\includegraphics[width=\textwidth]{charts/real_estate_scatter.pdf}

\vspace{0.2cm}
\small
Multiple features create complex relationships requiring mathematical modeling
\end{column}
\end{columns}
\end{frame}

\begin{frame}{2. Linear Regression as Baseline}
\begin{columns}
\begin{column}{0.48\textwidth}
\textbf{Mathematical Foundation}
\begin{itemize}
\item Model: $y = \beta_0 + \beta_1 x_1 + \beta_2 x_2 + ... + \epsilon$
\item Where $y$ = price, $x_i$ = features
\item Goal: Find best-fitting line/plane
\item Method: Minimize squared errors
\end{itemize}

\vspace{0.3cm}
\textbf{Assumptions}
\begin{itemize}
\item Linear relationship
\item Independent features
\item Constant variance
\item Normal errors
\end{itemize}
\end{column}
\begin{column}{0.48\textwidth}
\includegraphics[width=\textwidth]{charts/linear_regression_fit.pdf}

\vspace{0.2cm}
\small
Linear model assumes additive relationships between all features
\end{column}
\end{columns}
\end{frame}

\begin{frame}{3. Classification vs Regression}
\begin{columns}
\begin{column}{0.48\textwidth}
\textbf{Regression Problems}
\begin{itemize}
\item Predict continuous values
\item Examples: price, temperature, stock return
\item Output: Real numbers
\item Metrics: MSE, MAE, R-squared
\end{itemize}

\vspace{0.3cm}
\textbf{Classification Problems}
\begin{itemize}
\item Predict discrete categories
\item Examples: spam/ham, buy/sell/hold
\item Output: Class labels
\item Metrics: Accuracy, precision, recall
\end{itemize}
\end{column}
\begin{column}{0.48\textwidth}
\includegraphics[width=\textwidth]{charts/regression_vs_classification.pdf}

\vspace{0.2cm}
\small
Different problem types require different algorithms and evaluation metrics
\end{column}
\end{columns}
\end{frame}

\begin{frame}{4. The Curse of Dimensionality}
\begin{columns}
\begin{column}{0.48\textwidth}
\textbf{Feature Explosion Problem}
\begin{itemize}
\item Real estate: 20+ features
\item Interactions: $2^{20} = 1,048,576$ combinations
\item Sample: 10,000 data points
\item Ratio: 104 interactions per data point
\end{itemize}

\vspace{0.3cm}
\textbf{Mathematical Challenge}
\begin{itemize}
\item High-dimensional space is mostly empty
\item Distance metrics become meaningless
\item Overfitting becomes inevitable
\item ``Hughes phenomenon'' in pattern recognition
\end{itemize}
\end{column}
\begin{column}{0.48\textwidth}
\includegraphics[width=\textwidth]{charts/curse_dimensionality.pdf}

\vspace{0.2cm}
\small
As dimensions increase, all points become equidistant and patterns disappear
\end{column}
\end{columns}
\end{frame}

\begin{frame}{5. Feature Interactions Explode Combinatorially}
\begin{columns}
\begin{column}{0.48\textwidth}
\textbf{Combinatorial Mathematics}
\begin{itemize}
\item Linear terms: $n$ features
\item Pairwise: $\binom{n}{2} = \frac{n(n-1)}{2}$
\item Three-way: $\binom{n}{3} = \frac{n(n-1)(n-2)}{6}$
\item All subsets: $2^n - 1$
\end{itemize}

\vspace{0.3cm}
\textbf{Real Estate Example (n=20)}
\begin{itemize}
\item Linear: 20 terms
\item Pairwise: 190 interactions
\item Three-way: 1,140 interactions
\item Total possible: 1,048,575 terms
\end{itemize}
\end{column}
\begin{column}{0.48\textwidth}
\includegraphics[width=\textwidth]{charts/feature_combinations.pdf}

\vspace{0.2cm}
\small
{\color{mlred} Feature interactions grow exponentially, requiring regularization or feature selection}
\end{column}
\end{columns}
\end{frame}

% Act 2: Naive Approach
% ACT 2: THE NAIVE APPROACH (4 slides)
% Theme: Better prompts work... then fail
% CRITICAL: SUCCESS (Slide 7) BEFORE FAILURE (Slide 8)
% NO UNVERIFIED NUMBERS

\section{Act 2: The Naive Approach}

% Slide 5: Visual - The Obvious Solution
\begin{frame}[t]{The Obvious Solution: Just Write Better Prompts}
\vspace{-0.3cm}
\begin{center}
\includegraphics[width=0.85\textwidth]{prompt_engineering_patterns.pdf}
\end{center}

\begin{center}
\textbf{Key Insight}: If AI outputs are messy, clearer instructions should produce cleaner, more consistent results
\end{center}

\bottomnote{The naive hypothesis: Vague prompts cause chaos, detailed prompts create consistency}
\end{frame}

% Slide 6: Detail - How Prompt Engineering Works
\begin{frame}[t]{How Prompt Engineering Works: Five Common Techniques}
\small
\begin{columns}[T]
\column{0.49\textwidth}
\raggedright
\textbf{The Techniques}:

\vspace{0.2cm}
\textbf{1. Detailed Instructions}
\begin{itemize}
\item Specify exactly what to extract
\item List all required fields
\item Describe desired format
\end{itemize}

\vspace{0.2cm}
\textbf{2. Few-Shot Examples}
\begin{itemize}
\item Show 3-5 example outputs
\item Demonstrate desired format
\item Illustrate edge cases
\end{itemize}

\vspace{0.2cm}
\textbf{3. Role-Playing}
\begin{itemize}
\item "You are an expert data analyst..."
\item Sets context and expectations
\item Encourages professional output
\end{itemize}

\column{0.49\textwidth}
\raggedright
\textbf{The Techniques (continued)}:

\vspace{0.2cm}
\textbf{4. Step-by-Step Guidance}
\begin{itemize}
\item Break task into steps
\item "First identify..., then extract..."
\item Chain of thought reasoning
\end{itemize}

\vspace{0.2cm}
\textbf{5. Format Specification}
\begin{itemize}
\item "Return as JSON with keys..."
\item Describe field types
\item Request specific structure
\end{itemize}

\vspace{0.3cm}
\textbf{When It Helps}:
\begin{itemize}
\item Simple, clean inputs
\item Standard formats
\item Well-structured source data
\item Few edge cases
\end{itemize}
\end{columns}

\bottomnote{Prompt engineering improves quality through clearer communication - but is it enough for production?}
\end{frame}

% Slide 7: Visual - SUCCESS (CRITICAL - Build Hope)
\begin{frame}[t]{Success: When Prompt Engineering Works Beautifully}
\vspace{-0.3cm}
\begin{center}
\includegraphics[width=0.85\textwidth]{success_examples.pdf}
\end{center}

\begin{center}
\textcolor{mlgreen}{\textbf{On simple, well-formatted inputs, prompt engineering delivers consistent, high-quality results}}
\end{center}

\vspace{0.3cm}
\begin{columns}[T]
\column{0.32\textwidth}
\small
\textbf{Clean Data}:
\begin{itemize}
\item Standard formats
\item All fields present
\item No ambiguity
\item Expected patterns
\end{itemize}

\column{0.32\textwidth}
\small
\textbf{Consistent Output}:
\begin{itemize}
\item Correct extractions
\item Proper formatting
\item All fields captured
\item Parseable results
\end{itemize}

\column{0.32\textwidth}
\small
\textbf{Success Factors}:
\begin{itemize}
\item Simple inputs
\item Clear examples
\item Detailed prompts
\item Standard cases
\end{itemize}
\end{columns}

\bottomnote{This looks promising - detailed prompts produce reliable results on clean, simple data}
\end{frame}

% Slide 8: Visual - FAILURE (CRITICAL - Show Pattern)
\begin{frame}[t]{Failure: When Real-World Complexity Reveals Fundamental Limits}
\vspace{-0.3cm}
\begin{center}
\includegraphics[width=0.85\textwidth]{failure_pattern.pdf}
\end{center}

\begin{center}
\textcolor{mlred}{\textbf{On complex, messy real-world data, prompt engineering breaks down systematically}}
\end{center}

\vspace{0.3cm}
\begin{columns}[T]
\column{0.32\textwidth}
\small
\textbf{Messy Data}:
\begin{itemize}
\item Multiple formats mixed
\item Missing fields
\item Extra information
\item Ambiguous values
\item Unexpected patterns
\end{itemize}

\column{0.32\textwidth}
\small
\textbf{Inconsistent Output}:
\begin{itemize}
\item Variable formats
\item Wrong field names
\item Type mismatches
\item Parse failures
\item Unpredictable errors
\end{itemize}

\column{0.32\textwidth}
\small
\textbf{Root Cause}:
\begin{itemize}
\item Prompts are suggestions
\item No enforcement
\item AI can ignore format
\item No validation
\item No recovery mechanism
\end{itemize}
\end{columns}

\vspace{0.5cm}
\begin{center}
\textcolor{mlpurple}{\textbf{The Pattern:}} Simple cases work, complex cases fail - because prompts \textit{describe} but don't \textit{enforce}
\end{center}

\bottomnote{Failure pattern reveals fundamental limit: Suggestions can't guarantee structure when integrated systems demand it}
\end{frame}


% Act 3: The Breakthrough
% ACT 3: THE BREAKTHROUGH (6 slides)
% Theme: From human insight to technical solution
% Pattern: Visual + Detailed, with code examples
% NO UNVERIFIED NUMBERS

\section{Act 3: The Breakthrough}

% Slide 9: Human Introspection
\begin{frame}[t]{The Key Question: How Do YOU Ensure Data Consistency?}
\textbf{Before we design a solution, observe your own behavior:}

\vspace{0.5cm}

\begin{columns}[T]
\column{0.48\textwidth}
\textcolor{mlpurple}{\textbf{Scenario 1: Filling a Form}}

\small
You're entering customer data into a database:
\begin{itemize}
\item \textbf{First}: Check what fields are required
\item \textbf{Then}: Enter data matching field types
\item \textbf{Validate}: Form rejects if types don't match
\item \textbf{Fix}: Correct errors before submitting
\end{itemize}

\vspace{0.2cm}
\textbf{Key observation}: You \textit{validate against a schema} before submission

\column{0.48\textwidth}
\textcolor{mlpurple}{\textbf{Scenario 2: Creating a Spreadsheet}}

\small
You're standardizing product data:
\begin{itemize}
\item \textbf{First}: Define column headers (schema)
\item \textbf{Then}: Enter data in correct columns
\item \textbf{Validate}: Check types, ranges, required fields
\item \textbf{Enforce}: Use data validation rules
\end{itemize}

\vspace{0.2cm}
\textbf{Key observation}: You \textit{define structure first}, then fill it
\end{columns}

\vspace{0.5cm}

\begin{center}
\textcolor{mlgreen}{\Large \textbf{The Pattern}}
\end{center}

\begin{center}
\textbf{Humans ensure consistency by:}\\
1. Defining schema/structure FIRST\\
2. Validating data against that structure\\
3. Rejecting invalid entries\\
4. Fixing errors before proceeding
\end{center}

\bottomnote{Human introspection reveals the solution: Structure-first, validate-always, reject-invalid approach}
\end{frame}

% Slide 10: Visual - The Hypothesis
\begin{frame}[t]{The Hypothesis: Structure First, Then Generate}
\vspace{-0.3cm}
\begin{center}
\includegraphics[width=0.85\textwidth]{prompts_vs_schemas.pdf}
\end{center}

\begin{center}
\textbf{Key Insight}: Prompts suggest format (weak), schemas enforce structure (strong) - enforcement beats suggestion
\end{center}

\bottomnote{The breakthrough hypothesis: If we define schema first, AI can be forced to conform rather than suggest}
\end{frame}

% Slide 11: Detail - Zero-Jargon Explanation
\begin{frame}[t]{The Solution in Plain English: What It Does and Why It Works}
\small
\begin{columns}[T]
\column{0.49\textwidth}
\raggedright
\textbf{What It Does (No Technical Terms)}:

\vspace{0.2cm}
\textbf{Step 1: Define Contract}
\begin{itemize}
\item List exactly what fields you need
\item Specify types (text, number, date)
\item Mark which fields are required
\item Like a database table definition
\end{itemize}

\vspace{0.2cm}
\textbf{Step 2: Send to AI}
\begin{itemize}
\item Give AI the contract along with data
\item AI must return data matching contract
\item API-level enforcement (not just prompt)
\item AI literally cannot return wrong format
\end{itemize}

\vspace{0.2cm}
\textbf{Step 3: Validate and Recover}
\begin{itemize}
\item Check all required fields present
\item Verify types match specification
\item Retry if validation fails
\item Log errors for debugging
\end{itemize}

\column{0.49\textwidth}
\raggedright
\textbf{Why It Works}:

\vspace{0.2cm}
\textbf{Enforcement Mechanism}
\begin{itemize}
\item Not a suggestion - it's a requirement
\item API rejects non-conforming output
\item Like database rejecting bad INSERT
\item Guaranteed structure or explicit error
\end{itemize}

\vspace{0.2cm}
\textbf{Predictable Failures}
\begin{itemize}
\item Failures are caught immediately
\item Error messages are specific
\item Retry logic can handle failures
\item No silent corruption
\end{itemize}

\vspace{0.2cm}
\textbf{System Integration}
\begin{itemize}
\item Output is parseable (guaranteed)
\item Fields match database schema
\item Types are validated
\item Downstream systems accept input
\end{itemize}
\end{columns}

\vspace{0.3cm}
\begin{center}
\textcolor{mlpurple}{\textbf{Core Principle:}} Contract → Generate → Validate (not Hope → Generate → Fix)
\end{center}

\bottomnote{Zero-jargon explanation reveals the mechanism: Define structure, enforce at API level, validate before accepting}
\end{frame}

% Slide 12: Visual - The 3-Layer Architecture
\begin{frame}[t]{The 3-Layer Architecture: Schema, Generation, Validation}
\vspace{-0.3cm}
\begin{center}
\includegraphics[width=0.85\textwidth]{three_layer_architecture.pdf}
\end{center}

\begin{center}
\textbf{Key Insight}: Three layers working together transform unreliable text into reliable structured data
\end{center}

\bottomnote{Complete architecture: Schema defines contract, Function calling enforces it, Validation catches edge cases}
\end{frame}

% Slide 13: Code Example - Schema Definition
\begin{frame}[fragile,t]{Layer 1: Schema Definition with Pydantic}
\textbf{Real code defining a type-safe contract:}

\vspace{0.3cm}

\begin{lstlisting}[language=Python]
from pydantic import BaseModel, Field

class ProductExtraction(BaseModel):
    """Schema for structured product data extraction"""

    product_name: str = Field(
        description="Full product name"
    )

    price: float = Field(
        description="Price in USD",
        gt=0  # Must be positive
    )

    storage_gb: int = Field(
        description="Storage capacity in GB",
        ge=0  # Greater or equal to 0
    )

    confidence: float = Field(
        description="Extraction confidence score",
        ge=0.0, le=1.0  # Between 0 and 1
    )
\end{lstlisting}

\vspace{0.3cm}

\textbf{What this achieves}:
\begin{itemize}
\item Type safety: price must be float, storage must be int
\item Validation: automatic type checking and range constraints
\item Documentation: each field has clear description
\item Contract: downstream code knows exactly what to expect
\end{itemize}

\bottomnote{Pydantic schemas create type-safe contracts that validate automatically}
\end{frame}

% Slide 14: Code Example - Function Calling + Validation
\begin{frame}[fragile,t]{Layers 2 \& 3: Function Calling with Validation}
\textbf{Real code enforcing structure and validating output:}

\vspace{0.3cm}

\begin{lstlisting}[language=Python]
from openai import OpenAI

client = OpenAI()

# Convert Pydantic schema to JSON schema
schema = ProductExtraction.model_json_schema()

# Layer 2: Function calling (enforcement at API level)
response = client.chat.completions.create(
    model="gpt-4",
    messages=[{"role": "user", "content": product_text}],
    tools=[{
        "type": "function",
        "function": {
            "name": "extract_product",
            "description": "Extract product information",
            "parameters": schema
        }
    }]
)

# Layer 3: Validation and recovery
try:
    # Extract structured data
    args = response.choices[0].message.tool_calls[0].function.arguments

    # Validate against schema
    product = ProductExtraction.model_validate_json(args)

    # Success - guaranteed structure
    print(f"Product: {product.product_name}, Price: ${product.price}")

except Exception as e:
    # Validation failed - retry or escalate
    print(f"Validation error: {e}")
    # Implement retry logic here
\end{lstlisting}

\bottomnote{Function calling enforces schema at API level, validation catches any edge cases, retry handles failures}
\end{frame}

% Slide 15: Visual - Before/After Comparison (Qualitative)
\begin{frame}[t]{Before and After: The Transformation (Qualitative)}
\vspace{-0.3cm}
\begin{center}
\includegraphics[width=0.85\textwidth]{before_after_qualitative.pdf}
\end{center}

\begin{center}
\textcolor{mlgreen}{\textbf{Structured outputs with validation deliver reliable, production-ready results where prompt engineering alone breaks down}}
\end{center}

\vspace{0.3cm}

\begin{columns}[T]
\column{0.48\textwidth}
\small
\textcolor{mlred}{\textbf{Before: Prompt Engineering}}
\begin{itemize}
\item Works on simple cases
\item Breaks on complex, messy data
\item Variable output formats
\item Unpredictable failures
\item Manual intervention needed
\item Not production-ready
\end{itemize}

\column{0.48\textwidth}
\small
\textcolor{mlgreen}{\textbf{After: Structured Outputs}}
\begin{itemize}
\item Works across complexity levels
\item Handles messy, real-world data
\item Guaranteed consistent format
\item Predictable error modes
\item Automatic retry and recovery
\item Production-grade reliability
\end{itemize}
\end{columns}

\bottomnote{Qualitative improvement: From unreliable prototypes to production-ready systems through structure and validation}
\end{frame}


% Act 4: Synthesis
% Act 4: Synthesis (4 slides)

\section{\color{synthesispurple}Act 4: Synthesis}

\begin{frame}{Slide 21: Implementation: sklearn.cluster}
\begin{columns}[T]
\column{0.48\textwidth}
\textbf{\color{synthesispurple}Scikit-learn Implementation}

\textbf{K-means Implementation:}
\begin{itemize}
\item \texttt{from sklearn.cluster import KMeans}
\item \texttt{kmeans = KMeans(n\_clusters=3)}
\item \texttt{labels = kmeans.fit\_predict(X)}
\item \texttt{centroids = kmeans.cluster\_centers\_}
\end{itemize}

\textbf{DBSCAN Implementation:}
\begin{itemize}
\item \texttt{from sklearn.cluster import DBSCAN}
\item \texttt{dbscan = DBSCAN(eps=0.5, min\_samples=5)}
\item \texttt{labels = dbscan.fit\_predict(X)}
\end{itemize}

\textbf{Hierarchical Implementation:}
\begin{itemize}
\item \texttt{from sklearn.cluster import}
\item \texttt{AgglomerativeClustering}
\item \texttt{labels = hierarchical.fit\_predict(X)}
\end{itemize}

\column{0.48\textwidth}
\textbf{Production Pipeline:}
\begin{center}
\includegraphics[width=\textwidth]{charts/sklearn_pipeline.pdf}
\end{center}

\textbf{Evaluation Tools:}
\begin{itemize}
\item \texttt{from sklearn.metrics import}
\item \texttt{silhouette\_score, adjusted\_rand\_score}
\item \texttt{sil\_score = silhouette\_score(X, labels)}
\item \texttt{ari\_score = adjusted\_rand\_score(}
\item \texttt{true\_labels, predicted\_labels)}
\end{itemize}
\end{columns}

\vspace{\fill}
{\footnotesize \color{textgray}Implementation: Production-ready clustering with scikit-learn}
\end{frame}

\begin{frame}{Slide 22: Clustering Method Taxonomy}
\begin{columns}[T]
\column{0.48\textwidth}
\textbf{\color{synthesispurple}Clustering Algorithm Families}

\textbf{Centroid-Based:}
\begin{itemize}
\item K-means, K-medoids
\item Assumes spherical clusters
\item Fast, scalable
\item Requires k specification
\end{itemize}

\textbf{Density-Based:}
\begin{itemize}
\item DBSCAN, OPTICS, Mean-shift
\item Handles arbitrary shapes
\item Automatic noise detection
\item Parameter sensitive
\end{itemize}

\textbf{Hierarchical:}
\begin{itemize}
\item Agglomerative, Divisive
\item Creates cluster tree
\item No k pre-specification
\item Computationally expensive
\end{itemize}

\column{0.48\textwidth}
\textbf{Algorithm Taxonomy Tree:}
\begin{center}
\includegraphics[width=\textwidth]{charts/clustering_taxonomy.pdf}
\end{center}

\textbf{Modern Extensions:}
\begin{itemize}
\item Spectral clustering (graph-based)
\item Gaussian mixture models (probabilistic)
\item Deep clustering (neural networks)
\item Fuzzy clustering (soft assignments)
\end{itemize}

\textbf{Selection Criteria:} Data shape, size, interpretability needs.
\end{columns}

\vspace{\fill}
{\footnotesize \color{textgray}Taxonomy: Understanding the clustering algorithm landscape}
\end{frame}

\begin{frame}{Slide 23: Algorithm Selection Guide}
\begin{columns}[T]
\column{0.48\textwidth}
\textbf{\color{synthesispurple}Decision Framework}

\textbf{Ask These Questions:}

\textbf{1. What shapes do you expect?}
\begin{itemize}
\item Spherical \textrightarrow K-means
\item Arbitrary \textrightarrow DBSCAN
\item Unknown \textrightarrow Hierarchical
\end{itemize}

\textbf{2. Do you know k?}
\begin{itemize}
\item Yes \textrightarrow K-means/Hierarchical
\item No \textrightarrow DBSCAN
\end{itemize}

\textbf{3. How much noise?}
\begin{itemize}
\item Clean data \textrightarrow K-means
\item Noisy data \textrightarrow DBSCAN
\end{itemize}

\textbf{4. What's your dataset size?}
\begin{itemize}
\item Large (>10K) \textrightarrow K-means
\item Medium \textrightarrow Any method
\item Small (<1K) \textrightarrow Hierarchical
\end{itemize}

\column{0.48\textwidth}
\textbf{Selection Decision Tree:}
\begin{center}
\includegraphics[width=\textwidth]{charts/algorithm_selection.pdf}
\end{center}

\textbf{Business Considerations:}
\begin{itemize}
\item Interpretability: Hierarchical wins
\item Speed: K-means fastest
\item Robustness: DBSCAN handles outliers
\item Exploration: Try multiple methods
\end{itemize}

\textbf{Hybrid Approach:} Start with K-means, validate with DBSCAN.
\end{columns}

\vspace{\fill}
{\footnotesize \color{textgray}Practical guide: Choosing the right clustering algorithm for your problem}
\end{frame}

\begin{frame}{Slide 24: Modern Applications \& Neural Network Preview}
\begin{columns}[T]
\column{0.48\textwidth}
\textbf{\color{synthesispurple}Real-World Applications}

\textbf{Anomaly Detection:}
\begin{itemize}
\item Fraud detection in banking
\item Network intrusion detection
\item Quality control in manufacturing
\item Medical diagnosis outliers
\end{itemize}

\textbf{Recommendation Systems:}
\begin{itemize}
\item User behavior clustering
\item Product category discovery
\item Content similarity grouping
\item Market basket analysis
\end{itemize}

\textbf{Business Intelligence:}
\begin{itemize}
\item Customer segmentation
\item Market research
\item Operational optimization
\item Risk assessment
\end{itemize}

\column{0.48\textwidth}
\textbf{Modern Clustering Evolution:}
\begin{center}
\includegraphics[width=\textwidth]{charts/modern_applications.pdf}
\end{center}

\textbf{Neural Network Preview:}
\begin{itemize}
\item Autoencoders for dimensionality reduction
\item Self-organizing maps (SOMs)
\item Deep embedded clustering
\item Variational autoencoders
\end{itemize}

\textbf{Next Steps:} Deep learning approaches that learn representations and cluster simultaneously.
\end{columns}

\vspace{\fill}
{\footnotesize \color{textgray}Future directions: From classical clustering to neural network approaches}
\end{frame}

\end{document}
