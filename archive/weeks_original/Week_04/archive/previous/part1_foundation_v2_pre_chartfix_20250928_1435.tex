% Part 1: Foundation - Why Classification? (10 slides)
\section{Part 1: Foundation - Why Classification?}

% Slide 1: Hook - Opening visualization without title
\begin{frame}[plain]
\centering
\includegraphics[width=0.95\textwidth]{charts/innovation_success_dashboard.pdf}
\end{frame}

% Slide 2: Problem - The human judgment challenge
\begin{frame}{Why Human Judgment Fails at Scale}
\Large\textbf{The Definition Challenge}
\normalsize

\vspace{0.5em}

\begin{columns}[T]
\begin{column}{0.48\textwidth}
\textbf{Current Reality:}
\begin{itemize}
\item 1000+ innovation ideas per year
\item 10 evaluators = 10 different opinions
\item Decisions based on ``gut feel''
\item Success rate: 2-5\%
\item Millions lost on wrong bets
\end{itemize}
\end{column}
\begin{column}{0.48\textwidth}
\textbf{The Problem:}
\begin{itemize}
\item \textcolor{mlred}{Subjective}: ``I know it when I see it''
\item \textcolor{mlred}{Inconsistent}: Changes with mood/time
\item \textcolor{mlred}{Biased}: Favors familiar patterns
\item \textcolor{mlred}{Limited}: Can't process volume
\item \textcolor{mlred}{Expensive}: Expert time = \$\$\$
\end{itemize}
\end{column}
\end{columns}

\vspace{1em}
\begin{tcolorbox}[colback=mlred!10,colframe=mlred]
\centering
\textbf{Question:} Can we make innovation evaluation objective, consistent, and scalable?
\end{tcolorbox}
\end{frame}

% Slide 3: Evolution - Timeline of progress
\begin{frame}{The Journey from Intuition to Intelligence}
\Large\textbf{How Innovation Assessment Evolved}
\normalsize

\vspace{0.5em}

\begin{center}
\begin{tikzpicture}[scale=0.9, transform shape]
% Timeline arrow
\draw[ultra thick,->,mlgray] (0,0) -- (12,0);

% Time periods
\foreach \x/\year/\era in {1/1950s/Gut Feel, 4/1980s/Metrics, 7/2010s/Analytics, 10/2020s/ML, 12/Now/AI} {
    \draw[thick] (\x,0) -- (\x,-0.2);
    \node[below] at (\x,-0.3) {\small\textbf{\year}};
    \node[below] at (\x,-0.7) {\tiny \era};
}

% Evolution boxes
\node[draw,fill=mlred!20,rounded corners,minimum width=2cm] at (1,1.5) {Experience};
\node[draw,fill=mlorange!20,rounded corners,minimum width=2cm] at (4,1.5) {Spreadsheets};
\node[draw,fill=mlyellow!20,rounded corners,minimum width=2cm] at (7,1.5) {Dashboards};
\node[draw,fill=mlgreen!20,rounded corners,minimum width=2cm] at (10,1.5) {Algorithms};
\node[draw,fill=mlblue!20,rounded corners,minimum width=2cm] at (12,1.5) {Predictions};

% Success rates
\node[mlred] at (1,2.3) {\textbf{45\%}};
\node[mlorange] at (4,2.3) {\textbf{55\%}};
\node[mlyellow] at (7,2.3) {\textbf{65\%}};
\node[mlgreen] at (10,2.3) {\textbf{82\%}};
\node[mlblue] at (12,2.3) {\textbf{94\%}};

% Key innovations
\node[text width=2cm,align=center] at (1,0.7) {\tiny Individual expertise};
\node[text width=2cm,align=center] at (4,0.7) {\tiny Basic KPIs};
\node[text width=2cm,align=center] at (7,0.7) {\tiny Big Data};
\node[text width=2cm,align=center] at (10,0.7) {\tiny Machine Learning};
\node[text width=2cm,align=center] at (12,0.7) {\tiny Deep Learning};
\end{tikzpicture}
\end{center}

\vspace{0.5em}
\begin{center}
\Large\textit{Each leap forward = Better pattern recognition}
\end{center}
\end{frame}

% Slide 4: Patterns - Show that data reveals patterns
\begin{frame}{Success Patterns Hide in Plain Sight}
\Large\textbf{What 9,500 Innovations Taught Us}
\normalsize

\vspace{0.5em}

\begin{columns}[T]
\begin{column}{0.55\textwidth}
\includegraphics[width=\textwidth]{charts/feature_space_visualization.pdf}
\end{column}
\begin{column}{0.43\textwidth}
\textbf{Hidden Patterns Found:}
\begin{itemize}
\item \textcolor{mlgreen}{\textbf{Sweet spot:}} Novelty 70-85\%
\item \textcolor{mlblue}{\textbf{Team magic:}} Experience + Diversity
\item \textcolor{mlorange}{\textbf{Timing:}} 6-9 months optimal
\item \textcolor{mlred}{\textbf{Market size:}} \$1-5M best start
\end{itemize}

\vspace{0.5em}
\textbf{The Revelation:}
\begin{tcolorbox}[colback=mlgreen!10,colframe=mlgreen]
Success isn't random - it follows discoverable patterns
\end{tcolorbox}
\end{column}
\end{columns}
\end{frame}

% Slide 5: Dataset - Our laboratory
\begin{frame}{Our Innovation Laboratory}
\Large\textbf{Learning from Real-World Data}
\normalsize

\vspace{0.5em}

\begin{columns}[T]
\begin{column}{0.55\textwidth}
\includegraphics[width=\textwidth]{charts/innovation_dataset_overview.pdf}
\end{column}
\begin{column}{0.43\textwidth}
\textbf{The Dataset:}
\begin{itemize}
\item \textbf{9,500} innovation products
\item \textbf{8 years} of outcomes (2015-2023)
\item \textbf{27 features} per product
\item \textbf{6 segments} (Tech, Consumer, B2B...)
\end{itemize}

\vspace{0.5em}
\textbf{What We Track:}
\begin{enumerate}
\item Innovation metrics
\item Market conditions
\item Team composition
\item Development process
\item Financial indicators
\end{enumerate}

\vspace{0.5em}
\small\textcolor{mlgray}{Note: Simulated for educational purposes}
\end{column}
\end{columns}
\end{frame}

% Slide 6: Perspectives - Different ways to frame success
\begin{frame}{Two Lenses for Looking at Success}
\Large\textbf{Binary vs Multi-Class Perspectives}
\normalsize

\vspace{0.5em}

\begin{columns}[T]
\begin{column}{0.48\textwidth}
\textbf{Binary: Yes or No?}
\begin{center}
\begin{tikzpicture}[scale=0.8]
\draw[ultra thick,->] (0,0) -- (4,0);
\draw[fill=mlred!50] (0,-0.5) rectangle (1.8,0.5);
\draw[fill=mlgreen!50] (2.2,-0.5) rectangle (4,0.5);
\node at (0.9,0) {\textbf{FAIL}};
\node at (3.1,0) {\textbf{SUCCESS}};
\node at (0.9,-1) {40\%};
\node at (3.1,-1) {60\%};
\end{tikzpicture}
\end{center}

\textbf{Use When:}
\begin{itemize}
\item Go/No-go decisions
\item Limited resources
\item Clear threshold needed
\item Quick filtering required
\end{itemize}
\end{column}
\begin{column}{0.48\textwidth}
\textbf{Multi-Class: How Successful?}
\begin{center}
\begin{tikzpicture}[scale=0.7]
\draw[fill=mlred!50] (0,0) rectangle (1.5,1);
\node at (0.75,0.5) {\tiny Failed};
\node at (0.75,-0.3) {24\%};

\draw[fill=mlorange!50] (1.5,0) rectangle (3,1);
\node at (2.25,0.5) {\tiny Struggling};
\node at (2.25,-0.3) {16\%};

\draw[fill=mlblue!50] (0,1) rectangle (1.5,2);
\node at (0.75,1.5) {\tiny Growing};
\node at (0.75,2.3) {39\%};

\draw[fill=mlgreen!50] (1.5,1) rectangle (3,2);
\node at (2.25,1.5) {\tiny Breakthrough};
\node at (2.25,2.3) {21\%};
\end{tikzpicture}
\end{center}

\textbf{Use When:}
\begin{itemize}
\item Resource allocation
\item Risk assessment
\item Support prioritization
\item Nuanced understanding
\end{itemize}
\end{column}
\end{columns}

\vspace{0.5em}
\begin{center}
\textit{Same data, different questions → different insights}
\end{center}
\end{frame}

% Slide 7: Diamond - Connect to course theme
\begin{frame}{Classification Powers the Innovation Diamond}
\Large\textbf{From 5000 Ideas to 5 Winners}
\normalsize

\vspace{0.5em}

\begin{center}
\begin{tikzpicture}[scale=0.8, transform shape]
% Diamond shape
\coordinate (top) at (0,3);
\coordinate (left) at (-3,0);
\coordinate (right) at (3,0);
\coordinate (bottom) at (0,-3);

% Draw diamond with gradient effect
\draw[ultra thick,fill=challenge!10] (top) -- (left) -- (bottom) -- (right) -- cycle;

% Add classification stages
\node[draw,circle,fill=mlred!70,minimum size=1cm] at (-1.5,1.5) {\textbf{1}};
\node[draw,circle,fill=mlorange!70,minimum size=1cm] at (0,1) {\textbf{2}};
\node[draw,circle,fill=mlgreen!70,minimum size=1cm] at (1.5,1.5) {\textbf{3}};
\node[draw,circle,fill=mlblue!70,minimum size=1cm] at (0,-1) {\textbf{4}};

% Labels
\node[above] at (top) {\Large\textbf{5000 Ideas}};
\node[left] at (left) {\textbf{Diverge}};
\node[right] at (right) {\textbf{Converge}};
\node[below] at (bottom) {\Large\textbf{5 Strategies}};

% Classification roles
\node[right,text width=4cm] at (4,2) {
\textcolor{mlred}{\textbf{1. Quick Filter:}} Remove obvious failures (Binary)
};
\node[right,text width=4cm] at (4,0.5) {
\textcolor{mlorange}{\textbf{2. Categorize:}} Group by type (Multi-class)
};
\node[right,text width=4cm] at (4,-1) {
\textcolor{mlgreen}{\textbf{3. Rank:}} Success probability scores
};
\node[right,text width=4cm] at (4,-2.5) {
\textcolor{mlblue}{\textbf{4. Select:}} Top 5 with confidence
};
\end{tikzpicture}
\end{center}
\end{frame}

% Slide 8: Impact - Real-world examples
\begin{frame}{Real Companies, Real Results}
\Large\textbf{Classification in Action}
\normalsize

\vspace{0.5em}

\begin{columns}[T]
\begin{column}{0.32\textwidth}
\textbf{Amazon}
\begin{center}
\includegraphics[width=0.6\textwidth]{charts/precision_recall_curves.pdf}
\end{center}
\begin{itemize}
\small
\item 100M+ products classified
\item 35\% better recommendations
\item \$30B revenue impact
\end{itemize}
\end{column}
\begin{column}{0.32\textwidth}
\textbf{Netflix}
\begin{itemize}
\small
\item Content success prediction
\item User taste classification
\item 75\% of views from ML
\item Saved \$1B/year in content
\end{itemize}
\vspace{0.5em}
\begin{center}
\accuracy{93} prediction accuracy
\end{center}
\end{column}
\begin{column}{0.32\textwidth}
\textbf{Spotify}
\begin{itemize}
\small
\item 4B+ playlists evaluated
\item Music mood classification
\item Weekly discovery success
\item 40\% engagement increase
\end{itemize}
\vspace{0.5em}
\begin{center}
30M+ songs classified
\end{center}
\end{column}
\end{columns}

\vspace{1em}
\begin{center}
\begin{tikzpicture}
\node[draw,fill=mlblue!20,rounded corners,text width=10cm,align=center] {
\textbf{Common Thread:} Transform subjective taste into objective, scalable decisions
};
\end{tikzpicture}
\end{center}
\end{frame}

% Slide 9: Objectives - What students will learn
\begin{frame}{Your Learning Journey Today}
\Large\textbf{From Novice to Practitioner}
\normalsize

\vspace{0.5em}

\begin{columns}[T]
\begin{column}{0.48\textwidth}
\textbf{You Will Master:}
\begin{enumerate}
\item \textcolor{mlblue}{Understand} why classification matters
\item \textcolor{mlgreen}{Build} real classifiers
\item \textcolor{mlorange}{Evaluate} model performance
\item \textcolor{mlred}{Deploy} to production
\item \textcolor{mlpurple}{Design} user interfaces
\end{enumerate}

\vspace{0.5em}
\textbf{Algorithms You'll Use:}
\begin{itemize}
\item Logistic Regression
\item Decision Trees \& Random Forests
\item Support Vector Machines
\item Neural Networks
\end{itemize}
\end{column}
\begin{column}{0.48\textwidth}
\textbf{Skills You'll Gain:}
\begin{itemize}
\item Feature engineering
\item Model selection
\item Hyperparameter tuning
\item Cross-validation
\item Production deployment
\end{itemize}

\vspace{0.5em}
\textbf{Your Deliverable:}
\begin{tcolorbox}[colback=mlgreen!10,colframe=mlgreen]
Complete innovation success predictor ready for real-world use
\end{tcolorbox}
\end{column}
\end{columns}
\end{frame}

% Slide 10: Transition - Bridge to technical section
\begin{frame}{Ready to Learn How?}
\Large\textbf{From Problem to Solution}
\normalsize

\vspace{1em}

\begin{center}
We've seen the problem and opportunity.\\
\vspace{0.5em}
We have the data and motivation.\\
\vspace{1em}

\Huge\textbf{Now let's learn the algorithms!}

\vspace{1em}
\begin{tikzpicture}
\draw[ultra thick,->,mlblue] (0,0) -- (4,0);
\node[above] at (2,0.2) {\Large Next: Part 2 - How Classification Works};
\end{tikzpicture}

\vspace{1em}
\Large\textit{``Understanding the machine behind the magic''}
\end{center}
\end{frame}