% ACT 1: THE CHAOS OF UNSTRUCTURED PROTOTYPING (5 slides)
% Theme: Unstructured chaos = expensive, slow, kills innovation

% Slide 1: You Have 24 Hours (Statement title, Key Insight, Forward Question)
\begin{frame}[t]{The Unstructured Chaos: 24 Hours to Prototype Without a Plan}
\textbf{The scenario that reveals the chaos:}

\vspace{0.3em}

\begin{columns}[T]
\column{0.48\textwidth}
\textcolor{mlpurple}{\textbf{The Chaotic Reality}}

\small
Tomorrow: Startup pitch competition\\
Your idea: \textbf{EcoTrack} - Carbon footprint app\\
What you need: Working prototype

\vspace{0.3cm}
\textbf{Required (unstructured):}
\begin{itemize}
\item Logo (no brand guidelines)
\item UI mockups (no design system)
\item Copy (no voice guidelines)
\item Code (no architecture)
\item Video script (no structure)
\end{itemize}

\textcolor{mlred}{\textbf{What You Have:}}
\begin{itemize}
\item Yourself (no team)
\item A laptop (no tools)
\item Chaotic ideas (no structure)
\end{itemize}

\column{0.48\textwidth}
\textcolor{mlorange}{\textbf{The Chaos Tax}}

\small
\textbf{Unstructured Approach 1:}\\
Hire designer: \$5,000-15,000\\
Timeline: 1-2 weeks\\
\textcolor{mlgray}{(Expensive chaos)}

\vspace{0.3cm}
\textbf{Unstructured Approach 2:}\\
DIY with tools: Free\\
Learning: 40+ hours trial-error\\
\textcolor{mlgray}{(Time-consuming chaos)}

\vspace{0.3cm}
\textbf{Unstructured Approach 3:}\\
Use templates: \$50-200\\
Quality: Generic (no structure)\\
\textcolor{mlgray}{(Low-quality chaos)}

\vspace{0.5cm}
\textcolor{mlred}{\textbf{Chaos Problem:}}\\
No structure = No speed\\
No constraints = No consistency\\
No framework = No reliability
\end{columns}

\vspace{0.5em}
\begin{tcolorbox}[colback=mlblue!10, colframe=mlblue]
\textbf{Key Insight:} Unstructured prototyping is chaos - expensive, slow, kills most innovation before testing
\end{tcolorbox}

\vspace{0.5em}
\textbf{Key Question:} Can we escape this chaotic bottleneck with structure?

\bottomnote{Prototyping bottlenecks recur systematically - unstructured approaches create predictable resource drain across innovation cycles}
\end{frame}

% Slide 2: Sequential Workflow (Layout 10: Comparison, Statement title)
\begin{frame}[t]{The Sequential Chaos: Why Unstructured Methods Take Weeks}
\textbf{Unstructured workflow = sequential chaos = expensive}

\vspace{0.3em}

\begin{columns}[T]
\column{0.48\textwidth}
\textcolor{mlred}{\textbf{Traditional Unstructured Prototyping}}

\small
\textbf{Advantages (if structured):}
\begin{itemize}
\item Professional quality possible
\item Complete control achievable
\item Custom solutions available
\end{itemize}

\textbf{Disadvantages (when unstructured):}
\begin{itemize}
\item Takes 2-4 weeks (sequential)
\item Costs \$10,000-50,000 (specialists)
\item Limited iterations (3-5 max)
\item No parallelization (dependencies)
\item Requires multiple experts
\end{itemize}

\vspace{0.3cm}
\textcolor{mlred}{\textbf{Best for:}} Well-funded projects\\with time

\textcolor{mlgray}{\small But: 95\% of ideas don't\\have funding or time!}

\column{0.48\textwidth}
\textcolor{mlgreen}{\textbf{The Dream: Structured Instant Prototyping}}

\small
\textbf{Advantages (if it existed):}
\begin{itemize}
\item Minutes not weeks (parallel)
\item Dollars not thousands (automated)
\item Unlimited iterations (free)
\item No dependencies (structured)
\item No expert requirement
\end{itemize}

\textbf{Disadvantages:}
\begin{itemize}
\item Doesn't exist yet with\\unstructured approaches...
\item Needs structure to work
\end{itemize}

\vspace{0.3cm}
\textcolor{mlgreen}{\textbf{Best for:}} Everyone\\(if structure could be added!)

\vspace{0.3cm}
\textcolor{mlblue}{\small The gap between reality\\and dream: STRUCTURE}
\end{columns}

\vspace{0.5em}
\begin{tcolorbox}[colback=mlblue!10, colframe=mlblue]
\textbf{Key Insight:} Sequential unstructured workflow = no parallelization = expensive chaos
\end{tcolorbox}

\vspace{0.5em}
\textbf{Key Question:} What if we added structure to enable parallel creation?

\bottomnote{Comparative frameworks reveal trade-offs through juxtaposition - side-by-side presentation clarifies strengths and limitations simultaneously}
\end{frame}

% Slide 3: What Is a Prototype (Layout 9: Definition-Example, Statement title)
\begin{frame}[t]{A Prototype Is Your Testable Idea, Not Perfection}
\textbf{Building the "prototype" concept from scratch:}

\vspace{0.3em}

\begin{columns}[T]
\column{0.48\textwidth}
\textcolor{mlblue}{\textbf{Definition}}

\small
\textbf{Human Analogy: Cooking}

You want to cook new dish:\\
DON'T start with: Full restaurant,\\
perfect plating, 5-course meal

DO start with: Taste test,\\
basic version, learn what works

\vspace{0.3cm}
\textcolor{mlpurple}{\textbf{Computer Equivalent:}}

Testable version of idea that:
\begin{itemize}
\item Demonstrates core concept
\item Gets real feedback
\item Costs little to make
\item Easy to change
\end{itemize}

\vspace{0.3cm}
\textbf{Structure level matters:}\\
More structure = faster\\
Less structure = slower

\column{0.48\textwidth}
\textcolor{mlorange}{\textbf{Examples: Fidelity Spectrum}}

\small
\textbf{Example 1: Paper Sketch}\\
Time: 1 hour (unstructured)\\
Fidelity: 10\%, Learning: Basic\\
Structure: None

\textbf{Example 2: Clickable Mockup}\\
Time: 1 day (semi-structured)\\
Fidelity: 40\%, Learning: Interaction\\
Structure: Some constraints

\textbf{Example 3: Working MVP}\\
Time: 1-2 weeks (more structured)\\
Fidelity: 70\%, Learning: Real usage\\
Structure: Architecture defined

\textbf{Example 4: Production App}\\
Time: 3-6 months (fully structured)\\
Fidelity: 100\%, Learning: Market\\
Structure: Complete framework

\vspace{0.3cm}
\textcolor{mlgreen}{\textbf{Pattern:}} More structure\\= Less time per fidelity level
\end{columns}

\vspace{0.5em}
\begin{tcolorbox}[colback=mlblue!10, colframe=mlblue]
\textbf{Key Insight:} Prototype quality matters less than learning speed - structure enables speed
\end{tcolorbox}

\vspace{0.5em}
\textbf{Key Question:} How do we add structure to make prototypes instantly testable?

\bottomnote{Analogical reasoning bridges abstraction gaps - familiar examples scaffold understanding of novel technical concepts}
\end{frame}

% Slide 4: Creation Bottleneck (Statement title with formula)
\begin{frame}[t]{Skills $\times$ Time $\times$ Iterations = Exponential Chaos Cost}
\textbf{The unstructured chaos equation:}

\vspace{0.3em}

\begin{columns}[T]
\column{0.48\textwidth}
\textcolor{mlpurple}{\textbf{Three Sources of Chaos}}

\small
\textbf{1. The Skill Chaos}
\begin{itemize}
\item Design requires aesthetics
\item Code requires programming
\item Copy requires writing
\item Video requires editing
\end{itemize}
\textcolor{mlred}{Chaos:} Need 4+ unstructured skills

\vspace{0.3cm}
\textbf{2. The Time Chaos}
\begin{itemize}
\item Design: 20-40 hours
\item Development: 40-80 hours
\item Content: 10-20 hours
\item Testing: 10-20 hours
\end{itemize}
\textcolor{mlred}{Chaos:} 80-160 hours\\without structure

\vspace{0.3cm}
\textbf{3. The Iteration Chaos}
\begin{itemize}
\item Each change = hours rework
\item Coordination overhead
\item Changes compound
\end{itemize}
\textcolor{mlred}{Chaos:} Change is expensive

\column{0.48\textwidth}
\textcolor{mlblue}{\textbf{The Exponential Chaos Problem}}

\small
\textbf{Step-by-step calculation:}

Cost = Skills $\times$ Time $\times$ Iterations

\vspace{0.3cm}
For EcoTrack app prototype:

\textbf{Step 1: Count chaos sources}
\begin{itemize}
\item Unstructured skills: 4
\item Hours per skill: 20 average
\item Chaos iterations: 3 typical
\end{itemize}

\textbf{Step 2: Multiply chaos factors}

Total effort:\\
$4 \times 20 \times 3 = 240$ hours

\textbf{Step 3: Convert to cost}

At \$100/hour specialist rate:\\
$240 \times 100 = \$24,000$

\vspace{0.3cm}
\textcolor{mlorange}{\textbf{Chaos Growth:}}\\
Linear in each factor,\\
but \textcolor{mlred}{\textbf{multiplicative}} together!

Double skills: $2\times$ cost\\
Double time: $2\times$ cost\\
Double iterations: $2\times$ cost\\
\textcolor{mlred}{\textbf{All three: $8\times$ cost!}}
\end{columns}

\vspace{0.5em}
\begin{tcolorbox}[colback=mlblue!10, colframe=mlblue]
\textbf{Key Insight:} Unstructured chaos cost grows exponentially, not linearly - multiplicative disaster
\end{tcolorbox}

\vspace{0.5em}
\textbf{Key Question:} Can structure reduce this from exponential to linear growth?

\bottomnote{Multiplicative cost structures create exponential barriers - independent factors combine to amplify resource requirements beyond linear scaling}
\end{frame}

% Slide 5: Information Theory Quantification (Shannon entropy, complete calculation)
\begin{frame}[t]{The Harsh Math: 97\% of Ideas Die in Unstructured Chaos}
\textbf{Information theory reveals the true chaos cost:}

\vspace{0.3em}

\begin{columns}[T]
\column{0.55\textwidth}
\textcolor{mlred}{\textbf{Shannon Information Theory Analysis}}

\small
\textbf{Step 1: Calculate idea generation rate}

Average entrepreneur generates:\\
100 ideas per year

Information content (Shannon):\\
$H = \log_2(100) = 6.64$ bits/year

\vspace{0.3cm}
\textbf{Step 2: Calculate prototyping bandwidth}

Unstructured chaos allows:\\
3 prototypes per year (bottleneck)

Bandwidth capacity:\\
$B = \log_2(3) = 1.58$ bits/year

\vspace{0.3cm}
\textbf{Step 3: Calculate information loss}

Information bottleneck:\\
$Loss = H - B$\\
$= 6.64 - 1.58$\\
$= 5.06$ bits LOST to chaos

\vspace{0.3cm}
\textbf{Step 4: Calculate opportunity cost}

Ideas lost to chaos:\\
$2^{Loss} = 2^{5.06} \approx 33$ potential successes

\column{0.43\textwidth}
\textcolor{mlblue}{\textbf{The Chaos Opportunity Cost}}

\small
\textbf{If chaos removed:}\\
100 ideas $\times$ 33\% success = 33 wins

\textbf{Current (with chaos):}\\
3 ideas $\times$ 33\% success = 1 win

\textbf{Opportunity cost:}\\
\textcolor{mlred}{\textbf{32 successful products}}\\
\textcolor{mlred}{\textbf{never built!}}

\vspace{0.5cm}
\begin{center}
\begin{tabular}{lcc}
\toprule
\textbf{Stage} & \textbf{Count} & \textbf{Loss} \\
\midrule
Ideas generated & 100 & - \\
\textcolor{mlred}{Chaos bottleneck} & \textcolor{mlred}{3} & \textcolor{mlred}{-97\%} \\
Actually succeed & 1 & -66\% \\
\midrule
\textbf{Lost to chaos} & \textbf{32} & \textbf{-97\%} \\
\bottomrule
\end{tabular}
\end{center}

\vspace{0.3cm}
\textcolor{mlpurple}{\textbf{Structure could recover:}}\\
If structure enables 100 tests:\\
100 $\times$ 0.33 = 33 successes

\textcolor{mlgreen}{\textbf{Structure value:}}\\
$32\times$ more successful products!
\end{columns}

\vspace{0.5em}
\begin{tcolorbox}[colback=mlblue!10, colframe=mlblue]
\textbf{Key Insight:} 97\% information loss to chaos means 32 successes never discovered - structure recovers them
\end{tcolorbox}

\vspace{0.5em}
\textbf{Key Question:} Can we structure AI to eliminate this bottleneck?

\bottomnote{Information bottlenecks quantify opportunity cost - channel capacity constraints determine how many potential solutions remain unexplored}
\end{frame}