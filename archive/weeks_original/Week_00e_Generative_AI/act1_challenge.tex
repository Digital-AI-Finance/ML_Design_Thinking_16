% Slide 1: Want to CREATE, not just classify
\begin{frame}
\frametitle{The Creation Challenge}
\framesubtitle{Moving Beyond Classification}

\begin{columns}[T]
\column{0.48\textwidth}
\textbf{Traditional ML:} ``What is this?''

\small
\begin{itemize}
\item Email spam detector: Classify existing emails
\item Medical diagnosis: Analyze X-ray images
\item Sentiment analysis: Judge customer reviews
\end{itemize}

\textcolor{mlred}{\textbf{Limitation:}} Only analyzes, never creates

\column{0.48\textwidth}
\textbf{Generative AI:} ``Create something new''

\small
\begin{itemize}
\item Generate phishing emails for security training
\item Synthesize medical images for rare diseases
\item Write product descriptions automatically
\item Compose music for video backgrounds
\end{itemize}

\textcolor{mlgreen}{\textbf{Power:}} Creation enables innovation
\end{columns}

\vspace{\fill}
\footnotesize \textcolor{gray}{
Generative models learn full data distributions enabling sampling - classification learns boundaries, generation learns manifolds
}
\end{frame}

% Slide 2: Generative vs discriminative models distinction
\begin{frame}
\frametitle{Mathematical Foundation}
\framesubtitle{Two Approaches to Learning}

\small
\begin{columns}[T]
\column{0.48\textwidth}
\textbf{Discriminative Models}

Learn: $P(y|x)$ - Conditional probability

(Ng \& Jordan 2002: Defined distinction)

\textbf{What it does:}
\begin{itemize}
\item Given $x$, predict label $y$
\item Learns decision boundaries
\item Divides input space
\end{itemize}

\textbf{Examples:} Logistic, RF, SVM

\textbf{Can sample new $x$?} NO - only classifies existing data

\column{0.48\textwidth}
\textbf{Generative Models}

Learn: $P(x)$ - Joint or marginal distribution

\textbf{What it does:}
\begin{itemize}
\item Models entire data distribution
\item Sample via ancestral sampling or MCMC
\item Samples new $x \sim P(x)$
\item Creates novel instances
\end{itemize}

\textbf{Examples:} VAEs, GANs, Diffusion

\textbf{Can sample new $x$?} YES - generates from distribution
\end{columns}

\vspace{\fill}
\footnotesize \textcolor{gray}{
Discriminative models $P(y|x)$ learn boundaries while generative models $P(x)$ or $P(x,y)$ learn distributions - fundamental distinction enables creation
}
\end{frame}

% Slide 3: Capturing full data distribution (hard problem)
\begin{frame}
\frametitle{The Hard Problem}
\framesubtitle{Why Generation is Fundamentally Difficult}

\begin{center}
\includegraphics[width=0.55\textwidth]{charts/distribution_complexity.pdf}
\end{center}

\footnotesize
\begin{columns}[T]
\column{0.48\textwidth}
\textbf{Challenges:}
\begin{itemize}
\item High-dimensional spaces
\item Multimodal distributions
\item Curse of dimensionality (data lies on low-dimensional manifolds)
\item Sample complexity grows exponentially
\end{itemize}

\column{0.48\textwidth}
\textbf{Requirements:}
\begin{itemize}
\item Capture all patterns
\item Maintain realism
\item Computational tractability (exact inference intractable)
\end{itemize}
\end{columns}

\vspace{\fill}
\footnotesize \textcolor{gray}{
Manifold hypothesis: real data concentrates on low-dimensional manifolds in high-dimensional space - learning distribution requires exponential samples without structure
}
\end{frame}

% Slide 4: Realistic vs diverse tradeoff
\begin{frame}
\frametitle{The Fundamental Tradeoff}
\framesubtitle{Quality vs Diversity Dilemma}

\begin{center}
\includegraphics[width=0.45\textwidth]{charts/quality_diversity_tradeoff.pdf}
\end{center}

\footnotesize
\textcolor{mlred}{\textbf{High Quality:}} Mode collapse, repetitive

\textcolor{mlorange}{\textbf{Balanced:}} Realistic variety

\textcolor{mlblue}{\textbf{High Diversity:}} Unrealistic

\textbf{Explanations:}
\begin{itemize}
\item Mode Collapse: Generator produces limited variety (high quality, low diversity)
\item Coverage Issue: Generator spreads thin (high diversity, low quality)
\end{itemize}

\vspace{\fill}
\footnotesize \textcolor{gray}{
Quality-diversity trade-off is fundamental - mode collapse sacrifices coverage for fidelity, while mode coverage sacrifices fidelity for variety
}
\end{frame}

% Slide 5: Quantify: Inception Score, FID, perplexity
\begin{frame}
\frametitle{Measuring Generation Quality}
\framesubtitle{Metrics for Evaluating Generative Models}

\begin{center}
\includegraphics[width=0.65\textwidth]{charts/generation_metrics.pdf}
\end{center}

\footnotesize
\begin{columns}[T]
\column{0.32\textwidth}
\textbf{Inception Score (IS)}

(Salimans et al. 2016)

\begin{itemize}
\item Range: 1-1000
\item Higher = better
\item Quality \& diversity
\end{itemize}

\textbf{Interpretation:}
\begin{itemize}
\item $>$300: Excellent
\item 100-300: Good
\item $<$100: Poor
\end{itemize}

\column{0.32\textwidth}
\textbf{FID Score}

(Heusel et al. 2017: Fréchet Inception Distance)

\begin{itemize}
\item Range: 0-500
\item Lower = better
\item Feature distance
\end{itemize}

\textbf{Interpretation:}
\begin{itemize}
\item $<$10: Photorealistic
\item 10-50: Good quality
\item $>$50: Noticeable artifacts
\end{itemize}

\column{0.32\textwidth}
\textbf{Perplexity (Text)}
\begin{itemize}
\item Range: 1-10,000
\item Lower = better
\item Predictability
\end{itemize}

\textbf{Interpretation:}
\begin{itemize}
\item $<$20: Human-like
\item 20-100: Coherent
\item $>$100: Gibberish
\end{itemize}
\end{columns}

\textbf{Note:} Human evaluation remains gold standard

\vspace{\fill}
\footnotesize \textcolor{gray}{
Quantitative metrics approximate perceptual quality - IS measures label confidence and diversity, FID measures feature distribution distance, perplexity measures predictability
}
\end{frame}