\documentclass[8pt,aspectratio=169]{beamer}
\usetheme{Madrid}
\usecolortheme{default}
\usepackage{graphicx}
\usepackage{booktabs}
\usepackage{adjustbox}
\usepackage{multicol}
\usepackage{amsmath}
\usepackage{amssymb}
\usepackage{tcolorbox}
\usepackage{xcolor}
\usepackage{tikz}
\usetikzlibrary{shapes.geometric, arrows, positioning}
\usepackage{listings}

% Define colors matching the ML Design course theme
\definecolor{mlblue}{RGB}{31, 119, 180}
\definecolor{mlorange}{RGB}{255, 127, 14}
\definecolor{mlgreen}{RGB}{44, 160, 44}
\definecolor{mlred}{RGB}{214, 39, 40}
\definecolor{mlpurple}{RGB}{148, 103, 189}
\definecolor{mlbrown}{RGB}{140, 86, 75}
\definecolor{mlpink}{RGB}{227, 119, 194}
\definecolor{mlgray}{RGB}{127, 127, 127}
\definecolor{mlyellow}{RGB}{255, 187, 120}
\definecolor{mlcyan}{RGB}{23, 190, 207}

% Beamer template customization
\setbeamertemplate{navigation symbols}{}
\setbeamertemplate{footline}{
  \leavevmode%
  \hbox{%
  \begin{beamercolorbox}[wd=.333333\paperwidth,ht=2.25ex,dp=1ex,center]{author in head/foot}%
    \usebeamerfont{author in head/foot}Part \insertpartnumber/4
  \end{beamercolorbox}%
  \begin{beamercolorbox}[wd=.333333\paperwidth,ht=2.25ex,dp=1ex,center]{title in head/foot}%
    \usebeamerfont{title in head/foot}Week 2: Deep Empathy
  \end{beamercolorbox}%
  \begin{beamercolorbox}[wd=.333333\paperwidth,ht=2.25ex,dp=1ex,right]{date in head/foot}%
    \usebeamerfont{date in head/foot}Slide \insertframenumber{} of \inserttotalframenumber\hspace*{2ex} 
  \end{beamercolorbox}}%
  \vskip0pt%
}

% Code listing settings
\lstset{
    language=Python,
    basicstyle=\small\ttfamily,
    numbers=left,
    numberstyle=\tiny\color{gray},
    stepnumber=1,
    numbersep=5pt,
    backgroundcolor=\color{gray!5},
    showspaces=false,
    showstringspaces=false,
    showtabs=false,
    frame=single,
    rulecolor=\color{gray!30},
    tabsize=2,
    captionpos=b,
    breaklines=true,
    breakatwhitespace=false,
    keywordstyle=\color{mlblue},
    commentstyle=\color{mlgreen},
    stringstyle=\color{mlorange}
}

% Enhanced title information
\title{\Large\textbf{Machine Learning for Smarter Innovation}\\
\vspace{0.5em}
\Large Week 2: Clustering for Deep Empathy}
\subtitle{Understanding Users Through Data-Driven Segmentation\\
\small Discovering Hidden Patterns in Human Behavior}
\author{BSc Course in AI-Enhanced Innovation}
\institute{Department of Computer Science \& Design}
\date{2025}

\begin{document}

% Enhanced title slide
\begin{frame}[plain]
\titlepage
\vspace{-1em}
\begin{center}
\small\textit{This week: Transform behavioral data into deep user understanding through advanced clustering}
\end{center}
\end{frame}

% NEW: Table of Contents with Journey
\begin{frame}
\frametitle{Today's Journey: From Data Points to Human Insights}
\framesubtitle{Your Roadmap to Understanding Users Through Clustering}

\begin{columns}[T]
\begin{column}{0.48\textwidth}
\begin{tcolorbox}[colback=mlblue!10, colframe=mlblue!50, title=Morning Session]
\small
\textbf{Part 1: Foundation (25 min)}
\begin{itemize}
\item Why clustering for empathy?
\item Traditional vs data-driven personas
\item Setting objectives
\end{itemize}

\vspace{0.3em}
\textbf{Part 2: Technical Deep Dive (35 min)}
\begin{itemize}
\item Advanced K-means techniques
\item Finding optimal clusters
\item DBSCAN, Hierarchical, GMM
\item Algorithm selection guide
\end{itemize}
\end{tcolorbox}
\end{column}

\begin{column}{0.48\textwidth}
\begin{tcolorbox}[colback=mlgreen!10, colframe=mlgreen!50, title=Afternoon Session]
\small
\textbf{Part 3: Design Integration (30 min)}
\begin{itemize}
\item From clusters to personas
\item Building empathy maps
\item Journey mapping per segment
\item Innovation opportunities
\end{itemize}

\vspace{0.3em}
\textbf{Part 4: Practice (20 min)}
\begin{itemize}
\item Spotify case study
\item 5 music personas discovered
\item Implementation pipeline
\item Your turn: exercise
\end{itemize}
\end{tcolorbox}
\end{column}
\end{columns}

\vspace{0.5em}
\begin{center}
\begin{tcolorbox}[colback=mlpurple!20, colframe=mlpurple!60, width=0.9\textwidth]
\centering
\textbf{Goal:} Master data-driven empathy to understand users you've never met
\end{tcolorbox}
\end{center}
\end{frame}

% NEW: Prerequisites & Setup
\begin{frame}
\frametitle{Prerequisites \& What You'll Build}
\framesubtitle{Setting You Up for Success in User Understanding}

\begin{columns}[T]
\begin{column}{0.48\textwidth}
\begin{tcolorbox}[colback=mlgreen!10, colframe=mlgreen!50, title=Building on Week 1]
\normalsize
\textbf{You already know:}
\begin{itemize}
\item Basic K-means clustering
\item Elbow method for K
\item Silhouette scores
\item Distance metrics
\end{itemize}

\vspace{0.3em}
\textbf{New this week:}
\begin{itemize}
\item User behavior features
\item Persona creation methods
\item Empathy map construction
\item Journey differentiation
\end{itemize}
\end{tcolorbox}
\end{column}

\begin{column}{0.48\textwidth}
\begin{tcolorbox}[colback=mlorange!10, colframe=mlorange!50, title=What You'll Create]
\normalsize
\textbf{By end of today:}
\begin{itemize}
\item Data-driven user personas
\item Behavioral segment profiles
\item Empathy maps per segment
\item Differentiated journey maps
\item Innovation opportunity matrix
\end{itemize}

\vspace{0.3em}
\textbf{Tools we'll use:}
\begin{itemize}
\item Python + scikit-learn
\item Pandas for data processing
\item Matplotlib/Seaborn for viz
\item Jupyter notebooks
\end{itemize}
\end{tcolorbox}
\end{column}
\end{columns}

\vspace{0.5em}
\begin{center}
\begin{tcolorbox}[colback=mlyellow!20, colframe=mlorange!60, width=0.9\textwidth]
\centering
\textbf{Remember:} Every data point represents a real person with real needs
\end{tcolorbox}
\end{center}
\end{frame}

% PART 1: ENHANCED FOUNDATION
\begin{frame}[plain]
\begin{center}
\vspace{2em}
{\Huge\textcolor{mlblue}{\textbf{PART 1}}}\\
\vspace{0.5em}
{\Large\textbf{Foundation: Why Clustering for Empathy?}}\\
\vspace{1em}
\textit{Moving beyond assumptions to data-driven understanding}\\
\vspace{2em}
\Large
\textbf{Key Questions We'll Answer:}\\
\vspace{0.5em}
\normalsize
\begin{itemize}
\item How do we truly understand millions of users?
\item What patterns exist in user behavior?
\item How can data reveal emotional needs?
\item Where do traditional personas fail?
\end{itemize}
\vspace{1em}
\Large\textcolor{mlpurple}{\textbf{Let's discover hidden user segments}}
\end{center}
\end{frame}

% NEW: Learning Objectives for Part 1
\begin{frame}
\frametitle{Part 1: Learning Objectives}
\framesubtitle{Foundation Skills You'll Develop}

\begin{columns}[T]
\begin{column}{0.48\textwidth}
\begin{tcolorbox}[colback=mlblue!10, colframe=mlblue!50, title={By the end of Part 1, you will:}]
\normalsize
\begin{itemize}
\item \textbf{Understand} why clustering enables deep empathy
\item \textbf{Recognize} limitations of assumption-based personas
\item \textbf{Identify} behavioral patterns in user data
\item \textbf{Explain} the value of data-driven segmentation
\item \textbf{Connect} clustering to design thinking
\end{itemize}
\end{tcolorbox}
\end{column}

\begin{column}{0.48\textwidth}
\begin{tcolorbox}[colback=mlgreen!10, colframe=mlgreen!50, title=Success Criteria]
\normalsize
\begin{itemize}
\item Can articulate clustering's role in empathy
\item Understand behavioral vs demographic segmentation
\item Know when to use data-driven personas
\item Can identify suitable user features
\item Ready for technical deep dive
\end{itemize}
\end{tcolorbox}
\end{column}
\end{columns}

\vspace{0.5em}
\begin{center}
\textcolor{mlpurple}{\textbf{Foundation first, then we'll dive into the algorithms}}
\end{center}
\end{frame}

% Opening Power Chart
\begin{frame}
\frametitle{From Chaos to Clarity: The Power of User Clustering}
\framesubtitle{Watch 10,000 Users Self-Organize into Natural Groups}

\begin{center}
\includegraphics[width=0.85\textwidth]{charts/clustering_evolution.pdf}
\end{center}

\vspace{0.3em}
\begin{center}
\begin{tcolorbox}[colback=mlpurple!20, colframe=mlpurple!60, width=0.9\textwidth]
\centering
\Large\textcolor{mlpurple}{\textbf{Each dot = a real user, each cluster = shared needs}}
\end{tcolorbox}
\end{center}
\end{frame}

% Enhanced Problem Statement
\begin{frame}
\frametitle{The User Understanding Challenge: A Deep Dive}
\framesubtitle{Why Traditional Methods Fall Short at Scale}

\begin{columns}[T]
\begin{column}{0.48\textwidth}
\begin{tcolorbox}[colback=mlred!10, colframe=mlred!50, title=Traditional Challenges]
\normalsize
\textbf{Assumption-Based:}
\begin{itemize}
\item "Millennials want X"
\item "Power users need Y"
\item Based on 5-10 interviews
\end{itemize}

\textbf{Problems:}
\begin{itemize}
\item Generic personas
\item Missing hidden segments
\item Biased by loud voices
\item Static profiles
\item Limited samples
\end{itemize}

\textbf{Result:}\\
\textcolor{mlred}{70\% of features unused}
\end{tcolorbox}
\end{column}

\begin{column}{0.48\textwidth}
\begin{tcolorbox}[colback=mlgreen!10, colframe=mlgreen!50, title=ML-Enhanced Solutions]
\normalsize
\textbf{Data-Driven:}
\begin{itemize}
\item Behavioral patterns
\item Usage analytics
\item 1000s of data points
\end{itemize}

\textbf{Benefits:}
\begin{itemize}
\item Natural segments
\item Unexpected patterns
\item Balanced representation
\item Dynamic insights
\item Scale to millions
\end{itemize}

\textbf{Result:}\\
\textcolor{mlgreen}{2x feature adoption}
\end{tcolorbox}
\end{column}
\end{columns}

\vspace{0.5em}
\begin{center}
\begin{tcolorbox}[colback=mlyellow!20, colframe=mlorange!60, width=0.9\textwidth]
\centering
\textbf{Think about it:} How many user types are you missing with traditional personas?
\end{tcolorbox}
\end{center}
\end{frame}

% Why Clustering for Empathy
\begin{frame}
\frametitle{Why Clustering Creates Deeper Empathy}
\framesubtitle{Understanding the "Why" Behind User Behavior}

\begin{columns}[T]
\begin{column}{0.55\textwidth}
\begin{center}
\includegraphics[width=\textwidth]{charts/user_segmentation_main.pdf}
\end{center}
\end{column}

\begin{column}{0.43\textwidth}
\begin{tcolorbox}[colback=mlpurple!10, colframe=mlpurple!50, title=Clustering Reveals]
\small
\textbf{1. Natural Groupings}
\begin{itemize}
\item Users cluster by behavior
\item Not by demographics
\end{itemize}

\textbf{2. Hidden Patterns}
\begin{itemize}
\item Unexpected user types
\item Cross-demographic needs
\end{itemize}

\textbf{3. Emotional Context}
\begin{itemize}
\item Usage = emotional state
\item Patterns = needs
\end{itemize}

\textbf{4. Evolution}
\begin{itemize}
\item Users change over time
\item Clusters evolve
\end{itemize}
\end{tcolorbox}
\end{column}
\end{columns}

\begin{center}
\begin{tcolorbox}[colback=mlgreen!20, colframe=mlgreen!60, width=0.9\textwidth]
\centering
\textbf{Key Insight:} Behavior reveals needs better than demographics ever could
\end{tcolorbox}
\end{center}
\end{frame}

% Evolution Comparison
\begin{frame}
\frametitle{Evolution: From Assumptions to Data-Driven Insights}
\framesubtitle{The Journey of User Understanding Methods}

\begin{center}
\includegraphics[width=0.85\textwidth]{charts/empathy_evolution.pdf}
\end{center}

\begin{columns}[T]
\begin{column}{0.24\textwidth}
\centering
\textcolor{mlgray}{\textbf{1990s}}\\
\small Demographics\\
Age, Gender, Income\\
\textcolor{mlred}{0 behavioral data}
\end{column}

\begin{column}{0.24\textwidth}
\centering
\textcolor{mlgray}{\textbf{2000s}}\\
\small Personas\\
5-10 interviews\\
\textcolor{mlorange}{Limited scale}
\end{column}

\begin{column}{0.24\textwidth}
\centering
\textcolor{mlblue}{\textbf{2010s}}\\
\small Analytics\\
Usage tracking\\
\textcolor{mlblue}{Some patterns}
\end{column}

\begin{column}{0.24\textwidth}
\centering
\textcolor{mlgreen}{\textbf{2020s}}\\
\small ML Clustering\\
Behavioral segmentation\\
\textcolor{mlgreen}{Deep understanding}
\end{column}
\end{columns}

\vspace{0.5em}
\begin{center}
\textcolor{mlpurple}{\textbf{We're here: Using ML to understand users at scale with empathy}}
\end{center}
\end{frame}

% Checkpoint 1
\begin{frame}
\frametitle{Checkpoint: Foundation Understanding}
\framesubtitle{Quick Knowledge Check \hfill \textcolor{mlblue}{\textbf{Progress: Part 1/4}}}

\begin{columns}[T]
\begin{column}{0.48\textwidth}
\begin{tcolorbox}[colback=mlblue!10, colframe=mlblue!50, title=True or False?]
\normalsize
\begin{enumerate}
\item Clustering finds natural user groups \textcolor{mlgreen}{(T)}
\item Demographics > behavior for segmentation \textcolor{mlred}{(F)}
\item Traditional personas scale well \textcolor{mlred}{(F)}
\item Clustering reveals hidden patterns \textcolor{mlgreen}{(T)}
\item Users stay in same segment forever \textcolor{mlred}{(F)}
\end{enumerate}
\end{tcolorbox}
\end{column}

\begin{column}{0.48\textwidth}
\begin{tcolorbox}[colback=mlgreen!10, colframe=mlgreen!50, title=Can You Explain?]
\normalsize
\begin{itemize}
\item Why does behavior matter more than demographics?
\item How does clustering create empathy?
\item What patterns might we discover?
\item When should we re-cluster users?
\end{itemize}
\end{tcolorbox}
\end{column}
\end{columns}

\vspace{0.5em}
\begin{center}
\Large\textcolor{mlpurple}{\textbf{Ready for the technical deep dive?}}\\
\normalsize\textit{Next: Advanced clustering algorithms for user understanding}
\end{center}
\end{frame}

% PART 2: ENHANCED TECHNICAL
\begin{frame}[plain]
\begin{center}
\vspace{2em}
{\Huge\textcolor{mlgreen}{\textbf{PART 2}}}\\
\vspace{0.5em}
{\Large\textbf{Technical Deep Dive}}\\
\vspace{1em}
\textit{Advanced clustering for nuanced user understanding}\\
\vspace{2em}
\Large
\textbf{What You'll Master:}\\
\vspace{0.5em}
\normalsize
\begin{itemize}
\item Advanced K-means techniques
\item Optimal cluster validation
\item DBSCAN for outlier users
\item Hierarchical for user taxonomies
\item GMM for overlapping segments
\end{itemize}
\vspace{1em}
\Large\textcolor{mlpurple}{\textbf{From algorithms to insights}}
\end{center}
\end{frame}

% Learning Objectives Part 2
\begin{frame}
\frametitle{Part 2: Technical Learning Objectives}
\framesubtitle{Algorithm Mastery for User Understanding}

\begin{columns}[T]
\begin{column}{0.48\textwidth}
\begin{tcolorbox}[colback=mlgreen!10, colframe=mlgreen!50, title={Technical Skills}]
\normalsize
\begin{itemize}
\item \textbf{Implement} K-means variations
\item \textbf{Validate} cluster quality
\item \textbf{Choose} optimal K systematically
\item \textbf{Apply} DBSCAN for outliers
\item \textbf{Build} hierarchical taxonomies
\item \textbf{Use} GMM for soft clustering
\end{itemize}
\end{tcolorbox}
\end{column}

\begin{column}{0.48\textwidth}
\begin{tcolorbox}[colback=mlorange!10, colframe=mlorange!50, title={Application Skills}]
\normalsize
\begin{itemize}
\item Select right algorithm for user data
\item Handle outlier users properly
\item Validate segment stability
\item Interpret cluster characteristics
\item Scale to millions of users
\item Update clusters dynamically
\end{itemize}
\end{tcolorbox}
\end{column}
\end{columns}

\vspace{0.5em}
\begin{center}
\textcolor{mlpurple}{\textbf{Each algorithm reveals different aspects of user behavior}}
\end{center}
\end{frame}

% Enhanced K-means for Users
\begin{frame}
\frametitle{K-Means for User Segmentation: Advanced Techniques}
\framesubtitle{Beyond Basic Clustering - Understanding User Nuances}

\begin{columns}[T]
\begin{column}{0.48\textwidth}
\begin{tcolorbox}[colback=mlblue!10, colframe=mlblue!50, title=K-Means++ Initialization]
\small
\textbf{Problem:} Random init = poor segments

\textbf{Solution:} Smart initialization
\begin{enumerate}
\item Choose first center randomly
\item Next center: farthest from existing
\item Repeat until K centers
\end{enumerate}

\textbf{Result:} 
\begin{itemize}
\item Better convergence
\item More stable segments
\item Reproducible personas
\end{itemize}
\end{tcolorbox}
\end{column}

\begin{column}{0.48\textwidth}
\begin{tcolorbox}[colback=mlgreen!10, colframe=mlgreen!50, title=Mini-Batch K-Means]
\small
\textbf{Problem:} Millions of users = slow

\textbf{Solution:} Batch processing
\begin{itemize}
\item Sample random batches
\item Update centroids incrementally
\item Converge faster
\end{itemize}

\textbf{Performance:}
\begin{itemize}
\item 3-10x faster
\item Less than 5\% quality loss
\item Scales to billions
\end{itemize}
\end{tcolorbox}
\end{column}
\end{columns}

\vspace{0.5em}
\begin{center}
\begin{tcolorbox}[colback=mlyellow!20, colframe=mlorange!60, width=0.9\textwidth]
\centering
\textbf{Pro Tip:} Use K-means++ for quality, Mini-batch for scale, combine for production!
\end{tcolorbox}
\end{center}
\end{frame}

% Finding Optimal Clusters
\begin{frame}
\frametitle{Finding the Right Number of User Segments}
\framesubtitle{Data-Driven Approach to Persona Count}

\begin{center}
\includegraphics[width=0.85\textwidth]{charts/elbow_silhouette_analysis.pdf}
\end{center}

\begin{columns}[T]
\begin{column}{0.32\textwidth}
\begin{tcolorbox}[colback=mlblue!10, colframe=mlblue!50, title=Elbow Method]
\small
\textbf{Look for:}
\begin{itemize}
\item Sharp bend
\item Diminishing returns
\item Here: K=5
\end{itemize}

\textbf{Means:}\\
5 distinct user types
\end{tcolorbox}
\end{column}

\begin{column}{0.32\textwidth}
\begin{tcolorbox}[colback=mlgreen!10, colframe=mlgreen!50, title=Silhouette]
\small
\textbf{Measures:}
\begin{itemize}
\item Cohesion within
\item Separation between
\item Max at K=5
\end{itemize}

\textbf{Score 0.73:}\\
Strong segments!
\end{tcolorbox}
\end{column}

\begin{column}{0.32\textwidth}
\begin{tcolorbox}[colback=mlorange!10, colframe=mlorange!50, title=Business Logic]
\small
\textbf{Consider:}
\begin{itemize}
\item Team capacity
\item Product variants
\item Market reality
\end{itemize}

\textbf{Balance:}\\
Statistics + practicality
\end{tcolorbox}
\end{column}
\end{columns}
\end{frame}

% DBSCAN for Outlier Users
\begin{frame}
\frametitle{DBSCAN: Finding Unique Users and Edge Cases}
\framesubtitle{Not Everyone Fits in a Box - And That's Valuable!}

\begin{columns}[T]
\begin{column}{0.55\textwidth}
\begin{center}
\includegraphics[width=\textwidth]{charts/dbscan_detailed.pdf}
\end{center}
\end{column}

\begin{column}{0.43\textwidth}
\begin{tcolorbox}[colback=mlpurple!10, colframe=mlpurple!50, title=DBSCAN Benefits]
\small
\textbf{Finds:}
\begin{itemize}
\item Core user groups
\item Edge users
\item True outliers
\item Irregular shapes
\end{itemize}

\textbf{Perfect for:}
\begin{itemize}
\item Power users
\item Early adopters
\item Special needs
\item Fraud detection
\end{itemize}

\textbf{No K needed!}\\
Algorithm finds natural groups
\end{tcolorbox}
\end{column}
\end{columns}

\begin{center}
\begin{tcolorbox}[colback=mlgreen!20, colframe=mlgreen!60, width=0.9\textwidth]
\centering
\textbf{Insight:} Your most valuable users might be outliers - DBSCAN finds them!
\end{tcolorbox}
\end{center}
\end{frame}

% Hierarchical for User Taxonomy
\begin{frame}
\frametitle{Hierarchical Clustering: Building User Taxonomies}
\framesubtitle{Understanding User Relationships at Multiple Levels}

\begin{columns}[T]
\begin{column}{0.55\textwidth}
\begin{center}
\includegraphics[width=\textwidth]{charts/hierarchical_detailed.pdf}
\end{center}
\end{column}

\begin{column}{0.43\textwidth}
\begin{tcolorbox}[colback=mlblue!10, colframe=mlblue!50, title=Multi-Level Understanding]
\small
\textbf{Level 1: Broad}
\begin{itemize}
\item Active vs Passive users
\end{itemize}

\textbf{Level 2: Categories}
\begin{itemize}
\item Creators, Consumers, Curators
\end{itemize}

\textbf{Level 3: Specific}
\begin{itemize}
\item 8-10 detailed personas
\end{itemize}

\textbf{Benefits:}
\begin{itemize}
\item Flexible granularity
\item Natural hierarchy
\item Evolution tracking
\end{itemize}
\end{tcolorbox}
\end{column}
\end{columns}

\begin{center}
\begin{tcolorbox}[colback=mlyellow!20, colframe=mlorange!60, width=0.9\textwidth]
\centering
\textbf{Use Case:} Perfect for creating user documentation at different detail levels
\end{tcolorbox}
\end{center}
\end{frame}

% GMM for Overlapping Segments
\begin{frame}
\frametitle{Gaussian Mixture Models: When Users Don't Fit One Box}
\framesubtitle{Soft Clustering for Complex User Behaviors}

\begin{columns}[T]
\begin{column}{0.55\textwidth}
\begin{center}
\includegraphics[width=\textwidth]{charts/gmm_detailed.pdf}
\end{center}
\end{column}

\begin{column}{0.43\textwidth}
\begin{tcolorbox}[colback=mlgreen!10, colframe=mlgreen!50, title=Soft Assignments]
\small
\textbf{User A:}
\begin{itemize}
\item 70\% Power User
\item 20\% Creator
\item 10\% Casual
\end{itemize}

\textbf{User B:}
\begin{itemize}
\item 60\% Consumer
\item 40\% Curator
\end{itemize}

\textbf{Benefits:}
\begin{itemize}
\item Realistic modeling
\item Probability scores
\item Transition detection
\item Nuanced personas
\end{itemize}
\end{tcolorbox}
\end{column}
\end{columns}

\begin{center}
\begin{tcolorbox}[colback=mlpurple!20, colframe=mlpurple!60, width=0.9\textwidth]
\centering
\textbf{Reality:} Most users are combinations - GMM captures this complexity!
\end{tcolorbox}
\end{center}
\end{frame}

% Algorithm Selection Guide
\begin{frame}
\frametitle{Choosing the Right Algorithm: Decision Guide}
\framesubtitle{Match Your User Data to the Right Method}

\begin{center}
\includegraphics[width=0.85\textwidth]{charts/clustering_selection_guide.pdf}
\end{center}

\begin{columns}[T]
\begin{column}{0.48\textwidth}
\begin{tcolorbox}[colback=mlblue!10, colframe=mlblue!50, title=Quick Decision Tree]
\small
\textbf{Start here:}
\begin{itemize}
\item Know K? → K-means
\item Need outliers? → DBSCAN
\item Need hierarchy? → Hierarchical
\item Users overlap? → GMM
\item Huge scale? → Mini-batch
\end{itemize}
\end{tcolorbox}
\end{column}

\begin{column}{0.48\textwidth}
\begin{tcolorbox}[colback=mlgreen!10, colframe=mlgreen!50, title=Pro Combination]
\small
\textbf{Best practice:}
\begin{enumerate}
\item DBSCAN for outliers
\item K-means for main segments
\item GMM for refinement
\item Hierarchical for taxonomy
\end{enumerate}
\textbf{Result:} Complete user understanding
\end{tcolorbox}
\end{column}
\end{columns}
\end{frame}

% Python Implementation
\begin{frame}[fragile]
\frametitle{Implementation: User Clustering Pipeline}
\framesubtitle{Production-Ready Code for User Segmentation}

\begin{columns}[T]
\begin{column}{0.48\textwidth}
\begin{tcolorbox}[colback=gray!5, colframe=mlblue!50, title={\small Data Preparation}]
\begin{lstlisting}[language=Python, basicstyle=\tiny\ttfamily]
import pandas as pd
from sklearn.preprocessing import StandardScaler
from sklearn.cluster import KMeans

# Load user behavior data
users = pd.read_csv('user_behavior.csv')

# Select features
features = ['sessions', 'duration',
           'actions', 'shares', 
           'likes', 'comments']

# Scale features
scaler = StandardScaler()
X = scaler.fit_transform(users[features])

# Find optimal K
from sklearn.metrics import silhouette_score

scores = []
for k in range(2, 10):
    kmeans = KMeans(n_clusters=k)
    labels = kmeans.fit_predict(X)
    score = silhouette_score(X, labels)
    scores.append(score)

optimal_k = scores.index(max(scores)) + 2
\end{lstlisting}
\end{tcolorbox}
\end{column}

\begin{column}{0.48\textwidth}
\begin{tcolorbox}[colback=gray!5, colframe=mlgreen!50, title={\small Clustering \& Analysis}]
\begin{lstlisting}[language=Python, basicstyle=\tiny\ttfamily]
# Cluster with optimal K
kmeans = KMeans(
    n_clusters=optimal_k,
    init='k-means++',
    n_init=10,
    random_state=42
)
users['segment'] = kmeans.fit_predict(X)

# Analyze segments
for i in range(optimal_k):
    segment = users[users['segment'] == i]
    print(f"\nSegment {i} ({len(segment)} users):")
    print(segment[features].mean())

# Create persona profiles
personas = users.groupby('segment').agg({
    'sessions': ['mean', 'std'],
    'duration': ['mean', 'std'],
    'actions': ['mean', 'std']
})

# Export for design team
personas.to_csv('user_personas.csv')
users.to_csv('users_segmented.csv')
\end{lstlisting}
\end{tcolorbox}
\end{column}
\end{columns}

\begin{center}
\begin{tcolorbox}[colback=mlyellow!20, colframe=mlorange!60, width=0.9\textwidth]
\centering
\textbf{Note:} This code scales to millions of users with mini-batch modifications
\end{tcolorbox}
\end{center}
\end{frame}

% Checkpoint 2
\begin{frame}
\frametitle{Checkpoint: Technical Mastery}
\framesubtitle{Algorithm Understanding Check \hfill \textcolor{mlgreen}{\textbf{Progress: Part 2/4}}}

\begin{columns}[T]
\begin{column}{0.48\textwidth}
\begin{tcolorbox}[colback=mlblue!10, colframe=mlblue!50, title=Match the Algorithm]
\normalsize
Match use case to algorithm:
\begin{enumerate}
\item Finding power users\\
   → \textcolor{mlgreen}{DBSCAN}
\item Quick segmentation\\
   → \textcolor{mlblue}{K-means}
\item User taxonomy\\
   → \textcolor{mlorange}{Hierarchical}
\item Mixed behaviors\\
   → \textcolor{mlpurple}{GMM}
\end{enumerate}
\end{tcolorbox}
\end{column}

\begin{column}{0.48\textwidth}
\begin{tcolorbox}[colback=mlgreen!10, colframe=mlgreen!50, title=Can You Code?]
\normalsize
Write the code to:
\begin{itemize}
\item Load user data (Y)
\item Scale features (Y)
\item Find optimal K (Y)
\item Cluster users (Y)
\item Analyze segments (Y)
\end{itemize}
\end{tcolorbox}
\end{column}
\end{columns}

\vspace{0.5em}
\begin{center}
\Large\textcolor{mlpurple}{\textbf{Excellent! Now let's turn clusters into personas}}\\
\normalsize\textit{Next: Design integration - from data to human stories}
\end{center}
\end{frame}

% PART 3: ENHANCED DESIGN INTEGRATION
\begin{frame}[plain]
\begin{center}
\vspace{2em}
{\Huge\textcolor{mlorange}{\textbf{PART 3}}}\\
\vspace{0.5em}
{\Large\textbf{Design Integration}}\\
\vspace{1em}
\textit{Transforming clusters into empathetic understanding}\\
\vspace{2em}
\Large
\textbf{What You'll Create:}\\
\vspace{0.5em}
\normalsize
\begin{itemize}
\item Data-driven personas with narratives
\item Empathy maps per segment
\item Differentiated journey maps
\item Pain point matrices
\item Innovation opportunities
\end{itemize}
\vspace{1em}
\Large\textcolor{mlpurple}{\textbf{From numbers to human stories}}
\end{center}
\end{frame}

% From Clusters to Personas
\begin{frame}
\frametitle{From Clusters to Living Personas}
\framesubtitle{Breathing Life into Data Points}

\begin{columns}[T]
\begin{column}{0.55\textwidth}
\begin{center}
\includegraphics[width=\textwidth]{charts/cluster_to_empathy.pdf}
\end{center}
\end{column}

\begin{column}{0.43\textwidth}
\begin{tcolorbox}[colback=mlorange!10, colframe=mlorange!50, title=Persona Building]
\small
\textbf{Cluster 3 → "Creative Curator"}

\textbf{Data says:}
\begin{itemize}
\item High sharing rate
\item Medium session length
\item Peak usage evenings
\end{itemize}

\textbf{Persona becomes:}
\begin{itemize}
\item Sarah, 28, Designer
\item Discovers \& shares inspiration
\item Values quality over quantity
\item Needs: curation tools
\end{itemize}

\textbf{Key:} Data informs, empathy guides
\end{tcolorbox}
\end{column}
\end{columns}

\begin{center}
\begin{tcolorbox}[colback=mlpurple!20, colframe=mlpurple!60, width=0.9\textwidth]
\centering
\textbf{Remember:} Clusters are statistical, personas are human - bridge both worlds!
\end{tcolorbox}
\end{center}
\end{frame}

% Building Empathy Maps
\begin{frame}
\frametitle{Building Data-Driven Empathy Maps}
\framesubtitle{Understanding What Users Think, Feel, Say, and Do}

\begin{columns}[T]
\begin{column}{0.55\textwidth}
\begin{center}
\includegraphics[width=\textwidth]{charts/empathy_map_detailed.pdf}
\end{center}
\end{column}

\begin{column}{0.43\textwidth}
\begin{tcolorbox}[colback=mlgreen!10, colframe=mlgreen!50, title=From Data to Empathy]
\small
\textbf{Think (from search data):}
\begin{itemize}
\item "Is there a better way?"
\item "How do others do this?"
\end{itemize}

\textbf{Feel (from behavior):}
\begin{itemize}
\item Frustrated (rage clicks)
\item Delighted (shares)
\end{itemize}

\textbf{Say (from reviews):}
\begin{itemize}
\item "Love this feature!"
\item "Too complicated"
\end{itemize}

\textbf{Do (from analytics):}
\begin{itemize}
\item Abandons after 3 clicks
\item Returns daily
\end{itemize}
\end{tcolorbox}
\end{column}
\end{columns}

\begin{center}
\begin{tcolorbox}[colback=mlyellow!20, colframe=mlorange!60, width=0.9\textwidth]
\centering
\textbf{Pro Tip:} Combine quantitative cluster data with qualitative user research
\end{tcolorbox}
\end{center}
\end{frame}

% Journey Mapping per Segment
\begin{frame}
\frametitle{Differentiated Journey Maps: One Product, Many Paths}
\framesubtitle{How Different User Segments Experience Your Product}

\begin{center}
\includegraphics[width=0.85\textwidth]{charts/journey_comparison.pdf}
\end{center}

\begin{columns}[T]
\begin{column}{0.32\textwidth}
\begin{tcolorbox}[colback=mlblue!10, colframe=mlblue!50, title=Power Users]
\small
\textbf{Journey:}
\begin{itemize}
\item Skip onboarding
\item Deep dive features
\item Create workflows
\item Become advocates
\end{itemize}
\textbf{Need:} Advanced tools
\end{tcolorbox}
\end{column}

\begin{column}{0.32\textwidth}
\begin{tcolorbox}[colback=mlgreen!10, colframe=mlgreen!50, title=Casual Users]
\small
\textbf{Journey:}
\begin{itemize}
\item Need guidance
\item Use basics only
\item Periodic engagement
\item Quick tasks
\end{itemize}
\textbf{Need:} Simplicity
\end{tcolorbox}
\end{column}

\begin{column}{0.32\textwidth}
\begin{tcolorbox}[colback=mlorange!10, colframe=mlorange!50, title=Explorers]
\small
\textbf{Journey:}
\begin{itemize}
\item Try everything
\item Test limits
\item Share discoveries
\item Suggest features
\end{itemize}
\textbf{Need:} Playground
\end{tcolorbox}
\end{column}
\end{columns}
\end{frame}

% Innovation Opportunities
\begin{frame}
\frametitle{Innovation Opportunity Matrix}
\framesubtitle{Where Each Segment Needs Innovation}

\begin{center}
% \includegraphics[width=0.85\textwidth]{charts/innovation_opportunities.pdf}
\textit{[Innovation Opportunity Matrix Chart]}
\end{center}

\begin{center}
\begin{tcolorbox}[colback=mlpurple!20, colframe=mlpurple!60, width=0.9\textwidth]
\centering
\textbf{Insight:} Different segments = different innovation priorities
\end{tcolorbox}
\end{center}
\end{frame}

% PART 4: PRACTICE & CASE STUDY
\begin{frame}[plain]
\begin{center}
\vspace{2em}
{\Huge\textcolor{mlred}{\textbf{PART 4}}}\\
\vspace{0.5em}
{\Large\textbf{Practice \& Case Study}}\\
\vspace{1em}
\textit{Real-world application: Spotify's data-driven personas}\\
\vspace{2em}
\Large
\textbf{What We'll Explore:}\\
\vspace{0.5em}
\normalsize
\begin{itemize}
\item How Spotify segments 400M users
\item 5 core music personas discovered
\item Features designed per segment
\item Measurable impact
\item Your practice exercise
\end{itemize}
\vspace{1em}
\Large\textcolor{mlpurple}{\textbf{From theory to practice}}
\end{center}
\end{frame}

% Spotify Case Study
\begin{frame}
\frametitle{Case Study: Spotify's Data-Driven Personas}
\framesubtitle{How 400 Million Users Became 5 Core Personas}

\begin{columns}[T]
\begin{column}{0.55\textwidth}
\begin{center}
\includegraphics[width=\textwidth]{charts/persona_profiles.pdf}
\end{center}
\end{column}

\begin{column}{0.43\textwidth}
\begin{tcolorbox}[colback=mlgreen!10, colframe=mlgreen!50, title=The Challenge]
\small
\textbf{Problem:}
\begin{itemize}
\item 400M users
\item 180 countries
\item Diverse tastes
\item One app?
\end{itemize}

\textbf{Solution:}
\begin{itemize}
\item Behavioral clustering
\item 5 core personas found
\item Personalized features
\item Dynamic adaptation
\end{itemize}

\textbf{Impact:}
\begin{itemize}
\item 2x engagement
\item 40\% less churn
\item Higher satisfaction
\end{itemize}
\end{tcolorbox}
\end{column}
\end{columns}

\begin{center}
\begin{tcolorbox}[colback=mlyellow!20, colframe=mlorange!60, width=0.9\textwidth]
\centering
\textbf{Key Learning:} Same data, different algorithms = deeper understanding
\end{tcolorbox}
\end{center}
\end{frame}

% The 5 Spotify Personas
\begin{frame}
\frametitle{Spotify's 5 Music Personas: Data-Driven Discovery}
\framesubtitle{Each Persona Gets Different Features}

\begin{columns}[T]
\begin{column}{0.19\textwidth}
\begin{tcolorbox}[colback=mlblue!10, colframe=mlblue!50, title=Explorer]
\tiny
\textbf{Behavior:}
\begin{itemize}
\item New music daily
\item Diverse genres
\item Short sessions
\end{itemize}

\textbf{Feature:}\\
Discover Weekly
\end{tcolorbox}
\end{column}

\begin{column}{0.19\textwidth}
\begin{tcolorbox}[colback=mlgreen!10, colframe=mlgreen!50, title=Loyalist]
\tiny
\textbf{Behavior:}
\begin{itemize}
\item Same artists
\item Full albums
\item Long sessions
\end{itemize}

\textbf{Feature:}\\
Artist Radio
\end{tcolorbox}
\end{column}

\begin{column}{0.19\textwidth}
\begin{tcolorbox}[colback=mlorange!10, colframe=mlorange!50, title=Social]
\tiny
\textbf{Behavior:}
\begin{itemize}
\item Share playlists
\item Follow friends
\item Group sessions
\end{itemize}

\textbf{Feature:}\\
Blend Playlists
\end{tcolorbox}
\end{column}

\begin{column}{0.19\textwidth}
\begin{tcolorbox}[colback=mlred!10, colframe=mlred!50, title=Focused]
\tiny
\textbf{Behavior:}
\begin{itemize}
\item Background music
\item Mood playlists
\item Repeat mode
\end{itemize}

\textbf{Feature:}\\
Focus Playlists
\end{tcolorbox}
\end{column}

\begin{column}{0.19\textwidth}
\begin{tcolorbox}[colback=mlpurple!10, colframe=mlpurple!50, title=Curator]
\tiny
\textbf{Behavior:}
\begin{itemize}
\item Create playlists
\item Organize library
\item Edit metadata
\end{itemize}

\textbf{Feature:}\\
Enhanced Library
\end{tcolorbox}
\end{column}
\end{columns}

\vspace{0.5em}
\begin{center}
\Large\textcolor{mlpurple}{\textbf{Result: Each user feels Spotify was made for them}}
\end{center}
\end{frame}

% Practice Exercise
\begin{frame}
\frametitle{Practice Exercise: Segment Your Users}
\framesubtitle{Hands-On: From Raw Data to Personas}

\begin{columns}[T]
\begin{column}{0.48\textwidth}
\begin{tcolorbox}[colback=mlblue!10, colframe=mlblue!50, title=Your Task]
\normalsize
\textbf{Dataset:} E-commerce users\\
\textbf{Size:} 10,000 users\\
\textbf{Features:} 15 behavioral metrics

\vspace{0.3em}
\textbf{Steps:}
\begin{enumerate}
\item Load and explore data
\item Preprocess features
\item Find optimal K
\item Cluster users
\item Analyze segments
\item Create 3 personas
\item Build empathy maps
\item Design features
\end{enumerate}

\textbf{Time:} 45 minutes\\
\textbf{Deliverable:} Persona cards
\end{tcolorbox}
\end{column}

\begin{column}{0.48\textwidth}
\begin{tcolorbox}[colback=mlgreen!10, colframe=mlgreen!50, title=Starter Code]
\begin{lstlisting}[language=Python, basicstyle=\tiny\ttfamily]
import pandas as pd
from sklearn.cluster import KMeans
from sklearn.preprocessing import StandardScaler

# Load your data
users = pd.read_csv('ecommerce_users.csv')

# Explore
print(users.info())
print(users.describe())

# Select behavioral features
features = ['visits', 'duration', 
           'pages_viewed', 'items_bought',
           'cart_abandons', 'reviews']

# Your code here...
# 1. Scale features
# 2. Find optimal K
# 3. Cluster users
# 4. Analyze segments

# Template for persona
persona_template = {
    'name': '',
    'characteristics': [],
    'needs': [],
    'pain_points': [],
    'opportunities': []
}
\end{lstlisting}
\end{tcolorbox}
\end{column}
\end{columns}
\end{frame}

% Implementation Checklist
\begin{frame}
\frametitle{Implementation Checklist}
\framesubtitle{Your Step-by-Step Guide to User Segmentation Success}

\begin{columns}[T]
\begin{column}{0.32\textwidth}
\begin{tcolorbox}[colback=mlblue!10, colframe=mlblue!50, title=1. Data Prep]
\small
\textbf{Collect:}
\begin{itemize}
\item[$\square$] User behavior data
\item[$\square$] Engagement metrics
\item[$\square$] Feature usage
\item[$\square$] Feedback data
\end{itemize}

\textbf{Process:}
\begin{itemize}
\item[$\square$] Clean data
\item[$\square$] Handle missing
\item[$\square$] Feature engineering
\item[$\square$] Scale features
\end{itemize}
\end{tcolorbox}
\end{column}

\begin{column}{0.32\textwidth}
\begin{tcolorbox}[colback=mlgreen!10, colframe=mlgreen!50, title=2. Cluster]
\small
\textbf{Algorithm:}
\begin{itemize}
\item[$\square$] Choose method
\item[$\square$] Find optimal K
\item[$\square$] Validate quality
\item[$\square$] Check stability
\end{itemize}

\textbf{Analysis:}
\begin{itemize}
\item[$\square$] Segment profiles
\item[$\square$] Size distribution
\item[$\square$] Key differences
\item[$\square$] Edge cases
\end{itemize}
\end{tcolorbox}
\end{column}

\begin{column}{0.32\textwidth}
\begin{tcolorbox}[colback=mlorange!10, colframe=mlorange!50, title=3. Design]
\small
\textbf{Personas:}
\begin{itemize}
\item[$\square$] Create narratives
\item[$\square$] Build empathy maps
\item[$\square$] Map journeys
\item[$\square$] Identify needs
\end{itemize}

\textbf{Apply:}
\begin{itemize}
\item[$\square$] Design features
\item[$\square$] Prioritize roadmap
\item[$\square$] Test with users
\item[$\square$] Measure impact
\end{itemize}
\end{tcolorbox}
\end{column}
\end{columns}

\vspace{0.5em}
\begin{center}
\begin{tcolorbox}[colback=mlpurple!20, colframe=mlpurple!60, width=0.9\textwidth]
\centering
\textbf{Success Metric:} Users say "This feels made just for me!"
\end{tcolorbox}
\end{center}
\end{frame}

% Key Takeaways
\begin{frame}
\frametitle{Key Takeaways: Clustering for Deep Empathy}
\framesubtitle{What You've Mastered Today}

\begin{columns}[T]
\begin{column}{0.32\textwidth}
\begin{tcolorbox}[colback=mlblue!10, colframe=mlblue!50, title=Technical]
\small
\textbf{You can now:}
\begin{itemize}
\item Implement K-means++
\item Validate clusters
\item Use DBSCAN
\item Build hierarchies
\item Apply GMM
\item Scale to millions
\end{itemize}
\end{tcolorbox}
\end{column}

\begin{column}{0.32\textwidth}
\begin{tcolorbox}[colback=mlgreen!10, colframe=mlgreen!50, title=Design]
\small
\textbf{You create:}
\begin{itemize}
\item Data personas
\item Empathy maps
\item Journey maps
\item Pain matrices
\item Opportunity maps
\item Feature priorities
\end{itemize}
\end{tcolorbox}
\end{column}

\begin{column}{0.32\textwidth}
\begin{tcolorbox}[colback=mlorange!10, colframe=mlorange!50, title=Impact]
\small
\textbf{Results in:}
\begin{itemize}
\item Better products
\item Happy users
\item Less churn
\item More engagement
\item Innovation focus
\item Scalable empathy
\end{itemize}
\end{tcolorbox}
\end{column}
\end{columns}

\vspace{0.5em}
\begin{center}
\begin{tcolorbox}[colback=mlpurple!20, colframe=mlpurple!60, width=0.9\textwidth]
\centering\Large
\textbf{Remember: Every cluster represents real people with real needs}
\end{tcolorbox}
\end{center}

\begin{center}
\Large\textcolor{mlgreen}{\textbf{You now have the tools to understand millions of users with empathy!}}
\end{center}
\end{frame}

% Resources & Next Steps
\begin{frame}
\frametitle{Resources \& Next Week Preview}
\framesubtitle{Continue Your Journey in Data-Driven Empathy}

\begin{columns}[T]
\begin{column}{0.48\textwidth}
\begin{tcolorbox}[colback=mlblue!10, colframe=mlblue!50, title=This Week's Resources]
\small
\textbf{Readings:}
\begin{itemize}
\item "Persona Lifecycle" - Pruitt \& Adlin
\item "K-means++: The Advantages" - Arthur
\item Spotify Engineering Blog
\end{itemize}

\textbf{Tools:}
\begin{itemize}
\item Scikit-learn clustering guide
\item Persona template toolkit
\item Journey mapping tools
\end{itemize}

\textbf{Practice:}
\begin{itemize}
\item Kaggle customer segmentation
\item UCI ML Repository datasets
\end{itemize}
\end{tcolorbox}
\end{column}

\begin{column}{0.48\textwidth}
\begin{tcolorbox}[colback=mlgreen!10, colframe=mlgreen!50, title=Next Week: NLP for Emotion]
\small
\textbf{Week 3 Preview:}
\begin{itemize}
\item Sentiment analysis with BERT
\item Understanding user emotions
\item Context-aware NLP
\item Voice of customer analysis
\end{itemize}

\textbf{You'll learn:}
\begin{itemize}
\item Extract emotions from text
\item Detect sarcasm and nuance
\item Analyze reviews at scale
\item Build emotional journeys
\end{itemize}

\textbf{Homework:}\\
Segment a dataset of your choice!
\end{tcolorbox}
\end{column}
\end{columns}

\vspace{0.5em}
\begin{center}
\begin{tcolorbox}[colback=mlyellow!20, colframe=mlorange!60, width=0.9\textwidth]
\centering
\textbf{Office Hours:} Tuesday 2-4pm | Slack: \#ml-empathy | Questions welcome!
\end{tcolorbox}
\end{center}
\end{frame}

% Final Slide
\begin{frame}
\frametitle{Your Data-Driven Empathy Journey Continues!}
\framesubtitle{From Understanding Groups to Understanding Emotions}

\begin{center}
{\Large\textbf{This Week's Achievement:}}\\
\vspace{0.5em}
{\large You can now understand user segments at scale with deep empathy}\\
\vspace{1em}
{\Large\textbf{Next Week's Challenge:}}\\
\vspace{0.5em}
{\large Understanding what users feel through their words}
\end{center}

\vspace{1em}
\begin{columns}[T]
\begin{column}{0.48\textwidth}
\begin{tcolorbox}[colback=mlgreen!10, colframe=mlgreen!50, title=Your Homework]
\normalsize
\begin{enumerate}
\item Choose a dataset
\item Apply clustering pipeline
\item Create 3-5 personas
\item Build empathy maps
\item Share in Slack!
\end{enumerate}
\end{tcolorbox}
\end{column}

\begin{column}{0.48\textwidth}
\begin{tcolorbox}[colback=mlblue!10, colframe=mlblue!50, title=Success Tips]
\normalsize
\begin{itemize}
\item Start with K-means++
\item Always validate clusters
\item Combine with qualitative data
\item Focus on actionable insights
\item Remember: data serves empathy
\end{itemize}
\end{tcolorbox}
\end{column}
\end{columns}

\vspace{1em}
\begin{center}
\begin{tcolorbox}[colback=mlpurple!20, colframe=mlpurple!60, width=0.9\textwidth]
\centering\Large
\textbf{Questions? Let's discuss!}
\end{tcolorbox}
\end{center}
\end{frame}

\end{document}