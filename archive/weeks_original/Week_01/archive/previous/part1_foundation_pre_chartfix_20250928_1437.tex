% Part 1: Foundation - Understanding Why We Need ML for Innovation
\section{Foundation: The Innovation Challenge}

% PART 1 Section Divider
\begin{frame}[plain]
\begin{center}
\vspace{2em}
{\Huge\textcolor{mlblue}{\textbf{PART 1}}}\\
\vspace{0.5em}
{\Large\textbf{Foundation \& Context}}\\
\vspace{1em}
\textit{Understanding why we need ML for innovation}\\
\vspace{2em}
\Large
\textbf{Key Questions We'll Answer:}\\
\vspace{0.5em}
\normalsize
\begin{itemize}
\item Why do traditional methods fail at scale?
\item How does ML amplify human creativity?
\item What is the dual pipeline approach?
\item Where does clustering fit in innovation?
\end{itemize}
\vspace{1em}
\Large\textcolor{mlpurple}{\textbf{Let's build your foundation}}
\end{center}
\end{frame}

% Enhanced Innovation Discovery slide with detailed explanation
\begin{frame}
\frametitle{\Large Innovation Discovery: The Starting Point}
\framesubtitle{Finding Order in Chaos - Your First Challenge}

\begin{columns}[T]
\column{0.65\textwidth}
\begin{center}
\includegraphics[width=\textwidth]{charts/innovation_discovery.pdf}
\end{center}

\column{0.33\textwidth}
\begin{tcolorbox}[colback=mlpurple!10, colframe=mlpurple!50, title=The Challenge]
\small
\textbf{What you see:}
\begin{itemize}
\item 5000+ scattered ideas
\item No clear patterns
\item Hidden connections
\item Overwhelming complexity
\end{itemize}

\vspace{0.3em}
\textbf{What ML will find:}
\begin{itemize}
\item Natural groupings
\item Innovation types
\item Relationships
\item Opportunities
\end{itemize}
\end{tcolorbox}
\end{columns}

\vspace{0.3em}
\begin{center}
\begin{tcolorbox}[colback=mlyellow!20, colframe=mlorange!60, width=0.9\textwidth]
\centering
\textbf{Think about it:} How would YOU group 5000 ideas manually? How long would it take?
\end{tcolorbox}
\end{center}
\end{frame}

% Enhanced Innovation Challenge with detailed comparisons
\begin{frame}
\frametitle{\Large The Innovation Challenge: A Detailed Comparison}
\framesubtitle{Why Traditional Design Thinking Needs AI Enhancement}

\begin{columns}[T]
\column{0.48\textwidth}
\begin{tcolorbox}[colback=mlred!10, colframe=mlred!50, title=Traditional Limitations]
\small
\textbf{Scale Problems:}
\begin{itemize}
\item Can analyze 50-100 ideas manually
\item Takes weeks for basic insights
\item Limited to obvious patterns
\end{itemize}

\textbf{Human Biases:}
\begin{itemize}
\item Confirmation bias
\item Availability heuristic
\item Anchoring effects
\end{itemize}

\textbf{Process Issues:}
\begin{itemize}
\item Sequential analysis
\item Manual categorization
\item Static frameworks
\end{itemize}
\end{tcolorbox}

\column{0.48\textwidth}
\begin{tcolorbox}[colback=mlgreen!10, colframe=mlgreen!50, title=AI-Enhanced Capabilities]
\small
\textbf{Scale Advantages:}
\begin{itemize}
\item Process millions of data points
\item Real-time pattern recognition
\item Find non-obvious connections
\end{itemize}

\textbf{Objective Analysis:}
\begin{itemize}
\item Data-driven discovery
\item Statistical validation
\item Unbiased grouping
\end{itemize}

\textbf{Dynamic Process:}
\begin{itemize}
\item Parallel processing
\item Automatic clustering
\item Adaptive learning
\end{itemize}
\end{tcolorbox}
\end{columns}

\vspace{0.5em}
\begin{center}
\begin{tcolorbox}[colback=mlpurple!20, colframe=mlpurple!60, width=0.9\textwidth]
\centering\large
\textbf{The Promise:} 100x more insights, 10x faster innovation, 0 human bias
\end{tcolorbox}
\end{center}
\end{frame}

% Enhanced Dual Pipeline with detailed explanations
\begin{frame}
\frametitle{\Large The Dual Pipeline: A Revolutionary Approach}
\framesubtitle{Where Machine Learning Meets Design Thinking}

\begin{center}
\includegraphics[width=0.75\textwidth]{charts/dual_pipeline_overview.pdf}
\end{center}

\begin{columns}[T]
\column{0.48\textwidth}
\begin{tcolorbox}[colback=mlblue!10, colframe=mlblue!50, title=ML Pipeline Explained]
\small
\textbf{Data} → \textbf{Process} → \textbf{Model} → \textbf{Evaluate} → \textbf{Deploy}

\begin{itemize}
\item \textbf{Data:} Collect innovation metrics
\item \textbf{Process:} Clean and transform
\item \textbf{Model:} Apply clustering
\item \textbf{Evaluate:} Validate quality
\item \textbf{Deploy:} Scale insights
\end{itemize}
\end{tcolorbox}

\column{0.48\textwidth}
\begin{tcolorbox}[colback=mlorange!10, colframe=mlorange!50, title=Design Pipeline Explained]
\small
\textbf{Empathize} → \textbf{Define} → \textbf{Ideate} → \textbf{Prototype} → \textbf{Test}

\begin{itemize}
\item \textbf{Empathize:} Understand users
\item \textbf{Define:} Frame problems
\item \textbf{Ideate:} Generate solutions
\item \textbf{Prototype:} Build concepts
\item \textbf{Test:} Validate impact
\end{itemize}
\end{tcolorbox}
\end{columns}

\vspace{0.3em}
\begin{center}
\textcolor{mlpurple}{\textbf{Key Insight:} Each ML step enhances its design thinking counterpart}
\end{center}
\end{frame}

% Current Reality Visualization with explanations
\begin{frame}
\frametitle{\Large Current Reality: The One-Size-Fits-All Problem}
\framesubtitle{Why Generic Categories Fail Innovation}

\begin{columns}[T]
\column{0.65\textwidth}
\begin{center}
\includegraphics[width=\textwidth]{charts/current_reality_visual.pdf}
\end{center}

\column{0.33\textwidth}
\begin{tcolorbox}[colback=mlred!10, colframe=mlred!50, title=Problems]
\small
\textbf{Left Side Issues:}
\begin{itemize}
\item Square pegs, round holes
\item Forced categorization
\item Lost uniqueness
\item Missed patterns
\end{itemize}

\vspace{0.3em}
\textbf{Right Side Benefits:}
\begin{itemize}
\item Natural fit
\item Data-driven groups
\item Preserved characteristics
\item Revealed patterns
\end{itemize}
\end{tcolorbox}
\end{columns}

\begin{center}
\begin{tcolorbox}[colback=mlyellow!20, colframe=mlorange!60, width=0.9\textwidth]
\centering
\textbf{Real Example:} Netflix used to have 10 movie categories. Now they have 76,897 micro-genres thanks to clustering!
\end{tcolorbox}
\end{center}
\end{frame}

% Innovation Archetypes with detailed descriptions
\begin{frame}
\frametitle{\Large Innovation Archetypes: What We'll Discover}
\framesubtitle{Common Patterns Hidden in Your Data}

\begin{columns}[T]
\column{0.48\textwidth}
\begin{tcolorbox}[colback=mlblue!10, colframe=mlblue!50, title=Core Types]
\small
\textbf{1. Disruptive Innovation}
\begin{itemize}
\item Reshapes entire markets
\item High risk, high reward
\item Example: Uber vs taxis
\end{itemize}

\textbf{2. Incremental Innovation}
\begin{itemize}
\item Step-by-step improvements
\item Low risk, steady gains
\item Example: iPhone iterations
\end{itemize}

\textbf{3. Service Innovation}
\begin{itemize}
\item New delivery methods
\item Customer experience focus
\item Example: Amazon Prime
\end{itemize}
\end{tcolorbox}

\column{0.48\textwidth}
\begin{tcolorbox}[colback=mlgreen!10, colframe=mlgreen!50, title=Emerging Types]
\small
\textbf{4. Business Model Innovation}
\begin{itemize}
\item New value creation
\item Revenue model changes
\item Example: Freemium models
\end{itemize}

\textbf{5. Process Innovation}
\begin{itemize}
\item Efficiency improvements
\item Cost reduction focus
\item Example: Lean manufacturing
\end{itemize}

\textbf{6. Platform Innovation}
\begin{itemize}
\item Ecosystem creation
\item Network effects
\item Example: App stores
\end{itemize}
\end{tcolorbox}
\end{columns}

\vspace{0.3em}
\begin{center}
\begin{tcolorbox}[colback=mlpurple!20, colframe=mlpurple!60, width=0.9\textwidth]
\centering
\textbf{Clustering reveals:} Which type each of your 5000 ideas belongs to automatically!
\end{tcolorbox}
\end{center}
\end{frame}

% Innovation Diamond Framework
\begin{frame}
\frametitle{\Large The Innovation Diamond: Our Visual Framework}
\framesubtitle{From 1 Challenge to 5000 Ideas to 5 Strategic Solutions}

\begin{center}
\includegraphics[width=0.8\textwidth]{charts/innovation_diamond_complete.pdf}
\end{center}

\begin{columns}[T]
\column{0.48\textwidth}
\begin{tcolorbox}[colback=challenge!10, colframe=challenge!50, title=The Journey]
\small
\textbf{Start:} 1 business challenge\\
\textbf{Expand:} Divergent thinking → 5000 ideas\\
\textbf{Analyze:} ML clustering finds patterns\\
\textbf{Converge:} 5 strategic innovation areas\\
\textbf{Action:} Focused development
\end{tcolorbox}

\column{0.48\textwidth}
\begin{tcolorbox}[colback=mlblue!10, colframe=mlblue!50, title=Week 1 Focus]
\small
\textbf{This week we focus on:}
\begin{itemize}
\item The expansion phase (clustering)
\item Finding natural patterns
\item Understanding user segments
\item Building empathy maps
\end{itemize}
\end{tcolorbox}
\end{columns}

\vspace{\fill}
\footnotesize\textcolor{mlgray}{The Innovation Diamond will guide us through all 10 weeks of this course}
\end{frame}

% Course Context and Week 1 Position
\begin{frame}
\frametitle{\Large Where We Are: Week 1 in the 10-Week Journey}
\framesubtitle{Clustering \& Empathy - The Foundation of Everything}

\begin{columns}[T]
\column{0.48\textwidth}
\begin{tcolorbox}[colback=mlblue!10, colframe=mlblue!50, title=10-Week Overview]
\small
\textbf{Weeks 1-3: Empathize}
\begin{itemize}
\item Week 1: Clustering \& patterns
\item Week 2: Advanced clustering
\item Week 3: NLP \& emotional context
\end{itemize}

\textbf{Week 4: Define}
\begin{itemize}
\item Classification \& problem framing
\end{itemize}

\textbf{Week 5: Ideate}
\begin{itemize}
\item Topic modeling \& idea generation
\end{itemize}
\end{tcolorbox}

\column{0.48\textwidth}
\begin{tcolorbox}[colback=mlgreen!10, colframe=mlgreen!50, title=Week 1 Learning Goals]
\small
\textbf{By the end of today:}
\begin{itemize}
\item Understand clustering fundamentals
\item Apply K-means to real data
\item Find optimal cluster numbers
\item Create user personas from clusters
\item Build empathy maps
\item Identify innovation opportunities
\end{itemize}

\textbf{You'll be ready for:}
\begin{itemize}
\item Week 2's advanced techniques
\item Real-world clustering projects
\end{itemize}
\end{tcolorbox}
\end{columns}

\vspace{\fill}
\footnotesize\textcolor{mlgray}{Strong clustering foundation = success in all subsequent weeks}
\end{frame}