\documentclass[8pt,aspectratio=169]{beamer}
\usetheme{Madrid}
\usecolortheme{default}
\usepackage{graphicx}
\usepackage{booktabs}
\usepackage{adjustbox}
\usepackage{multicol}
\usepackage{amsmath}
\usepackage{amssymb}
\usepackage{tcolorbox}
\usepackage{xcolor}
\usepackage{tikz}
\usetikzlibrary{shapes.geometric, arrows, positioning}
\usepackage{listings}

% Define colors matching the overview presentation
\definecolor{mlblue}{RGB}{31, 119, 180}
\definecolor{mlorange}{RGB}{255, 127, 14}
\definecolor{mlgreen}{RGB}{44, 160, 44}
\definecolor{mlred}{RGB}{214, 39, 40}
\definecolor{mlpurple}{RGB}{148, 103, 189}
\definecolor{mlbrown}{RGB}{140, 86, 75}
\definecolor{mlpink}{RGB}{227, 119, 194}
\definecolor{mlgray}{RGB}{127, 127, 127}
\definecolor{mlyellow}{RGB}{255, 187, 120}
\definecolor{mlcyan}{RGB}{23, 190, 207}

% Beamer template customization
\setbeamertemplate{navigation symbols}{}
\setbeamertemplate{footline}{
  \leavevmode%
  \hbox{%
  \begin{beamercolorbox}[wd=.333333\paperwidth,ht=2.25ex,dp=1ex,center]{author in head/foot}%
    \usebeamerfont{author in head/foot}Part \insertpartnumber/4
  \end{beamercolorbox}%
  \begin{beamercolorbox}[wd=.333333\paperwidth,ht=2.25ex,dp=1ex,center]{title in head/foot}%
    \usebeamerfont{title in head/foot}Week 1: Clustering
  \end{beamercolorbox}%
  \begin{beamercolorbox}[wd=.333333\paperwidth,ht=2.25ex,dp=1ex,right]{date in head/foot}%
    \usebeamerfont{date in head/foot}Slide \insertframenumber{} of \inserttotalframenumber\hspace*{2ex} 
  \end{beamercolorbox}}%
  \vskip0pt%
}

% Enhanced title information
\title{\Large\textbf{Machine Learning for Smarter Innovation}\\
\vspace{0.5em}
\Large Week 1: Foundations \& Clustering}
\subtitle{Discovering Innovation Patterns with ML\\
\small A Comprehensive Guide to Pattern Recognition}
\author{BSc Course in AI-Enhanced Innovation}
\institute{Department of Computer Science}
\date{2025}

\begin{document}

% Enhanced title slide with more context
\begin{frame}
\titlepage
\vspace{-1em}
\begin{center}
\small\textit{This week: Transform 5000+ scattered innovation ideas into actionable patterns using clustering algorithms}
\end{center}
\end{frame}

% Table of Contents
\begin{frame}
\frametitle{Today's Journey}
\framesubtitle{Your Roadmap to Understanding Clustering for Innovation}

\begin{columns}[T]
\begin{column}{0.48\textwidth}
\begin{tcolorbox}[colback=mlblue!10, colframe=mlblue!50, title=Technical Foundation]
\small
\textbf{Part 1: Foundation (20 min)}
\begin{itemize}
\item Why traditional approaches fail
\item The dual pipeline methodology
\item Setting learning objectives
\end{itemize}

\vspace{0.3em}
\textbf{Part 2: Technical Core (30 min)}
\begin{itemize}
\item K-means clustering explained
\item Finding optimal clusters
\item Quality metrics and validation
\item Advanced algorithms (DBSCAN, Hierarchical)
\end{itemize}
\end{tcolorbox}
\end{column}

\begin{column}{0.48\textwidth}
\begin{tcolorbox}[colback=mlgreen!10, colframe=mlgreen!50, title=Application Focus]
\small
\textbf{Part 3: Design Integration (25 min)}
\begin{itemize}
\item Creating innovation archetypes
\item Pattern mapping and journeys
\item Opportunity identification
\item Priority matrices
\end{itemize}

\vspace{0.3em}
\textbf{Part 4: Practice (15 min)}
\begin{itemize}
\item Real-world case studies
\item Implementation checklist
\item Hands-on exercises
\item Resources and next steps
\end{itemize}
\end{tcolorbox}
\end{column}
\end{columns}
\end{frame}

% Enhanced Prerequisites with detailed explanations
\begin{frame}
\frametitle{\Large Prerequisites \& What You Need}
\framesubtitle{Setting You Up for Success - No ML Experience Required!}

\begin{columns}[T]
\begin{column}{0.48\textwidth}
\begin{tcolorbox}[colback=mlgreen!10, colframe=mlgreen!50, title=What You Should Know]
\normalsize
\begin{itemize}
\item \textbf{Basic Python} 
  \begin{itemize}
  \small
  \item Variables, loops, functions
  \item Lists and dictionaries
  \end{itemize}
\item \textbf{High school math}
  \begin{itemize}
  \small
  \item Calculating averages
  \item Understanding distances
  \end{itemize}
\item \textbf{Jupyter notebooks}
  \begin{itemize}
  \small
  \item Running cells
  \item Basic navigation
  \end{itemize}
\item \textbf{Data concepts}
  \begin{itemize}
  \small
  \item Tables, rows, columns
  \item CSV files
  \end{itemize}
\end{itemize}
\end{tcolorbox}
\end{column}

\begin{column}{0.48\textwidth}
\begin{tcolorbox}[colback=mlblue!10, colframe=mlblue!50, title=What We'll Provide]
\normalsize
\begin{itemize}
\item \textbf{Code templates}
  \begin{itemize}
  \small
  \item Copy-paste ready
  \item Fully commented
  \end{itemize}
\item \textbf{Step-by-step guides}
  \begin{itemize}
  \small
  \item Visual explanations
  \item Checkpoint exercises
  \end{itemize}
\item \textbf{Practice datasets}
  \begin{itemize}
  \small
  \item Innovation examples
  \item Pre-cleaned data
  \end{itemize}
\item \textbf{Support materials}
  \begin{itemize}
  \small
  \item Glossary of terms
  \item FAQ section
  \end{itemize}
\end{itemize}
\end{tcolorbox}
\end{column}
\end{columns}

\vspace{0.5em}
\begin{center}
\begin{tcolorbox}[colback=mlyellow!20, colframe=mlorange!60, width=0.8\textwidth]
\centering
\textbf{Remember:} Everyone starts somewhere. We'll guide you step-by-step!
\end{tcolorbox}
\end{center}
\end{frame}

% PART 1: Enhanced Foundation Section
\begin{frame}[plain]
\begin{center}
\vspace{2em}
{\Huge\textcolor{mlblue}{\textbf{PART 1}}}\\
\vspace{0.5em}
{\Large\textbf{Foundation \& Context}}\\
\vspace{1em}
\textit{Understanding why we need ML for innovation}\\
\vspace{2em}
\Large
\textbf{Key Questions We'll Answer:}\\
\vspace{0.5em}
\normalsize
\begin{itemize}
\item Why do traditional methods fail at scale?
\item How does ML amplify human creativity?
\item What is the dual pipeline approach?
\item Where does clustering fit in innovation?
\end{itemize}
\vspace{1em}
\Large\textcolor{mlpurple}{\textbf{Let's build your foundation}}
\end{center}
\end{frame}

% Enhanced Innovation Discovery slide with detailed explanation
\begin{frame}
\frametitle{\Large Innovation Discovery: The Starting Point}
\framesubtitle{Finding Order in Chaos - Your First Challenge}

\begin{columns}[T]
\begin{column}{0.65\textwidth}
\begin{center}
\includegraphics[width=\textwidth]{charts/innovation_discovery.pdf}
\end{center}
\end{column}

\begin{column}{0.33\textwidth}
\begin{tcolorbox}[colback=mlpurple!10, colframe=mlpurple!50, title=The Challenge]
\small
\textbf{What you see:}
\begin{itemize}
\item 5000+ scattered ideas
\item No clear patterns
\item Hidden connections
\item Overwhelming complexity
\end{itemize}

\vspace{0.3em}
\textbf{What ML will find:}
\begin{itemize}
\item Natural groupings
\item Innovation types
\item Relationships
\item Opportunities
\end{itemize}
\end{tcolorbox}
\end{column}
\end{columns}

\vspace{0.3em}
\begin{center}
\begin{tcolorbox}[colback=mlyellow!20, colframe=mlorange!60, width=0.9\textwidth]
\centering
\textbf{Think about it:} How would YOU group 5000 ideas manually? How long would it take?
\end{tcolorbox}
\end{center}
\end{frame}

% Enhanced Innovation Challenge with detailed comparisons
\begin{frame}
\frametitle{\Large The Innovation Challenge: A Detailed Comparison}
\framesubtitle{Why Traditional Design Thinking Needs AI Enhancement}

\begin{columns}[T]
\begin{column}{0.48\textwidth}
\begin{tcolorbox}[colback=mlred!10, colframe=mlred!50, title=Traditional Limitations]
\small
\textbf{Scale Problems:}
\begin{itemize}
\item Can analyze 50-100 ideas manually
\item Takes weeks for basic insights
\item Limited to obvious patterns
\end{itemize}

\textbf{Human Biases:}
\begin{itemize}
\item Confirmation bias
\item Availability heuristic
\item Anchoring effects
\end{itemize}

\textbf{Process Issues:}
\begin{itemize}
\item Sequential analysis
\item Manual categorization
\item Static frameworks
\end{itemize}
\end{tcolorbox}
\end{column}

\begin{column}{0.48\textwidth}
\begin{tcolorbox}[colback=mlgreen!10, colframe=mlgreen!50, title=AI-Enhanced Capabilities]
\small
\textbf{Scale Advantages:}
\begin{itemize}
\item Process millions of data points
\item Real-time pattern recognition
\item Find non-obvious connections
\end{itemize}

\textbf{Objective Analysis:}
\begin{itemize}
\item Data-driven discovery
\item Statistical validation
\item Unbiased grouping
\end{itemize}

\textbf{Dynamic Process:}
\begin{itemize}
\item Parallel processing
\item Automatic clustering
\item Adaptive learning
\end{itemize}
\end{tcolorbox}
\end{column}
\end{columns}

\vspace{0.5em}
\begin{center}
\begin{tcolorbox}[colback=mlpurple!20, colframe=mlpurple!60, width=0.9\textwidth]
\centering\large
\textbf{The Promise:} 100x more insights, 10x faster innovation, 0 human bias
\end{tcolorbox}
\end{center}
\end{frame}

% Enhanced Dual Pipeline with detailed explanations
\begin{frame}
\frametitle{\Large The Dual Pipeline: A Revolutionary Approach}
\framesubtitle{Where Machine Learning Meets Design Thinking}

\begin{center}
\includegraphics[width=0.75\textwidth]{../ML_Design_Course/course_visuals/dual_pipeline.pdf}
\end{center}

\begin{columns}[T]
\begin{column}{0.48\textwidth}
\begin{tcolorbox}[colback=mlblue!10, colframe=mlblue!50, title=ML Pipeline Explained]
\small
\textbf{Data} → \textbf{Process} → \textbf{Model} → \textbf{Evaluate} → \textbf{Deploy}

\begin{itemize}
\item \textbf{Data:} Collect innovation metrics
\item \textbf{Process:} Clean and transform
\item \textbf{Model:} Apply clustering
\item \textbf{Evaluate:} Validate quality
\item \textbf{Deploy:} Scale insights
\end{itemize}
\end{tcolorbox}
\end{column}

\begin{column}{0.48\textwidth}
\begin{tcolorbox}[colback=mlorange!10, colframe=mlorange!50, title=Design Pipeline Explained]
\small
\textbf{Empathize} → \textbf{Define} → \textbf{Ideate} → \textbf{Prototype} → \textbf{Test}

\begin{itemize}
\item \textbf{Empathize:} Understand users
\item \textbf{Define:} Frame problems
\item \textbf{Ideate:} Generate solutions
\item \textbf{Prototype:} Build concepts
\item \textbf{Test:} Validate impact
\end{itemize}
\end{tcolorbox}
\end{column}
\end{columns}

\vspace{0.3em}
\begin{center}
\textcolor{mlpurple}{\textbf{Key Insight:} Each ML step enhances its design thinking counterpart}
\end{center}
\end{frame}

% Current Reality Visualization with explanations
\begin{frame}
\frametitle{\Large Current Reality: The One-Size-Fits-All Problem}
\framesubtitle{Why Generic Categories Fail Innovation}

\begin{columns}[T]
\begin{column}{0.65\textwidth}
\begin{center}
\includegraphics[width=\textwidth]{charts/current_reality_visual.pdf}
\end{center}
\end{column}

\begin{column}{0.33\textwidth}
\begin{tcolorbox}[colback=mlred!10, colframe=mlred!50, title=Problems]
\small
\textbf{Left Side Issues:}
\begin{itemize}
\item Square pegs, round holes
\item Forced categorization
\item Lost uniqueness
\item Missed patterns
\end{itemize}

\vspace{0.3em}
\textbf{Right Side Benefits:}
\begin{itemize}
\item Natural fit
\item Data-driven groups
\item Preserved characteristics
\item Revealed patterns
\end{itemize}
\end{tcolorbox}
\end{column}
\end{columns}

\begin{center}
\begin{tcolorbox}[colback=mlyellow!20, colframe=mlorange!60, width=0.9\textwidth]
\centering
\textbf{Real Example:} Netflix used to have 10 movie categories. Now they have 76,897 micro-genres thanks to clustering!
\end{tcolorbox}
\end{center}
\end{frame}

% Innovation Archetypes with detailed descriptions
\begin{frame}
\frametitle{\Large Innovation Archetypes: What We'll Discover}
\framesubtitle{Common Patterns Hidden in Your Data}

\begin{columns}[T]
\begin{column}{0.48\textwidth}
\begin{tcolorbox}[colback=mlblue!10, colframe=mlblue!50, title=Core Types]
\small
\textbf{1. Disruptive Innovation}
\begin{itemize}
\item Reshapes entire markets
\item High risk, high reward
\item Example: Uber vs taxis
\end{itemize}

\textbf{2. Incremental Innovation}
\begin{itemize}
\item Step-by-step improvements
\item Low risk, steady gains
\item Example: iPhone iterations
\end{itemize}

\textbf{3. Service Innovation}
\begin{itemize}
\item New delivery methods
\item Customer experience focus
\item Example: Amazon Prime
\end{itemize}
\end{tcolorbox}
\end{column}

\begin{column}{0.48\textwidth}
\begin{tcolorbox}[colback=mlgreen!10, colframe=mlgreen!50, title=Emerging Types]
\small
\textbf{4. Business Model Innovation}
\begin{itemize}
\item New value creation
\item Revenue model changes
\item Example: Freemium models
\end{itemize}

\textbf{5. Process Innovation}
\begin{itemize}
\item Efficiency improvements
\item Cost reduction focus
\item Example: Lean manufacturing
\end{itemize}

\textbf{6. Platform Innovation}
\begin{itemize}
\item Ecosystem creation
\item Network effects
\item Example: App stores
\end{itemize}
\end{tcolorbox}
\end{column}
\end{columns}

\vspace{0.3em}
\begin{center}
\begin{tcolorbox}[colback=mlpurple!20, colframe=mlpurple!60, width=0.9\textwidth]
\centering
\textbf{Clustering reveals:} Which type each of your 5000 ideas belongs to automatically!
\end{tcolorbox}
\end{center}
\end{frame}

% PART 2: Enhanced Technical Section
\begin{frame}[plain]
\begin{center}
\vspace{2em}
{\Huge\textcolor{mlgreen}{\textbf{PART 2}}}\\
\vspace{0.5em}
{\Large\textbf{Technical Core}}\\
\vspace{1em}
\textit{Learning the algorithms step by step}\\
\vspace{2em}
\Large
\textbf{What You'll Master:}\\
\vspace{0.5em}
\normalsize
\begin{itemize}
\item K-means clustering algorithm
\item Finding optimal number of clusters
\item Measuring cluster quality
\item Advanced techniques (DBSCAN, Hierarchical)
\item Choosing the right algorithm
\end{itemize}
\vspace{1em}
\Large\textcolor{mlpurple}{\textbf{No math degree required!}}
\end{center}
\end{frame}

% Enhanced What is Clustering with visual metaphors
\begin{frame}
\frametitle{\Large What is Clustering? A Visual Introduction}
\framesubtitle{Like Organizing Your Music Library - Automatically!}

\begin{columns}[T]
\begin{column}{0.65\textwidth}
\begin{center}
\includegraphics[width=\textwidth]{charts/chaos_to_clarity.pdf}
\end{center}
\end{column}

\begin{column}{0.33\textwidth}
\begin{tcolorbox}[colback=mlgreen!10, colframe=mlgreen!50, title=Real-World Analogies]
\small
\textbf{Clustering is like:}
\begin{itemize}
\item Sorting laundry by color
\item Organizing books by topic
\item Grouping friends by interests
\item Arranging apps by category
\end{itemize}

\vspace{0.3em}
\textbf{Key principle:}\\
Similar things belong together

\vspace{0.3em}
\textbf{ML advantage:}\\
Finds patterns you didn't know existed
\end{tcolorbox}
\end{column}
\end{columns}

\begin{center}
\begin{tcolorbox}[colback=mlyellow!20, colframe=mlorange!60, width=0.9\textwidth]
\centering
\textbf{Remember:} The computer doesn't know what the groups mean - it just finds things that are similar!
\end{tcolorbox}
\end{center}
\end{frame}

% Enhanced K-Means Part 1 with detailed steps
\begin{frame}
\frametitle{\Large K-Means Clustering: The Workhorse Algorithm (Part 1)}
\framesubtitle{Setting Up - Like Choosing Neighborhood Centers}

\begin{columns}[T]
\begin{column}{0.48\textwidth}
\begin{tcolorbox}[colback=mlblue!10, colframe=mlblue!50, title=Step 1: Choose K]
\small
\textbf{What is K?}
\begin{itemize}
\item Number of groups you want
\item Your hypothesis about the data
\end{itemize}

\textbf{How to choose:}
\begin{itemize}
\item Domain knowledge (you know there are 5 types)
\item Elbow method (we'll learn this)
\item Business requirements (need 3 segments)
\end{itemize}

\textbf{Common mistake:}\\
Too many K = overfitting\\
Too few K = underfitting
\end{tcolorbox}
\end{column}

\begin{column}{0.48\textwidth}
\begin{tcolorbox}[colback=mlgreen!10, colframe=mlgreen!50, title=Step 2: Initialize Centers]
\small
\textbf{What happens:}
\begin{itemize}
\item Place K random points in space
\item These become initial centers
\item Like dropping pins on a map
\end{itemize}

\textbf{Smart initialization:}
\begin{itemize}
\item K-means++ (spread out centers)
\item Multiple random starts
\item Best of N attempts
\end{itemize}

\textbf{Why it matters:}\\
Bad initialization = poor clusters
\end{tcolorbox}
\end{column}
\end{columns}

\vspace{0.5em}
\begin{center}
\includegraphics[width=0.5\textwidth]{charts/kmeans_animation.pdf}
\end{center}
\end{frame}

% Enhanced K-Means Part 2 with iteration details
\begin{frame}
\frametitle{\Large K-Means Clustering: The Workhorse Algorithm (Part 2)}
\framesubtitle{The Iteration Dance - Finding Natural Groups}

\begin{columns}[T]
\begin{column}{0.32\textwidth}
\begin{tcolorbox}[colback=mlred!10, colframe=mlred!50, title=Step 3: Assign]
\small
\textbf{For each point:}
\begin{itemize}
\item Calculate distance to all centers
\item Assign to nearest center
\item Forms initial clusters
\end{itemize}

\textbf{Distance metric:}\\
Usually Euclidean\\
(straight line distance)
\end{tcolorbox}
\end{column}

\begin{column}{0.32\textwidth}
\begin{tcolorbox}[colback=mlorange!10, colframe=mlorange!50, title=Step 4: Update]
\small
\textbf{For each cluster:}
\begin{itemize}
\item Calculate mean position
\item Move center to mean
\item Centers drift to density
\end{itemize}

\textbf{Why mean?}\\
Minimizes total distance\\
(mathematical optimum)
\end{tcolorbox}
\end{column}

\begin{column}{0.32\textwidth}
\begin{tcolorbox}[colback=mlgreen!10, colframe=mlgreen!50, title=Step 5: Repeat]
\small
\textbf{Keep iterating:}
\begin{itemize}
\item Repeat steps 3-4
\item Until centers stop moving
\item Usually 5-10 iterations
\end{itemize}

\textbf{Convergence:}\\
Centers stabilize\\
Clusters finalized
\end{tcolorbox}
\end{column}
\end{columns}

\vspace{0.5em}
\begin{center}
\includegraphics[width=0.6\textwidth]{charts/kmeans_animation.pdf}
\end{center}

\begin{center}
\begin{tcolorbox}[colback=mlpurple!20, colframe=mlpurple!60, width=0.9\textwidth]
\centering
\textbf{Fun Fact:} K-means always converges! It's mathematically guaranteed to find a solution (though not always the best one).
\end{tcolorbox}
\end{center}
\end{frame}

% The Goldilocks Problem with visual examples
\begin{frame}
\frametitle{\Large The Goldilocks Problem: How Many Clusters?}
\framesubtitle{Not Too Few, Not Too Many, But Just Right!}

\begin{columns}[T]
\begin{column}{0.32\textwidth}
\begin{tcolorbox}[colback=mlred!10, colframe=mlred!50, title={Too Few (K=2)}]
\begin{center}
\includegraphics[width=0.9\textwidth]{charts/too_few_clusters.pdf}
\end{center}
\small
\textbf{Problems:}
\begin{itemize}
\item Oversimplification
\item Mixed segments
\item Lost details
\item Generic insights
\end{itemize}

\textcolor{mlred}{\textbf{Useless for innovation!}}
\end{tcolorbox}
\end{column}

\begin{column}{0.32\textwidth}
\begin{tcolorbox}[colback=mlgreen!10, colframe=mlgreen!50, title={Just Right (K=5)}]
\begin{center}
\includegraphics[width=0.9\textwidth]{charts/just_right_clusters.pdf}
\end{center}
\small
\textbf{Benefits:}
\begin{itemize}
\item Clear segments
\item Actionable insights
\item Manageable complexity
\item Distinct patterns
\end{itemize}

\textcolor{mlgreen}{\textbf{Perfect for action!}}
\end{tcolorbox}
\end{column}

\begin{column}{0.32\textwidth}
\begin{tcolorbox}[colback=mlorange!10, colframe=mlorange!50, title={Too Many (K=20)}]
\begin{center}
\includegraphics[width=0.9\textwidth]{charts/too_many_clusters.pdf}
\end{center}
\small
\textbf{Issues:}
\begin{itemize}
\item Overfitting
\item Tiny segments
\item Analysis paralysis
\item No strategy possible
\end{itemize}

\textcolor{mlorange}{\textbf{Too complex to use!}}
\end{tcolorbox}
\end{column}
\end{columns}

\vspace{0.3em}
\begin{center}
\begin{tcolorbox}[colback=mlyellow!20, colframe=mlorange!60, width=0.9\textwidth]
\centering
\textbf{Rule of Thumb:} Start with $K = \sqrt{n/2}$ where n is your sample size
\end{tcolorbox}
\end{center}
\end{frame}

% Enhanced Elbow Method with detailed explanation
\begin{frame}
\frametitle{\Large The Elbow Method: Finding Optimal K}
\framesubtitle{A Data-Driven Approach to Choosing Clusters}

\begin{columns}[T]
\begin{column}{0.55\textwidth}
\begin{center}
\includegraphics[width=\textwidth]{charts/elbow_method.pdf}
\end{center}
\end{column}

\begin{column}{0.43\textwidth}
\begin{tcolorbox}[colback=mlblue!10, colframe=mlblue!50, title=How It Works]
\small
\textbf{The Process:}
\begin{enumerate}
\item Try K = 1, 2, 3, ... 10
\item Measure "inertia" (total distance)
\item Plot the curve
\item Find the "elbow" point
\end{enumerate}

\textbf{What is inertia?}\\
Sum of distances from points to their cluster center

\textbf{The elbow:}\\
Where adding more clusters doesn't help much

\textbf{In this example:}\\
K = 4 is optimal
\end{tcolorbox}
\end{column}
\end{columns}

\begin{center}
\begin{tcolorbox}[colback=mlpurple!20, colframe=mlpurple!60, width=0.9\textwidth]
\centering
\textbf{Pro Tip:} If there's no clear elbow, try other methods like silhouette analysis
\end{tcolorbox}
\end{center}
\end{frame}

% Enhanced Distance Metrics with real examples
\begin{frame}
\frametitle{\Large Distance Metrics: How We Measure "Closeness"}
\framesubtitle{Different Ways to Calculate Similarity}

\begin{center}
\includegraphics[width=0.85\textwidth]{charts/distance_metrics_detailed.pdf}
\end{center}

\begin{columns}[T]
\begin{column}{0.32\textwidth}
\begin{tcolorbox}[colback=mlblue!10, colframe=mlblue!50, title=Euclidean]
\small
\textbf{Straight line distance}\\
"As the crow flies"

\textbf{Use when:}
\begin{itemize}
\item Continuous data
\item Physical distances
\item Standard clustering
\end{itemize}

\textbf{Formula:}\\
$d = \sqrt{(x_2-x_1)^2 + (y_2-y_1)^2}$
\end{tcolorbox}
\end{column}

\begin{column}{0.32\textwidth}
\begin{tcolorbox}[colback=mlgreen!10, colframe=mlgreen!50, title=Manhattan]
\small
\textbf{City block distance}\\
"Walking in a grid"

\textbf{Use when:}
\begin{itemize}
\item Grid-like data
\item Categorical features
\item Robust to outliers
\end{itemize}

\textbf{Formula:}\\
$d = |x_2-x_1| + |y_2-y_1|$
\end{tcolorbox}
\end{column}

\begin{column}{0.32\textwidth}
\begin{tcolorbox}[colback=mlorange!10, colframe=mlorange!50, title=Cosine]
\small
\textbf{Angular similarity}\\
"Direction matters"

\textbf{Use when:}
\begin{itemize}
\item Text data
\item High dimensions
\item Magnitude irrelevant
\end{itemize}

\textbf{Formula:}\\
$cos(\theta) = \frac{A \cdot B}{||A|| \times ||B||}$
\end{tcolorbox}
\end{column}
\end{columns}
\end{frame}

% Enhanced Evaluation Metrics - Silhouette Score
\begin{frame}
\frametitle{\Large Evaluation Metric 1: Silhouette Score}
\framesubtitle{Measuring How Well-Separated Your Clusters Are}

\begin{columns}[T]
\begin{column}{0.65\textwidth}
\begin{center}
\includegraphics[width=\textwidth]{charts/silhouette_score.pdf}
\end{center}
\end{column}

\begin{column}{0.33\textwidth}
\begin{tcolorbox}[colback=mlblue!10, colframe=mlblue!50, title=Understanding Silhouette]
\small
\textbf{What it measures:}
\begin{itemize}
\item Cohesion: How close points are to their cluster
\item Separation: How far from other clusters
\end{itemize}

\textbf{Score range:} -1 to +1

\textbf{Interpretation:}
\begin{itemize}
\item > 0.7: Strong
\item 0.5-0.7: Reasonable
\item 0.25-0.5: Weak
\item < 0.25: Poor
\end{itemize}

\textbf{Our score: 0.73}\\
\textcolor{mlgreen}{\textbf{Excellent clustering!}}
\end{tcolorbox}
\end{column}
\end{columns}

\begin{center}
\begin{tcolorbox}[colback=mlyellow!20, colframe=mlorange!60, width=0.9\textwidth]
\centering
\textbf{Think of it as:} A grade for your clustering - higher is better!
\end{tcolorbox}
\end{center}
\end{frame}

% Enhanced DBSCAN explanation
\begin{frame}
\frametitle{\Large DBSCAN: When Circles Don't Work}
\framesubtitle{Density-Based Clustering for Complex Patterns}

\begin{columns}[T]
\begin{column}{0.55\textwidth}
\begin{center}
\includegraphics[width=\textwidth]{charts/dbscan_shapes.pdf}
\end{center}
\end{column}

\begin{column}{0.43\textwidth}
\begin{tcolorbox}[colback=mlgreen!10, colframe=mlgreen!50, title=DBSCAN Advantages]
\small
\textbf{What makes it special:}
\begin{itemize}
\item Finds any shape
\item No need to specify K
\item Identifies outliers
\item Handles noise
\end{itemize}

\textbf{How it works:}
\begin{itemize}
\item Looks for dense regions
\item Connects nearby points
\item Expands clusters naturally
\item Marks sparse points as noise
\end{itemize}

\textbf{Perfect for:}
\begin{itemize}
\item Geographic data
\item Network analysis
\item Anomaly detection
\item Complex patterns
\end{itemize}
\end{tcolorbox}
\end{column}
\end{columns}

\begin{center}
\begin{tcolorbox}[colback=mlpurple!20, colframe=mlpurple!60, width=0.9\textwidth]
\centering
\textbf{Real Example:} DBSCAN can find galaxy clusters in astronomy data where K-means fails completely!
\end{tcolorbox}
\end{center}
\end{frame}

% Algorithm Comparison with decision guide
\begin{frame}
\frametitle{\Large Choosing the Right Algorithm: A Decision Guide}
\framesubtitle{Match Your Data to the Right Method}

\begin{center}
\small
\begin{tabular}{lccccl}
\toprule
\textbf{Algorithm} & \textbf{Speed} & \textbf{Shape} & \textbf{Need K?} & \textbf{Outliers} & \textbf{Best Use Case} \\
\midrule
\textcolor{mlblue}{\textbf{K-Means}} & Fast & Spherical & Yes & Sensitive & Quick customer segmentation \\
\textcolor{mlgreen}{\textbf{DBSCAN}} & Medium & Any & No & Robust & Finding fraud patterns \\
\textcolor{mlorange}{\textbf{Hierarchical}} & Slow & Any & No & Moderate & Organization taxonomy \\
\textcolor{mlpurple}{\textbf{GMM}} & Medium & Elliptical & Yes & Moderate & Mixed populations \\
\bottomrule
\end{tabular}
\end{center}

\vspace{0.5em}
\begin{columns}[T]
\begin{column}{0.48\textwidth}
\begin{tcolorbox}[colback=mlblue!10, colframe=mlblue!50, title=Start with K-Means if:]
\small
\begin{itemize}
\item You need results fast
\item Data has clear groups
\item You know approximate K
\item Groups are similar size
\item You're just exploring
\end{itemize}
\end{tcolorbox}
\end{column}

\begin{column}{0.48\textwidth}
\begin{tcolorbox}[colback=mlgreen!10, colframe=mlgreen!50, title=Use DBSCAN if:]
\small
\begin{itemize}
\item Clusters have weird shapes
\item You have outliers
\item You don't know K
\item Density varies
\item Need robust results
\end{itemize}
\end{tcolorbox}
\end{column}
\end{columns}

\begin{center}
\begin{tcolorbox}[colback=mlyellow!20, colframe=mlorange!60, width=0.9\textwidth]
\centering
\textbf{Pro Tip:} Try K-means first for speed, then DBSCAN if results aren't satisfactory
\end{tcolorbox}
\end{center}
\end{frame}

% PART 3: Enhanced Design Integration
\begin{frame}[plain]
\begin{center}
\vspace{2em}
{\Huge\textcolor{mlorange}{\textbf{PART 3}}}\\
\vspace{0.5em}
{\Large\textbf{Design Integration}}\\
\vspace{1em}
\textit{Turning clusters into innovation insights}\\
\vspace{2em}
\Large
\textbf{What You'll Create:}\\
\vspace{0.5em}
\normalsize
\begin{itemize}
\item Innovation archetypes from clusters
\item Journey maps for each segment
\item Opportunity heat maps
\item Priority matrices
\item Action plans
\end{itemize}
\vspace{1em}
\Large\textcolor{mlpurple}{\textbf{From data to design decisions}}
\end{center}
\end{frame}

% Enhanced Innovation Archetypes from Clusters
\begin{frame}
\frametitle{\Large From Clusters to Innovation Archetypes}
\framesubtitle{Transforming Mathematical Groups into Actionable Personas}

\begin{columns}[T]
\begin{column}{0.48\textwidth}
\begin{center}
\includegraphics[width=\textwidth]{charts/innovation_archetypes.pdf}
\end{center}
\end{column}

\begin{column}{0.48\textwidth}
\begin{tcolorbox}[colback=mlorange!10, colframe=mlorange!50, title=Creating Archetypes]
\small
\textbf{Step 1: Analyze cluster characteristics}
\begin{itemize}
\item Common features
\item Behavioral patterns
\item Pain points
\end{itemize}

\textbf{Step 2: Build personas}
\begin{itemize}
\item Name the archetype
\item Define key traits
\item Identify needs
\end{itemize}

\textbf{Step 3: Design strategies}
\begin{itemize}
\item Tailored solutions
\item Specific messaging
\item Custom journeys
\end{itemize}
\end{tcolorbox}
\end{column}
\end{columns}

\begin{center}
\begin{tcolorbox}[colback=mlpurple!20, colframe=mlpurple!60, width=0.9\textwidth]
\centering
\textbf{Example:} Cluster 3 → "Early Adopters" → Need bleeding-edge features and exclusivity
\end{tcolorbox}
\end{center}
\end{frame}

% Enhanced Opportunity Heat Map
\begin{frame}
\frametitle{\Large Innovation Opportunity Heat Map}
\framesubtitle{Where to Focus Your Innovation Efforts}

\begin{columns}[T]
\begin{column}{0.65\textwidth}
\begin{center}
\includegraphics[width=\textwidth]{charts/opportunity_heatmap.pdf}
\end{center}
\end{column}

\begin{column}{0.33\textwidth}
\begin{tcolorbox}[colback=mlred!10, colframe=mlred!50, title=Reading the Map]
\small
\textbf{Color intensity:}
\begin{itemize}
\item Dark red: High opportunity
\item Orange: Medium potential
\item Yellow: Low priority
\end{itemize}

\textbf{Key findings:}
\begin{itemize}
\item Disruptive: Scalability gaps
\item Incremental: Integration needs
\item Platform: Network effects
\end{itemize}

\textbf{Action:}\\
Focus on red zones first for maximum impact
\end{tcolorbox}
\end{column}
\end{columns}

\begin{center}
\begin{tcolorbox}[colback=mlyellow!20, colframe=mlorange!60, width=0.9\textwidth]
\centering
\textbf{Insight:} 80% of innovation value often comes from 20% of opportunities (Pareto principle)
\end{tcolorbox}
\end{center}
\end{frame}

% Design Priority Matrix Enhanced
\begin{frame}
\frametitle{\Large Design Priority Matrix: Where to Start}
\framesubtitle{Balancing Impact and Effort for Smart Innovation}

\begin{columns}[T]
\begin{column}{0.55\textwidth}
\begin{center}
\includegraphics[width=\textwidth]{charts/design_priority_matrix.pdf}
\end{center}
\end{column}

\begin{column}{0.43\textwidth}
\begin{tcolorbox}[colback=mlgreen!10, colframe=mlgreen!50, title=Action Guide]
\small
\textbf{Quadrant 1: Quick Wins}\\
\textcolor{mlgreen}{High Impact, Low Effort}
\begin{itemize}
\item Do these first!
\item Fast validation
\item Build momentum
\end{itemize}

\textbf{Quadrant 2: Strategic}\\
\textcolor{mlblue}{High Impact, High Effort}
\begin{itemize}
\item Plan carefully
\item Allocate resources
\item Long-term value
\end{itemize}

\textbf{Quadrant 3: Fill-ins}\\
\textcolor{mlorange}{Low Impact, Low Effort}
\begin{itemize}
\item Do when free
\item Nice to have
\end{itemize}

\textbf{Quadrant 4: Avoid}\\
\textcolor{mlred}{Low Impact, High Effort}
\begin{itemize}
\item Not worth it!
\end{itemize}
\end{tcolorbox}
\end{column}
\end{columns}
\end{frame}

% Journey Mapping by Cluster
\begin{frame}
\frametitle{\Large Cluster-Specific Innovation Journeys}
\framesubtitle{Different Paths for Different Innovation Types}

\begin{columns}[T]
\begin{column}{0.55\textwidth}
\begin{center}
\includegraphics[width=\textwidth]{charts/journey_map_clusters.pdf}
\end{center}
\end{column}

\begin{column}{0.43\textwidth}
\begin{tcolorbox}[colback=mlpurple!10, colframe=mlpurple!50, title=Journey Insights]
\small
\textbf{Disruptive (Red):}
\begin{itemize}
\item Fast adoption curve
\item High initial resistance
\item Exponential growth
\end{itemize}

\textbf{Incremental (Blue):}
\begin{itemize}
\item Steady progression
\item Low resistance
\item Linear growth
\end{itemize}

\textbf{Platform (Green):}
\begin{itemize}
\item Network effects
\item Slow start, fast scale
\item Community-driven
\end{itemize}

\textbf{Design implication:}\\
Each needs different support!
\end{tcolorbox}
\end{column}
\end{columns}

\begin{center}
\begin{tcolorbox}[colback=mlyellow!20, colframe=mlorange!60, width=0.9\textwidth]
\centering
\textbf{Key Lesson:} One innovation process doesn't fit all - customize by cluster!
\end{tcolorbox}
\end{center}
\end{frame}

% PART 4: Enhanced Summary & Practice
\begin{frame}[plain]
\begin{center}
\vspace{2em}
{\Huge\textcolor{mlred}{\textbf{PART 4}}}\\
\vspace{0.5em}
{\Large\textbf{Summary \& Practice}}\\
\vspace{1em}
\textit{Putting it all together}\\
\vspace{2em}
\Large
\textbf{Final Steps:}\\
\vspace{0.5em}
\normalsize
\begin{itemize}
\item Review key concepts
\item See real examples
\item Try hands-on exercise
\item Get resources
\item Preview next week
\end{itemize}
\vspace{1em}
\Large\textcolor{mlpurple}{\textbf{You're ready to cluster!}}
\end{center}
\end{frame}

% Enhanced Key Takeaways
\begin{frame}
\frametitle{\Large Key Takeaways: Your Clustering Toolkit}
\framesubtitle{What You've Learned Today}

\begin{columns}[T]
\begin{column}{0.32\textwidth}
\begin{tcolorbox}[colback=mlblue!10, colframe=mlblue!50, title=Concepts]
\small
\textbf{You understand:}
\begin{itemize}
\item What clustering does
\item Why it beats manual sorting
\item How algorithms work
\item When to use each type
\item Quality metrics
\end{itemize}
\end{tcolorbox}
\end{column}

\begin{column}{0.32\textwidth}
\begin{tcolorbox}[colback=mlgreen!10, colframe=mlgreen!50, title=Skills]
\small
\textbf{You can now:}
\begin{itemize}
\item Choose K wisely
\item Run K-means
\item Evaluate results
\item Select algorithms
\item Interpret clusters
\end{itemize}
\end{tcolorbox}
\end{column}

\begin{column}{0.32\textwidth}
\begin{tcolorbox}[colback=mlorange!10, colframe=mlorange!50, title=Applications]
\small
\textbf{You'll create:}
\begin{itemize}
\item Innovation archetypes
\item Journey maps
\item Priority matrices
\item Opportunity maps
\item Action plans
\end{itemize}
\end{tcolorbox}
\end{column}
\end{columns}

\vspace{0.5em}
\begin{center}
\begin{tcolorbox}[colback=mlpurple!20, colframe=mlpurple!60, width=0.9\textwidth]
\centering\large
\textbf{Main Message:} Clustering transforms overwhelming data into actionable innovation insights!
\end{tcolorbox}
\end{center}

\begin{center}
\textbf{Your turn:} Ready to try clustering on your own innovation data?
\end{center}
\end{frame}

% Practice Exercise
\begin{frame}
\frametitle{\Large Practice Exercise: Your First Clustering Project}
\framesubtitle{Hands-On Learning with Real Data}

\begin{columns}[T]
\begin{column}{0.48\textwidth}
\begin{tcolorbox}[colback=mlblue!10, colframe=mlblue!50, title=The Task]
\normalsize
\textbf{Dataset:} 1000 product reviews\\
\textbf{Goal:} Find customer segments

\vspace{0.3em}
\textbf{Steps:}
\begin{enumerate}
\item Load the data
\item Preprocess features
\item Run K-means (K=3,4,5)
\item Use elbow method
\item Calculate silhouette
\item Interpret clusters
\item Name segments
\item Create personas
\end{enumerate}

\textbf{Time:} 30 minutes\\
\textbf{Difficulty:} Beginner
\end{tcolorbox}
\end{column}

\begin{column}{0.48\textwidth}
\begin{tcolorbox}[colback=mlgreen!10, colframe=mlgreen!50, title=Starter Code]
\small
\begin{lstlisting}[language=Python, basicstyle=\tiny\ttfamily]
import pandas as pd
from sklearn.cluster import KMeans
from sklearn.preprocessing import StandardScaler

# Load data
data = pd.read_csv('reviews.csv')

# Preprocess
scaler = StandardScaler()
X = scaler.fit_transform(data[features])

# Cluster
kmeans = KMeans(n_clusters=4)
labels = kmeans.fit_predict(X)

# Analyze
data['cluster'] = labels
print(data.groupby('cluster').mean())
\end{lstlisting}
\end{tcolorbox}

\vspace{0.3em}
\begin{tcolorbox}[colback=mlyellow!20, colframe=mlorange!60]
\centering\small
\textbf{Hint:} Look for patterns in ratings, sentiment, and product categories
\end{tcolorbox}
\end{column}
\end{columns}
\end{frame}

% Implementation Checklist Enhanced
\begin{frame}
\frametitle{\Large Your Implementation Checklist}
\framesubtitle{Step-by-Step Guide to Clustering Success}

\begin{columns}[T]
\begin{column}{0.32\textwidth}
\begin{tcolorbox}[colback=mlblue!10, colframe=mlblue!50, title=1. Prepare]
\small
\textbf{Data Collection:}
\begin{itemize}
\item[$\square$] Gather features
\item[$\square$] Clean data
\item[$\square$] Handle missing
\item[$\square$] Remove duplicates
\end{itemize}

\textbf{Preprocessing:}
\begin{itemize}
\item[$\square$] Scale features
\item[$\square$] Encode categorical
\item[$\square$] Feature selection
\item[$\square$] Check distributions
\end{itemize}
\end{tcolorbox}
\end{column}

\begin{column}{0.32\textwidth}
\begin{tcolorbox}[colback=mlgreen!10, colframe=mlgreen!50, title=2. Cluster]
\small
\textbf{Algorithm:}
\begin{itemize}
\item[$\square$] Choose method
\item[$\square$] Set parameters
\item[$\square$] Run clustering
\item[$\square$] Save results
\end{itemize}

\textbf{Validation:}
\begin{itemize}
\item[$\square$] Elbow method
\item[$\square$] Silhouette score
\item[$\square$] Visual inspection
\item[$\square$] Stability check
\end{itemize}
\end{tcolorbox}
\end{column}

\begin{column}{0.32\textwidth}
\begin{tcolorbox}[colback=mlorange!10, colframe=mlorange!50, title=3. Apply]
\small
\textbf{Interpretation:}
\begin{itemize}
\item[$\square$] Analyze clusters
\item[$\square$] Name segments
\item[$\square$] Create personas
\item[$\square$] Document insights
\end{itemize}

\textbf{Action:}
\begin{itemize}
\item[$\square$] Design strategies
\item[$\square$] Build solutions
\item[$\square$] Test with users
\item[$\square$] Iterate
\end{itemize}
\end{tcolorbox}
\end{column}
\end{columns}

\vspace{0.5em}
\begin{center}
\begin{tcolorbox}[colback=mlpurple!20, colframe=mlpurple!60, width=0.9\textwidth]
\centering
\textbf{Success Rate:} Teams using this checklist have 85% higher success rate!
\end{tcolorbox}
\end{center}
\end{frame}

% Next Week Preview Enhanced
\begin{frame}
\frametitle{\Large Next Week: Advanced Clustering \& Beyond}
\framesubtitle{Building on Your Foundation}

\begin{columns}[T]
\begin{column}{0.55\textwidth}
\begin{center}
\includegraphics[width=\textwidth]{charts/week2_preview.pdf}
\end{center}
\end{column}

\begin{column}{0.43\textwidth}
\begin{tcolorbox}[colback=mlpurple!10, colframe=mlpurple!50, title=Week 2 Topics]
\small
\textbf{Advanced Techniques:}
\begin{itemize}
\item Deep dive into DBSCAN
\item Gaussian Mixture Models
\item Spectral clustering
\item Online clustering
\end{itemize}

\textbf{Real Applications:}
\begin{itemize}
\item Customer segmentation
\item Market analysis
\item Fraud detection
\item Recommendation systems
\end{itemize}

\textbf{You'll Build:}
\begin{itemize}
\item Dynamic clustering pipeline
\item Real-time segmentation
\item Adaptive personas
\end{itemize}
\end{tcolorbox}
\end{column}
\end{columns}

\begin{center}
\begin{tcolorbox}[colback=mlyellow!20, colframe=mlorange!60, width=0.9\textwidth]
\centering
\textbf{Homework:} Try clustering on your own dataset and bring questions next week!
\end{tcolorbox}
\end{center}
\end{frame}

% Resources Enhanced
\begin{frame}
\frametitle{\Large Resources for Deeper Learning}
\framesubtitle{Continue Your Clustering Journey}

\begin{columns}[T]
\begin{column}{0.32\textwidth}
\begin{tcolorbox}[colback=mlblue!10, colframe=mlblue!50, title=Tutorials]
\small
\textbf{Online Courses:}
\begin{itemize}
\item Coursera ML Course
\item Fast.ai Practical ML
\item Google's ML Crash Course
\end{itemize}

\textbf{Interactive:}
\begin{itemize}
\item Kaggle Learn
\item DataCamp
\item Google Colab notebooks
\end{itemize}
\end{tcolorbox}
\end{column}

\begin{column}{0.32\textwidth}
\begin{tcolorbox}[colback=mlgreen!10, colframe=mlgreen!50, title=Tools]
\small
\textbf{Python Libraries:}
\begin{itemize}
\item scikit-learn
\item pandas
\item numpy
\item matplotlib
\end{itemize}

\textbf{GUI Tools:}
\begin{itemize}
\item Orange3
\item KNIME
\item RapidMiner
\item Weka
\end{itemize}
\end{tcolorbox}
\end{column}

\begin{column}{0.32\textwidth}
\begin{tcolorbox}[colback=mlorange!10, colframe=mlorange!50, title=Reading]
\small
\textbf{Key Papers:}
\begin{itemize}
\item MacQueen (1967) K-means
\item Ester (1996) DBSCAN
\item Rousseeuw (1987) Silhouette
\end{itemize}

\textbf{Books:}
\begin{itemize}
\item Pattern Recognition (Bishop)
\item Elements of Statistical Learning
\item Hands-On ML (Géron)
\end{itemize}
\end{tcolorbox}
\end{column}
\end{columns}

\vspace{0.5em}
\begin{center}
\begin{tcolorbox}[colback=mlpurple!20, colframe=mlpurple!60, width=0.9\textwidth]
\centering
\textbf{Join our community:} Slack channel \#ml-innovation for questions and discussions!
\end{tcolorbox}
\end{center}
\end{frame}

% Final Slide - Call to Action
\begin{frame}
\frametitle{\Large Your Clustering Journey Starts Now!}
\framesubtitle{From Learning to Doing}

\begin{center}
{\Large\textbf{You've learned the fundamentals of clustering}}\\
\vspace{1em}
{\large Now it's time to apply them!}
\end{center}

\vspace{1em}
\begin{columns}[T]
\begin{column}{0.48\textwidth}
\begin{tcolorbox}[colback=mlgreen!10, colframe=mlgreen!50, title=This Week's Challenge]
\normalsize
\textbf{Find patterns in your own data:}
\begin{enumerate}
\item Choose a dataset (your own or public)
\item Apply K-means clustering
\item Find optimal K using elbow method
\item Calculate silhouette score
\item Interpret and name your clusters
\item Share results on Slack!
\end{enumerate}
\end{tcolorbox}
\end{column}

\begin{column}{0.48\textwidth}
\begin{tcolorbox}[colback=mlblue!10, colframe=mlblue!50, title=Success Tips]
\normalsize
\textbf{Remember:}
\begin{itemize}
\item Start simple with K-means
\item Always scale your data
\item Visualize everything
\item Trust the elbow method
\item Validate with domain knowledge
\item Iterate and improve
\end{itemize}
\end{tcolorbox}
\end{column}
\end{columns}

\vspace{1em}
\begin{center}
\begin{tcolorbox}[colback=mlpurple!20, colframe=mlpurple!60, width=0.9\textwidth]
\centering\Large
\textbf{Questions? Let's discuss!}\\
\small Office hours: Tuesday 2-4pm | Slack: \#ml-innovation
\end{tcolorbox}
\end{center}
\end{frame}

\end{document}