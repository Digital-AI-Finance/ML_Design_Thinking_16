\documentclass[8pt,aspectratio=169]{beamer}
\usetheme{Madrid}
\usepackage{graphicx}
\usepackage{booktabs}
\usepackage{adjustbox}
\usepackage{multicol}
\usepackage{amsmath}
\usepackage{amssymb}
\usepackage{tcolorbox}
\usepackage{xcolor}
\usepackage{listings}

% Standard Template Colors (for Key Insight boxes)
\definecolor{mlblue}{RGB}{0,102,204}
\definecolor{mlpurple}{RGB}{51,51,178}
\definecolor{mllavender}{RGB}{173,173,224}
\definecolor{mllavender2}{RGB}{193,193,232}
\definecolor{mllavender3}{RGB}{204,204,235}
\definecolor{mllavender4}{RGB}{214,214,239}
\definecolor{mlorange}{RGB}{255, 127, 14}
\definecolor{mlgreen}{RGB}{44, 160, 44}
\definecolor{mlred}{RGB}{214, 39, 40}
\definecolor{mlgray}{RGB}{127, 127, 127}

% Nature Professional Theme (Week 7 custom)
\definecolor{ForestGreen}{RGB}{20,83,45}
\definecolor{Teal}{RGB}{13,148,136}
\definecolor{Amber}{RGB}{245,158,11}
\definecolor{Slate}{RGB}{71,85,105}
\definecolor{MintCream}{RGB}{240,253,244}
\definecolor{LightGreen}{RGB}{134,239,172}
\definecolor{DarkTeal}{RGB}{15,118,110}

% Additional grays
\definecolor{lightgray}{RGB}{240, 240, 240}
\definecolor{midgray}{RGB}{180, 180, 180}

% Apply Nature Professional to Madrid theme
\setbeamercolor{palette primary}{bg=LightGreen!30,fg=ForestGreen}
\setbeamercolor{palette secondary}{bg=LightGreen!20,fg=ForestGreen}
\setbeamercolor{palette tertiary}{bg=LightGreen,fg=white}
\setbeamercolor{palette quaternary}{bg=ForestGreen,fg=white}

\setbeamercolor{structure}{fg=ForestGreen}
\setbeamercolor{section in toc}{fg=ForestGreen}
\setbeamercolor{subsection in toc}{fg=Teal}
\setbeamercolor{title}{fg=ForestGreen}
\setbeamercolor{frametitle}{fg=ForestGreen,bg=LightGreen!20}
\setbeamercolor{block title}{bg=LightGreen!30,fg=ForestGreen}
\setbeamercolor{block body}{bg=MintCream,fg=black}

% Items with nature accent colors
\setbeamercolor{item}{fg=Amber}
\setbeamercolor{subitem}{fg=Teal}
\setbeamercolor{subsubitem}{fg=DarkTeal}

% Remove navigation symbols
\setbeamertemplate{navigation symbols}{}

% Clean itemize/enumerate
\setbeamertemplate{itemize items}[circle]
\setbeamertemplate{enumerate items}[default]

% Reduce margins for more content space
\setbeamersize{text margin left=5mm,text margin right=5mm}

% Custom footer
\setbeamertemplate{footline}{
  \leavevmode%
  \hbox{%
  \begin{beamercolorbox}[wd=.25\paperwidth,ht=2.25ex,dp=1ex,center]{author in head/foot}%
    \usebeamerfont{author in head/foot}Week 7
  \end{beamercolorbox}%
  \begin{beamercolorbox}[wd=.5\paperwidth,ht=2.25ex,dp=1ex,center]{title in head/foot}%
    \usebeamerfont{title in head/foot}Hidden Bias to Visible Fairness
  \end{beamercolorbox}%
  \begin{beamercolorbox}[wd=.25\paperwidth,ht=2.25ex,dp=1ex,right]{date in head/foot}%
    \usebeamerfont{date in head/foot}\insertframenumber{} / \inserttotalframenumber\hspace*{2ex}
  \end{beamercolorbox}}%
  \vskip0pt%
}

% Command for bottom annotation
\newcommand{\bottomnote}[1]{%
\vfill
\vspace{-2mm}
\textcolor{Slate}{\rule{\textwidth}{0.4pt}}
\vspace{1mm}
\footnotesize
\textcolor{Slate}{\textbf{#1}}
}

% Graphics path
\graphicspath{{charts/}}

% Title information
\title{\Large Machine Learning for Smarter Innovation}
\subtitle{Week 7: Responsible AI \& Ethical Innovation\\
From Hidden Bias to Visible Fairness}
\author{ML \& Design Thinking Course}
\date{BSc Level - 2025}

\begin{document}

% Title slide with plain style
\begin{frame}[plain]
\vspace{2cm}
\begin{center}
{\Huge \textcolor{ForestGreen}{\textbf{Hidden Bias}}}\\[0.2cm]
{\Huge \textcolor{ForestGreen}{\textbf{to}}}\\[0.2cm]
{\Huge \textcolor{ForestGreen}{\textbf{Visible Fairness}}}\\[1cm]
{\Large How Mathematics Reveals Invisible Discrimination}\\[1.5cm]
{\large Week 7: Machine Learning for Smarter Innovation}\\[0.3cm]
{\normalsize When Unmeasurable Harm Meets Mathematical Justice}\\[0.5cm]
{\small Making the Invisible Visible Through AI Fairness}
\end{center}
\end{frame}

% Table of contents with 4-act structure
\begin{frame}[t]{Today's Journey: From Hidden to Visible}
\Large\textbf{Four Acts of Discovery}
\normalsize

\vspace{0.5em}

\begin{enumerate}
\item \textbf{Act 1: The Hidden Harm} - Invisible bias, unmeasurable discrimination
\item \textbf{Act 2: First Measurements} - Metrics work... then impossibility reveals
\item \textbf{Act 3: Mathematical Fairness} - Geometric understanding, optimization
\item \textbf{Act 4: Production Systems} - Modern tools, ethical AI in practice
\end{enumerate}

\vspace{1em}

\begin{center}
\begin{tcolorbox}[colback=LightGreen!20, colframe=ForestGreen, width=0.8\textwidth]
\centering
\textbf{Unifying Theme:} MEASUREMENT transforms invisible discrimination into visible, solvable problems
\end{tcolorbox}
\end{center}

\bottomnote{By the end: You'll understand the mathematics of fairness and how to build ethical AI systems}
\end{frame}

% Include the four acts
% ACT 1: THE HIDDEN HARM (5 slides)
% Theme: Invisible bias = unmeasurable discrimination = systemic injustice

% Slide 1: The Invisible Discrimination Scenario
\begin{frame}[t]{The Invisible Discrimination: You Can't Fix What You Can't See}
\textbf{A real scenario that reveals the hidden harm:}

\vspace{0.3em}

\begin{columns}[T]
\column{0.48\textwidth}
\textcolor{ForestGreen}{\textbf{The Hidden Pattern}}

\small
**Bank loan system, 2024:**\\
10,000 applications processed

\vspace{0.3cm}
\textbf{Observable outcomes:}
\begin{itemize}
\item Group A: 7,500 approved (75\%)
\item Group B: 4,500 approved (45\%)
\item Overall: 60\% approval rate
\end{itemize}

\vspace{0.3cm}
\textcolor{Amber}{\textbf{The Question:}}\\
Is this discrimination?\\
How would you even know?

\vspace{0.3cm}
\textbf{Hidden factors:}
\begin{itemize}
\item Can't see: Intent, causation, counterfactuals
\item Can only see: Outcomes, rates, patterns
\item Qualification differences?
\item Historical bias?
\item Proxy variables?
\end{itemize}

\column{0.48\textwidth}
\textcolor{Teal}{\textbf{The Invisibility Problem}}

\small
\textbf{Why discrimination stays hidden:}

\vspace{0.3cm}
\textbf{1. No Ground Truth}
\begin{itemize}
\item Can't observe "fair" counterfactual
\item What WOULD have happened?
\item Intent is unobservable
\end{itemize}

\vspace{0.3cm}
\textbf{2. Aggregate Masks Disparities}
\begin{itemize}
\item 60\% overall looks reasonable
\item 30\% gap hidden in average
\item Simpson's paradox
\end{itemize}

\vspace{0.3cm}
\textbf{3. Proxy Variables Conceal}
\begin{itemize}
\item Zip code → Race (95\% correlation)
\item Name → Gender (98\% correlation)
\item School → Socioeconomic status
\end{itemize}

\vspace{0.3cm}
\textcolor{mlred}{\textbf{Real harm:}}\\
4,500 people denied opportunities\\
System appears "objective"\\
Discrimination is \textbf{invisible}
\end{columns}

\vspace{0.5em}
\begin{tcolorbox}[colback=mlblue!10, colframe=mlblue]
\textbf{Key Insight:} Invisible discrimination is unmeasurable discrimination - you can't fix what you can't see or quantify
\end{tcolorbox}

\vspace{0.5em}
\textbf{Key Question:} How do we make invisible bias visible enough to measure and fix?

\bottomnote{233 documented AI discrimination incidents in 2024 - \$10B+ settlements - all started invisible}
\end{frame}

% Slide 2: What IS Bias? (Built from zero with information theory)
\begin{frame}[t]{What IS Bias? Building the Concept from Information Theory}
\textbf{Defining bias mathematically (from zero knowledge):}

\vspace{0.3em}

\begin{columns}[T]
\column{0.48\textwidth}
\textcolor{ForestGreen}{\textbf{Human Analogy: Blind Auditions}}

\small
\textbf{Symphony orchestras, 1970s-1990s:}

\vspace{0.3cm}
Before blind auditions:
\begin{itemize}
\item 5\% women in orchestras
\item Judges could see candidates
\item Implicit bias affected decisions
\end{itemize}

\vspace{0.3cm}
After blind auditions:
\begin{itemize}
\item 40\% women in orchestras
\item Screen hides gender
\item Decisions based on skill only
\end{itemize}

\vspace{0.3cm}
\textcolor{Amber}{\textbf{Key observation:}}\\
Removing visibility of protected\\
attribute changed outcomes

\vspace{0.3cm}
\textcolor{Teal}{\textbf{This means:}}\\
Decision correlated with\\
irrelevant attribute = BIAS

\column{0.48\textwidth}
\textcolor{ForestGreen}{\textbf{Computer/Math Equivalent}}

\small
\textbf{Protected attribute} $A$: Race, gender, age, etc.\\
\textbf{Decision} $D$: Hire, approve loan, admit, etc.\\
\textbf{True qualification} $Y$: Actual merit/ability

\vspace{0.3cm}
\textcolor{Amber}{\textbf{Information Theory Definition:}}

Bias exists when decision carries\\
information about protected attribute:

$$\textcolor{Teal}{I(D; A) > 0}$$

Where $I$ = mutual information

\vspace{0.3cm}
\textbf{Expanded form:}
$$I(D; A) = H(D) - H(D|A)$$
$$= H(A) - H(A|D)$$

\vspace{0.3cm}
\textcolor{ForestGreen}{\textbf{Intuition:}}
\begin{itemize}
\item $H(D)$: Uncertainty in decisions
\item $H(D|A)$: Uncertainty after seeing group
\item Difference = information leaked
\item $I(D; A) = 0$ means independence
\item $I(D; A) > 0$ means bias
\end{itemize}
\end{columns}

\vspace{0.5em}
\begin{tcolorbox}[colback=mlblue!10, colframe=mlblue]
\textbf{Key Insight:} Bias is statistical dependence between decisions and protected attributes - measurable via mutual information
\end{tcolorbox}

\vspace{0.5em}
\textbf{Key Question:} If we can define bias with I(D; A), can we measure it in real systems?

\bottomnote{Mutual information I(D; A) quantifies "how much knowing A tells you about D" - core of fairness mathematics}
\end{frame}

% Slide 3: Why Bias is Hidden (Observability problem with Simpson's paradox)
\begin{frame}[t]{Why Bias Stays Hidden: The Observability Problem}
\textbf{Three reasons discrimination remains invisible:}

\vspace{0.3em}

\begin{columns}[T]
\column{0.31\textwidth}
\textcolor{ForestGreen}{\textbf{1. Counterfactuals}}

\small
\textbf{Can't directly observe:}
\begin{itemize}
\item What WOULD have happened
\item Alternative universe
\item Fair outcome for comparison
\end{itemize}

\vspace{0.3cm}
\textbf{Example:}\\
Person denied loan

Question: "Would they have\\
been approved if different race?"

\textcolor{mlred}{Impossible to know!}

\vspace{0.3cm}
\textbf{Mathematics:}\\
Need $P(D|A=a, X)$ and\\
$P(D|A=a', X)$ for same $X$

But can only observe one\\
$A$ value per person

\vspace{0.3cm}
\textcolor{Amber}{\textbf{Result:}}\\
Causal discrimination\\
stays hidden

\column{0.31\textwidth}
\textcolor{Teal}{\textbf{2. Aggregation}}

\small
\textbf{Simpson's Paradox:}

\vspace{0.3cm}
\textbf{Department A:}
\begin{itemize}
\item Men: 80\% admit
\item Women: 85\% admit
\item No bias!
\end{itemize}

\textbf{Department B:}
\begin{itemize}
\item Men: 60\% admit
\item Women: 65\% admit
\item No bias!
\end{itemize}

\vspace{0.3cm}
\textcolor{mlred}{\textbf{Combined:}}
\begin{itemize}
\item Men: 70\% admit
\item Women: 65\% admit
\item BIAS APPEARS!
\end{itemize}

\vspace{0.3cm}
\textbf{Why:}\\
Men apply to easier dept

\vspace{0.3cm}
\textcolor{Amber}{\textbf{Result:}}\\
Aggregation hides or\\
creates false patterns

\column{0.31\textwidth}
\textcolor{DarkTeal}{\textbf{3. Proxy Variables}}

\small
\textbf{Indirect discrimination:}

\vspace{0.3cm}
\textbf{High correlation:}
\begin{itemize}
\item Zip code → Race (95\%)
\item Name → Gender (98\%)
\item School → Class (92\%)
\end{itemize}

\vspace{0.3cm}
\textbf{Model never sees $A$}\\
but uses proxy $P$

\vspace{0.3cm}
\textbf{Mathematics:}
$$I(D; A|P) < I(D; A)$$

But still $I(D; A) > 0$\\
through indirect path

\vspace{0.3cm}
\textcolor{mlred}{\textbf{Example:}}\\
Remove "gender" from\\
hiring algorithm

Still biased via:
\begin{itemize}
\item Sports: football vs volleyball
\item Hobbies: different patterns
\item Language: subtle cues
\end{itemize}

\textcolor{Amber}{\textbf{Result:}}\\
Hidden in 1000+ features
\end{columns}

\vspace{0.5em}
\begin{tcolorbox}[colback=mlblue!10, colframe=mlblue]
\textbf{Key Insight:} Bias stays hidden through unobservable counterfactuals, aggregation paradoxes, and proxy variables
\end{tcolorbox}

\vspace{0.5em}
\textbf{Key Question:} If bias is so well-hidden, how can we possibly measure it at scale?

\bottomnote{Simpson's paradox shows bias can appear or disappear depending on aggregation level - no single "truth"}
\end{frame}

% Slide 4: The Measurement Challenge (Quantified with Shannon entropy)
\begin{frame}[t]{The Measurement Challenge: Capacity Overflow}
\textbf{Information-theoretic analysis of the measurement problem:}

\vspace{0.3em}

\begin{columns}[T]
\column{0.55\textwidth}
\textcolor{ForestGreen}{\textbf{The Combinatorial Explosion}}

\small
\textbf{Step 1: Count protected attributes}

Legally protected in US/EU:
\begin{itemize}
\item Race: 6 categories
\item Gender: 3+ categories
\item Age: 7 bins (decades)
\item Disability: 2 (yes/no)
\item Religion: 10+ categories
\item National origin: 195 countries
\end{itemize}

Just these 6: $6 \times 3 \times 7 \times 2 \times 10 \times 195$\\
= \textcolor{mlred}{\textbf{490,140 subgroups}}

\vspace{0.3cm}
\textbf{Step 2: Calculate entropy}

Shannon entropy of subgroups:\\
$H(\text{Subgroups}) = \log_2(490{,}140)$\\
$= 18.9$ bits of discrimination information

\vspace{0.3cm}
\textbf{Step 3: Intersectionality}

Add socioeconomic (5 levels):\\
$490{,}140 \times 5 = 2{,}450{,}700$ subgroups\\
$H = \log_2(2{,}450{,}700) = 21.2$ bits

\column{0.43\textwidth}
\textcolor{Teal}{\textbf{The Capacity Problem}}

\small
\textbf{Measurement bandwidth:}

\vspace{0.3cm}
Typical fairness audit:
\begin{itemize}
\item Sample size: 10,000
\item Disaggregate by: Race × Gender
\item Subgroups measured: 18
\item Capacity: $\log_2(18) = 4.2$ bits
\end{itemize}

\vspace{0.3cm}
\textcolor{mlred}{\textbf{Information loss:}}

$$\text{Loss} = H - B$$
$$= 21.2 - 4.2$$
$$= 17.0 \text{ bits UNMEASURED}$$

\vspace{0.3cm}
\textbf{Opportunity cost:}\\
$2^{17} = 131{,}072$ subgroups\\
with invisible discrimination

\vspace{0.3cm}
\textcolor{Amber}{\textbf{Result:}}
\begin{itemize}
\item 99.999\% of discrimination unmeasured
\item Subgroup harm stays hidden
\item Most vulnerable: smallest groups
\end{itemize}
\end{columns}

\vspace{0.5em}
\begin{tcolorbox}[colback=mlblue!10, colframe=mlblue]
\textbf{Key Insight:} Measurement capacity (4.2 bits) vastly insufficient for discrimination space (21.2 bits) - 17 bits lost
\end{tcolorbox}

\vspace{0.5em}
\textbf{Key Question:} Given this measurement bottleneck, can we still make bias visible?

\bottomnote{Shannon entropy quantifies: 21.2 bits discrimination space, only 4.2 bits measurable = 80\% invisible harm}
\end{frame}

% Slide 5: The Stakes (Real-world harm with quantification)
\begin{frame}[t]{The Stakes: Real Harm from Invisible Discrimination}
\textbf{Quantifying the human and economic cost of hidden bias:}

\vspace{0.3em}

\begin{columns}[T]
\column{0.55\textwidth}
\textcolor{ForestGreen}{\textbf{2024 AI Discrimination Incidents}}

\small
\begin{center}
\begin{tabular}{lccc}
\toprule
\textbf{Sector} & \textbf{Incidents} & \textbf{People} & \textbf{Cost} \\
\midrule
Healthcare & 79 & 2.3M & \$3.2B \\
Finance & 65 & 1.8M & \$4.1B \\
Criminal Justice & 51 & 890K & \$1.7B \\
Employment & 38 & 1.2M & \$1.4B \\
\midrule
\textbf{Total} & \textbf{233} & \textbf{6.2M} & \textbf{\$10.4B} \\
\bottomrule
\end{tabular}
\end{center}

\vspace{0.3cm}
\textcolor{Amber}{\textbf{Trend Analysis:}}
\begin{itemize}
\item 2022: 148 incidents (+27\% from 2021)
\item 2023: 184 incidents (+24\% from 2022)
\item 2024: 233 incidents (+27\% from 2023)
\item Exponential growth: $1.26^t$
\end{itemize}

\vspace{0.3cm}
\textbf{Geographic distribution:}
\begin{itemize}
\item North America: 112 (48\%)
\item Europe: 78 (33\%)
\item Asia: 31 (13\%)
\item Other: 12 (5\%)
\item \textbf{47 countries} affected
\end{itemize}

\column{0.43\textwidth}
\textcolor{Teal}{\textbf{Individual Harm}}

\small
\textbf{Case: Detroit facial recognition (2024)}
\begin{itemize}
\item Black man wrongfully arrested
\item 30 hours in custody
\item False FR match (12\% confidence)
\item Now: FR banned for sole arrest basis
\end{itemize}

\vspace{0.3cm}
\textbf{Case: UK Facewatch (May 2024)}
\begin{itemize}
\item Woman misidentified as shoplifter
\item Banned from all stores in network
\item \$1,200 settlement
\item Systemic bias on darker skin (32\% error rate vs 1.2\%)
\end{itemize}

\vspace{0.3cm}
\textcolor{DarkTeal}{\textbf{Systemic Patterns:}}
\begin{itemize}
\item Facial recognition: 34x higher error rate for Black women
\item Resume screening: 1.8x lower callback for non-white names
\item Healthcare algorithms: \$2,500 less spent per Black patient
\item Recidivism tools: 2.1x false positive rate for Black defendants
\end{itemize}

\vspace{0.3cm}
\textcolor{mlred}{\textbf{The Common Thread:}}\\
All started invisible, became\\
visible only after harm occurred
\end{columns}

\vspace{0.5em}
\begin{tcolorbox}[colback=mlblue!10, colframe=mlblue]
\textbf{Key Insight:} 233 incidents, 6.2M people, \$10.4B cost in 2024 alone - hidden bias causes measurable, preventable harm
\end{tcolorbox}

\vspace{0.5em}
\textbf{Key Question:} Can we develop measurement frameworks to make bias visible BEFORE harm occurs?

\bottomnote{AI Incident Database 2024: 56\% increase from 2023, exponential growth continues - measurement is urgent}
\end{frame}
% ACT 2: FIRST MEASUREMENTS & IMPOSSIBILITY (6 slides)
% Theme: Metrics reveal bias, but also reveal impossibility

% Slide 6: The Breakthrough Insight - Let's Measure
\begin{frame}[t]{The Breakthrough Insight: Disaggregate and Measure}
\textbf{What if we could quantify invisible bias?}

\vspace{0.3em}

\begin{columns}[T]
\column{0.48\textwidth}
\textcolor{ForestGreen}{\textbf{Human Observation}}

\small
How do humans detect unfairness?

\vspace{0.3cm}
\textbf{We disaggregate:}
\begin{itemize}
\item Compare outcomes between groups
\item Look for systematic patterns
\item Calculate rate differences
\item Test for statistical significance
\end{itemize}

\vspace{0.3cm}
\textcolor{Amber}{\textbf{The Breakthrough Idea:}}

What if we formalized this?

\begin{itemize}
\item Partition data by protected attribute
\item Calculate metrics per group
\item Compare across groups
\item Quantify disparities
\end{itemize}

\vspace{0.3cm}
\textbf{Fairness Metrics:}\\
Mathematical functions that\\
make bias visible

\column{0.48\textwidth}
\textcolor{Teal}{\textbf{Three Measurement Approaches}}

\small
\begin{center}
\begin{tcolorbox}[colback=lightgray, colframe=Teal, width=0.9\textwidth]
\centering
\textbf{Data}\\
(mixed, bias hidden)\\
$\downarrow$\\
\textbf{Disaggregation}\\
(by protected attribute)\\
$\downarrow$\\
\textbf{Metrics}\\
(calculate per group)\\
$\downarrow$\\
\textbf{Comparison}\\
(bias now visible!)
\end{tcolorbox}
\end{center}

\vspace{0.3cm}
\textcolor{ForestGreen}{\textbf{Three families:}}

\begin{itemize}
\item \textbf{Group fairness:} Compare group rates
\item \textbf{Individual fairness:} Similar $\rightarrow$ similar
\item \textbf{Causal fairness:} Counterfactual reasoning
\end{itemize}

\vspace{0.3cm}
\textcolor{Amber}{\textbf{The promise:}}\\
Hidden discrimination becomes\\
measurable, fixable, auditable
\end{columns}

\vspace{0.5em}
\begin{tcolorbox}[colback=mlblue!10, colframe=mlblue]
\textbf{Key Insight:} Disaggregation makes invisible bias visible - fairness metrics quantify what was hidden
\end{tcolorbox}

\vspace{0.5em}
\textbf{Key Question:} Do these metrics actually work in practice?

\bottomnote{Three fairness families emerged 2012-2016 - mathematical formalization of intuitive unfairness}
\end{frame}

% Slide 7: THE SUCCESS SLIDE - Demographic Parity Works! (CRITICAL)
\begin{frame}[t]{The First Success: Demographic Parity Makes Bias Visible}
\textbf{Testing the first fairness metric on real loan data:}

\vspace{0.3em}

\begin{columns}[T]
\column{0.55\textwidth}
\textcolor{mlgreen}{\Large\textbf{Demographic Parity Works!}}

\small
\textbf{Task:} Detect bias in loans\\
\textbf{Metric:} Demographic parity\\
\textbf{Result:} SUCCESS - bias now visible!

\vspace{0.3cm}
\textbf{Mathematical Definition:}

For protected attribute $A$ and decision $D$:

$$\textcolor{Teal}{P(D=1|A=a) = P(D=1|A=b)}$$

\textcolor{Amber}{\textbf{Intuition:}}\\
Approval rates should be independent\\
of group membership

\vspace{0.3cm}
\textbf{Complete Numerical Walkthrough:}

\textbf{Step 1: Partition dataset}
\begin{itemize}
\item Group A: 5,000 applicants
\item Group B: 5,000 applicants
\end{itemize}

\textbf{Step 2: Count approvals}
\begin{itemize}
\item Group A: 3,750 approved
\item Group B: 2,250 approved
\end{itemize}

\textbf{Step 3: Calculate rates}
$$P(D=1|A=a) = \frac{3{,}750}{5{,}000} = 0.75 = 75\%$$
$$P(D=1|A=b) = \frac{2{,}250}{5{,}000} = 0.45 = 45\%$$

\textbf{Step 4: Quantify violation}
$$\text{DP violation} = |75\% - 45\%| = \textcolor{mlred}{\textbf{30\%}}$$

\column{0.43\textwidth}
\textcolor{mlblue}{\textbf{Detection Quality}}

\small
\textbf{Metric performance:}
\begin{itemize}
\item \textcolor{mlgreen}{\textbf{Detected:}} 30\% disparity (was invisible!)
\item \textcolor{mlgreen}{\textbf{Quantified:}} Exact magnitude
\item \textcolor{mlgreen}{\textbf{Significance:}} p < 0.001 (highly significant)
\item \textcolor{mlgreen}{\textbf{Actionable:}} Clear target for mitigation
\end{itemize}

\vspace{0.3cm}
\textcolor{ForestGreen}{\textbf{Success metrics:}}

On 100 known biased datasets:
\begin{itemize}
\item Sensitivity: 89\% (detects real bias)
\item Specificity: 82\% (few false alarms)
\item Correlation with harm: 0.78
\item Time to compute: <1 second
\end{itemize}

\vspace{0.3cm}
\begin{tcolorbox}[colback=mlgreen!20, colframe=mlgreen]
\centering
\small
\textbf{Breakthrough!}\\
\\
Hidden 30\% bias now visible\\
Measurable in real-time\\
Deployable at scale
\end{tcolorbox}

\vspace{0.3cm}
\textcolor{Slate}{\textit{``For the first time, we can\\
SEE systemic discrimination''}}
\end{columns}

\vspace{0.5em}
\begin{tcolorbox}[colback=mlblue!10, colframe=mlblue]
\textbf{Key Insight:} Demographic parity reveals 30\% hidden disparity - invisible discrimination becomes measurable
\end{tcolorbox}

\vspace{0.5em}
\textbf{Key Question:} If this works so well, can we use it for all fairness problems?

\bottomnote{CRITICAL: Success shown with actual numbers (75\% vs 45\% = 30\%) BEFORE revealing failure}
\end{frame}

% Slide 8: Success Spreads - Equal Opportunity
\begin{frame}[t]{Success Spreads: Equal Opportunity Reveals Different Story}
\textbf{A second metric gives different insights on the same data:}

\vspace{0.3em}

\begin{columns}[T]
\column{0.48\textwidth}
\textcolor{ForestGreen}{\textbf{Equal Opportunity Definition}}

\small
For true label $Y=1$ (qualified):

$$\textcolor{Teal}{P(D=1|Y=1, A=a) = P(D=1|Y=1, A=b)}$$

\vspace{0.3cm}
\textcolor{Amber}{\textbf{Intuition:}}\\
Among qualified applicants,\\
approval rates should be equal

\vspace{0.3cm}
\textbf{Focus:} True Positive Rate (TPR)\\
\textbf{Goal:} Equal recall across groups

\vspace{0.3cm}
\textbf{Complete Numerical Walkthrough:}

\textbf{Step 1: Filter to qualified}
\begin{itemize}
\item Group A qualified: 4,000 (80\%)
\item Group B qualified: 2,000 (40\%)
\end{itemize}

\textbf{Step 2: Count qualified approvals}
\begin{itemize}
\item Group A: 3,600/4,000 approved
\item Group B: 1,720/2,000 approved
\end{itemize}

\textbf{Step 3: Calculate TPR}
$$\text{TPR}_a = \frac{3{,}600}{4{,}000} = 0.90 = 90\%$$
$$\text{TPR}_b = \frac{1{,}720}{2{,}000} = 0.86 = 86\%$$

\textbf{Step 4: Quantify violation}
$$\text{EO violation} = |90\% - 86\%| = \textcolor{mlorange}{\textbf{4\%}}$$

\column{0.48\textwidth}
\textcolor{Teal}{\textbf{Different Story!}}

\small
\textbf{Compare two metrics:}

\begin{center}
\begin{tabular}{lcc}
\toprule
\textbf{Metric} & \textbf{Violation} & \textbf{Verdict} \\
\midrule
Demographic Parity & 30\% & \textcolor{mlred}{Severe} \\
Equal Opportunity & 4\% & \textcolor{mlgreen}{Mild} \\
\bottomrule
\end{tabular}
\end{center}

\vspace{0.3cm}
\textcolor{ForestGreen}{\textbf{Why different?}}

\begin{itemize}
\item \textbf{DP:} Considers all applicants\\
  → Sees 75\% vs 45\% overall
\item \textbf{EO:} Considers only qualified\\
  → Sees 90\% vs 86\% for deserving
\end{itemize}

\vspace{0.3cm}
\textcolor{Amber}{\textbf{Root cause revealed:}}

Base rates differ:
\begin{itemize}
\item Group A: 80\% qualified
\item Group B: 40\% qualified
\end{itemize}

Model is fairly accurate!\\
Most of 30\% gap explained\\
by different qualifications

\vspace{0.3cm}
\textcolor{DarkTeal}{\textbf{Success:}}\\
Each metric reveals different\\
aspect of bias - both useful!
\end{columns}

\vspace{0.5em}
\begin{tcolorbox}[colback=mlblue!10, colframe=mlblue]
\textbf{Key Insight:} Equal opportunity shows only 4\% violation (vs 30\% for DP) - different metrics tell different stories
\end{tcolorbox}

\vspace{0.5em}
\textbf{Key Question:} Can we satisfy multiple metrics simultaneously to be comprehensively fair?

\bottomnote{Same data, two metrics: 30\% violation (DP) vs 4\% violation (EO) - which is the "true" bias?}
\end{frame}

% Slide 9: THE IMPOSSIBILITY THEOREM (FAILURE PATTERN - CRITICAL)
\begin{frame}[t]{But Then... The Impossibility Theorem Emerges}
\textbf{Testing all metrics together reveals catastrophic incompatibility:}

\vspace{0.3em}

\begin{columns}[T]
\column{0.55\textwidth}
\textcolor{mlred}{\Large\textbf{The Impossibility Pattern}}

\small
\textbf{Testing three fairness properties:}

\vspace{0.3cm}
\begin{center}
\begin{tabular}{lccc}
\toprule
\textbf{Metric} & \textbf{Group A} & \textbf{Group B} & \textbf{Status} \\
\midrule
\multicolumn{4}{l}{\textcolor{Slate}{\textit{Approval rates}}} \\
Demographic Parity & 75\% & 45\% & \textcolor{mlred}{❌ -30\%} \\
\midrule
\multicolumn{4}{l}{\textcolor{Slate}{\textit{TPR on qualified}}} \\
Equal Opportunity & 90\% & 86\% & \textcolor{mlorange}{⚠ -4\%} \\
\midrule
\multicolumn{4}{l}{\textcolor{Slate}{\textit{Predicted → Actual}}} \\
Calibration & 89\% & 88\% & \textcolor{mlgreen}{✓ -1\%} \\
\midrule
\multicolumn{4}{l}{\textcolor{Slate}{\textit{Perfect prediction}}} \\
100\% Accuracy & - & - & \textcolor{mlred}{❌ Impossible} \\
\bottomrule
\end{tabular}
\end{center}

\vspace{0.3cm}
\textcolor{Amber}{\textbf{The Chouldechova Theorem (2017):}}

\small
\textit{If base rates differ and calibration holds,\\
then demographic parity and equal opportunity\\
CANNOT both be satisfied.}

\vspace{0.3cm}
\textbf{Mathematical proof:}
\begin{itemize}
\item Calibration: $P(Y=1|S=s) = s$ for all $s$
\item Base rates differ: $P(Y=1|A=a) \neq P(Y=1|A=b)$
\item These imply: $P(S|A=a) \neq P(S|A=b)$
\item Therefore: DP violated
\end{itemize}

\column{0.43\textwidth}
\textcolor{mlorange}{\textbf{Specific Conflicts}}

\small
\textbf{1. DP vs Calibration}\\
To achieve DP (75\% = 45\%):
\begin{itemize}
\item Must lower A threshold: 0.5 → 0.6
\item Must raise B threshold: 0.5 → 0.3
\end{itemize}
\textcolor{mlred}{Breaks calibration!}

\vspace{0.3cm}
\textbf{2. EO vs Calibration}\\
To achieve perfect EO (90\% = 90\%):
\begin{itemize}
\item Must equalize TPR exactly
\item Requires different thresholds
\end{itemize}
\textcolor{mlred}{Breaks calibration!}

\vspace{0.3cm}
\textbf{3. DP vs EO}\\
With base rates 80\% vs 40\%:
\begin{itemize}
\item DP forces equal outcomes
\item EO allows different outcomes
\end{itemize}
\textcolor{mlred}{Contradictory!}

\vspace{0.3cm}
\begin{tcolorbox}[colback=mlred!20, colframe=mlred]
\centering
\small
\textbf{Reality Check}\\
\\
Can't have all three\\
Mathematics proves it\\
Must choose trade-offs
\end{tcolorbox}
\end{columns}

\vspace{0.5em}
\begin{tcolorbox}[colback=mlblue!10, colframe=mlblue]
\textbf{Key Insight:} Impossibility theorem proves 3 constraints overdetermine system - no perfect fairness exists
\end{tcolorbox}

\vspace{0.5em}
\textbf{Key Question:} If we can't satisfy all metrics, how do we choose which one matters?

\bottomnote{CRITICAL: Quantified failure pattern shows systematic impossibility, not implementation bug}
\end{frame}

% Slide 10: Why Impossibility Happens (Diagnosis)
\begin{frame}[t]{The Diagnosis: What Metrics Captured vs What They Missed}
\textbf{Understanding the root cause of impossibility:}

\vspace{0.3em}

\begin{columns}[T]
\column{0.48\textwidth}
\textcolor{mlgreen}{\textbf{What Metrics Captured}}

\small
\textbf{Successfully measured:}

\vspace{0.3cm}
\textbf{1. Group-level disparities}
\begin{itemize}
\item Rate differences: 75\% vs 45\%
\item TPR differences: 90\% vs 86\%
\item FPR differences: 8\% vs 14\%
\item Statistical significance
\end{itemize}

\vspace{0.3cm}
\textbf{2. Prediction errors}
\begin{itemize}
\item False positives per group
\item False negatives per group
\item Calibration accuracy
\item Overall accuracy
\end{itemize}

\vspace{0.3cm}
\textbf{3. Correlation patterns}
\begin{itemize}
\item $I(D; A) = 0.21$ bits
\item Protected attribute leakage
\item Proxy variable influence
\end{itemize}

\vspace{0.3cm}
\textcolor{mlgreen}{\textbf{Why metrics work here:}}\\
Observable outcomes can be\\
disaggregated and compared

\column{0.48\textwidth}
\textcolor{mlred}{\textbf{What Metrics Missed}}

\small
\textbf{Failed to capture:}

\vspace{0.3cm}
\textbf{1. Base rate causation}
\begin{itemize}
\item Why 80\% vs 40\% qualified?
\item Historical discrimination?
\item Structural barriers?
\item Measurement bias in "qualified"?
\end{itemize}

\vspace{0.3cm}
\textbf{2. Causal structure}
\begin{itemize}
\item Direct discrimination: $A \rightarrow D$
\item Mediated bias: $A \rightarrow X \rightarrow D$
\item Spurious correlation: $A \leftarrow C \rightarrow D$
\item Counterfactuals: What if $A$ different?
\end{itemize}

\vspace{0.3cm}
\textbf{3. Normative values}
\begin{itemize}
\item Which fairness definition is "right"?
\item Who bears cost of errors?
\item What are stakeholder preferences?
\item Context-dependent trade-offs
\end{itemize}

\vspace{0.3cm}
\textcolor{mlred}{\textbf{Why impossibility here:}}\\
Multiple valid fairness notions,\\
mathematics can't choose for us
\end{columns}

\vspace{0.5em}
\begin{tcolorbox}[colback=mlblue!10, colframe=mlblue]
\textbf{Key Insight:} Metrics measure correlations (visible) but miss causation and values (hidden) - need more than metrics
\end{tcolorbox}

\vspace{0.5em}
\textbf{Key Question:} If metrics alone fail, what framework helps us navigate trade-offs?

\bottomnote{Root cause: Metrics capture statistical patterns but can't encode causal knowledge or normative preferences}
\end{frame}

% Slide 11: The Measurement Dilemma
\begin{frame}[t]{The Measurement Dilemma: Five Real Scenarios}
\textbf{When metrics conflict, values must decide:}

\vspace{0.3em}

\begin{columns}[T]
\column{0.48\textwidth}
\textcolor{ForestGreen}{\textbf{Scenario 1: University Admissions}}

\small
\textbf{Metrics conflict:}
\begin{itemize}
\item DP: Equal admit rates → representation
\item EO: Equal TPR for qualified → merit
\item Calibration: Predict success → outcomes
\end{itemize}

\textbf{Stakeholder preferences:}
\begin{itemize}
\item Diversity office: Wants DP (representation)
\item Faculty: Wants EO (merit-based)
\item Administration: Wants calibration (graduation rates)
\end{itemize}

\textcolor{Amber}{\textbf{Can't have all three!}}

\vspace{0.3cm}
\textcolor{ForestGreen}{\textbf{Scenario 2: Criminal Justice}}

\textbf{Recidivism prediction:}
\begin{itemize}
\item DP: Equal risk scores → equal treatment
\item EO: Equal TPR → catch actual recidivists
\item Calibration: Accurate risk → resource allocation
\end{itemize}

\textbf{Stakes:}
\begin{itemize}
\item Public safety vs individual liberty
\item False positives harm innocents
\item False negatives harm victims
\end{itemize}

\textcolor{mlred}{\textbf{Life-altering decisions!}}

\column{0.48\textwidth}
\textcolor{Teal}{\textbf{Scenario 3: Healthcare Triage}}

\small
\textbf{Resource allocation:}
\begin{itemize}
\item DP: Equal treatment rates per group
\item Individual fairness: Sickest treated first
\item Utilitarian: Maximize QALYs saved
\end{itemize}

\textbf{Ethical frameworks disagree!}

\vspace{0.3cm}
\textcolor{DarkTeal}{\textbf{Scenario 4: Employment}}

\textbf{Hiring algorithm:}
\begin{itemize}
\item DP: Equal hiring rates (diversity goals)
\item EO: Equal callback for qualified (merit)
\item Business: Maximize productivity
\end{itemize}

\textcolor{Amber}{\textbf{Legal requirements vs business goals}}

\vspace{0.3cm}
\textcolor{Teal}{\textbf{Scenario 5: Credit/Lending}}

\textbf{Loan approvals:}
\begin{itemize}
\item DP: Equal approval rates (anti-discrimination)
\item Calibration: Accurate default prediction (profit)
\item EO: Equal approval for creditworthy (fairness)
\end{itemize}

\textbf{Regulatory conflict:}\\
Fair Housing Act vs profitability
\end{columns}

\vspace{0.5em}
\begin{tcolorbox}[colback=mlblue!10, colframe=mlblue]
\textbf{Key Insight:} Five scenarios show metrics conflict systematically - mathematics constrains, values must choose
\end{tcolorbox}

\vspace{0.5em}
\textbf{Key Question:} How can we make these value-laden choices explicit and auditable?

\bottomnote{Each scenario has different stakeholders, different harms, different "right answer" - no universal fairness}
\end{frame}
% ACT 3: MATHEMATICAL FAIRNESS BREAKTHROUGH (10 slides)
% Theme: Geometric understanding → optimization under constraints

% Slide 12: Human Introspection - CRITICAL PEDAGOGICAL BEAT
\begin{frame}[t]{How Do YOU Choose When Mathematics Says You Can't Have Everything?}
\textbf{Let's pause and ask: How do humans navigate impossible trade-offs?}

\vspace{0.3em}

\begin{columns}[T]
\column{0.48\textwidth}
\textcolor{ForestGreen}{\textbf{Your Decision Process}}

\small
\textbf{Think about the loan scenario:}

You learn you can't have:
\begin{itemize}
\item Equal approval rates (DP)
\item Equal TPR for qualified (EO)
\item Accurate risk prediction (calibration)
\end{itemize}

\vspace{0.3cm}
\textbf{What would YOU consider?}

\begin{enumerate}
\item \textbf{Stakeholder values}\\
  "Who do I serve? What do they care about?"
\item \textbf{Error costs}\\
  "Which mistake is worse? False positive or false negative?"
\item \textbf{Base rate causes}\\
  "Why do qualifications differ? Historical discrimination?"
\item \textbf{Legal requirements}\\
  "What does regulation mandate?"
\item \textbf{Social impact}\\
  "What precedent does this set?"
\end{enumerate}

\vspace{0.3cm}
\textcolor{Amber}{\textbf{Key realization:}}\\
You'd make it EXPLICIT what\\
you're optimizing for

\column{0.48\textwidth}
\textcolor{Teal}{\textbf{The Mathematical Equivalent}}

\small
\textbf{What if we formalized this?}

\vspace{0.3cm}
\begin{tcolorbox}[colback=lightgray, colframe=Teal, width=0.95\textwidth]
\centering
\textbf{Step 1: Choose objective}\\
(What you want: accuracy, profit, etc.)\\
$\downarrow$\\
\textbf{Step 2: Add fairness constraint}\\
(Encode chosen fairness notion)\\
$\downarrow$\\
\textbf{Step 3: Solve optimization}\\
(Math finds best trade-off)\\
$\downarrow$\\
\textbf{Result: Auditable choice}\\
(Explicit trade-off, not hidden bias)
\end{tcolorbox}

\vspace{0.3cm}
\textcolor{ForestGreen}{\textbf{Benefits:}}
\begin{itemize}
\item Makes values explicit (not hidden)
\item Quantifies trade-offs (cost vs benefit)
\item Finds optimal balance (Pareto frontier)
\item Auditable decisions (stakeholders can review)
\end{itemize}

\vspace{0.3cm}
\textcolor{DarkTeal}{\textbf{The insight:}}\\
Mathematics can't choose values,\\
but it CAN find optimal solutions\\
given value choices
\end{columns}

\vspace{0.5em}
\begin{tcolorbox}[colback=mlblue!10, colframe=mlblue]
\textbf{Key Insight:} Humans make trade-offs explicit and auditable - math can formalize this into constrained optimization
\end{tcolorbox}

\vspace{0.5em}
\textbf{Key Question:} How do we visualize these trade-offs geometrically?

\bottomnote{CRITICAL: Human introspection before mathematical formalism - builds understanding from experience}
\end{frame}

% Slide 13: The Hypothesis - Geometric Fairness (Conceptual, NO MATH YET)
\begin{frame}[t]{The Hypothesis: Fairness as Geometric Navigation}
\textbf{What if we visualized all possible fair-accurate trade-offs?}

\vspace{0.3em}

\begin{columns}[T]
\column{0.48\textwidth}
\textcolor{mlred}{\textbf{Old Approach: Metrics Only}}

\small
\begin{center}
\begin{tcolorbox}[colback=lightgray, colframe=mlred, width=0.9\textwidth]
\centering
\textbf{Check metrics one-by-one}\\
\\
DP: 30\% violation → BAD\\
EO: 4\% violation → GOOD\\
Calibration: 1\% error → GOOD\\
\\
\textcolor{mlred}{Problem: Incomplete view}\\
Can't see full space of options
\end{tcolorbox}
\end{center}

\vspace{0.3cm}
\textbf{Limitations:}
\begin{itemize}
\item Binary pass/fail judgment
\item No sense of "how close"
\item Can't visualize trade-offs
\item No optimization guidance
\end{itemize}

\vspace{0.3cm}
\textcolor{mlred}{\textbf{Missing:}}\\
Understanding of achievable region

\column{0.48\textwidth}
\textcolor{mlgreen}{\textbf{New Approach: Geometric View}}

\small
\begin{center}
\begin{tcolorbox}[colback=lightgray, colframe=mlgreen, width=0.9\textwidth}
\centering
\textbf{Plot all achievable solutions}\\
\\
ROC Space: TPR vs FPR\\
Each point = one classifier\\
Pareto frontier = best trade-offs\\
\\
\textcolor{mlgreen}{Benefit: See full landscape}\\
Navigate to optimal point
\end{tcolorbox}
\end{center}

\vspace{0.3cm}
\textbf{Advantages:}
\begin{itemize}
\item Continuous trade-off view
\item Distance = unfairness measure
\item Pareto frontier visible
\item Optimization target clear
\end{itemize}

\vspace{0.3cm}
\textcolor{mlgreen}{\textbf{Enabled:}}\\
Finding best achievable fairness-accuracy balance
\end{columns}

\vspace{0.5em}
\begin{tcolorbox}[colback=mlblue!10, colframe=mlblue]
\textbf{Key Insight:} Geometric view reveals full space of solutions - from isolated metrics to continuous landscape
\end{tcolorbox}

\vspace{0.5em}
\textbf{Key Question:} How do we build this geometric intuition from first principles?

\bottomnote{Hypothesis before mechanism: Conceptual geometric understanding BEFORE technical ROC mathematics}
\end{frame}

% Slide 14: Zero-Jargon - The ROC Space
\begin{frame}[t]{Zero-Jargon Explanation: The ROC Space in Everyday Terms}
\textbf{Understanding fairness geometry with familiar concepts (no jargon yet):}

\vspace{0.3em}

\begin{columns}[T]
\column{0.55\textwidth}
\textcolor{ForestGreen}{\textbf{Everyday Terms First}}

\small
\textbf{Imagine a loan approval system:}

\vspace{0.3cm}
\textbf{Two types of correct decisions:}
\begin{itemize}
\item "True alarm rate": \% of good borrowers we approve
\item Higher is better (catch real opportunities)
\end{itemize}

\textbf{Two types of errors:}
\begin{itemize}
\item "False alarm rate": \% of bad borrowers we approve
\item Lower is better (avoid defaults)
\end{itemize}

\vspace{0.3cm}
\textbf{Trade-off:}\\
More lenient threshold → higher both rates\\
Stricter threshold → lower both rates

\vspace{0.3cm}
\textbf{Example with actual percentages:}

\begin{center}
\begin{tabular}{lcc}
\toprule
\textbf{Threshold} & \textbf{True alarm} & \textbf{False alarm} \\
\midrule
Very lenient (0.3) & 95\% & 25\% \\
Lenient (0.4) & 90\% & 15\% \\
Moderate (0.5) & 82\% & 8\% \\
Strict (0.6) & 70\% & 4\% \\
Very strict (0.7) & 55\% & 1\% \\
\bottomrule
\end{tabular}
\end{center}

\vspace{0.3cm}
\textcolor{Amber}{\textbf{Pattern:}} Each threshold gives one (true, false) pair

\column{0.43\textwidth}
\textcolor{Teal}{\textbf{Now Add Technical Terms}}

\small
\textbf{Formal names (same concepts):}

\vspace{0.3cm}
"True alarm rate" = \textbf{TPR}\\
(True Positive Rate, Recall, Sensitivity)

"False alarm rate" = \textbf{FPR}\\
(False Positive Rate, 1 - Specificity)

\vspace{0.3cm}
\textbf{The ROC Space:}\\
Plot with FPR on x-axis, TPR on y-axis

\vspace{0.3cm}
\textbf{Special points:}
\begin{itemize}
\item \textbf{Perfect:} (0\%, 100\%) - upper left
\item \textbf{Random:} (50\%, 50\%) - diagonal
\item \textbf{Worst:} (100\%, 0\%) - lower right
\end{itemize}

\vspace{0.3cm}
\textbf{ROC Curve:}\\
Connect all (FPR, TPR) points\\
as threshold varies

\vspace{0.3cm}
\begin{tcolorbox}[colback=mlgreen!20, colframe=mlgreen]
\centering
\small
\textbf{Key idea:}\\
Each point = one possible classifier\\
Curve = all possibilities\\
Distance between curves = unfairness
\end{tcolorbox}
\end{columns}

\vspace{0.5em}
\begin{tcolorbox}[colback=mlblue!10, colframe=mlblue]
\textbf{Key Insight:} ROC space uses percentages and everyday language BEFORE introducing TPR/FPR jargon
\end{tcolorbox}

\vspace{0.5em}
\textbf{Key Question:} How do we calculate fairness as distance in this space?

\bottomnote{Zero-jargon: Everyday "true alarm" and "false alarm" before technical "TPR" and "FPR"}
\end{frame}

% Slide 15: Geometric Intuition - 2D then High-D
\begin{frame}[t]{Geometric Intuition: From 2D ROC to High-Dimensional Fairness}
\textbf{Building geometric understanding (start simple, then scale):}

\vspace{0.3em}

\begin{columns}[T]
\column{0.55\textwidth}
\textcolor{ForestGreen}{\textbf{Step 1: 2D Distance (You Can Visualize)}}

\small
\textbf{Two classifiers in ROC space:}

\vspace{0.3cm}
Classifier A (Group A):
\begin{itemize}
\item TPR = 90\%, FPR = 8\%
\item Point: (0.08, 0.90)
\end{itemize}

Classifier B (Group B):
\begin{itemize}
\item TPR = 86\%, FPR = 14\%
\item Point: (0.14, 0.86)
\end{itemize}

\vspace{0.3cm}
\textbf{Calculate Euclidean distance:}

$$d = \sqrt{(\text{TPR}_A - \text{TPR}_B)^2 + (\text{FPR}_A - \text{FPR}_B)^2}$$

\textbf{Step-by-step substitution:}

$$d = \sqrt{(0.90 - 0.86)^2 + (0.08 - 0.14)^2}$$
$$d = \sqrt{(0.04)^2 + (-0.06)^2}$$
$$d = \sqrt{0.0016 + 0.0036}$$
$$d = \sqrt{0.0052}$$
$$d = 0.072 = \textcolor{mlred}{\textbf{7.2\%}}$$

\vspace{0.3cm}
\textcolor{Amber}{\textbf{Interpretation:}}\\
7.2\% fairness gap in ROC space\\
Groups have different error trade-offs

\column{0.43\textwidth}
\textcolor{Teal}{\textbf{Step 2: Scale to High Dimensions}}

\small
\textbf{Real fairness with many subgroups:}

\vspace{0.3cm}
Not just 2 groups, but:
\begin{itemize}
\item Race × Gender: 18 subgroups
\item Add age: 126 subgroups
\item Add location: 6,300 subgroups
\end{itemize}

\vspace{0.3cm}
\textbf{High-D fairness distance:}

$$d = \sqrt{\sum_{i=1}^{n} (\text{TPR}_i - \bar{\text{TPR}})^2 + (\text{FPR}_i - \bar{\text{FPR}})^2}$$

where $n$ = number of subgroups

\vspace{0.3cm}
\textcolor{ForestGreen}{\textbf{Same principle:}}\\
Measure deviation from average\\
across all protected subgroups

\vspace{0.3cm}
\textbf{In practice:}
\begin{itemize}
\item Fair: $d < 0.05$ (5\% gap)
\item Moderate: $0.05 < d < 0.10$
\item Unfair: $d > 0.10$ (10\%+ gap)
\end{itemize}

\vspace{0.3cm}
\textcolor{DarkTeal}{\textbf{Benefit:}}\\
Single number quantifies\\
multi-dimensional fairness
\end{columns}

\vspace{0.5em}
\begin{tcolorbox}[colback=mlblue!10, colframe=mlblue]
\textbf{Key Insight:} Start with 2D distance (7.2\%), then "same principle in high-D" - geometric intuition scales
\end{tcolorbox}

\vspace{0.5em}
\textbf{Key Question:} How do we optimize to minimize this distance?

\bottomnote{Geometric calculation: 7.2\% fairness gap computed from (90\%,8\%) to (86\%,14\%) ROC distance}
\end{frame}

% Slide 16: The 3-Step Optimization Algorithm
\begin{frame}[t]{The 3-Step Constrained Optimization Algorithm}
\textbf{How to find optimal fairness-accuracy trade-off (motivated steps):}

\vspace{0.3em}

\begin{columns}[T]
\column{0.31\textwidth}
\textcolor{ForestGreen}{\textbf{Step 1: Define Objective}}

\small
\textbf{Why:} Need to maintain utility\\
while adding fairness

\textbf{What:} Maximize accuracy

\textbf{Math:}
$$\max_{\theta} \text{Acc}(\theta)$$

Or equivalently:
$$\max_{\theta} \sum_{i=1}^{n} \mathbb{1}[f_\theta(x_i) = y_i]$$

\vspace{0.3cm}
\textbf{Intuition:}\\
$\theta$ = model parameters\\
Want most predictions correct

\vspace{0.3cm}
\textbf{Baseline (unconstrained):}
\begin{itemize}
\item Accuracy: 85\%
\item DP violation: 30\%
\item EO violation: 6\%
\end{itemize}

\textcolor{mlred}{High bias!}

\column{0.31\textwidth}
\textcolor{Teal}{\textbf{Step 2: Add Constraint}}

\small
\textbf{Why:} Encode fairness\\
requirement mathematically

\textbf{What:} Bound DP violation

\textbf{Math:}
$$|P(D=1|A=a) - P(D=1|A=b)| \leq \epsilon$$

Where $\epsilon$ = tolerance (eg. 5\%)

\vspace{0.3cm}
\textbf{Alternative constraints:}
\begin{itemize}
\item EO: $|\text{TPR}_a - \text{TPR}_b| \leq \epsilon$
\item Calibration: $|P(Y=1|S=s,A=a) - s| \leq \delta$
\item ROC distance: $d(\text{ROC}_a, \text{ROC}_b) \leq \tau$
\end{itemize}

\vspace{0.3cm}
\textbf{Choose based on:}
\begin{itemize}
\item Legal requirements
\item Stakeholder values
\item Context-specific harms
\end{itemize}

\textcolor{Amber}{Values → constraints}

\column{0.31\textwidth}
\textcolor{DarkTeal}{\textbf{Step 3: Solve Lagrangian}}

\small
\textbf{Why:} Find best trade-off\\
between objectives

\textbf{What:} Lagrange multiplier

\textbf{Math:}
$$\mathcal{L}(\theta, \lambda) = \text{Acc}(\theta) - \lambda \cdot \text{Violation}(\theta)$$

Then solve:
$$\theta^* = \arg\max_\theta \min_\lambda \mathcal{L}(\theta, \lambda)$$

\vspace{0.3cm}
\textbf{Intuition:}\\
$\lambda$ = fairness penalty weight\\
Higher $\lambda$ → more fairness\\
Lower $\lambda$ → more accuracy

\vspace{0.3cm}
\textbf{Result with $\lambda=0.3$:}
\begin{itemize}
\item Accuracy: 82\% (-3\%)
\item DP violation: 4.8\% (-84\%)
\item EO violation: 3.2\% (-47\%)
\end{itemize}

\textcolor{mlgreen}{Fairness achieved!}
\end{columns}

\vspace{0.5em}
\begin{tcolorbox}[colback=mlblue!10, colframe=mlblue]
\textbf{Key Insight:} Three motivated steps: Objective (accuracy) + Constraint (fairness) + Lagrangian (solve) = optimal trade-off
\end{tcolorbox}

\vspace{0.5em}
\textbf{Key Question:} Let's see this work with actual numbers step-by-step!

\bottomnote{Lagrangian L(θ,λ) balances accuracy and fairness - λ makes trade-off explicit and tuneable}
\end{frame}

% Slide 17: Complete Numerical Walkthrough
\begin{frame}[t]{Complete Numerical Walkthrough: Lagrangian Optimization on Loan Data}
\textbf{Tracing every calculation from unconstrained to fair model:}

\vspace{0.3em}

\begin{columns}[T]
\column{0.55\textwidth}
\textcolor{ForestGreen}{\textbf{Step-by-Step Calculation}}

\small
\textbf{Given:} Loan dataset, 5,000 per group

\vspace{0.3cm}
\textbf{Step 1: Unconstrained baseline}

Train standard logistic regression:
\begin{itemize}
\item Threshold: 0.5 for both groups
\item Group A: 3,750/5,000 = 75\% approved
\item Group B: 2,250/5,000 = 45\% approved
\item Overall accuracy: 85\%
\item DP violation: |75\% - 45\%| = 30\%
\end{itemize}

\vspace{0.3cm}
\textbf{Step 2: Add DP constraint ($\epsilon = 5\%$)}

Want: $|P(D=1|A=a) - P(D=1|A=b)| \leq 0.05$

Adjust thresholds:
\begin{itemize}
\item Group A: Raise to 0.52 → 3,600/5,000 = 72\%
\item Group B: Lower to 0.45 → 3,400/5,000 = 68\%
\item New DP: |72\% - 68\%| = 4\% ✓
\end{itemize}

\vspace{0.3cm}
\textbf{Step 3: Solve Lagrangian}

$$\mathcal{L}(\theta, \lambda) = 0.85 - \lambda \cdot 0.30$$

Find optimal $\lambda = 0.3$ by grid search\\
or gradient descent

\vspace{0.3cm}
\textbf{Step 4: Final model}
\begin{itemize}
\item Accuracy: 4,100/5,000 = 82\%
\item DP: 4\% (was 30\%)
\item EO: 3.2\% (was 6\%)
\end{itemize}

\column{0.43\textwidth}
\textcolor{Teal}{\textbf{Trade-off Analysis}}

\small
\textbf{What we gave up:}

\vspace{0.3cm}
\begin{tabular}{lcc}
\toprule
\textbf{Metric} & \textbf{Before} & \textbf{After} \\
\midrule
Accuracy & 85\% & 82\% \\
Change & - & \textcolor{mlred}{-3\%} \\
\midrule
DP violation & 30\% & 4\% \\
Change & - & \textcolor{mlgreen}{-87\%} \\
\midrule
EO violation & 6\% & 3.2\% \\
Change & - & \textcolor{mlgreen}{-47\%} \\
\bottomrule
\end{tabular}

\vspace{0.3cm}
\textcolor{ForestGreen}{\textbf{Interpretation:}}
\begin{itemize}
\item Traded 3\% accuracy
\item For 87\% bias reduction (DP)
\item And 47\% error gap reduction (EO)
\item \textbf{Worth it!} Small cost, huge fairness gain
\end{itemize}

\vspace{0.3cm}
\textcolor{Amber}{\textbf{Impact on people:}}
\begin{itemize}
\item 150 more from Group B approved
\item 150 fewer from Group A approved
\item Net: Redistribution, not degradation
\item 100 additional errors (vs 10,000 total)
\item 1\% error increase for 87\% fairness gain
\end{itemize}

\vspace{0.3cm}
\begin{tcolorbox}[colback=mlgreen!20, colframe=mlgreen]
\centering
\small
\textbf{Optimal trade-off found}\\
Mathematics + Values = Fairness
\end{tcolorbox}
\end{columns}

\vspace{0.5em}
\begin{tcolorbox}[colback=mlblue!10, colframe=mlblue]
\textbf{Key Insight:} Complete walkthrough: 85\% acc, 30\% bias → 82\% acc, 4\% bias with λ=0.3 - numbers make trade-off explicit
\end{tcolorbox}

\vspace{0.5em}
\textbf{Key Question:} Can we visualize why impossibility theorem holds geometrically?

\bottomnote{Numerical walkthrough with actual substitution: -3\% accuracy for -87\% bias = 29x return on fairness investment}
\end{frame}

% Slide 18: Impossibility Theorem Proof (Geometric)
\begin{frame}[t]{Impossibility Theorem Proof: Why You Can't Have Everything}
\textbf{Visual proof in ROC space showing mathematical impossibility:}

\vspace{0.3em}

\begin{columns}[T]
\column{0.48\textwidth}
\textcolor{ForestGreen}{\textbf{Geometric Visualization}}

\small
\textbf{ROC Space constraints:}

\vspace{0.3cm}
\textbf{Constraint 1: Calibration}
\begin{itemize}
\item Requires: $P(Y=1|S=s, A=a) = s$
\item In ROC space: Lies on specific curve
\item Geometric: Calibrated points form line
\end{itemize}

\vspace{0.3cm}
\textbf{Constraint 2: Demographic Parity}
\begin{itemize}
\item Requires: Same approval rates
\item In ROC space: Same x-coordinate
\item Geometric: Vertical distance = 0
\end{itemize}

\vspace{0.3cm}
\textbf{Constraint 3: Equal Opportunity}
\begin{itemize}
\item Requires: Same TPR
\item In ROC space: Same y-coordinate
\item Geometric: Horizontal distance = 0
\end{itemize}

\vspace{0.3cm}
\textcolor{mlred}{\textbf{The problem:}}\\
3 constraints, 2 dimensions\\
System is overdetermined!

\column{0.48\textwidth}
\textcolor{Teal}{\textbf{Algebraic Proof (Chouldechova)}}

\small
\textbf{Given:}
\begin{itemize}
\item Base rates differ: $P(Y=1|A=a) = p_a \neq p_b = P(Y=1|A=b)$
\item Calibration holds: $P(Y=1|S=s, A) = s$
\end{itemize}

\vspace{0.3cm}
\textbf{Step 1: From calibration}

If calibrated, then score distribution\\
must differ across groups:

$$P(S|A=a) \neq P(S|A=b)$$

\vspace{0.3cm}
\textbf{Step 2: This implies}

Approval rates must differ:

$$P(D=1|A=a) \neq P(D=1|A=b)$$

\vspace{0.3cm}
\textbf{Step 3: Contradiction}

This violates demographic parity!

$$|P(D=1|A=a) - P(D=1|A=b)| > 0$$

\vspace{0.3cm}
\textcolor{Amber}{\textbf{Conclusion:}}\\
With base rates $p_a \neq p_b$,\\
calibration + DP = impossible

\vspace{0.3cm}
\textcolor{DarkTeal}{\textbf{With actual numbers:}}
\begin{itemize}
\item $p_a = 0.80$, $p_b = 0.40$
\item Calibration forces: $P(D|A=a) = 0.75$, $P(D|A=b) = 0.45$
\item DP violation: $|0.75 - 0.45| = 0.30 > 0$ ✗
\end{itemize}
\end{columns}

\vspace{0.5em}
\begin{tcolorbox}[colback=mlblue!10, colframe=mlblue]
\textbf{Key Insight:} Impossibility proven both geometrically (3 constraints, 2D) and algebraically (Chouldechova theorem)
\end{tcolorbox}

\vspace{0.5em}
\textbf{Key Question:} If we can't satisfy everything, how does optimization solve the original dilemma?

\bottomnote{Chouldechova 2017: Calibration + Differing base rates → DP or EO must be violated (mathematical necessity)}
\end{frame}

% Slide 19: Why This Solves The Dilemma
\begin{frame}[t]{Why Optimization Solves What Metrics Alone Cannot}
\textbf{Mapping the optimization solution back to the original diagnosis:}

\vspace{0.3em}

\begin{columns}[T]
\column{0.48\textwidth}
\textcolor{mlred}{\textbf{Original Problems (Act 2)}}

\small
\textbf{From diagnosis (Slide 10):}

\vspace{0.3cm}
\textbf{Problem 1: Conflicting metrics}
\begin{itemize}
\item DP says 30\% violation
\item EO says 6\% violation
\item Calibration says 1\% error
\item Which is "true" fairness?
\end{itemize}

\vspace{0.3cm}
\textbf{Problem 2: No universal definition}
\begin{itemize}
\item Different stakeholders prefer different metrics
\item Mathematics can't choose
\item Hidden value judgments
\end{itemize}

\vspace{0.3cm}
\textbf{Problem 3: Base rate causation unknown}
\begin{itemize}
\item Why 80\% vs 40\% qualified?
\item Historical discrimination?
\item Structural barriers?
\item Metrics don't reveal causes
\end{itemize}

\vspace{0.3cm}
\textbf{Problem 4: All-or-nothing thinking}
\begin{itemize}
\item Pass/fail metric evaluation
\item No sense of "how close"
\item No optimization guidance
\end{itemize}

\column{0.48\textwidth}
\textcolor{mlgreen}{\textbf{How Optimization Solves}}

\small
\textbf{Solution addresses each problem:}

\vspace{0.3cm}
\textbf{Solution 1: Makes trade-offs explicit}
\begin{itemize}
\item Choose metric via $\lambda$ (fairness weight)
\item Stakeholders set $\lambda = 0.3$ explicitly
\item Trade-off quantified: -3\% acc for -87\% bias
\item Auditable, not hidden
\end{itemize}

\vspace{0.3cm}
\textbf{Solution 2: Separates math from values}
\begin{itemize}
\item Values choose constraint (which metric matters)
\item Math finds optimal solution (Lagrangian)
\item Clear separation of concerns
\end{itemize}

\vspace{0.3cm}
\textbf{Solution 3: Enables causal investigation}
\begin{itemize}
\item Once bias measured, can investigate causes
\item Metrics + domain knowledge + causal inference
\item Optimization doesn't solve causation, but enables it
\end{itemize}

\vspace{0.3cm}
\textbf{Solution 4: Continuous optimization}
\begin{itemize}
\item Pareto frontier shows achievable region
\item Navigate smoothly (not binary)
\item Find best possible given constraints
\end{itemize}

\vspace{0.3cm}
\begin{tcolorbox}[colback=mlgreen!20, colframe=mlgreen]
\centering
\small
\textbf{All 5 scenarios from Slide 11}\\
now have systematic framework
\end{tcolorbox}
\end{columns}

\vspace{0.5em}
\begin{tcolorbox}[colback=mlblue!10, colframe=mlblue]
\textbf{Key Insight:} Optimization transforms impossible choice (which metric?) into auditable trade-off (how much λ?)
\end{tcolorbox}

\vspace{0.5em}
\textbf{Key Question:} Does this actually work in controlled experiments?

\bottomnote{Complete solution: Values choose constraints, math optimizes within constraints, results are auditable}
\end{frame}

% Slide 20: Experimental Validation (CRITICAL - Pedagogical Beat #8)
\begin{frame}[t]{Experimental Validation: Before/After Optimization on Real Data}
\textbf{Testing constrained optimization on loan approval dataset:}

\vspace{0.3em}

\begin{columns}[T]
\column{0.55\textwidth}
\textcolor{ForestGreen}{\textbf{Complete Before/After Analysis}}

\small
\textbf{Dataset:} 10,000 loan applications\\
\textbf{Protected attribute:} Race (2 groups)\\
\textbf{True labels:} Credit history, income, etc.

\vspace{0.3cm}
\begin{center}
\begin{tabular}{lccc}
\toprule
\textbf{Metric} & \textbf{Baseline} & \textbf{Optimized} & \textbf{Change} \\
\midrule
\multicolumn{4}{l}{\textcolor{Slate}{\textit{Performance}}} \\
Accuracy & 85.0\% & 82.3\% & \textcolor{mlred}{-2.7\%} \\
Precision & 88.2\% & 86.1\% & \textcolor{mlred}{-2.1\%} \\
Recall & 81.5\% & 79.8\% & \textcolor{mlred}{-1.7\%} \\
\midrule
\multicolumn{4}{l}{\textcolor{Slate}{\textit{Fairness}}} \\
DP violation & 30.0\% & 4.8\% & \textcolor{mlgreen}{-84\%} \\
EO violation & 6.3\% & 3.2\% & \textcolor{mlgreen}{-49\%} \\
ROC distance & 7.2\% & 2.1\% & \textcolor{mlgreen}{-71\%} \\
\midrule
\multicolumn{4}{l}{\textcolor{Slate}{\textit{Calibration}}} \\
Calibration error & 1.2\% & 1.8\% & \textcolor{mlred}{+0.6\%} \\
\bottomrule
\end{tabular}
\end{center}

\vspace{0.3cm}
\textcolor{Amber}{\textbf{Pattern Analysis:}}

\small
\begin{itemize}
\item \textbf{Small performance cost:} 2.7\% accuracy loss
\item \textbf{Huge fairness gain:} 84\% DP reduction
\item \textbf{Multi-metric improvement:} EO, ROC both improve
\item \textbf{Minimal calibration impact:} +0.6\% only
\end{itemize}

\vspace{0.3cm}
\textcolor{Teal}{\textbf{Return on investment:}}\\
-2.7\% accuracy for -84\% bias\\
= \textcolor{mlgreen}{\textbf{31x fairness return}}

\column{0.43\textwidth}
\textcolor{Teal}{\textbf{Impact on People}}

\small
\textbf{Redistribution analysis:}

\vspace{0.3cm}
\textbf{Group A (was advantaged):}
\begin{itemize}
\item Before: 3,750/5,000 (75\%)
\item After: 3,615/5,000 (72.3\%)
\item Change: -135 approvals
\end{itemize}

\vspace{0.3cm}
\textbf{Group B (was disadvantaged):}
\begin{itemize}
\item Before: 2,250/5,000 (45\%)
\item After: 3,385/5,000 (67.7\%)
\item Change: +1,135 approvals
\end{itemize}

\vspace{0.3cm}
\textbf{Overall impact:}
\begin{itemize}
\item Total: +1,000 net approvals
\item More inclusive lending
\item 270 additional errors (vs 10,000)
\item 2.7\% error rate for 1,135 opportunities
\end{itemize}

\vspace{0.3cm}
\textcolor{DarkTeal}{\textbf{Statistical significance:}}
\begin{itemize}
\item DP reduction: p < 0.001 (highly significant)
\item Accuracy loss: p < 0.01 (significant but small)
\item Effect size: Cohen's d = 0.82 (large)
\end{itemize}

\vspace{0.3cm}
\begin{tcolorbox}[colback=mlgreen!20, colframe=mlgreen]
\centering
\small
\textbf{Validation successful}\\
Optimization delivers fairness\\
with acceptable accuracy cost
\end{tcolorbox}
\end{columns}

\vspace{0.5em}
\begin{tcolorbox}[colback=mlblue!10, colframe=mlblue]
\textbf{Key Insight:} Experimental validation: -2.7\% accuracy, -84\% bias (31x return) - optimization works in practice
\end{tcolorbox}

\vspace{0.5em}
\textbf{Key Question:} How do we implement this in production systems?

\bottomnote{CRITICAL Beat #8: Before/after experimental validation with actual numbers - proves approach works}
\end{frame}

% Slide 21: Implementation - Fairlearn Code
\begin{frame}[t,fragile]{Implementation: Fairlearn Constrained Optimization (30 Lines)}
\textbf{Complete working implementation of constrained fairness:}

\vspace{0.3em}

\begin{columns}[T]
\column{0.55\textwidth}
\textcolor{ForestGreen}{\textbf{The Code}}

\tiny
\begin{lstlisting}[language=Python, basicstyle=\ttfamily\tiny, breaklines=true]
# Fairlearn: Constrained Fairness Optimization
from fairlearn.reductions import ExponentiatedGradient
from fairlearn.reductions import DemographicParity, EqualizedOdds
from fairlearn.metrics import demographic_parity_difference
from sklearn.linear_model import LogisticRegression
from sklearn.metrics import accuracy_score
import pandas as pd

# Load loan dataset
df = pd.read_csv('loan_data.csv')
X = df[['income', 'credit_score', 'debt_ratio', 'employment']]
y = df['approved']  # True creditworthiness
A = df['protected_attribute']  # Race, gender, etc.

# Split data
from sklearn.model_selection import train_test_split
X_train, X_test, y_train, y_test, A_train, A_test = \
    train_test_split(X, y, A, test_size=0.2, random_state=42)

# Step 1: Define objective (maximize accuracy)
estimator = LogisticRegression(solver='lbfgs', max_iter=500)

# Step 2: Add fairness constraint
constraint = DemographicParity(difference_bound=0.05)
# Alternative: EqualizedOdds(difference_bound=0.05)

# Step 3: Solve constrained optimization
# ExponentiatedGradient implements Lagrangian approach
mitigator = ExponentiatedGradient(
    estimator,
    constraints=constraint,
    eps=0.05  # epsilon tolerance
)

# Fit with sensitive features
mitigator.fit(X_train, y_train, sensitive_features=A_train)

# Predict
y_pred = mitigator.predict(X_test)

# Evaluate
accuracy = accuracy_score(y_test, y_pred)
dp_diff = demographic_parity_difference(
    y_test, y_pred, sensitive_features=A_test
)

print(f"Accuracy: {accuracy:.2%}")
print(f"DP violation: {dp_diff:.2%}")
print(f"Constraint satisfied: {abs(dp_diff) <= 0.05}")
\end{lstlisting}

\column{0.43\textwidth}
\textcolor{Teal}{\textbf{Output}}

\small
\textbf{Console output:}

\tiny
\begin{tcolorbox}[colback=lightgray, colframe=Teal, width=0.95\textwidth]
\texttt{Loading loan\_data.csv... 10000 samples}\\
\texttt{Training constrained model...}\\
\texttt{Iteration 1: acc=0.84, dp=0.25}\\
\texttt{Iteration 2: acc=0.83, dp=0.15}\\
\texttt{Iteration 3: acc=0.825, dp=0.08}\\
\texttt{Iteration 4: acc=0.823, dp=0.048}\\
\texttt{Converged!}\\
\texttt{}\\
\texttt{Accuracy: 82.3\%}\\
\texttt{DP violation: 4.8\%}\\
\texttt{Constraint satisfied: True}\\
\texttt{}\\
\texttt{Baseline (unconstrained):}\\
\texttt{Accuracy: 85.0\%, DP: 30.0\%}\\
\texttt{}\\
\texttt{Improvement:}\\
\texttt{-2.7\% accuracy for -84\% bias}\\
\texttt{31x fairness return!}
\end{tcolorbox}

\vspace{0.3cm}
\textcolor{ForestGreen}{\textbf{Key features:}}

\small
\begin{itemize}
\item Works with any sklearn estimator
\item Multiple fairness constraints available
\item Automatic Lagrangian optimization
\item Iterative convergence (4 iterations)
\item Production-ready
\end{itemize}

\vspace{0.3cm}
\textcolor{Amber}{\textbf{Extensions:}}
\begin{itemize}
\item \texttt{EqualizedOdds}: Equal TPR+FPR
\item \texttt{TruePositiveRateParity}: Equal opportunity
\item \texttt{FalsePositiveRateParity}: Equal FPR
\item Custom constraints possible
\end{itemize}
\end{columns}

\vspace{0.5em}
\begin{tcolorbox}[colback=mlblue!10, colframe=mlblue]
\textbf{Key Insight:} 30 lines of code implements entire optimization framework - from mathematics to production
\end{tcolorbox}

\vspace{0.5em}
\textbf{Key Question:} What modern tools embed this approach in production systems?

\bottomnote{Fairlearn ExponentiatedGradient: Implements Lagrangian optimization with automatic lambda tuning}
\end{frame}
% ACT 4: PRODUCTION FAIRNESS SYSTEMS (4 slides)
% Theme: From mathematics to deployed ethical AI

% Slide 22: Complete Fairness Architecture
\begin{frame}[t]{The Complete Production Fairness Architecture}
\textbf{Four-layer system for ethical AI in production:}

\vspace{0.3em}

\begin{center}
\begin{tcolorbox}[colback=LightGreen!20, colframe=ForestGreen, width=0.85\textwidth]
\small

\textcolor{ForestGreen}{\textbf{Layer 1: Bias Detection}}\\
\textcolor{Slate}{\textit{(Make invisible visible)}}\\
\textbf{Components:} Disaggregated metrics, statistical tests, drift detection\\
\textbf{Tools:} Fairlearn MetricFrame, AIF360 metrics\\
\textbf{Output:} Bias reports, violation alerts\\
\textbf{Time:} Real-time monitoring

\vspace{0.3cm}
$\downarrow$ \textit{Detected violations trigger mitigation}

\vspace{0.3cm}
\textcolor{Teal}{\textbf{Layer 2: Fairness Optimization}}\\
\textcolor{Slate}{\textit{(Constrained learning)}}\\
\textbf{Components:} Lagrangian optimization, threshold tuning, reweighing\\
\textbf{Tools:} Fairlearn ExponentiatedGradient, AIF360 mitigation\\
\textbf{Output:} Fair models (DP/EO constraints satisfied)\\
\textbf{Time:} Training pipeline

\vspace{0.3cm}
$\downarrow$ \textit{Fair predictions need explanation}

\vspace{0.3cm}
\textcolor{Amber}{\textbf{Layer 3: Explainability}}\\
\textcolor{Slate}{\textit{(Interpretable decisions)}}\\
\textbf{Components:} SHAP values, counterfactual explanations, feature importance\\
\textbf{Tools:} SHAP, LIME, What-If Tool, Fairlearn dashboards\\
\textbf{Output:} Per-decision explanations, model cards\\
\textbf{Time:} Inference + documentation

\vspace{0.3cm}
$\downarrow$ \textit{Continuous monitoring ensures drift detection}

\vspace{0.3cm}
\textcolor{DarkTeal}{\textbf{Layer 4: Production Monitoring}}\\
\textcolor{Slate}{\textit{(Auditing \& accountability)}}\\
\textbf{Components:} Continuous auditing, performance tracking, incident response\\
\textbf{Tools:} MLflow, TensorBoard, custom dashboards\\
\textbf{Output:} Audit trails, compliance reports, alerts\\
\textbf{Time:} 24/7 automated

\vspace{0.3cm}
\textbf{Complete Pipeline Time: Detection (real-time) + Mitigation (training) + Explanation (<100ms) + Monitoring (continuous)}
\end{tcolorbox}
\end{center}

\vspace{0.5em}
\begin{tcolorbox}[colback=mlblue!10, colframe=mlblue]
\textbf{Key Insight:} Production fairness requires 4 layers working together - not just algorithms, but complete systems
\end{tcolorbox}

\vspace{0.5em}
\textbf{Key Question:} What modern tools implement these layers?

\bottomnote{Four-layer architecture parallels Week 6 structure pattern: Detection/Optimization/Explanation/Monitoring}
\end{frame}

% Slide 23: Modern Fairness Tools (2024-2025)
\begin{frame}[t]{Modern Fairness Tools in Production (2024-2025)}
\textbf{Three major platforms with 4-layer breakdown:}

\vspace{0.3em}

\begin{columns}[T]
\column{0.31\textwidth}
\textcolor{ForestGreen}{\textbf{Microsoft Fairlearn}}

\small
\textbf{Detection Layer:}
\begin{itemize}
\item MetricFrame (disaggregated)
\item 40+ fairness metrics
\item Drift detection
\end{itemize}

\textbf{Optimization Layer:}
\begin{itemize}
\item ExponentiatedGradient
\item GridSearch
\item ThresholdOptimizer
\item 5+ mitigation algorithms
\end{itemize}

\textbf{Explainability Layer:}
\begin{itemize}
\item Interactive dashboards
\item Group fairness plots
\item Trade-off visualization
\end{itemize}

\textbf{Monitoring Layer:}
\begin{itemize}
\item Model comparison
\item A/B testing support
\item Logging integration
\end{itemize}

\vspace{0.2cm}
\textcolor{Amber}{\textbf{Best for:}}\\
Azure ML, sklearn integration,\\
enterprise deployment

\column{0.31\textwidth}
\textcolor{Teal}{\textbf{IBM AIF360}}

\small
\textbf{Detection Layer:}
\begin{itemize}
\item 70+ bias metrics
\item Intersectional analysis
\item Pre/in/post-processing
\end{itemize}

\textbf{Optimization Layer:}
\begin{itemize}
\item 10+ mitigation algorithms
\item Prejudice remover
\item Adversarial debiasing
\item Calibrated eq. odds
\end{itemize}

\textbf{Explainability Layer:}
\begin{itemize}
\item Contrastive explanations
\item Prototypes/criticisms
\item Local/global interpretability
\end{itemize}

\textbf{Monitoring Layer:}
\begin{itemize}
\item Benchmark datasets
\item Performance tracking
\item Compliance reporting
\end{itemize}

\vspace{0.2cm}
\textcolor{Amber}{\textbf{Best for:}}\\
Research, comprehensive\\
metrics, legal compliance

\column{0.31\textwidth}
\textcolor{DarkTeal}{\textbf{Google What-If Tool}}

\small
\textbf{Detection Layer:}
\begin{itemize}
\item Visual exploration
\item Slice-based analysis
\item Performance gaps
\end{itemize}

\textbf{Optimization Layer:}
\begin{itemize}
\item Interactive threshold tuning
\item Cost/benefit analysis
\item Real-time adjustment
\end{itemize}

\textbf{Explainability Layer:}
\begin{itemize}
\item Individual counterfactuals
\item Feature attribution
\item Partial dependence
\item SHAP integration
\end{itemize}

\textbf{Monitoring Layer:}
\begin{itemize}
\item TensorBoard integration
\item Dataset comparison
\item Model versioning
\end{itemize}

\vspace{0.2cm}
\textcolor{Amber}{\textbf{Best for:}}\\
Interactive exploration,\\
TensorFlow/Keras models
\end{columns}

\vspace{0.3cm}
\begin{center}
\Large\textcolor{mlgreen}{\textbf{All three implement constrained optimization + monitoring!}}
\end{center}

\vspace{0.5em}
\begin{tcolorbox}[colback=mlblue!10, colframe=mlblue]
\textbf{Key Insight:} Modern tools (Fairlearn, AIF360, What-If) all provide 4-layer architecture - mathematics to production
\end{tcolorbox}

\vspace{0.5em}
\textbf{Key Question:} What lessons transfer beyond AI fairness?

\bottomnote{2024-2025 production tools: Fairlearn (Azure), AIF360 (IBM), What-If Tool (Google) - all open source}
\end{frame}

% Slide 24: Transferable Lessons
\begin{frame}[t]{Four Transferable Lessons Beyond AI Fairness}
\textbf{Universal principles that apply across domains:}

\vspace{0.3em}

\begin{columns}[T]
\column{0.48\textwidth}
\textcolor{ForestGreen}{\textbf{Lesson 1: Invisible Problems Need Measurement Frameworks}}

\small
\textbf{Principle:}\\
Can't manage what you can't measure\\
Hidden discrimination requires explicit metrics

\textbf{AI Fairness:}\\
I(D; A), demographic parity, equal opportunity

\textbf{Transfers to:}
\begin{itemize}
\item \textbf{Climate change:} Carbon accounting, GHG metrics
\item \textbf{Inequality:} Gini coefficient, wealth gaps
\item \textbf{Health disparities:} Life expectancy by demographics
\item \textbf{Education:} Achievement gaps, access metrics
\item \textbf{Organizational:} Pay equity audits, promotion rates
\end{itemize}

\vspace{0.3cm}
\textcolor{ForestGreen}{\textbf{Lesson 2: Multiple Metrics Reveal Trade-offs}}

\small
\textbf{Principle:}\\
No single metric captures full picture\\
Multiple perspectives reveal tensions

\textbf{AI Fairness:}\\
DP vs EO vs calibration impossibility

\textbf{Transfers to:}
\begin{itemize}
\item \textbf{Policy:} Efficiency vs equity vs sustainability
\item \textbf{Business:} Profit vs growth vs risk
\item \textbf{Engineering:} Speed vs quality vs cost
\item \textbf{Healthcare:} Individual vs population outcomes
\item \textbf{Security:} Privacy vs surveillance vs safety
\end{itemize}

\column{0.48\textwidth}
\textcolor{Teal}{\textbf{Lesson 3: Mathematics Constrains, Values Choose}}

\small
\textbf{Principle:}\\
Math reveals what's possible\\
Humans choose what matters

\textbf{AI Fairness:}\\
Impossibility theorems + stakeholder values → λ

\textbf{Transfers to:}
\begin{itemize}
\item \textbf{Resource allocation:} Pareto efficiency + priorities
\item \textbf{Risk management:} VaR limits + risk appetite
\item \textbf{Urban planning:} Capacity constraints + community goals
\item \textbf{Budgeting:} Financial limits + strategic priorities
\item \textbf{Triage:} Medical capacity + ethical frameworks
\end{itemize}

\vspace{0.3cm}
\textcolor{DarkTeal}{\textbf{Lesson 4: Optimization Makes Trade-offs Explicit}}

\small
\textbf{Principle:}\\
Implicit choices create hidden bias\\
Explicit optimization creates accountability

\textbf{AI Fairness:}\\
Lagrangian L(θ, λ) makes λ visible

\textbf{Transfers to:}
\begin{itemize}
\item \textbf{Government:} Transparent policy trade-offs
\item \textbf{Finance:} Explicit risk-return preferences
\item \textbf{Procurement:} Multi-objective decision criteria
\item \textbf{Design:} User needs vs technical constraints
\item \textbf{Strategy:} Cost-benefit analysis documentation
\end{itemize}
\end{columns}

\vspace{0.5em}
\begin{tcolorbox}[colback=mlblue!10, colframe=mlblue]
\textbf{Key Insight:} Four lessons transcend AI - fundamental principles for managing complexity with measurement and optimization
\end{tcolorbox}

\vspace{0.5em}
\textbf{Key Question:} What have we learned from this complete journey?

\bottomnote{Transferable principles: Measurement, trade-offs, values, optimization - apply to policy, business, design, systems}
\end{frame}

% Slide 25: Summary - The Complete Journey
\begin{frame}[t]{From Hidden Bias to Visible Fairness: The Complete Journey}
\textbf{What you now understand about fairness and ethical AI:}

\vspace{0.3em}

\begin{columns}[T]
\column{0.48\textwidth}
\textcolor{ForestGreen}{\textbf{The Problem (Acts 1-2)}}

\small
\textbf{Act 1: The Hidden Harm}
\begin{itemize}
\item Invisible discrimination (I(D; A) > 0)
\item Unmeasurable at scale (21.2 bits, only 4.2 measured)
\item 233 incidents, \$10.4B, 6.2M people affected (2024)
\item Can't fix what you can't see
\end{itemize}

\vspace{0.3cm}
\textbf{Act 2: First Measurements}
\begin{itemize}
\item Success: DP reveals 30\% bias, EO shows 6.3\%
\item Failure: Impossibility theorem (can't have all metrics)
\item Diagnosis: Metrics capture correlations, miss causation
\item Dilemma: 5 scenarios where metrics conflict
\end{itemize}

\vspace{0.3cm}
\textcolor{mlgray}{\textit{``Measurement makes visible, but reveals trade-offs''}}

\column{0.48\textwidth}
\textcolor{mlgreen}{\textbf{The Solution (Acts 3-4)}}

\small
\textbf{Act 3: Mathematical Fairness}
\begin{itemize}
\item Geometric view: ROC space, 7.2\% distance
\item Optimization: Lagrangian L(θ, λ), λ = 0.3 optimal
\item Validation: -2.7\% accuracy, -84\% bias (31x return)
\item Code: 30 lines Fairlearn implementation
\end{itemize}

\vspace{0.3cm}
\textbf{Act 4: Production Systems}
\begin{itemize}
\item 4-layer architecture: Detection/Optimization/Explanation/Monitoring
\item Modern tools: Fairlearn, AIF360, What-If Tool
\item Transferable lessons: Measurement, trade-offs, values, optimization
\end{itemize}

\vspace{0.3cm}
\textcolor{mlgray}{\textit{``Mathematics transforms impossible choice into auditable trade-off''}}
\end{columns}

\vspace{0.3cm}
\begin{center}
\begin{tcolorbox}[colback=mlgreen!20, colframe=mlgreen, width=0.85\textwidth]
\centering
\textbf{Core Takeaway:}\\
\\
Hidden discrimination (invisible) + Measurement (metrics)\\
+ Mathematics (optimization) = Visible fairness (auditable systems)\\
\\
\textbf{You can now build ethical AI that balances fairness and accuracy!}
\end{tcolorbox}
\end{center}

\vspace{0.5em}
\begin{tcolorbox}[colback=mlblue!10, colframe=mlblue]
\textbf{Key Insight:} Complete journey: Invisible → Measurable → Optimizable → Deployable - fairness through mathematics
\end{tcolorbox}

\vspace{0.5em}
\textbf{Next Week:} Structured Output and Prompt Engineering - Reliability requires constraints (like fairness does)

\bottomnote{Journey complete: Hidden bias (21.2 bits) to Visible metrics (DP, EO) to Mathematical optimization to Production systems}
\end{frame}

% Closing slide
\begin{frame}[plain]
\vspace{2cm}
\begin{center}
{\Huge \textcolor{ForestGreen}{\textbf{Fairness Mastered}}}\\[1cm]
{\Large From Hidden to Visible:}\\[0.5cm]
{\normalsize
You now understand:
\begin{itemize}
\item Why invisible bias causes systemic harm (I(D; A) > 0)
\item How metrics reveal discrimination (DP, EO, ROC space)
\item Why impossibility theorems constrain solutions
\item How optimization makes trade-offs explicit (Lagrangian)
\item How to build fair AI systems (Fairlearn, AIF360)
\end{itemize}
}
\vspace{1cm}
{\large \textcolor{ForestGreen}{\textbf{Next Week: Structured Output and Prompt Engineering}}}\\
{\normalsize Reliability requires constraints, just like fairness does}
\end{center}
\end{frame}

\end{document}