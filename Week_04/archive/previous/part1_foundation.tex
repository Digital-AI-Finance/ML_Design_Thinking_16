% Part 1: Foundation - Problem & Context (10 slides)
\section{Foundation: The Definition Challenge}

% Slide 1: Opening Power Chart
\begin{frame}[plain]
\centering
\includegraphics[width=0.85\textwidth]{charts/innovation_success_dashboard.pdf}
\end{frame}

% Slide 2: The Definition Challenge
\begin{frame}{The Definition Challenge}
\Large\textbf{Why Classification Matters in Innovation}
\normalsize

\vspace{0.5em}

\begin{columns}[T]
\begin{column}{0.48\textwidth}
\textbf{Human Intuition}
\begin{itemize}
\item ``I know success when I see it''
\item Based on experience
\item Subjective criteria
\item Limited by bias
\item Inconsistent across evaluators
\end{itemize}
\end{column}
\begin{column}{0.48\textwidth}
\textbf{Machine Classification}
\begin{itemize}
\item Data-driven definitions
\item Objective metrics
\item Consistent criteria
\item Learns from patterns
\item Scalable evaluation
\end{itemize}
\end{column}
\end{columns}

\vspace{1em}
\begin{tcolorbox}[colback=mlblue!10,colframe=mlblue]
\centering
Classification transforms subjective judgments into objective, repeatable decisions
\end{tcolorbox}
\end{frame}

% Slide 3: From Intuition to Data
\begin{frame}{From Intuition to Data-Driven Decisions}
\Large\textbf{The Evolution of Innovation Assessment}
\normalsize

\vspace{0.5em}

\begin{tikzpicture}[scale=0.9, transform shape]
% Timeline
\draw[thick,->] (0,0) -- (12,0);
\foreach \x/\year/\label in {0/Past/Gut Feel, 4/Present/Metrics, 8/Today/ML Models, 12/Future/AI Prediction} {
    \draw (\x,0) -- (\x,-0.1) node[below] {\small \year};
    \draw (\x,0.5) node[above,text width=2.5cm,align=center] {\small \label};
}

% Boxes above timeline
\node[draw,fill=mlred!20,rounded corners] at (0,1.5) {Experience-Based};
\node[draw,fill=mlorange!20,rounded corners] at (4,1.5) {KPI-Driven};
\node[draw,fill=mlgreen!20,rounded corners] at (8,1.5) {Pattern Recognition};
\node[draw,fill=mlblue!20,rounded corners] at (12,1.5) {Predictive Analytics};

% Accuracy percentages
\node at (0,2.5) {\accuracy{45}};
\node at (4,2.5) {\accuracy{65}};
\node at (8,2.5) {\accuracy{82}};
\node at (12,2.5) {\accuracy{94}};
\end{tikzpicture}

\vspace{1em}
\begin{columns}[T]
\begin{column}{0.32\textwidth}
\small
\textbf{Traditional}: Individual expertise, limited scale
\end{column}
\begin{column}{0.32\textwidth}
\small
\textbf{Analytics}: Dashboards, retrospective analysis
\end{column}
\begin{column}{0.32\textwidth}
\small
\textbf{ML-Powered}: Real-time, predictive, adaptive
\end{column}
\end{columns}
\end{frame}

% Slide 4: Success Patterns in Innovation
\begin{frame}{Success Patterns in Innovation}
\Large\textbf{What Makes Innovation Succeed?}
\normalsize

\begin{columns}[T]
\begin{column}{0.48\textwidth}
\includegraphics[width=0.85\textwidth]{charts/feature_space_visualization.pdf}

\small
Different features reveal different patterns
\end{column}
\begin{column}{0.48\textwidth}
\textbf{Key Success Factors}
\begin{itemize}
\item \textcolor{mlblue}{Novelty Score}: How unique?
\item \textcolor{mlgreen}{Market Size}: How big is opportunity?
\item \textcolor{mlorange}{Team Experience}: Who's building?
\item \textcolor{mlred}{Development Time}: How fast?
\item \textcolor{mlpurple}{User Testing}: How validated?
\end{itemize}

\vspace{0.5em}
\textbf{The Classification Question:}
\begin{tcolorbox}[colback=mlyellow!20,colframe=mlorange]
Can we predict success from early indicators?
\end{tcolorbox}
\end{column}
\end{columns}
\end{frame}

% Slide 5: Binary vs Multi-Class Perspective
\begin{frame}{Binary vs Multi-Class Perspectives}
\Large\textbf{Different Lenses for Success}
\normalsize

\vspace{0.5em}

\begin{columns}[T]
\begin{column}{0.48\textwidth}
\textbf{Binary Classification}
\begin{center}
\begin{tikzpicture}[scale=0.8]
\draw[fill=mlred!30] (0,0) circle (1.5cm);
\draw[fill=mlgreen!30] (3,0) circle (1.5cm);
\node at (0,0) {\textbf{Fail}};
\node at (3,0) {\textbf{Success}};
\node at (0,-2) {40\%};
\node at (3,-2) {60\%};
\end{tikzpicture}
\end{center}

\textbf{Simple Decision:}
\begin{itemize}
\item Launch or not?
\item Invest or pass?
\item Continue or pivot?
\end{itemize}
\end{column}
\begin{column}{0.48\textwidth}
\textbf{Multi-Class Classification}
\begin{center}
\begin{tikzpicture}[scale=0.8]
\draw[fill=mlred!30] (0,0) -- (1,0) -- (1,1) -- (0,1) -- cycle;
\draw[fill=mlorange!30] (1,0) -- (2,0) -- (2,1) -- (1,1) -- cycle;
\draw[fill=mlblue!30] (0,1) -- (1,1) -- (1,2) -- (0,2) -- cycle;
\draw[fill=mlgreen!30] (1,1) -- (2,1) -- (2,2) -- (1,2) -- cycle;
\node at (0.5,0.5) {\small Failed};
\node at (1.5,0.5) {\small Struggling};
\node at (0.5,1.5) {\small Growing};
\node at (1.5,1.5) {\small Breakthrough};
\end{tikzpicture}
\end{center}

\textbf{Nuanced Understanding:}
\begin{itemize}
\item Resource allocation
\item Support intensity
\item Risk assessment
\end{itemize}
\end{column}
\end{columns}

\vspace{0.5em}
\begin{center}
\textit{Same data, different questions → different insights}
\end{center}
\end{frame}

% Slide 6: The Innovation Dataset
\begin{frame}{The Innovation Dataset}
\Large\textbf{9,500 Real Innovation Products}
\normalsize

\vspace{0.5em}

\begin{columns}[T]
\begin{column}{0.55\textwidth}
\includegraphics[width=0.85\textwidth]{charts/innovation_dataset_overview.pdf}
\end{column}
\begin{column}{0.43\textwidth}
\textbf{Dataset Characteristics:}
\begin{itemize}
\item \textbf{Products}: 9,500 innovations
\item \textbf{Features}: 27 dimensions
\item \textbf{Timespan}: 2015-2023
\item \textbf{Segments}: 6 categories
\end{itemize}

\vspace{0.5em}
\textbf{Feature Categories:}
\begin{enumerate}
\small
\item Innovation metrics
\item Market factors
\item Team attributes
\item Development process
\item Financial indicators
\end{enumerate}

\vspace{0.5em}
\textcolor{mlgray}{\small Note: Simulated for educational purposes}
\end{column}
\end{columns}
\end{frame}

% Slide 7: Learning Objectives
\begin{frame}{Learning Objectives}
\Large\textbf{What You'll Master Today}
\normalsize

\vspace{1em}

\begin{columns}[T]
\begin{column}{0.48\textwidth}
\textbf{Technical Skills}
\begin{itemize}
\item Build classification models
\item Evaluate model performance
\item Handle imbalanced data
\item Tune hyperparameters
\item Interpret predictions
\end{itemize}

\vspace{0.5em}
\textbf{Algorithms Covered}
\begin{itemize}
\item Logistic Regression
\item Decision Trees
\item Random Forests
\item Support Vector Machines
\item Neural Networks
\end{itemize}
\end{column}
\begin{column}{0.48\textwidth}
\textbf{Design Applications}
\begin{itemize}
\item Product success prediction
\item User segment classification
\item Risk assessment tools
\item A/B test evaluation
\item Portfolio optimization
\end{itemize}

\vspace{0.5em}
\textbf{Key Metrics}
\begin{itemize}
\item Accuracy \& Precision
\item Recall \& F1-Score
\item ROC-AUC
\item Confusion Matrices
\item Learning Curves
\end{itemize}
\end{column}
\end{columns}

\vspace{1em}
\begin{tcolorbox}[colback=mlgreen!10,colframe=mlgreen]
\centering
By session end: Build a complete innovation success predictor
\end{tcolorbox}
\end{frame}

% Slide 8: Real-World Impact
\begin{frame}{Real-World Impact}
\Large\textbf{Where Classification Drives Innovation}
\normalsize

\vspace{0.5em}

\begin{columns}[T]
\begin{column}{0.32\textwidth}
\textbf{Startup Funding}
\begin{center}
\includegraphics[width=0.8\textwidth]{charts/precision_recall_curves.pdf}
\end{center}
\small
VCs use ML to predict unicorns from pitch decks
\end{column}
\begin{column}{0.32\textwidth}
\textbf{Product Launch}
\begin{itemize}
\small
\item Feature prioritization
\item Market timing
\item Resource allocation
\item Risk mitigation
\end{itemize}
\vspace{0.5em}
\accuracy{85} prediction accuracy
\end{column}
\begin{column}{0.32\textwidth}
\textbf{User Experience}
\begin{itemize}
\small
\item Personalization
\item Churn prediction
\item Satisfaction scoring
\item Behavior clustering
\end{itemize}
\vspace{0.5em}
3x improvement in retention
\end{column}
\end{columns}

\vspace{1em}
\begin{center}
\begin{tikzpicture}
\node[draw,fill=challenge!20,rounded corners,text width=10cm,align=center] {
\textbf{Classification in Practice:} Amazon uses 100+ classifiers for product recommendations,
Netflix for content suggestions, Spotify for music discovery
};
\end{tikzpicture}
\end{center}
\end{frame}

% Slide 9: The Innovation Diamond Connection
\begin{frame}{Connecting to the Innovation Diamond}
\Large\textbf{Classification in the Innovation Journey}
\normalsize

\vspace{0.5em}

\begin{center}
\begin{tikzpicture}[scale=0.8, transform shape]
% Diamond shape
\coordinate (top) at (0,3);
\coordinate (left) at (-3,0);
\coordinate (right) at (3,0);
\coordinate (bottom) at (0,-3);

% Draw diamond
\draw[thick,fill=challenge!10] (top) -- (left) -- (bottom) -- (right) -- cycle;

% Labels at vertices
\node[above] at (top) {\textbf{5000 Ideas}};
\node[left] at (left) {\textbf{Diverge}};
\node[right] at (right) {\textbf{Converge}};
\node[below] at (bottom) {\textbf{5 Strategies}};

% Classification points
\node[draw,circle,fill=mlred!50] at (-1.5,1.5) {1};
\node[draw,circle,fill=mlorange!50] at (0,1) {2};
\node[draw,circle,fill=mlgreen!50] at (1.5,1.5) {3};
\node[draw,circle,fill=mlblue!50] at (0,-1) {4};

% Legend
\node[right,text width=4cm] at (4,2) {
\small
\textcolor{mlred}{\textbf{1. Initial Filter:}} Binary classification of viable ideas
};
\node[right,text width=4cm] at (4,0.5) {
\small
\textcolor{mlorange}{\textbf{2. Category Sort:}} Multi-class by innovation type
};
\node[right,text width=4cm] at (4,-1) {
\small
\textcolor{mlgreen}{\textbf{3. Quality Score:}} Probability of success
};
\node[right,text width=4cm] at (4,-2.5) {
\small
\textcolor{mlblue}{\textbf{4. Final Selection:}} Top 5 predictions
};
\end{tikzpicture}
\end{center}
\end{frame}

% Slide 10: Section Summary
\begin{frame}{Foundation Summary}
\Large\textbf{Key Takeaways}
\normalsize

\vspace{1em}

\begin{columns}[T]
\begin{column}{0.48\textwidth}
\textbf{The Problem}
\begin{itemize}
\item Innovation success is hard to predict
\item Human judgment is biased
\item Scale requires automation
\item Patterns exist in data
\end{itemize}

\vspace{0.5em}
\textbf{The Opportunity}
\begin{itemize}
\item Learn from historical data
\item Identify success patterns
\item Make consistent decisions
\item Scale evaluation process
\end{itemize}
\end{column}
\begin{column}{0.48\textwidth}
\textbf{The Approach}
\begin{itemize}
\item Classification algorithms
\item Multiple perspectives (binary/multi)
\item Rigorous evaluation
\item Interpretable results
\end{itemize}

\vspace{0.5em}
\textbf{Next: Technical Deep Dive}
\begin{tcolorbox}[colback=mlblue!10,colframe=mlblue]
How do classification algorithms actually work?
\end{tcolorbox}
\end{column}
\end{columns}

\vspace{1em}
\begin{center}
\Large
\textit{``Prediction is difficult, especially about the future''} \\
\normalsize
- But ML makes it systematically better
\end{center}
\end{frame}