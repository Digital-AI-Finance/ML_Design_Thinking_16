\documentclass[8pt,aspectratio=169]{beamer}
\usetheme{Madrid}
\usepackage{graphicx}
\usepackage{booktabs}
\usepackage{adjustbox}
\usepackage{multicol}
\usepackage{amsmath}
\usepackage{xcolor}
\usepackage{tcolorbox}

% Color definitions
\definecolor{mlblue}{RGB}{31, 119, 180}
\definecolor{mlorange}{RGB}{255, 127, 14}
\definecolor{mlgreen}{RGB}{44, 160, 44}
\definecolor{mlred}{RGB}{214, 39, 40}
\definecolor{mlpurple}{RGB}{148, 103, 189}

% Remove navigation symbols
\setbeamertemplate{navigation symbols}{}

% Clean itemize/enumerate
\setbeamertemplate{itemize items}[circle]
\setbeamertemplate{enumerate items}[default]

% Title information
\title{Week 1: How do we truly understand our users?}
\subtitle{From overwhelming data to clear insights}
\author{ML/AI/GenAI for Design Thinking}
\institute{BSc Course - 12 Week Program}
\date{2024}

\begin{document}

% Title slide
\begin{frame}
\titlepage
\end{frame}

% Section 1: The Problem (5 slides)

% Slide 2: The Challenge
\begin{frame}[t]{The Challenge: Information Overload}
\begin{center}
\includegraphics[width=0.7\textwidth]{charts/clustering_demo.pdf}
\end{center}
\vspace{0.5em}
\begin{itemize}
\item A startup has \textbf{10,000 product reviews}
\item Reading all = 83 hours = \textbf{2 full weeks}
\item Critical insights hidden in the noise
\end{itemize}
\end{frame}

% Slide 3: Traditional Approach
\begin{frame}[t]{Traditional Approach: Manual Sampling}
\begin{columns}[T]
\column{0.5\textwidth}
\textbf{What we typically do:}
\begin{itemize}
\item Read 100 random reviews (1\%)
\item Take notes on patterns
\item Draw conclusions
\item Hope it represents all users
\end{itemize}

\column{0.5\textwidth}
\textbf{The problems:}
\begin{itemize}
\item Miss 99\% of insights
\item Confirmation bias
\item Random sampling misses minorities
\item Time-consuming even for 1\%
\end{itemize}
\end{columns}

\vspace{1em}
\begin{tcolorbox}[colback=mlred!20]
\centering
\textbf{Result: Incomplete understanding, missed opportunities}
\end{tcolorbox}
\end{frame}

% Slide 4: The Cost
\begin{frame}[t]{The Hidden Cost: What We're Missing}
\begin{columns}[T]
\column{0.33\textwidth}
\textbf{Hidden Segments}
\begin{itemize}
\item Power users (5\%)
\item Edge cases (8\%)
\item Silent majority (40\%)
\end{itemize}

\column{0.33\textwidth}
\textbf{Unexpected Uses}
\begin{itemize}
\item Creative workarounds
\item Unintended features
\item New market opportunities
\end{itemize}

\column{0.33\textwidth}
\textbf{Critical Issues}
\begin{itemize}
\item Rare but severe bugs
\item Accessibility problems
\item Cultural differences
\end{itemize}
\end{columns}

\vspace{1em}
\begin{center}
\Large\textcolor{mlred}{\textbf{We're designing blind!}}
\end{center}
\end{frame}

% Slide 5: Today's Mission
\begin{frame}[t]{Today's Mission: ML-Powered Understanding}
\begin{center}
\Large\textbf{From This...}\\
\vspace{0.5em}
\normalsize
Reading 100 reviews → Guessing about 10,000 users\\
\vspace{1em}
\Large\textcolor{mlgreen}{\textbf{...To This}}\\
\vspace{0.5em}
\normalsize
ML analyzes 10,000 reviews → Clear user segments\\
\vspace{1em}
\end{center}

\begin{columns}[T]
\column{0.5\textwidth}
\centering
\textbf{What you'll learn:}
\begin{itemize}
\item K-means clustering
\item Digital empathy at scale
\item Pattern discovery
\end{itemize}

\column{0.5\textwidth}
\centering
\textbf{What you'll achieve:}
\begin{itemize}
\item Hear ALL voices
\item Find hidden patterns
\item Make data-driven decisions
\end{itemize}
\end{columns}
\end{frame}

% Section 2: ML Concept - Clustering (10 slides)

% Slide 6: What is Clustering
\begin{frame}[t]{What is Clustering?}
\begin{center}
\Large\textbf{Finding natural groups without labels}
\end{center}

\vspace{1em}
\textbf{Real-world analogy: Sorting cookies}
\begin{itemize}
\item You have 100 mixed cookies
\item No labels or categories given
\item You naturally group by similarity:
\begin{itemize}
\item Chocolate chip together
\item Sugar cookies together
\item Oatmeal raisin together
\end{itemize}
\item Groups emerge from characteristics
\end{itemize}

\vspace{0.5em}
\begin{tcolorbox}[colback=mlblue!20]
\textbf{Clustering does this automatically with data!}
\end{tcolorbox}
\end{frame}

% Slide 7: K-means Step 1
\begin{frame}[t]{How K-means Works: Step 1 - Initialize}
\begin{center}
\includegraphics[width=0.7\textwidth]{charts/kmeans_process.pdf}
\end{center}

\textbf{Step 1: Pick K random centers}
\begin{itemize}
\item K = number of groups we want
\item Start with random positions
\item These are ``centroids'' (group centers)
\end{itemize}

\textcolor{mlblue}{Example: K=3 for three user types}
\end{frame}

% Slide 8: K-means Step 2
\begin{frame}[t]{How K-means Works: Step 2 - Assign}
\textbf{Step 2: Assign points to nearest center}

\begin{columns}[T]
\column{0.5\textwidth}
\textbf{For each data point:}
\begin{enumerate}
\item Calculate distance to all centers
\item Assign to closest center
\item Point joins that cluster
\end{enumerate}

\column{0.5\textwidth}
\textbf{Distance = Similarity}
\begin{itemize}
\item Close = similar
\item Far = different
\item Uses Euclidean distance
\end{itemize}
\end{columns}

\vspace{1em}
\begin{center}
\colorbox{mlgreen!20}{\textbf{Points naturally group with similar neighbors}}
\end{center}
\end{frame}

% Slide 9: K-means Step 3
\begin{frame}[t]{How K-means Works: Step 3 - Update and Repeat}
\textbf{Step 3: Recalculate centers and repeat}

\begin{enumerate}
\item Find new center of each cluster (mean position)
\item Centers move to better positions
\item Repeat assignment step
\item Continue until centers stop moving
\end{enumerate}

\vspace{0.5em}
\textbf{Convergence:}
\begin{itemize}
\item Usually 10-20 iterations
\item Centers stabilize
\item Clusters are final
\end{itemize}

\begin{tcolorbox}[colback=mlblue!20]
\textbf{Result: Natural groups discovered automatically!}
\end{tcolorbox}
\end{frame}

% Slide 10: The Math (Simplified)
\begin{frame}[t]{The Math (Simplified)}
\textbf{Only one concept to understand: Distance = Similarity}

\vspace{1em}
\begin{center}
\Large
$\text{Distance} = \sqrt{(x_1 - x_2)^2 + (y_1 - y_2)^2}$
\end{center}

\vspace{1em}
\textbf{In plain English:}
\begin{itemize}
\item How different are two reviews?
\item Smaller distance = more similar
\item We minimize total distance within clusters
\end{itemize}

\vspace{0.5em}
\textbf{That's all the math you need!}
\end{frame}

% Slide 11: Choosing K
\begin{frame}[t]{Choosing K: The Elbow Method}
\begin{center}
\includegraphics[width=0.6\textwidth]{charts/empathy_scale.pdf}
\end{center}

\textbf{How many clusters?}
\begin{itemize}
\item Try different K values (2, 3, 4, 5...)
\item Measure cluster quality
\item Look for the ``elbow'' - where improvement slows
\item Usually between 3-8 for user segmentation
\end{itemize}
\end{frame}

% Slide 12: Real Example
\begin{frame}[t]{Real Example: Clustering Product Reviews}
\textbf{Input: 10,000 smartphone reviews}

\vspace{0.5em}
\textbf{K-means discovers 5 clusters:}
\begin{enumerate}
\item \textcolor{mlgreen}{\textbf{Happy Users}} (35\%): ``love'', ``perfect'', ``amazing''
\item \textcolor{mlred}{\textbf{Frustrated Users}} (20\%): ``broken'', ``disappointed'', ``waste''
\item \textcolor{mlblue}{\textbf{Feature Requests}} (15\%): ``wish'', ``should'', ``needs''
\item \textcolor{mlorange}{\textbf{Bug Reports}} (10\%): ``crash'', ``freeze'', ``error''
\item \textcolor{mlpurple}{\textbf{Comparison Shoppers}} (20\%): ``better than'', ``compared to'', ``vs''
\end{enumerate}

\vspace{0.5em}
\begin{tcolorbox}[colback=mlgreen!20]
\textbf{Found without reading a single review!}
\end{tcolorbox}
\end{frame}

% Slide 13: What Clustering Finds
\begin{frame}[t]{What Clustering Reveals}
\textbf{Patterns invisible to humans:}

\begin{columns}[T]
\column{0.5\textwidth}
\textbf{Expected findings:}
\begin{itemize}
\item Happy vs unhappy
\item New vs returning users
\item Price-sensitive segments
\end{itemize}

\column{0.5\textwidth}
\textbf{Surprise discoveries:}
\begin{itemize}
\item Gift purchasers (different language)
\item Weekend vs weekday users
\item Seasonal patterns
\item Cultural differences
\end{itemize}
\end{columns}

\vspace{1em}
\begin{center}
\Large\textcolor{mlblue}{\textbf{ML sees what we miss}}
\end{center}
\end{frame}

% Slide 14: Unsupervised Learning
\begin{frame}[t]{Unsupervised Learning: No Labels Needed}
\begin{columns}[T]
\column{0.5\textwidth}
\textbf{Supervised Learning}
\begin{itemize}
\item Needs labeled examples
\item ``This is spam'' / ``This is not''
\item Learns from answers
\item Predicts categories
\end{itemize}

\column{0.5\textwidth}
\textbf{Unsupervised Learning}
\begin{itemize}
\item No labels required
\item Discovers structure
\item Finds patterns
\item Reveals unknown groups
\end{itemize}
\end{columns}

\vspace{1em}
\begin{tcolorbox}[colback=mlpurple!20]
\textbf{Perfect for exploration when you don't know what you're looking for!}
\end{tcolorbox}
\end{frame}

% Slide 15: When to Use Clustering
\begin{frame}[t]{When to Use Clustering in Design Thinking}
\textbf{Perfect for:}
\begin{itemize}
\item \textbf{User Segmentation}: Find distinct user groups
\item \textbf{Behavior Analysis}: Discover usage patterns
\item \textbf{Content Organization}: Group similar items
\item \textbf{Anomaly Detection}: Find outliers
\item \textbf{Feature Discovery}: Identify themes
\end{itemize}

\vspace{0.5em}
\textbf{Not great for:}
\begin{itemize}
\item When you need specific categories
\item Very small datasets (<100 points)
\item When groups overlap heavily
\end{itemize}
\end{frame}

% Section 3: Design - Empathize (8 slides)

% Slide 16: What is Empathy in Design
\begin{frame}[t]{What is Empathy in Design?}
\begin{center}
\Large\textbf{Understanding users' needs, feelings, and contexts}
\end{center}

\vspace{1em}
\textbf{The foundation of human-centered design:}
\begin{itemize}
\item Walk in users' shoes
\item Feel their frustrations
\item Understand their goals
\item Recognize their constraints
\end{itemize}

\vspace{0.5em}
\textbf{Why it matters:}
\begin{itemize}
\item Solve real problems, not assumed ones
\item Create products people actually want
\item Build emotional connections
\end{itemize}

\begin{tcolorbox}[colback=mlorange!20]
\textbf{``Fall in love with the problem, not the solution''}
\end{tcolorbox}
\end{frame}

% Slide 17: Traditional Empathy Methods
\begin{frame}[t]{Traditional Empathy Methods}
\begin{columns}[T]
\column{0.33\textwidth}
\textbf{Interviews}
\begin{itemize}
\item Deep insights
\item Personal stories
\item 5-20 people
\item 2 weeks effort
\end{itemize}

\column{0.33\textwidth}
\textbf{Observations}
\begin{itemize}
\item Natural behavior
\item Context matters
\item Time-intensive
\item Small sample
\end{itemize}

\column{0.33\textwidth}
\textbf{Surveys}
\begin{itemize}
\item Larger scale
\item Structured data
\item Surface-level
\item Response bias
\end{itemize}
\end{columns}

\vspace{1em}
\begin{center}
\textbf{The Tradeoff: Depth vs Scale}
\end{center}
\end{frame}

% Slide 18: The Scale Problem
\begin{frame}[t]{The Scale Problem in Modern Design Thinking}
\textbf{The Challenge of Digital Scale:}
\begin{itemize}
\item \textbf{1995}: Average product had 100 users to understand
\item \textbf{2024}: Average app has 100,000+ users globally
\item That's a \textbf{1000x increase} in complexity
\end{itemize}

\vspace{0.5em}
\textbf{Why Traditional Methods Break:}
\begin{itemize}
\item \textbf{Interviews}: 20 users = 0.02\% coverage
\item \textbf{Surveys}: 1000 responses = 1\% coverage  
\item \textbf{Focus Groups}: 50 people = 0.05\% coverage
\end{itemize}

\vspace{0.5em}
\begin{tcolorbox}[colback=mlred!20]
\textbf{We're making decisions based on less than 1\% of our users!}
\end{tcolorbox}
\end{frame}

% Slide 19: The Scale Challenge
\begin{frame}[t]{The Scale Challenge: Numbers Don't Lie}
\begin{center}
\includegraphics[width=0.75\textwidth]{charts/results_comparison.pdf}
\end{center}

\textbf{Modern Design Reality:}
\begin{columns}[T]
\column{0.5\textwidth}
\begin{itemize}
\item Millions of users globally
\item 50+ languages and cultures
\item 24/7 usage patterns
\item Mobile, desktop, tablet, voice
\end{itemize}

\column{0.5\textwidth}
\begin{itemize}
\item Rapid feedback cycles needed
\item Can't interview everyone
\item Can't observe everyone
\item But we CAN analyze everyone
\end{itemize}
\end{columns}
\end{frame}

% Slide 20: Digital Empathy
\begin{frame}[t]{Digital Empathy: Data as Voice}
\begin{center}
\Large\textbf{``Every click, review, and interaction tells a story''}
\end{center}

\vspace{1em}
\textbf{Digital footprints reveal:}
\begin{columns}[T]
\column{0.5\textwidth}
\begin{itemize}
\item What users do (behavior)
\item What they say (reviews)
\item When they struggle (errors)
\item What they want (searches)
\end{itemize}

\column{0.5\textwidth}
\begin{itemize}
\item How they feel (sentiment)
\item What they value (choices)
\item Where they quit (dropoff)
\item Why they return (loyalty)
\end{itemize}
\end{columns}

\vspace{0.5em}
\begin{tcolorbox}[colback=mlgreen!20]
\textbf{ML helps us listen to thousands of voices simultaneously}
\end{tcolorbox}
\end{frame}

% Slide 20: Human + Machine Partnership
\begin{frame}[t]{Human + Machine Partnership}
\begin{columns}[T]
\column{0.5\textwidth}
\textbf{What ML Does Well:}
\begin{itemize}
\item Process volume
\item Find patterns
\item Remove bias
\item Work 24/7
\item Quantify sentiment
\end{itemize}

\column{0.5\textwidth}
\textbf{What Humans Do Well:}
\begin{itemize}
\item Interpret meaning
\item Understand context
\item Feel emotion
\item Make connections
\item Create solutions
\end{itemize}
\end{columns}

\vspace{1em}
\begin{center}
\Large\textcolor{mlblue}{\textbf{Together: Scalable Empathy}}
\end{center}

\begin{itemize}
\item ML finds the patterns
\item Humans understand why they matter
\item Result: Deep insights at scale
\end{itemize}
\end{frame}

% Slide 21: Empathy at Scale Framework
\begin{frame}[t]{Empathy at Scale Framework}
\begin{center}
\Large\textbf{The 4-Step Process}
\end{center}

\vspace{0.5em}
\begin{enumerate}
\item \textbf{COLLECT} - Gather all user data
\begin{itemize}
\item Reviews, support tickets, analytics
\item The more diverse, the better
\end{itemize}

\item \textbf{CLUSTER} - Find natural groups
\begin{itemize}
\item Apply K-means or similar
\item Discover segments automatically
\end{itemize}

\item \textbf{INTERPRET} - Understand each segment
\begin{itemize}
\item What defines each group?
\item What are their needs?
\end{itemize}

\item \textbf{ACT} - Design for each segment
\begin{itemize}
\item Tailored solutions
\item Prioritized roadmap
\end{itemize}
\end{enumerate}
\end{frame}

% Slide 22: Real Company Example
\begin{frame}[t]{Real Example: Spotify's Discovery}
\textbf{The Challenge:}
\begin{itemize}
\item Millions of users, diverse tastes
\item How to personalize for everyone?
\end{itemize}

\vspace{0.5em}
\textbf{The ML Solution:}
\begin{itemize}
\item Clustered listening patterns
\item Found unexpected segment: ``Cooking Music'' (12\% of users)
\item Never explicitly requested
\item Pattern: Upbeat, no explicit lyrics, 30-45 min sessions
\end{itemize}

\vspace{0.5em}
\textbf{The Result:}
\begin{itemize}
\item Created ``Cooking'' playlist category
\item 8 million+ followers
\item Increased engagement 23\%
\end{itemize}

\begin{tcolorbox}[colback=mlgreen!20]
\textbf{ML revealed what users never told them directly}
\end{tcolorbox}
\end{frame}

% Slide 23: Avoiding Empathy Pitfalls
\begin{frame}[t]{Avoiding Digital Empathy Pitfalls}
\textbf{Remember:}
\begin{itemize}
\item \textbf{Data shows WHAT, not WHY}
\begin{itemize}
\item Always validate findings with real users
\item Follow up clusters with interviews
\end{itemize}

\item \textbf{Correlation is not causation}
\begin{itemize}
\item Ice cream sales and crime both increase in summer
\item Doesn't mean ice cream causes crime!
\end{itemize}

\item \textbf{Don't lose the human touch}
\begin{itemize}
\item Numbers tell a story, but people live it
\item Always connect data back to real humans
\end{itemize}

\item \textbf{Beware of echo chambers}
\begin{itemize}
\item Vocal minorities can dominate
\item Silent majority matters too
\end{itemize}
\end{itemize}
\end{frame}

% Section 4: Integration Demo (5 slides)

% Slide 24: Live Demo Setup
\begin{frame}[t]{Live Demo: Analyzing 10,000 Amazon Reviews}
\textbf{Our Dataset:}
\begin{itemize}
\item 10,000 product reviews for wireless headphones
\item Mix of 1-5 star ratings
\item Various review lengths
\item Real customer feedback
\end{itemize}

\vspace{0.5em}
\textbf{Preprocessing Steps:}
\begin{enumerate}
\item Convert text to numbers (TF-IDF)
\item Normalize features
\item Apply K-means with K=5
\item Analyze results
\end{enumerate}

\vspace{0.5em}
\begin{center}
\colorbox{mlblue!20}{\textbf{Let's see what patterns emerge...}}
\end{center}
\end{frame}

% Slide 25-26: Combined Demo Results
\begin{frame}[t]{Demo Results: 5 Segments Including Hidden Gem}
\begin{columns}[T]
\column{0.55\textwidth}
\textbf{K-means discovers:}
\begin{enumerate}
\item \textbf{Price-conscious} (22\%): ``value'', ``cheap''
\item \textbf{Quality focus} (28\%): ``sound'', ``bass''
\item \textbf{Convenience} (18\%): ``easy'', ``quick''
\item \textbf{Technical} (17\%): ``bluetooth'', ``battery''
\item \textbf{Gift buyers} (15\%): ``gift'', ``birthday''
\end{enumerate}

\column{0.45\textwidth}
\textbf{The Hidden 15\%:}
\begin{itemize}
\item Never said ``gift'' directly
\item Found via language patterns
\item Third-person references
\item Price insensitive
\item Seasonal patterns
\end{itemize}
\end{columns}

\vspace{0.5em}
\begin{tcolorbox}[colback=mlgreen!20]
\textbf{A whole market segment hiding in plain sight - worth millions in revenue!}
\end{tcolorbox}
\end{frame}

% Slide 26: From Discovery to Action  
\begin{frame}[t]{From Discovery to Design Decisions}
\textbf{Segment-specific strategies:}

\begin{columns}[T]
\column{0.5\textwidth}
\textbf{For Price-conscious (22\%):}
\begin{itemize}
\item Highlight value propositions
\item Compare to competitors
\item Offer bundles
\end{itemize}

\textbf{For Quality enthusiasts (28\%):}
\begin{itemize}
\item Technical specifications
\item Audio samples
\item Expert reviews
\end{itemize}

\column{0.5\textwidth}
\textbf{For Gift buyers (15\%):}
\begin{itemize}
\item Gift wrapping options
\item Gift message features
\item Holiday promotions
\item Gift guides
\end{itemize}

\textbf{For Technical users (17\%):}
\begin{itemize}
\item Detailed specs
\item Compatibility info
\item Firmware updates
\end{itemize}
\end{columns}
\end{frame}

% Slide 27: From Clusters to Action
\begin{frame}[t]{From Clusters to Action: Design Implications}
\textbf{Immediate actions based on clustering:}

\begin{enumerate}
\item \textbf{Personalized landing pages}
\begin{itemize}
\item Different messaging per segment
\item Relevant features highlighted
\end{itemize}

\item \textbf{Targeted email campaigns}
\begin{itemize}
\item Price alerts for value seekers
\item New features for tech enthusiasts
\end{itemize}

\item \textbf{Product development priorities}
\begin{itemize}
\item 28\% care most about sound quality
\item Focus R\&D on audio improvements
\end{itemize}

\item \textbf{Customer support training}
\begin{itemize}
\item Different segments, different needs
\item Tailored support approaches
\end{itemize}
\end{enumerate}
\end{frame}

% Section 5: Bridge (2 slides)

% Slide 28: What We Achieved
\begin{frame}[t]{What We Achieved Today}
\begin{columns}[T]
\column{0.5\textwidth}
\textbf{Traditional Approach:}
\begin{itemize}
\item 100 reviews read
\item 2 segments identified
\item 2 weeks of work
\item \$5,000 cost
\item Lots of guessing
\end{itemize}

\column{0.5\textwidth}
\textbf{ML-Powered Approach:}
\begin{itemize}
\item 10,000 reviews analyzed
\item 5 segments discovered
\item 4 hours total
\item \$50 compute cost
\item Data-driven insights
\end{itemize}
\end{columns}

\vspace{1em}
\begin{center}
\Large\textcolor{mlgreen}{\textbf{100x more data, 50x faster, 100x cheaper}}
\end{center}

\begin{tcolorbox}[colback=mlblue!20]
\textbf{Key Learning: ML + Empathy = Understanding at Scale}
\end{tcolorbox}
\end{frame}

% Slide 29: Next Week's Challenge
\begin{frame}[t]{Next Week's Challenge}
\begin{center}
\Large\textbf{We found 5 user segments...}\\
\vspace{1em}
\normalsize
We know they exist. We see their patterns.\\
We understand their keywords.\\
\vspace{1em}
\textbf{But...}\\
\vspace{1em}
\Large\textcolor{mlred}{\textbf{What are they actually FEELING?}}\\
\vspace{1em}
\normalsize
Happy? Frustrated? Confused? Excited?\\
\vspace{1em}
\end{center}

\begin{tcolorbox}[colback=mlorange!20]
\textbf{Next Week: Transformers and NLP - Understanding Emotion in Text}\\
How BERT reads between the lines to understand sentiment
\end{tcolorbox}
\end{frame}

\end{document}