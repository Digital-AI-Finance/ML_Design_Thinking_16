\documentclass[8pt,aspectratio=169]{beamer}
\usetheme{Madrid}
\usepackage{graphicx}
\usepackage{booktabs}
\usepackage{adjustbox}
\usepackage{multicol}
\usepackage{amsmath}
\usepackage{xcolor}
\usepackage{tcolorbox}

% Color definitions - consistent emotion colors
\definecolor{mlblue}{RGB}{31, 119, 180}
\definecolor{mlorange}{RGB}{255, 127, 14}
\definecolor{mlgreen}{RGB}{44, 160, 44}
\definecolor{mlred}{RGB}{214, 39, 40}
\definecolor{mlpurple}{RGB}{148, 103, 189}
\definecolor{mlgray}{RGB}{140, 86, 75}

% Remove navigation symbols
\setbeamertemplate{navigation symbols}{}

% Clean itemize/enumerate
\setbeamertemplate{itemize items}[circle]
\setbeamertemplate{enumerate items}[default]

% Title information
\title{Week 2: Understanding Emotions in Text}
\subtitle{BERT + Empathize = What Users Really Mean}
\author{ML/AI/GenAI for Design Thinking}
\institute{BSc Course - 12 Week Program}
\date{2024}

\begin{document}

% ============================================
% TITLE SLIDE
% ============================================

% Slide 1: Title
\begin{frame}
\titlepage
\end{frame}

% ============================================
% PART 1: THE HIDDEN PROBLEM (Slides 2-6)
% ============================================

% Section Divider
\begin{frame}
\begin{center}
\Huge\textbf{Part 1: The Hidden Problem}\\
\vspace{1em}
\Large\textit{What we miss in user feedback}\\
\vspace{2em}
\normalsize\textcolor{mlgray}{Part 1 of 4}
\end{center}
\end{frame}

% Slide 2: Colorful Sentiment Clusters
\begin{frame}{The Sentiment Landscape}
\begin{center}
\includegraphics[width=0.85\textwidth]{charts/sentiment_clusters.pdf}
\end{center}
\begin{tcolorbox}[colback=mlblue!20]
\centering
\textbf{Challenge: How do we navigate this complex emotional landscape at scale?}
\end{tcolorbox}
\end{frame}

% Slide 3: Hidden Emotions in Text
\begin{frame}{The Problem: Hidden Emotions in Text}
\begin{columns}
\column{0.5\textwidth}
\textbf{What users write:}
\begin{itemize}
\item ``Great product... if you like disappointment''
\item ``Not bad at all''
\item ``Fine.''
\item ``Can't complain''
\end{itemize}

\vspace{1em}
\textbf{Design Thinking blind spot:}
\begin{itemize}
\item Missing real pain points
\item Building wrong features
\item Misreading user satisfaction
\end{itemize}

\column{0.5\textwidth}
\textbf{What they actually mean:}
\begin{itemize}
\item Angry (sarcasm)
\item Happy (double negative)
\item Unhappy (short response)
\item Forced acceptance
\end{itemize}

\vspace{1em}
\textbf{Design Thinking opportunity:}
\begin{itemize}
\item Understand true feelings
\item Identify hidden frustrations
\item Discover unspoken needs
\end{itemize}
\end{columns}

\vspace{1em}
\begin{tcolorbox}[colback=mlred!20]
\centering
\textbf{For Design Thinking: Words alone miss 45\% of user emotions}
\end{tcolorbox}
\end{frame}

% Slide 4: Why Keywords Fail
\begin{frame}{Why Keyword Matching Fails}
\begin{columns}
\column{0.5\textwidth}
\textbf{The ``Not Bad'' Problem:}
\begin{center}
\small
\begin{tabular}{lcc}
\toprule
\textbf{Text} & \textbf{Keywords} & \textbf{Reality} \\
\midrule
``Not bad'' & Negative & Positive \\
``Terribly good'' & Mixed & Very Positive \\
``Love waiting'' & Positive & Sarcastic \\
``Could be worse'' & Negative & Neutral \\
\bottomrule
\end{tabular}
\end{center}

\vspace{0.5em}
\textbf{Why it fails:}
\begin{itemize}
\item Counts words, ignores relationships
\item Misses context completely
\item Can't detect sarcasm
\end{itemize}

\column{0.5\textwidth}
\textbf{Design Thinking Impact of Failures:}
\begin{itemize}
\item \textbf{False positives:} Thinking users are happy when they're not
\item \textbf{Missed sarcasm:} Building on ``praised'' features that users hate
\item \textbf{Wrong priorities:} Focusing on the wrong problems
\end{itemize}

\vspace{0.5em}
\textbf{Real cost:}
\begin{itemize}
\item 68\% of users leave due to perceived indifference
\item Wrong features = wasted development
\item Missed insights = lost opportunities
\end{itemize}
\end{columns}
\end{frame}

% Slide 5: The Challenge with Design Focus
\begin{frame}{The Challenge: Understanding Context for Design Thinking}
\begin{columns}
\column{0.5\textwidth}
\textbf{Current Problems:}
\begin{itemize}
\item Manual analysis: 100 reviews/day max
\item Digital products: 10,000+ reviews/day
\item Each review: Unique human experience
\item Context lost in aggregation
\end{itemize}

\vspace{1em}
\textbf{What we need:}
\begin{enumerate}
\item See word relationships
\item Understand order matters
\item Detect sarcasm and tone
\item Process at scale
\end{enumerate}

\column{0.5\textwidth}
\textbf{Design Thinking Needs:}
\begin{itemize}
\item \textbf{Empathize:} Feel what thousands feel
\item \textbf{Define:} Find real problems, not symptoms
\item \textbf{Ideate:} Generate solutions for actual needs
\item \textbf{Test:} Measure emotional impact
\end{itemize}

\vspace{1em}
\textbf{The Goal:}\\
Scale empathy without losing humanity
\end{columns}

\vspace{0.5em}
\begin{tcolorbox}[colback=mlgreen!20]
\centering
\textbf{Solution: BERT - Reading text like humans, at machine scale}
\end{tcolorbox}
\end{frame}

% ============================================
% PART 2: BERT - THE TECHNOLOGY (Slides 7-15)
% ============================================

% Section Divider
\begin{frame}
\begin{center}
\Huge\textbf{Part 2: BERT - The Technology}\\
\vspace{1em}
\Large\textit{Understanding context like humans do}\\
\vspace{2em}
\normalsize\textcolor{mlgray}{Part 2 of 4}
\end{center}
\end{frame}

% Slide 6: What is BERT with Design Focus
\begin{frame}{What is BERT?}
\textbf{BERT = Bidirectional Encoder Representations from Transformers}

\vspace{0.5em}
Simple explanation: \textbf{BERT reads all words at once, not one by one}

\begin{columns}[t]
\column{0.48\textwidth}
\begin{tcolorbox}[colback=mlred!10]
\textbf{Traditional (Sequential):}\\
The $\rightarrow$ movie $\rightarrow$ was $\rightarrow$ not $\rightarrow$ bad\\
\small(Reads left to right, misses connections)
\end{tcolorbox}

\vspace{0.5em}
\begin{tcolorbox}[colback=mlgreen!10]
\textbf{BERT (Parallel):}\\
The movie was not bad\\
\small(Sees everything, understands ``not bad'' = good)
\end{tcolorbox}

\column{0.48\textwidth}
\textbf{For Designers, this means:}
\begin{itemize}
\item Understand user feelings in context
\item Catch subtle frustrations
\item Identify what users really want
\item No more keyword guessing
\end{itemize}

\vspace{0.5em}
\textbf{Design Thinking Impact:}
\begin{itemize}
\item 87\% sarcasm detection
\item Find hidden pain points
\item Understand feature requests
\end{itemize}
\end{columns}
\end{frame}

% Slide 7: Bidirectional Reading with Design Impact
\begin{frame}{Bidirectional: Seeing Past AND Future}
\begin{columns}
\column{0.5\textwidth}
\textbf{Example: ``The app was \_\_\_ frustrating''}

\vspace{0.5em}
\textbf{Old Way (Left to Right):}
\begin{itemize}
\item Sees: ``The app was''
\item Guesses: good? bad? slow?
\item Can't use ``frustrating'' as hint
\item Often wrong
\end{itemize}

\vspace{0.5em}
\textbf{BERT (Both Directions):}
\begin{itemize}
\item Sees: ``The app was'' + ``frustrating''
\item Knows: probably ``very'' or ``incredibly''
\item Uses full context
\item Much more accurate
\end{itemize}

\column{0.5\textwidth}
\textbf{Design Thinking Implications:}
\begin{itemize}
\item \textbf{Intensity matters:} ``Very frustrating'' vs ``Slightly frustrating''
\item \textbf{Context reveals priority:} What made it frustrating?
\item \textbf{Emotional nuance:} Frustrated vs Angry vs Disappointed
\end{itemize}

\vspace{0.5em}
\textbf{Real Example:}\\
``The checkout process was \_\_\_ confusing''
\begin{itemize}
\item BERT finds: ``incredibly''
\item Design action: Simplify checkout
\item Result: 23\% fewer cart abandonments
\end{itemize}
\end{columns}

\vspace{0.5em}
\begin{tcolorbox}[colback=mlorange!20]
\centering
\textbf{Design Thinking Benefit: Catches problems keyword analysis misses completely}
\end{tcolorbox}
\end{frame}

% Slide 8: Attention Mechanism with Design Insights
\begin{frame}{Attention: BERT Focuses on Emotional Triggers}
\begin{center}
\includegraphics[width=0.65\textwidth]{charts/bert_attention_heatmap.pdf}
\end{center}

\begin{columns}
\column{0.5\textwidth}
\textbf{What attention shows:}
\begin{itemize}
\item ``Not'' strongly links to ``bad''
\item ``Great'' connects to ``disappointment''
\item Reveals sarcasm patterns
\end{itemize}

\column{0.5\textwidth}
\textbf{Design Thinking Insights:}
\begin{itemize}
\item Which features trigger emotions
\item What words signal problems
\item Where users struggle most
\end{itemize}
\end{columns}

\vspace{0.5em}
\begin{tcolorbox}[colback=mlpurple!10]
\centering
\textbf{Preview: Week 3 dives deep into attention mechanisms for prioritization}
\end{tcolorbox}
\end{frame}

% Slide 9: Context Changes Meaning - Design Implications
\begin{frame}{Context Changes Everything in Design Thinking}
\begin{columns}
\column{0.5\textwidth}
\textbf{Same Word, Different Meanings:}
\begin{center}
\small
\begin{tabular}{ll}
\toprule
\textbf{Word} & \textbf{Contexts} \\
\midrule
``Fast'' & Quick delivery (good) \\
        & Battery drains fast (bad) \\
``Simple'' & Easy to use (good) \\
          & Too basic (bad) \\
``Light'' & Portable (good) \\
         & Feels cheap (bad) \\
\bottomrule
\end{tabular}
\end{center}

\textbf{BERT understands context:}
\begin{itemize}
\item Different meanings per use
\item Surrounding words determine sentiment
\item No fixed good/bad words
\end{itemize}

\column{0.5\textwidth}
\textbf{Design Thinking Implications:}
\begin{itemize}
\item Same feature, different contexts = different user needs
\item ``Simple'' for beginners vs power users
\item ``Fast'' performance vs battery life
\end{itemize}

\vspace{0.5em}
\textbf{Real Design Decision:}\\
Spotify discovered ``shuffle'' meant:
\begin{itemize}
\item Random (tech users)
\item Variety (casual users)
\item Discover (new users)
\end{itemize}
Result: Three different shuffle modes
\end{columns}
\end{frame}

% Slide 10: Training BERT with Customization Focus
\begin{frame}{Training BERT for Your Product}
\begin{columns}
\column{0.5\textwidth}
\begin{tcolorbox}[colback=mlgreen!10]
\textbf{Step 1: General Training}\\
3.3 billion words from books/web\\
Learns language, grammar, facts
\end{tcolorbox}

\vspace{0.5em}
\begin{center}
\Large $\downarrow$
\end{center}
\vspace{0.5em}

\begin{tcolorbox}[colback=mlblue!10]
\textbf{Step 2: Your Product}\\
Your reviews and feedback\\
Learns your users' language
\end{tcolorbox}

\column{0.5\textwidth}
\textbf{Design Thinking Benefits:}
\begin{itemize}
\item Customizable to your domain
\item Learns your product's jargon
\item Adapts to user base
\item Improves over time
\end{itemize}

\vspace{0.5em}
\textbf{Example Customization:}
\begin{itemize}
\item Gaming: ``lag'' = critical issue
\item Fashion: ``fit'' = top priority
\item SaaS: ``integration'' = key need
\end{itemize}
\end{columns}

\vspace{0.5em}
\begin{tcolorbox}[colback=mlpurple!20]
\centering
\textbf{Result: BERT speaks your users' language}
\end{tcolorbox}
\end{frame}

% Slide 11: BERT Process for Sentiment - Design Focused
\begin{frame}{How BERT Detects Emotions for Design Thinking}
\begin{columns}
\column{0.5\textwidth}
\textbf{BERT's Process:}
\begin{enumerate}
\item \textbf{Read everything:} All words at once
\item \textbf{Connect words:} Find relationships
\item \textbf{Build understanding:} Recognize patterns
\item \textbf{Output emotion:} With confidence score
\end{enumerate}

\vspace{0.5em}
\textbf{Example:}\\
``Not bad for the price''
\begin{itemize}
\item Links: ``not'' + ``bad'' = positive
\item Context: ``for the price'' = qualified
\item Output: Moderately positive (0.65)
\end{itemize}

\column{0.5\textwidth}
\textbf{Design Thinking Application:}
\begin{enumerate}
\item \textbf{Emotion detected:} Moderate satisfaction
\item \textbf{Qualifier found:} Price-sensitive
\item \textbf{Design insight:} Value perception issue
\item \textbf{Action:} Highlight value props
\end{enumerate}

\vspace{0.5em}
\textbf{Confidence helps prioritize:}
\begin{itemize}
\item High confidence = Clear issue
\item Low confidence = Investigate more
\item Mixed signals = User conflict
\end{itemize}
\end{columns}
\end{frame}

% Slide 12: Sarcasm Detection with Design Warning
\begin{frame}{BERT Catches Sarcasm - A Design Thinking Warning}
\begin{columns}
\column{0.5\textwidth}
\textbf{Sarcasm Patterns BERT Detects:}
\begin{itemize}
\item Positive words + negative context
\item Exaggerated praise
\item Contradiction signals
\item Timing mismatches
\end{itemize}

\vspace{0.5em}
\textbf{Examples Found:}
\begin{itemize}
\item ``Great! It crashed again''
\item ``Love the 3-hour load time''
\item ``Perfect... if you like broken''
\item ``Fantastic customer service'' (1 star)
\end{itemize}

\column{0.5\textwidth}
\textbf{Design Thinking Warning:}
\begin{tcolorbox}[colback=mlred!20]
15\% of ``positive'' reviews contain sarcastic criticism
\end{tcolorbox}

\textbf{What this means:}
\begin{itemize}
\item Your satisfaction scores are inflated
\item Real problems hidden in ``praise''
\item Users resort to sarcasm when frustrated
\item Critical issues being missed
\end{itemize}

\textbf{Design Response:}
\begin{itemize}
\item Check all 5-star reviews for sarcasm
\item Look for feature ``praise'' patterns
\item Identify frustration triggers
\end{itemize}
\end{columns}
\end{frame}

% Slide 13: Performance with Design Implications
\begin{frame}{Performance: What 23\% Accuracy Means for Design Thinking}
\begin{center}
\includegraphics[width=0.7\textwidth]{charts/sentiment_comparison.pdf}
\end{center}

\begin{columns}
\column{0.5\textwidth}
\textbf{Accuracy Improvements:}
\begin{itemize}
\item Overall: 72\% $\rightarrow$ 95\% (+23\%)
\item Sarcasm: 15\% $\rightarrow$ 87\% (+72\%)
\item Context: 60\% $\rightarrow$ 94\% (+34\%)
\end{itemize}

\column{0.5\textwidth}
\textbf{Design Thinking Implications:}
\begin{itemize}
\item Better user personas (95\% accurate)
\item Real pain points found (87\% sarcasm)
\item Nuanced feature requests understood
\item Confidence in design decisions
\end{itemize}
\end{columns}
\end{frame}

% Slide 14: Try It Yourself - Practical Code
\begin{frame}{Try It Yourself: Quick BERT Implementation}
\begin{tcolorbox}[colback=gray!10]
\footnotesize
\texttt{from transformers import pipeline}\\
\texttt{}\\
\texttt{\# Load pre-trained BERT for sentiment}\\
\texttt{analyzer = pipeline("sentiment-analysis")}\\
\texttt{}\\
\texttt{\# Analyze your product reviews}\\
\texttt{reviews = [}\\
\texttt{\ \ "This product is not bad at all",}\\
\texttt{\ \ "Love waiting 3 hours for support",}\\
\texttt{\ \ "Simple to use but too basic"}\\
\texttt{]}\\
\texttt{}\\
\texttt{for review in reviews:}\\
\texttt{\ \ result = analyzer(review)}\\
\texttt{\ \ print(f"Text: \{review\}")}\\
\texttt{\ \ print(f"Result: \{result\}")}\\
\texttt{}\\
\texttt{\# Output shows sentiment and confidence score}
\end{tcolorbox}

\begin{columns}
\column{0.5\textwidth}
\textbf{Try with your data:}
\begin{itemize}
\item Export your reviews to CSV
\item Run this simple script
\item Compare with manual analysis
\end{itemize}

\column{0.5\textwidth}
\textbf{Next steps:}
\begin{itemize}
\item Fine-tune on your domain
\item Add emotion categories
\item Build into your workflow
\end{itemize}
\end{columns}
\end{frame}

% ============================================
% PART 3: EMPATHY AT SCALE (Expanded with real examples)
% ============================================

% Section Divider
\begin{frame}
\begin{center}
\Huge\textbf{Part 3: Empathy at Scale}\\
\vspace{1em}
\Large\textit{From data to real-world design decisions}\\
\vspace{2em}
\normalsize\textcolor{mlgray}{Part 3 of 4}
\end{center}
\end{frame}

% Slide 15: Empathy at Scale - Two Columns
\begin{frame}{Empathize at Scale: Understanding Thousands}
\begin{columns}
\column{0.5\textwidth}
\textbf{Traditional Empathy Methods:}
\begin{itemize}
\item User interviews: 20 people/week
\item Surveys: Low response, biased
\item Observation: Time-intensive
\item Focus groups: Groupthink issues
\end{itemize}

\vspace{0.5em}
\textbf{Limitations:}
\begin{itemize}
\item Small sample sizes
\item Geographic constraints
\item Time and cost barriers
\item Vocal minority bias
\end{itemize}

\column{0.5\textwidth}
\textbf{BERT-Enhanced Empathy:}
\begin{itemize}
\item Process 10,000+ reviews/day
\item Understand global users
\item Find silent majority opinions
\item Detect emotional patterns
\end{itemize}

\vspace{0.5em}
\textbf{Advantages:}
\begin{itemize}
\item Every user voice heard
\item Real-time emotional pulse
\item Unbiased pattern detection
\item Cultural nuance preserved
\end{itemize}
\end{columns}

\vspace{0.5em}
\begin{tcolorbox}[colback=mlorange!20]
\centering
\textbf{Design Thinking Power: Feel what thousands feel, understand what they can't articulate}
\end{tcolorbox}
\end{frame}

% Slide 16a: Emotional Spectrum Part 1
\begin{frame}{Emotional Spectrum in User Feedback}
\begin{columns}
\column{0.5\textwidth}
\begin{center}
\includegraphics[width=\textwidth]{charts/sentiment_analysis_demo.pdf}
\end{center}

\column{0.5\textwidth}
\textbf{Beyond Binary Sentiment:}
\begin{itemize}
\item \textbf{Joy:} Delight, satisfaction, excitement
\item \textbf{Anger:} Frustration, annoyance, rage
\item \textbf{Fear:} Anxiety, concern, worry
\item \textbf{Surprise:} Amazement, shock, confusion
\item \textbf{Sadness:} Disappointment, regret
\item \textbf{Trust:} Confidence, security, faith
\end{itemize}

\vspace{0.5em}
\textbf{Emotions Blend:}
\begin{itemize}
\item Joy + Surprise = Delight
\item Fear + Sadness = Despair
\item Anger + Disgust = Contempt
\item Trust + Joy = Love
\end{itemize}
\end{columns}
\end{frame}

% Slide 16b: Emotional Journey Mapping
\begin{frame}{The User's Emotional Journey}
\begin{center}
\textbf{Mapping Emotions Through the Product Experience}
\end{center}

\begin{columns}
\column{0.5\textwidth}
\textbf{Discovery Phase:}
\begin{itemize}
\item \textcolor{mlgreen}{Curiosity} $\rightarrow$ Interest
\item \textcolor{mlorange}{Confusion} $\rightarrow$ Need help
\item \textcolor{mlblue}{Excitement} $\rightarrow$ Engagement
\end{itemize}

\vspace{0.5em}
\textbf{Usage Phase:}
\begin{itemize}
\item \textcolor{mlgreen}{Satisfaction} $\rightarrow$ Continue
\item \textcolor{mlred}{Frustration} $\rightarrow$ Seek support
\item \textcolor{mlpurple}{Delight} $\rightarrow$ Share
\end{itemize}

\column{0.5\textwidth}
\textbf{Retention Phase:}
\begin{itemize}
\item \textcolor{mlgreen}{Trust} $\rightarrow$ Loyalty
\item \textcolor{mlred}{Disappointment} $\rightarrow$ Churn risk
\item \textcolor{mlblue}{Love} $\rightarrow$ Advocacy
\end{itemize}

\vspace{0.5em}
\begin{tcolorbox}[colback=mlblue!10]
\textbf{Design Insight:}\\
Track emotional transitions to identify critical intervention points
\end{tcolorbox}
\end{columns}

\vspace{0.5em}
\begin{center}
\small\textit{Remember: Like the sentiment clusters, emotions are interconnected and fluid}
\end{center}
\end{frame}

% Slide 17: From Emotions to Design Actions
\begin{frame}{From Emotions to Design Thinking Insights}
\begin{columns}
\column{0.5\textwidth}
\textbf{Emotion-Driven Design Actions:}
\begin{itemize}
\item \textbf{Joy (45\%):} Amplify successful features
\item \textbf{Frustration (25\%):} Simplify workflows
\item \textbf{Confusion (15\%):} Improve onboarding
\item \textbf{Delight (10\%):} Create memorable moments
\item \textbf{Anxiety (5\%):} Add reassurance, guidance
\end{itemize}

\vspace{0.5em}
\textbf{Priority Matrix:}
\begin{itemize}
\item High frequency + High intensity = Fix now
\item High frequency + Low intensity = Improve
\item Low frequency + High intensity = Investigate
\end{itemize}

\column{0.5\textwidth}
\textbf{Real Design Decisions:}

\vspace{0.5em}
\begin{tcolorbox}[colback=mlgreen!10]
\textbf{Joy $\rightarrow$ Enhance:}\\
Users love quick checkout\\
Action: Make it more prominent
\end{tcolorbox}

\vspace{0.3em}
\begin{tcolorbox}[colback=mlred!10]
\textbf{Frustration $\rightarrow$ Simplify:}\\
Login process causes anger\\
Action: Add social login
\end{tcolorbox}

\vspace{0.3em}
\begin{tcolorbox}[colback=mlorange!10]
\textbf{Confusion $\rightarrow$ Guide:}\\
New users lost in features\\
Action: Progressive disclosure
\end{tcolorbox}
\end{columns}
\end{frame}

% Slide 18: Netflix Real Example (moved from Part 4)
\begin{frame}{Real World: Netflix Emotion-Driven Design}
\begin{columns}
\column{0.5\textwidth}
\textbf{The Challenge:}
\begin{itemize}
\item Users: ``Nothing to watch'' paradox
\item Reality: 15,000+ titles available
\item Problem: Choice overload
\item Need: Mood-based discovery
\end{itemize}

\vspace{0.5em}
\textbf{BERT Analysis Process:}
\begin{enumerate}
\item Analyzed 50M+ subtitles
\item Mapped emotional arcs
\item Studied viewing patterns
\item Correlated mood to content
\end{enumerate}

\column{0.5\textwidth}
\textbf{Design Decisions Made:}
\begin{itemize}
\item \textbf{Mood categories:} Feel-good, Thrilling, Thought-provoking
\item \textbf{Emotional thumbnails:} Show mood not just genre
\item \textbf{Sentiment trajectory:} ``Starts sad, ends happy''
\item \textbf{Mood continuity:} Next episode emotional preview
\end{itemize}

\vspace{0.5em}
\textbf{Results:}
\begin{itemize}
\item 15\% increase in completion
\item 23\% fewer browse abandonments
\item ``Mood match'' top-rated feature
\end{itemize}
\end{columns}

\vspace{0.5em}
\begin{tcolorbox}[colback=mlblue!20]
\centering
\textbf{Key Learning: Understanding emotional needs drives better design than demographics}
\end{tcolorbox}
\end{frame}

% Slide 19: More Success Stories (moved from Part 4)
\begin{frame}{More Success Stories: BERT in Action}
\begin{columns}
\column{0.5\textwidth}
\textbf{Airbnb: Trust Building}
\begin{itemize}
\item Analyzed host descriptions
\item Found trust-building language
\item Created writing guidelines
\item Result: 18\% more bookings
\end{itemize}

\vspace{0.5em}
\textbf{Duolingo: Motivation Patterns}
\begin{itemize}
\item Detected frustration points
\item Identified celebration moments
\item Adjusted difficulty curves
\item Result: 25\% better retention
\end{itemize}

\column{0.5\textwidth}
\textbf{Slack: Feature Discovery}
\begin{itemize}
\item Found confusion about features
\item Detected ``aha'' moments
\item Redesigned onboarding
\item Result: 40\% faster adoption
\end{itemize}

\vspace{0.5em}
\textbf{Common Pattern:}
\begin{tcolorbox}[colback=mlblue!10]
All used BERT to understand emotional context, not just sentiment scores
\end{tcolorbox}
\end{columns}
\end{frame}

% Slide 20: Human + AI Collaboration
\begin{frame}{Human + AI: Collaborative Design Intelligence}
\begin{columns}
\column{0.5\textwidth}
\textbf{BERT Strengths:}
\begin{itemize}
\item Process massive volume
\item Find hidden patterns
\item Consistent analysis 24/7
\item Unbiased detection
\item Quantify emotions
\item Track sentiment trends
\end{itemize}

\vspace{0.5em}
\textbf{BERT Provides:}
\begin{itemize}
\item The ``what'' - patterns found
\item The ``where'' - problem areas
\item The ``how much'' - severity
\end{itemize}

\column{0.5\textwidth}
\textbf{Human Strengths:}
\begin{itemize}
\item Understand context deeply
\item Creative problem solving
\item Ethical judgment
\item Cultural sensitivity
\item Intuitive leaps
\item Empathetic response
\end{itemize}

\vspace{0.5em}
\textbf{Humans Provide:}
\begin{itemize}
\item The ``why'' - root causes
\item The ``how'' - solutions
\item The ``should we'' - ethics
\end{itemize}
\end{columns}

\vspace{0.5em}
\begin{tcolorbox}[colback=mlgreen!20]
\centering
\textbf{Best Practice: BERT finds patterns, humans interpret meaning, together create solutions}
\end{tcolorbox}
\end{frame}

% Slide 21: Key Takeaway (moved from Part 4)
\begin{frame}{The Key Insight for Designers}
\begin{center}
\Huge\textbf{Context Matters More Than Keywords}
\end{center}

\vspace{1.5em}
\begin{columns}
\column{0.5\textwidth}
\textbf{Old Design Research:}
\begin{itemize}
\item Count positive/negative words
\item Average star ratings
\item Tag cloud analysis
\item Sentiment percentages
\end{itemize}

\column{0.5\textwidth}
\textbf{BERT-Powered Research:}
\begin{itemize}
\item Understand relationships
\item Detect hidden emotions
\item Find real problems
\item Scale human empathy
\end{itemize}
\end{columns}

\vspace{1.5em}
\begin{tcolorbox}[colback=mlgreen!20]
\centering
\Large\textbf{BERT + Design Thinking = Understanding users at scale with human insight}
\end{tcolorbox}
\end{frame}

% ============================================
% PART 4: THEORETICAL FOUNDATIONS (New theory section)
% ============================================

% Section Divider
\begin{frame}
\begin{center}
\Huge\textbf{Part 4: Theoretical Foundations}\\
\vspace{1em}
\Large\textit{The mathematics behind the magic}\\
\vspace{2em}
\normalsize\textcolor{mlgray}{Part 4 of 4}
\end{center}
\end{frame}

% Slide 22: Mathematical Foundation of Attention
\begin{frame}{Mathematical Foundation of Attention}
\begin{columns}
\column{0.5\textwidth}
\textbf{The Attention Formula:}
\begin{center}
\Large
$\text{Attention}(Q,K,V) = \text{softmax}\left(\frac{QK^T}{\sqrt{d_k}}\right)V$
\end{center}

\vspace{0.5em}
\textbf{Components:}
\begin{itemize}
\item \textbf{Q (Query):} What am I looking for?
\item \textbf{K (Key):} What information exists?
\item \textbf{V (Value):} What should I retrieve?
\item \textbf{$\sqrt{d_k}$:} Scaling factor for stability
\end{itemize}

\column{0.5\textwidth}
\textbf{Why This Matters for Design:}
\begin{itemize}
\item Attention scores = importance weights
\item Higher scores = stronger relationships
\item Shows which words influence meaning
\item Reveals hidden connections
\end{itemize}

\vspace{0.5em}
\textbf{Example:}\\
``Not bad'' $\rightarrow$ High attention between ``not'' and ``bad''\\
Result: BERT knows this means ``good''
\end{columns}

\vspace{0.5em}
\begin{tcolorbox}[colback=mlpurple!10]
\centering
\textbf{Design Impact: Attention reveals what users focus on in their feedback}
\end{tcolorbox}
\end{frame}

% Slide 23: Embedding Space Theory
\begin{frame}{Words as Vectors: The Embedding Space}
\begin{center}
\includegraphics[width=0.8\textwidth]{charts/embedding_space_visualization.pdf}
\end{center}

\begin{columns}
\column{0.5\textwidth}
\textbf{Key Concepts:}
\begin{itemize}
\item Words = 768-dimensional vectors
\item Similar meanings cluster together
\item Context changes position
\item Relationships are geometric
\end{itemize}

\column{0.5\textwidth}
\textbf{Design Applications:}
\begin{itemize}
\item Find similar feedback automatically
\item Group related complaints
\item Discover unexpected connections
\item Map emotional territories
\end{itemize}
\end{columns}
\end{frame}

% Slide 24: How BERT Learns - Training Theory
\begin{frame}{How BERT Learns: Self-Supervised Training}
\begin{columns}
\column{0.5\textwidth}
\textbf{Masked Language Modeling (MLM):}
\begin{itemize}
\item Hide 15\% of words randomly
\item BERT predicts missing words
\item Example: ``The [MASK] was delicious''
\item BERT learns: [MASK] = ``food''
\end{itemize}

\vspace{0.5em}
\textbf{Why This Works:}
\begin{itemize}
\item Forces understanding of context
\item No labeled data needed
\item Learns from massive text
\item Captures language patterns
\end{itemize}

\column{0.5\textwidth}
\textbf{Next Sentence Prediction (NSP):}
\begin{itemize}
\item Given: Sentence A and B
\item Task: Are they consecutive?
\item Learns discourse relationships
\item Understands conversation flow
\end{itemize}

\vspace{0.5em}
\textbf{Design Thinking Connection:}
\begin{itemize}
\item MLM = Understanding context
\item NSP = Following user journeys
\item Combined = Full comprehension
\end{itemize}
\end{columns}

\vspace{0.5em}
\begin{tcolorbox}[colback=mlgreen!10]
\centering
\textbf{Result: BERT learns language like children do - by filling gaps and making connections}
\end{tcolorbox}
\end{frame}

% Slide 25: Transfer Learning Theory
\begin{frame}{Transfer Learning: From General to Specific}
\begin{columns}
\column{0.5\textwidth}
\textbf{Two-Stage Process:}

\vspace{0.5em}
\begin{tcolorbox}[colback=mlblue!10]
\textbf{Stage 1: Pre-training}\\
\small
\begin{itemize}
\item 3.3 billion words
\item General language understanding
\item No task-specific labels
\item Learns universal patterns
\end{itemize}
\end{tcolorbox}

\vspace{0.5em}
\begin{tcolorbox}[colback=mlorange!10]
\textbf{Stage 2: Fine-tuning}\\
\small
\begin{itemize}
\item Your specific data
\item Task-focused learning
\item Much smaller dataset
\item Adapts to your domain
\end{itemize}
\end{tcolorbox}

\column{0.5\textwidth}
\textbf{Why Transfer Learning Works:}
\begin{itemize}
\item Language basics are universal
\item Specific tasks build on basics
\item Reduces data requirements
\item Faster training time
\end{itemize}

\vspace{0.5em}
\textbf{Design Thinking Analogy:}
\begin{itemize}
\item Pre-training = Learning to read
\item Fine-tuning = Understanding your users
\item Result = Expert in your domain
\end{itemize}
\end{columns}

\vspace{0.5em}
\begin{tcolorbox}[colback=mlpurple!20]
\centering
\textbf{Efficiency: 1,000 labeled examples can achieve 90\%+ accuracy with transfer learning}
\end{tcolorbox}
\end{frame}

% Slide 26: Why Bidirectional Matters - Theory
\begin{frame}{Why Bidirectional Matters: The Theory}
\begin{columns}
\column{0.5\textwidth}
\textbf{Traditional (Left-to-Right):}
\begin{itemize}
\item Sequential processing
\item P(word$_n$ | word$_1$...word$_{n-1}$)
\item Can't see future context
\item Limited understanding
\end{itemize}

\vspace{0.5em}
\textbf{BERT (Bidirectional):}
\begin{itemize}
\item Parallel processing
\item P(word$_n$ | all other words)
\item Sees full context
\item Complete understanding
\end{itemize}

\column{0.5\textwidth}
\textbf{Mathematical Advantage:}
\begin{itemize}
\item More information per prediction
\item Richer representations
\item Better disambiguation
\item Stronger contextual embeddings
\end{itemize}

\vspace{0.5em}
\textbf{Real Impact:}
\begin{center}
\small
\begin{tabular}{lcc}
\toprule
\textbf{Task} & \textbf{Uni} & \textbf{Bi} \\
\midrule
Sentiment & 82\% & 95\% \\
Sarcasm & 45\% & 87\% \\
Context & 71\% & 94\% \\
\bottomrule
\end{tabular}
\end{center}
\end{columns}

\vspace{0.5em}
\begin{tcolorbox}[colback=mlred!20]
\centering
\textbf{Key Insight: Bidirectional = 2x the context = dramatically better understanding}
\end{tcolorbox}
\end{frame}

% Slide 27: Bridge to Next Week
\begin{frame}{Next Week: From Understanding to Focusing}
\begin{columns}
\column{0.5\textwidth}
\textbf{This Week's Achievement:}
\begin{itemize}
\item Understand all emotions in text
\item Process thousands of reviews
\item Detect sarcasm and context
\item Scale empathy
\item Grasp theoretical foundations
\end{itemize}

\vspace{0.5em}
\textbf{The New Challenge:}
\begin{itemize}
\item Information overload
\item Too many insights
\item Which emotions matter most?
\item How to prioritize?
\end{itemize}

\column{0.5\textwidth}
\textbf{Week 3: Attention Mechanisms}
\begin{itemize}
\item \textbf{Focus} on critical emotions
\item \textbf{Prioritize} user pain points
\item \textbf{Filter} signal from noise
\item \textbf{Zoom in} on what matters
\end{itemize}

\vspace{0.5em}
\begin{tcolorbox}[colback=mlpurple!10]
\textbf{Preview:}\\
Attention = AI's way of saying\\
``This is important, look here!''
\end{tcolorbox}
\end{columns}

\vspace{0.5em}
\begin{tcolorbox}[colback=mlorange!20]
\centering
\textbf{Journey: Week 1 (Cluster) $\rightarrow$ Week 2 (Understand) $\rightarrow$ Week 3 (Focus)}
\end{tcolorbox}
\end{frame}

% Slide 28: Summary
\begin{frame}{Week 2 Summary: Theory Meets Practice}
\begin{columns}
\column{0.33\textwidth}
\textbf{Technical Learning:}
\begin{itemize}
\item BERT bidirectional
\item Attention mechanism
\item Embedding spaces
\item Transfer learning
\item 95\% accuracy
\end{itemize}

\column{0.33\textwidth}
\textbf{Theory Mastered:}
\begin{itemize}
\item Attention formula
\item Vector semantics
\item Self-supervision
\item Context importance
\item Mathematical foundation
\end{itemize}

\column{0.33\textwidth}
\textbf{Design Applications:}
\begin{itemize}
\item Scale empathy
\item Find pain points
\item Real examples
\item Proven results
\item Clear ROI
\end{itemize}
\end{columns}

\vspace{1em}
\begin{center}
\textbf{The Journey: From Chaos to Clarity}\\
\vspace{0.5em}
\begin{tabular}{ccccc}
\textcolor{mlred}{Scattered} & $\rightarrow$ & \textcolor{mlorange}{Clustered} & $\rightarrow$ & \textcolor{mlgreen}{Understood} \\
\small(Raw data) & & \small(Week 1) & & \small(Week 2) \\
\end{tabular}
\end{center}

\vspace{0.5em}
\begin{tcolorbox}[colback=mlblue!20]
\centering
\Large\textbf{BERT + Theory + Practice = Design with deep understanding}
\end{tcolorbox}
\end{frame}

% ============================================
% APPENDIX: TECHNICAL DETAILS (Selected slides)
% ============================================

% Appendix Section Divider
\begin{frame}
\begin{center}
\Huge\textbf{Appendix: Technical Deep Dive}\\
\vspace{1em}
\Large\textit{For those who want to go deeper}\\
\vspace{2em}
\normalsize\textcolor{mlgray}{Optional Material}
\end{center}
\end{frame}

% Keep selected technical appendix slides for reference
% A1: NLP Evolution Timeline
\begin{frame}{Appendix A1: NLP Evolution Timeline}
\textbf{History of Natural Language Processing:}

\begin{itemize}
\item \textbf{1950s - Rule-Based:} Hand-coded grammar rules
\item \textbf{1980s - Statistical:} Probabilistic models
\item \textbf{1990s - Machine Learning:} Naive Bayes, SVM
\item \textbf{2013 - Word2Vec:} Words as vectors
\item \textbf{2017 - Transformers:} Attention is all you need
\item \textbf{2018 - BERT:} Bidirectional pre-training
\item \textbf{2019 - GPT-2:} Large-scale generation
\item \textbf{2020+ - Giant Models:} GPT-3, PaLM, Claude
\end{itemize}

Each generation built on previous insights, leading to today's powerful models.
\end{frame}

% A2: Model Comparisons
\begin{frame}{Appendix A2: BERT vs Other Models}
\begin{center}
\begin{tabular}{lccc}
\toprule
\textbf{Model} & \textbf{Direction} & \textbf{Use Case} & \textbf{Params} \\
\midrule
BERT & Bidirectional & Understanding & 110M \\
GPT-2 & Left-to-right & Generation & 1.5B \\
GPT-3 & Left-to-right & Generation & 175B \\
RoBERTa & Bidirectional & Better BERT & 355M \\
ALBERT & Bidirectional & Efficient BERT & 12M \\
XLNet & Permutation & Best of both & 340M \\
\bottomrule
\end{tabular}
\end{center}

\textbf{Key Differences:}
\begin{itemize}
\item GPT: Autoregressive (good for generation)
\item BERT: Autoencoding (good for understanding)
\item RoBERTa: BERT with more data, no NSP
\item ALBERT: Parameter sharing for efficiency
\end{itemize}
\end{frame}

% A3: Implementation Code
\begin{frame}{Appendix A3: Complete BERT Implementation}
\textbf{Full Working Example:}

\begin{tcolorbox}[colback=gray!10]
\tiny
\texttt{from transformers import pipeline, AutoTokenizer}\\
\texttt{import pandas as pd}\\
\texttt{}\\
\texttt{\# Load BERT for sentiment analysis}\\
\texttt{analyzer = pipeline("sentiment-analysis",}\\
\texttt{\ \ \ \ model="nlptown/bert-base-multilingual-uncased-sentiment")}\\
\texttt{}\\
\texttt{\# Analyze product reviews}\\
\texttt{reviews = pd.read\_csv("reviews.csv")}\\
\texttt{results = []}\\
\texttt{}\\
\texttt{for review in reviews['text']:}\\
\texttt{\ \ \ \ result = analyzer(review)}\\
\texttt{\ \ \ \ results.append(\{}\\
\texttt{\ \ \ \ \ \ \ \ 'text': review,}\\
\texttt{\ \ \ \ \ \ \ \ 'sentiment': result[0]['label'],}\\
\texttt{\ \ \ \ \ \ \ \ 'confidence': result[0]['score']}\\
\texttt{\ \ \ \ \})}\\
\texttt{}\\
\texttt{\# Find sarcasm patterns}\\
\texttt{sarcasm\_indicators = ['great', 'love', 'fantastic', 'perfect']}\\
\texttt{low\_ratings = reviews[reviews['rating'] <= 2]}\\
\texttt{potential\_sarcasm = low\_ratings[}\\
\texttt{\ \ \ \ low\_ratings['text'].str.contains('|'.join(sarcasm\_indicators))}\\
\texttt{]}
\end{tcolorbox}

Full notebook available in course repository with fine-tuning examples.
\end{frame}

\end{document}