\documentclass[11pt,a4paper]{article}
\usepackage[margin=2cm]{geometry}
\usepackage{graphicx}
\usepackage{xcolor}
\usepackage{tcolorbox}
\usepackage{multicol}
\usepackage{enumitem}
\usepackage{amsmath}
\usepackage{amssymb}

% Colors from main presentation
\definecolor{mlblue}{RGB}{31, 119, 180}
\definecolor{mlorange}{RGB}{255, 127, 14}
\definecolor{mlgreen}{RGB}{44, 160, 44}
\definecolor{mlred}{RGB}{214, 39, 40}
\definecolor{mlpurple}{RGB}{148, 103, 189}

% Box styles
\tcbset{
    theorybox/.style={colback=mlblue!10, colframe=mlblue!50, title=#1},
    discoverybox/.style={colback=mlgreen!10, colframe=mlgreen!50, title=#1},
    hintbox/.style={colback=mlorange!10, colframe=mlorange!50, title=#1}
}

\title{\textbf{Week 2: Discovery Handout}\\
\Large Understanding Emotions in Text\\
\normalsize ML/AI for Design Thinking}
\author{Pre-Class Exploration}
\date{}

\begin{document}
\maketitle

\begin{tcolorbox}[discoverybox={Your Mission}]
\textbf{Before we dive into theory, let's discover why understanding emotions in text is harder than it seems!}\\
Work through these exercises. Use the theory hints when stuck. Discuss findings with your partner.
\end{tcolorbox}

% ============================================
% PART A: SENTIMENT DISCOVERY
% ============================================

\section*{Part A: Sentiment Discovery (10 min)}

\begin{tcolorbox}[colback=white, colframe=black!50]
\textbf{Exercise 1:} Rate these reviews as Positive (+), Negative (-), or Neutral (0):

\begin{enumerate}[label=\alph*)]
\item ``This app is absolutely perfect if you enjoy frustration'' \hfill Your rating: \_\_\_
\item ``Not bad at all'' \hfill Your rating: \_\_\_
\item ``It works.'' \hfill Your rating: \_\_\_
\item ``Can't complain about the price, you get what you pay for'' \hfill Your rating: \_\_\_
\item ``I've never been so impressed... by how slow an app can be'' \hfill Your rating: \_\_\_
\end{enumerate}

\textbf{Question:} Which ones were hardest to classify? Why?\\
\vspace{1.5cm}
\end{tcolorbox}

\begin{tcolorbox}[hintbox={Theory Hint 1}]
\textbf{Sarcasm} uses positive words to express negative sentiment. Look for:
\begin{itemize}[noitemsep]
\item Exaggerated positive language (``absolutely perfect'')
\item Unexpected endings (``...if you enjoy frustration'')
\item Context reversal
\end{itemize}
\textbf{Double negatives} (``not bad'') often express mild positivity.
\end{tcolorbox}

% ============================================
% PART B: CONTEXT MATTERS
% ============================================

\section*{Part B: Context Changes Everything (10 min)}

\begin{tcolorbox}[colback=white, colframe=black!50]
\textbf{Exercise 2:} The word ``fast'' appears in these reviews. Determine the sentiment:

\begin{enumerate}[label=\alph*)]
\item ``The delivery was incredibly fast'' \hfill Sentiment: \_\_\_
\item ``The battery drains fast'' \hfill Sentiment: \_\_\_
\item ``Fast customer service response!'' \hfill Sentiment: \_\_\_
\item ``Too fast to be thorough'' \hfill Sentiment: \_\_\_
\end{enumerate}

\textbf{Discovery Question:} Can a single word have fixed sentiment? Explain:\\
\vspace{2cm}
\end{tcolorbox}

\begin{tcolorbox}[hintbox={Theory Hint 2}]
\textbf{Bidirectional Understanding:} BERT reads text in both directions simultaneously:
\begin{itemize}[noitemsep]
\item Traditional: Read left $\rightarrow$ right only
\item BERT: Read left $\rightarrow$ right AND right $\rightarrow$ left
\item Result: ``fast'' + ``battery drain'' = negative context
\item Result: ``fast'' + ``delivery'' = positive context
\end{itemize}
This is why BERT understands ``not bad'' as positive (sees ``not'' + ``bad'' together).
\end{tcolorbox}

% ============================================
% PART C: ATTENTION MECHANISM
% ============================================

\section*{Part C: What Words Matter Most? (10 min)}

\begin{tcolorbox}[colback=white, colframe=black!50]
\textbf{Exercise 3:} In this review, underline the 3 most important words for understanding sentiment:

\textit{``The interface is clean and intuitive, but the app crashes constantly, making it completely unusable despite the beautiful design and smooth animations when it actually works.''}

Your top 3 words: \_\_\_\_\_\_\_\_ , \_\_\_\_\_\_\_\_ , \_\_\_\_\_\_\_\_

\textbf{Partner Discussion:} Did you choose the same words? Why might they differ?
\end{tcolorbox}

\begin{tcolorbox}[hintbox={Theory Hint 3}]
\textbf{Attention Mechanism:} BERT assigns importance scores to words:
\begin{itemize}[noitemsep]
\item High attention: ``crashes'', ``unusable'', ``but''
\item Lower attention: ``the'', ``and'', ``is''
\item Context-dependent: ``works'' gets high attention near ``actually''
\end{itemize}
Formula preview: $\text{Attention} = \text{How much should I focus on each word?}$
\end{tcolorbox}

\newpage

% ============================================
% PART D: EMOTION SPECTRUM
% ============================================

\section*{Part D: Beyond Positive/Negative (10 min)}

\begin{tcolorbox}[colback=white, colframe=black!50]
\textbf{Exercise 4:} Match these reviews to emotions (Joy, Anger, Fear, Surprise, Sadness):

\begin{enumerate}[label=\alph*)]
\item ``I'm worried my data isn't secure'' \hfill Emotion: \_\_\_\_\_\_\_\_
\item ``Can't believe how much this has improved!'' \hfill Emotion: \_\_\_\_\_\_\_\_
\item ``Disappointed it doesn't work on older phones'' \hfill Emotion: \_\_\_\_\_\_\_\_
\item ``This is absolutely unacceptable!'' \hfill Emotion: \_\_\_\_\_\_\_\_
\item ``Exceeded all my expectations!'' \hfill Emotion: \_\_\_\_\_\_\_\_
\end{enumerate}

\textbf{Design Question:} How would you redesign a product differently for users feeling fear vs anger?\\
\vspace{2cm}
\end{tcolorbox}

\begin{tcolorbox}[hintbox={Theory Hint 4}]
\textbf{Emotion Categories:} Modern NLP goes beyond binary sentiment:
\begin{multicols}{2}
\begin{itemize}[noitemsep]
\item \textbf{Joy:} excitement, satisfaction, delight
\item \textbf{Anger:} frustration, irritation, rage
\item \textbf{Fear:} worry, concern, anxiety
\item \textbf{Surprise:} amazement, shock, unexpected
\item \textbf{Sadness:} disappointment, regret, loss
\item \textbf{Disgust:} rejection, aversion, distaste
\end{itemize}
\end{multicols}
Each emotion $\rightarrow$ different design response!
\end{tcolorbox}

% ============================================
% PART E: WORD EMBEDDINGS
% ============================================

\section*{Part E: Words in Space (5 min)}

\begin{tcolorbox}[colback=white, colframe=black!50]
\textbf{Exercise 5:} If words were points in space, which pairs would be closest?

\begin{multicols}{2}
\begin{enumerate}[label=\alph*)]
\item good $\leftrightarrow$ excellent
\item good $\leftrightarrow$ bad  
\item fast $\leftrightarrow$ quick
\item fast $\leftrightarrow$ slow
\item happy $\leftrightarrow$ joyful
\item happy $\leftrightarrow$ sad
\end{enumerate}
\end{multicols}

Circle the closest pairs. \textbf{Bonus:} Can you draw a simple 2D map showing these relationships?
\end{tcolorbox}

\begin{tcolorbox}[hintbox={Theory Hint 5}]
\textbf{Word Embeddings:} BERT converts words to 768-dimensional vectors:
\begin{itemize}[noitemsep]
\item Similar meaning = closer in space
\item ``good'' is closer to ``excellent'' than to ``bad''
\item Context changes position: ``bank'' (river) vs ``bank'' (money)
\item We use t-SNE to visualize in 2D (like a map projection)
\end{itemize}
\end{tcolorbox}

% ============================================
% SYNTHESIS SECTION
% ============================================

\section*{Synthesis: Connect to Design (5 min)}

\begin{tcolorbox}[discoverybox={Design Thinking Connection}]
\textbf{Final Reflection:} Based on your discoveries, answer:

\begin{enumerate}
\item \textbf{Empathy Challenge:} If you only counted positive/negative keywords, what emotions would you miss?\\
\vspace{1.5cm}

\item \textbf{Scale Problem:} You have 10,000 reviews. How long to read them all? (Assume 30 seconds each)\\
\vspace{1cm}

\item \textbf{Design Impact:} Name one feature you'd build differently after understanding hidden emotions:\\
\vspace{1.5cm}
\end{enumerate}
\end{tcolorbox}

% ============================================
% PREPARATION BOX
% ============================================

\begin{tcolorbox}[theorybox={Prepare for Class}]
\textbf{What we'll explore today:}
\begin{itemize}[noitemsep]
\item How BERT achieves 87\% sarcasm detection (vs 15\% for keywords)
\item Why bidirectional reading changes everything
\item Attention mechanisms: the key to understanding
\item Real examples: Netflix emotions, Airbnb hostile review detection
\item Hands-on: Using BERT to analyze product reviews
\end{itemize}

\textbf{Key Terms to Remember:}
BERT, Transformer, Attention, Embedding, Bidirectional, Context, Fine-tuning
\end{tcolorbox}

\vfill
\begin{center}
\textcolor{gray}{\small Week 2: NLP with Transformers + Empathize | ML/AI for Design Thinking}
\end{center}

\end{document}