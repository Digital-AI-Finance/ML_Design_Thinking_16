% Part 2: Linear + Regularization (6 slides)

\begin{frame}{6. OLS with Worked Example}
\begin{columns}
\begin{column}{0.48\textwidth}
\textbf{Ordinary Least Squares}
\begin{itemize}
\item Minimize: $\sum_{i=1}^n (y_i - \hat{y}_i)^2$
\item Solution: $\beta = (X^T X)^{-1} X^T y$
\item Assumptions: $X^T X$ is invertible
\item Unbiased estimator under Gauss-Markov
\end{itemize}

\vspace{0.3cm}
\textbf{Worked Example}
\small
Data: $(1,2,3)$ predicts $(2,4,6)$
\begin{align}
X &= \begin{pmatrix} 1 & 1 \\ 1 & 2 \\ 1 & 3 \end{pmatrix}, y = \begin{pmatrix} 2 \\ 4 \\ 6 \end{pmatrix} \\
\beta &= \begin{pmatrix} 0 \\ 2 \end{pmatrix}
\end{align}
\end{column}
\begin{column}{0.48\textwidth}
\includegraphics[width=\textwidth]{charts/ols_example.pdf}
\end{column}
\end{columns}

\vspace{\fill}
\small\textcolor{gray}{OLS provides minimum variance unbiased estimators under Gauss-Markov assumptions - closed-form solution enables fast computation for linear models}
\end{frame}

\begin{frame}{7. Ridge and LASSO Regularization}
\begin{columns}
\begin{column}{0.48\textwidth}
\textbf{Ridge Regression (L2)}
\begin{itemize}
\item Minimize: $||y - X\beta||^2 + \lambda ||\beta||^2$
\item Shrinks coefficients toward zero
\item Keeps all features
\item Solution: $\beta = (X^T X + \lambda I)^{-1} X^T y$
\end{itemize}

\vspace{0.3cm}
\textbf{LASSO Regression (L1)}
\begin{itemize}
\item Minimize: $||y - X\beta||^2 + \lambda ||\beta||_1$
\item Sets some coefficients to exactly zero
\item Automatic feature selection
\item No closed-form solution
\end{itemize}

\textbf{Elastic Net}
\begin{itemize}
\item Combines L1 + L2: $\alpha||\beta||_1 + (1-\alpha)||\beta||^2$
\item Balances selection and shrinkage
\end{itemize}
\end{column}
\begin{column}{0.48\textwidth}
\includegraphics[width=\textwidth]{charts/ridge_lasso_comparison.pdf}
\end{column}
\end{columns}

\vspace{\fill}
\small\textcolor{gray}{Regularization adds penalty terms to loss function - L2 shrinks coefficients, L1 enforces sparsity, Elastic Net combines both strategies}
\end{frame}

\begin{frame}{8. {\color{mlgreen} SUCCESS: Perfect on Linearly Separable Data}}
\begin{columns}
\begin{column}{0.48\textwidth}
\textbf{Linear Model Triumphs}
\begin{itemize}
\item {\color{mlgreen} Iris setosa classification: 100\% accuracy}
\item {\color{mlgreen} House price in suburbs: R² = 0.94}
\item {\color{mlgreen} Linear trend prediction: MSE = 0.01}
\item {\color{mlgreen} Feature importance: Interpretable}
\end{itemize}

\vspace{0.3cm}
\textbf{Why It Works}
\begin{itemize}
\item Underlying relationship is linear
\item Features are independent
\item Low noise in measurements
\item Sufficient training data
\end{itemize}
\end{column}
\begin{column}{0.48\textwidth}
\includegraphics[width=\textwidth]{charts/linear_success_cases.pdf}

\vspace{0.2cm}
\small
{\color{mlgreen} When assumptions hold, linear models are optimal and interpretable}
\end{column}
\end{columns}
\end{frame}

\begin{frame}{9. {\color{mlred} FAILURE: Terrible on XOR, Nonlinear Boundaries}}
\begin{columns}
\begin{column}{0.48\textwidth}
\textbf{{\color{mlred} Linear Model Failures}}

\small
\begin{tabular}{|c|c|c|c|}
\hline
\textbf{Dataset} & \textbf{Linear Acc} & \textbf{Tree Acc} & \textbf{Gap} \\
\hline
XOR & 50\% & 100\% & 50\% \\
Circles & 52\% & 98\% & 46\% \\
Moons & 58\% & 94\% & 36\% \\
Spirals & 48\% & 89\% & 41\% \\
\hline
\end{tabular}

\vspace{0.3cm}
\textbf{XOR Truth Table}
\begin{tabular}{|c|c|c|}
\hline
$x_1$ & $x_2$ & $y$ \\
\hline
0 & 0 & 0 \\
0 & 1 & 1 \\
1 & 0 & 1 \\
1 & 1 & 0 \\
\hline
\end{tabular}
\end{column}
\begin{column}{0.48\textwidth}
\includegraphics[width=\textwidth]{charts/linear_failure_cases.pdf}

\vspace{0.2cm}
\small
{\color{mlred} Linear boundaries cannot separate XOR or curved patterns}
\end{column}
\end{columns}
\end{frame}

\begin{frame}{10. Root Cause: Linear Assumption Too Restrictive}
\begin{columns}
\begin{column}{0.48\textwidth}
\textbf{Mathematical Limitation}
\begin{itemize}
\item Linear model: $y = w^T x + b$
\item Decision boundary: hyperplane
\item Cannot curve or bend
\item Cannot create islands or holes
\end{itemize}

\vspace{0.3cm}
\textbf{Real-World Examples}
\begin{itemize}
\item Customer behavior (nonlinear)
\item Stock market patterns (chaotic)
\item Medical diagnosis (complex interactions)
\item Image recognition (hierarchical)
\end{itemize}
\end{column}
\begin{column}{0.48\textwidth}
\includegraphics[width=\textwidth]{charts/linear_vs_nonlinear_boundaries.pdf}

\vspace{0.2cm}
\small
Most real-world phenomena require nonlinear decision boundaries
\end{column}
\end{columns}
\end{frame}

\begin{frame}{11. Regularization Tradeoff}
\begin{columns}
\begin{column}{0.48\textwidth}
\textbf{Bias-Variance Tradeoff}
\begin{itemize}
\item $\lambda = 0$: High variance, low bias
\item $\lambda \to \infty$: Low variance, high bias
\item Optimal $\lambda$: Minimizes test error
\item Cross-validation finds optimum
\end{itemize}

\vspace{0.3cm}
\textbf{Practical Guidelines}
\begin{itemize}
\item Start with Ridge for stability
\item Use LASSO for feature selection
\item Elastic Net combines both
\item Grid search for $\lambda$
\end{itemize}
\end{column}
\begin{column}{0.48\textwidth}
\includegraphics[width=\textwidth]{charts/regularization_tradeoff.pdf}

\vspace{0.2cm}
\small
Regularization strength controls the complexity-accuracy tradeoff
\end{column}
\end{columns}
\end{frame}