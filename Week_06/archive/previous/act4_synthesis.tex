% ACT 4: SYNTHESIS (4 slides)
% Connect everything, show broader context, look forward

% Slide 22: Unified Architecture - All Components Together
\begin{frame}[t]{The Complete System: All Pieces Together}
\begin{center}
\begin{tikzpicture}[
  node distance=1.2cm,
  bigbox/.style={rectangle, draw, fill=mllavender3, text width=3cm, text centered, minimum height=1cm, font=\small},
  smallbox/.style={rectangle, draw, fill=mllavender4, text width=2cm, text centered, minimum height=0.7cm, font=\tiny},
  arrow/.style={->, >=stealth, thick, mlpurple}
]

% Top level
\node[bigbox, fill=challenge!20] (challenge) {\textbf{Challenge}\\97\% ideas untested};

% Middle level - Solution components
\node[bigbox, below=1.5cm of challenge, fill=mlblue!20] (context) {\textbf{1. Context Layer}\\RAG System};
\node[bigbox, right=0.5cm of context, fill=mlpurple!20] (prompt) {\textbf{2. Generation Layer}\\Structured Prompts};
\node[bigbox, right=0.5cm of prompt, fill=mlgreen!20] (eval) {\textbf{3. Evaluation Layer}\\Validation};
\node[bigbox, right=0.5cm of eval, fill=mlorange!20] (integrate) {\textbf{4. Integration Layer}\\Coherence};

% Bottom level
\node[bigbox, below=1.5cm of prompt, fill=strategy!20] (output) {\textbf{Solution}\\100 prototypes tested};

% Arrows
\draw[arrow] (challenge) -- (context);
\draw[arrow] (challenge) -- (prompt);
\draw[arrow] (challenge) -- (eval);
\draw[arrow] (challenge) -- (integrate);
\draw[arrow] (context) -- (output);
\draw[arrow] (prompt) -- (output);
\draw[arrow] (eval) -- (output);
\draw[arrow] (integrate) -- (output);

\end{tikzpicture}
\end{center}

\vspace{0.5cm}
\begin{columns}[T]
\column{0.48\textwidth}
\textcolor{mlblue}{\textbf{The 5 Key Innovations}}

\small
\begin{enumerate}
\item \textbf{RAG provides missing context}\\
Retrieves project-specific knowledge
\item \textbf{Structured prompts ensure consistency}\\
Enforces brand and style guidelines
\item \textbf{Multiple variants enable selection}\\
Explore options, pick best
\item \textbf{Validation prevents bad outputs}\\
Automated quality control
\item \textbf{Integration creates coherence}\\
All pieces work together
\end{enumerate}

\column{0.48\textwidth}
\textcolor{mlorange}{\textbf{Information Flow}}

\small
\textbf{Input:} 1 idea (10 words)\\
$\downarrow$\\
\textbf{Context:} +500 words from RAG\\
$\downarrow$\\
\textbf{Prompt:} 510 words structured\\
$\downarrow$\\
\textbf{Generation:} 10 variants\\
$\downarrow$\\
\textbf{Evaluation:} Ranked by quality\\
$\downarrow$\\
\textbf{Integration:} Coherent prototype\\
$\downarrow$\\
\textbf{Output:} Validated, ready to test

\vspace{0.3cm}
\textcolor{mlgray}{\small
Each layer adds value,\\
together they transform\\
raw AI into reliable tool
}
\end{columns}

\bottomnote{Complete system: Challenge → 4 layers → Solution addresses every diagnosed problem}
\end{frame}

% Slide 23: Conceptual Lessons - Beyond Implementation
\begin{frame}[t]{What We Learned: Transferable Principles}
\begin{columns}[T]
\column{0.48\textwidth}
\textcolor{mlpurple}{\Large\textbf{Principle 1}}

\small
\textbf{AI as Collaborator, Not Magic Button}

\vspace{0.3cm}
Human provides:
\begin{itemize}
\item Structure (context, constraints)
\item Judgment (evaluation, selection)
\item Integration (coherence)
\end{itemize}

AI provides:
\begin{itemize}
\item Speed (seconds vs hours)
\item Scale (10 variants easily)
\item Execution (actual creation)
\end{itemize}

\textcolor{mlblue}{\textbf{Why it matters:}}\\
Removes fear of ``AI replacing designers''\\
Humans and AI have complementary strengths

\vspace{0.5cm}
\textcolor{mlcyan}{\Large\textbf{Principle 2}}

\small
\textbf{Structure > Raw Power}

\vspace{0.3cm}
GPT-4 with bad prompts\\
< GPT-3.5 with good structure

\vspace{0.3cm}
\textcolor{mlblue}{\textbf{Why it matters:}}\\
Technique beats technology\\
Focus on framework, not model size

\column{0.48\textwidth}
\textcolor{mlgreen}{\Large\textbf{Principle 3}}

\small
\textbf{Iteration is Now Free}

\vspace{0.3cm}
Traditional cost of change:\\
\$5,000 + 2 weeks

AI cost of change:\\
\$0.50 + 30 seconds

\vspace{0.3cm}
\textcolor{mlblue}{\textbf{Why it matters:}}\\
Unlocks exploration mindset\\
Test 100 ideas instead of 3\\
Fail fast, learn faster

\vspace{0.5cm}
\textcolor{mlorange}{\Large\textbf{Principle 4}}

\small
\textbf{Validation Still Human}

\vspace{0.3cm}
AI creates options\\
Humans decide

\vspace{0.3cm}
Framework:
\begin{itemize}
\item AI generates 10 variants
\item Auto-filter rules (60\% pass)
\item Human selects from top 3
\item AI refines based on feedback
\end{itemize}

\vspace{0.3cm}
\textcolor{mlblue}{\textbf{Why it matters:}}\\
Quality control in human hands\\
Build evaluation criteria, not blind trust
\end{columns}

\bottomnote{These principles apply beyond prototyping - to any AI-augmented creative work}
\end{frame}

% Slide 24: Modern Applications (2024 Production Systems)
\begin{frame}[t]{Modern Applications: 2024 Production Systems}
\begin{columns}[T]
\column{0.32\textwidth}
\textcolor{mlblue}{\textbf{Code Generation}}

\small
\textbf{GitHub Copilot}
\begin{itemize}
\item 1M+ developers
\item 40\% code AI-generated
\item \$10/month
\item GPT-4 Turbo powered
\end{itemize}

\vspace{0.3cm}
\textbf{Replit Agent}
\begin{itemize}
\item Full apps from description
\item Node.js, Python, React
\item Deploy in minutes
\item Context-aware generation
\end{itemize}

\vspace{0.3cm}
\textbf{Cursor}
\begin{itemize}
\item AI-first IDE
\item Codebase understanding
\item Multi-file edits
\item Test generation
\end{itemize}

\vspace{0.3cm}
\textcolor{mlgray}{\tiny All use structured generation\\with codebase context (RAG)}

\column{0.32\textwidth}
\textcolor{mlpurple}{\textbf{Design Tools}}

\small
\textbf{Figma AI}
\begin{itemize}
\item Auto-layout suggestions
\item Component generation
\item Design system enforcement
\item 5M+ users
\end{itemize}

\vspace{0.3cm}
\textbf{v0 by Vercel}
\begin{itemize}
\item UI from description
\item React + Tailwind
\item Interactive prototypes
\item Copy-paste code
\end{itemize}

\vspace{0.3cm}
\textbf{Midjourney v6}
\begin{itemize}
\item Commercial-quality images
\item Style consistency
\item 15M+ users
\item Structured prompting critical
\end{itemize}

\vspace{0.3cm}
\textcolor{mlgray}{\tiny Consistency through\\prompt engineering}

\column{0.32\textwidth}
\textcolor{mlgreen}{\textbf{Content \& Prototyping}}

\small
\textbf{Claude Artifacts}
\begin{itemize}
\item Interactive prototypes
\item HTML/React/SVG
\item Iterative refinement
\item Context preservation
\end{itemize}

\vspace{0.3cm}
\textbf{ChatGPT Canvas}
\begin{itemize}
\item Collaborative writing
\item Side-by-side editing
\item Version control
\item Structured feedback
\end{itemize}

\vspace{0.3cm}
\textbf{Notion AI}
\begin{itemize}
\item Content generation
\item Brand voice training
\item 10M+ users
\item Workspace context
\end{itemize}

\vspace{0.3cm}
\textcolor{mlgray}{\tiny Integration + context\\= production quality}
\end{columns}

\vspace{0.5cm}
\begin{center}
\textcolor{mlblue}{\textbf{Timeline:} 2022 (GPT-3 toys) → 2024 (Production tools)}\\
\textcolor{mlorange}{\small Evolution from novelty to necessity - \$1T market by 2030 (McKinsey)}
\end{center}

\bottomnote{All 2024 tools use the framework principles: context + structure + validation + integration}
\end{frame}

% Slide 25: Summary & Next Week
\begin{frame}[t]{Summary \& Next Week}
\begin{columns}[T]
\column{0.48\textwidth}
\textcolor{mlpurple}{\Large\textbf{What You Now Understand}}

\small
\textbf{The Theory:}
\begin{itemize}
\item Generative AI learns probability distributions $P(x)$
\item Latent space = compressed knowledge (512D)
\item Sampling = creation (controlled randomness)
\item Context narrows distribution → better outputs
\end{itemize}

\vspace{0.3cm}
\textbf{The Framework:}
\begin{itemize}
\item Context layer (RAG)
\item Generation layer (structured prompts)
\item Evaluation layer (validation)
\item Integration layer (coherence)
\end{itemize}

\vspace{0.3cm}
\textbf{The Reality:}
\begin{itemize}
\item Structure beats raw power
\item Human-AI collaboration > either alone
\item Iteration now free
\item 540x faster prototyping
\end{itemize}

\column{0.48\textwidth}
\textcolor{mlblue}{\Large\textbf{What You Can Now Do}}

\small
\begin{enumerate}
\item Build RAG-enhanced prototyping pipelines
\item Write effective structured prompts
\item Evaluate AI outputs systematically
\item Prototype 100 ideas instead of 3
\item Test complex concepts in hours
\item Iterate based on real feedback
\item Integrate AI into creative workflow
\end{enumerate}

\vspace{0.5cm}
\textcolor{mlgreen}{\Large\textbf{Next Week: Responsible AI}}

\small
\textbf{With great power comes great responsibility:}
\begin{itemize}
\item Bias detection and mitigation
\item Fairness metrics and evaluation
\item Transparency and explainability
\item Legal and ethical considerations
\item Building trustworthy AI systems
\end{itemize}

\vspace{0.3cm}
\textcolor{mlorange}{\textbf{Lab Assignment:}}\\
Build complete prototype of your app idea\\
using structured generation framework.\\
Due: Next week
\end{columns}

\vspace{0.3cm}
\begin{center}
\begin{tcolorbox}[colback=mlpurple!20, colframe=mlpurple, width=0.9\textwidth]
\centering
\textbf{The Innovation Bottleneck is Broken}\\
\\
\small
You can now test every idea. The question is no longer\\
``Can I afford to prototype?'' but ``Which ideas should I test first?''
\end{tcolorbox}
\end{center}

\bottomnote{From 3 ideas/year to 100 ideas/year - innovation at scale is now possible}
\end{frame}