% ==================== PART 4: DESIGN SYNTHESIS ====================
\section{Part 4: From ML Insights to Design Action}

% Slide 22: Emotion Taxonomy Discovery
\begin{frame}[t]{Discovering Your Emotion Taxonomy}
\Large\textbf{What BERT Found in 50,000 Reviews}
\normalsize

\begin{columns}[T]
\column{0.55\textwidth}
\begin{center}
\includegraphics[width=0.95\textwidth]{charts/emotion_taxonomy.pdf}
\end{center}

\column{0.43\textwidth}
\textbf{6 Core Emotion Clusters:}

\footnotesize
\begin{enumerate}
\item \textbf{Frustration (28\%)}
   \begin{itemize}
   \footnotesize
   \item Long wait times
   \item Complex interfaces
   \item Missing features
   \end{itemize}

\item \textbf{Delight (22\%)}
   \begin{itemize}
   \footnotesize
   \item Unexpected features
   \item Beautiful design
   \item Fast performance
   \end{itemize}

\item \textbf{Confusion (18\%)}
   \begin{itemize}
   \footnotesize
   \item Unclear instructions
   \item Hidden functions
   \item Inconsistent behavior
   \end{itemize}

\item \textbf{Trust (15\%)}
   \begin{itemize}
   \footnotesize
   \item Data security
   \item Reliable service
   \item Transparent pricing
   \end{itemize}

\item \textbf{Disappointment (12\%)}
   \begin{itemize}
   \footnotesize
   \item Unmet expectations
   \item Quality issues
   \item Broken promises
   \end{itemize}

\item \textbf{Satisfaction (5\%)}
   \begin{itemize}
   \footnotesize
   \item Met expectations
   \item Good value
   \item Works as intended
   \end{itemize}
\end{enumerate}
\end{columns}

\vspace{0.3em}
\begin{tcolorbox}[colback=mllavender4, colframe=mlpurple, width=0.9\textwidth]
\centering
\textbf{Key Insight:} Only 5\% express simple satisfaction - 95\% have complex emotional responses
\end{tcolorbox}

\bottomnote{BERT's attention mechanism reveals emotional nuances that keyword counting would miss entirely}
\end{frame}

% Slide 23: User Journey Emotional Mapping
\begin{frame}[t]{Mapping the Emotional User Journey}
\Large\textbf{When and Why Emotions Shift}
\normalsize

\begin{center}
\includegraphics[width=0.85\textwidth]{charts/user_journey_emotions.pdf}
\end{center}

\begin{columns}[T]
\column{0.32\textwidth}
\textbf{Onboarding (Day 1-7):}
\begin{itemize}
\footnotesize
\item Hope → Confusion
\item ``Excited but lost''
\item 43\% mention complexity
\item Design focus: Simplify
\end{itemize}

\column{0.32\textwidth}
\textbf{Learning (Week 2-4):}
\begin{itemize}
\footnotesize
\item Confusion → Mastery
\item ``Starting to get it''
\item 67\% show progress
\item Design focus: Guide
\end{itemize}

\column{0.32\textwidth}
\textbf{Routine (Month 2+):}
\begin{itemize}
\footnotesize
\item Satisfaction → Boredom
\item ``Same old thing''
\item 31\% seek novelty
\item Design focus: Engage
\end{itemize}
\end{columns}

\vspace{0.5em}
\begin{center}
\textcolor{mlpurple}{\textbf{Emotion-driven design: Address the right feeling at the right time}}
\end{center}

\bottomnote{Time-series analysis of reviews reveals critical emotional transition points}
\end{frame}

% Slide 24: Priority Matrix - What Drives Churn
\begin{frame}[t]{The Churn Priority Matrix}
\Large\textbf{Which Emotions Predict User Loss?}
\normalsize

\begin{columns}[T]
\column{0.55\textwidth}
\begin{center}
\includegraphics[width=0.95\textwidth]{charts/churn_priority_matrix.pdf}
\end{center}

\column{0.43\textwidth}
\textbf{High Impact on Churn:}

\begin{tcolorbox}[colback=mlred!10, colframe=mlred]
\footnotesize
\textbf{1. Frustration + Confusion (72\% quit)}\\
``Can't figure it out and support doesn't help''\\[0.3em]
\textbf{Design Action:} Better onboarding + in-app help
\end{tcolorbox}

\vspace{0.3em}
\begin{tcolorbox}[colback=mlorange!10, colframe=mlorange]
\footnotesize
\textbf{2. Disappointment + Distrust (64\% quit)}\\
``Not what was promised, feels sketchy''\\[0.3em]
\textbf{Design Action:} Align marketing with reality
\end{tcolorbox}

\vspace{0.3em}
\textbf{Low Impact on Churn:}

\begin{tcolorbox}[colback=mlgreen!10, colframe=mlgreen]
\footnotesize
\textbf{3. Minor Frustrations (8\% quit)}\\
``Annoying but I deal with it''\\[0.3em]
\textbf{Design Action:} Fix in regular updates
\end{tcolorbox}
\end{columns}

\bottomnote{Emotion combinations matter more than individual emotions - BERT captures these interactions}
\end{frame}

% Slide 25: From Clusters to Personas
\begin{frame}[t]{Data-Driven Persona Development}
\Large\textbf{Real Users, Not Imagined Ones}
\normalsize

\begin{columns}[T]
\column{0.25\textwidth}
\textbf{The Enthusiast}\\
\footnotesize
15\% of users\\[0.2em]
\textbf{Language:}
\begin{itemize}
\footnotesize
\item ``Love it!''
\item ``Game changer''
\item ``Can't wait for...''
\end{itemize}
\textbf{Emotions:}
\begin{itemize}
\footnotesize
\item Delight: 78\%
\item Anticipation: 22\%
\end{itemize}
\textbf{Design Need:}\\
Advanced features

\column{0.25\textwidth}
\textbf{The Struggler}\\
\footnotesize
35\% of users\\[0.2em]
\textbf{Language:}
\begin{itemize}
\footnotesize
\item ``Trying to...''
\item ``Can't find...''
\item ``How do I...''
\end{itemize}
\textbf{Emotions:}
\begin{itemize}
\footnotesize
\item Confusion: 61\%
\item Frustration: 39\%
\end{itemize}
\textbf{Design Need:}\\
Better guidance

\column{0.25\textwidth}
\textbf{The Pragmatist}\\
\footnotesize
30\% of users\\[0.2em]
\textbf{Language:}
\begin{itemize}
\footnotesize
\item ``It works''
\item ``Does the job''
\item ``Fair price''
\end{itemize}
\textbf{Emotions:}
\begin{itemize}
\footnotesize
\item Satisfaction: 82\%
\item Neutral: 18\%
\end{itemize}
\textbf{Design Need:}\\
Reliability

\column{0.25\textwidth}
\textbf{The Critic}\\
\footnotesize
20\% of users\\[0.2em]
\textbf{Language:}
\begin{itemize}
\footnotesize
\item ``Should have...''
\item ``Compared to X...''
\item ``Missing...''
\end{itemize}
\textbf{Emotions:}
\begin{itemize}
\footnotesize
\item Disappointment: 54\%
\item Frustration: 46\%
\end{itemize}
\textbf{Design Need:}\\
Feature parity
\end{columns}

\vspace{0.5em}
\begin{center}
\begin{tcolorbox}[colback=mllavender4, colframe=mlpurple, width=0.9\textwidth]
\centering
\textbf{These personas emerged from BERT clustering - not designer assumptions}
\end{tcolorbox}
\end{center}

\bottomnote{BERT groups users by emotional language patterns, revealing authentic user segments}
\end{frame}

% Slide 26: Empathy at Scale
\begin{frame}[t]{Empathy at Scale: Understanding Millions Individually}
\Large\textbf{Personalized Understanding, Automated Delivery}
\normalsize

\begin{columns}[T]
\column{0.48\textwidth}
\textbf{The Scale Challenge:}
\begin{itemize}
\item 1 million users
\item 100 reviews each
\item 100 million opinions
\item Impossible to read manually
\end{itemize}

\vspace{0.5em}
\textbf{BERT's Solution:}
\begin{itemize}
\item Process all in 24 hours
\item Understand each individually
\item Group by emotional need
\item Generate targeted responses
\end{itemize}

\vspace{0.5em}
\textbf{Personalization Examples:}
\begin{itemize}
\footnotesize
\item Frustrated user → Proactive support offer
\item Confused user → Tutorial recommendation
\item Delighted user → Feature beta invitation
\item Disappointed user → Feedback survey
\end{itemize}

\column{0.48\textwidth}
\begin{center}
\includegraphics[width=0.95\textwidth]{charts/empathy_scale_pyramid.pdf}
\end{center}

\begin{tcolorbox}[colback=mlgreen!10, colframe=mlgreen]
\textbf{Real Impact:}\\[0.3em]
\footnotesize
• Response time: 48hr → 2hr\\
• User satisfaction: +34\%\\
• Support tickets: -52\%\\
• Retention: +18\%\\[0.3em]
\normalsize
Empathy becomes scalable
\end{tcolorbox}
\end{columns}

\bottomnote{True empathy means understanding each user's unique emotional context - now possible at scale}
\end{frame}

% Slide 27: Case Study - Spotify Wrapped
\begin{frame}[t]{Case Study: Spotify Wrapped}
\Large\textbf{Emotion-Driven Engagement at Scale}
\normalsize

\begin{columns}[T]
\column{0.55\textwidth}
\textbf{The Challenge:}
\begin{itemize}
\item 400 million users
\item Make each feel special
\item Drive social sharing
\item Increase engagement
\end{itemize}

\vspace{0.3em}
\textbf{NLP Analysis Revealed:}
\begin{itemize}
\item Users want validation of taste
\item Nostalgia drives sharing
\item Uniqueness matters most
\item Discovery excites users
\end{itemize}

\vspace{0.3em}
\textbf{Design Response:}
\begin{itemize}
\item Personal emotion words: ``Your year was Rebellious''
\item Unique statistics: ``Top 0.5\% of fans''
\item Nostalgic moments: ``You played X 47 times in March''
\item Social proof: ``Share your unique taste''
\end{itemize}

\column{0.43\textwidth}
\begin{center}
\includegraphics[width=0.9\textwidth]{charts/spotify_wrapped_emotions.pdf}
\end{center}

\begin{tcolorbox}[colback=mlgreen!10, colframe=mlgreen]
\textbf{Results:}\\[0.3em]
• 120M users engaged\\
• 60M social shares\\
• 40\% increase in app usage\\
• 21\% increase in subscriptions\\[0.3em]
\textbf{ROI: 400\% on NLP investment}
\end{tcolorbox}
\end{columns}

\vspace{0.3em}
\begin{center}
\textcolor{mlpurple}{\textbf{Emotion understanding → Personalization → Engagement → Business value}}
\end{center}

\bottomnote{Spotify uses NLP to understand emotional connections to music, creating deeply personal experiences}
\end{frame}

% Slide 28: Workshop Preview
\begin{frame}[t]{Your Turn: Emotion Mining Workshop}
\Large\textbf{Apply These Techniques to Real Data}
\normalsize

\begin{columns}[T]
\column{0.48\textwidth}
\textbf{Workshop Exercise (45 minutes):}

\vspace{0.3em}
\textbf{Dataset:}
\begin{itemize}
\item 5,000 app store reviews
\item Your choice of app category
\item Mix of ratings (1-5 stars)
\item Real user language
\end{itemize}

\vspace{0.3em}
\textbf{Your Tasks:}
\begin{enumerate}
\item Load pre-trained BERT model
\item Fine-tune on 500 labeled reviews
\item Analyze remaining 4,500 reviews
\item Discover emotion clusters
\item Identify top 3 pain points
\item Generate design recommendations
\end{enumerate}

\vspace{0.3em}
\textbf{Tools Provided:}
\begin{itemize}
\item Jupyter notebook template
\item Pre-processed data
\item BERT model loaded
\item Visualization functions
\end{itemize}

\column{0.48\textwidth}
\textbf{Expected Outputs:}

\begin{tcolorbox}[colback=mllavender4, colframe=mlpurple]
\footnotesize
\textbf{1. Emotion Distribution Chart}\\
Show percentages of each emotion\\[0.3em]
\textbf{2. Pain Point Priority List}\\
Ranked by impact on ratings\\[0.3em]
\textbf{3. User Segment Personas}\\
Data-driven, not assumed\\[0.3em]
\textbf{4. Design Recommendations}\\
Specific, actionable, prioritized
\end{tcolorbox}

\vspace{0.3em}
\textbf{Learning Objectives:}
\begin{itemize}
\footnotesize
\item Use BERT for emotion analysis
\item Interpret attention weights
\item Convert ML insights to design actions
\item Experience the power of scale
\end{itemize}

\vspace{0.3em}
\begin{center}
\textcolor{mlgreen}{\textbf{From 5,000 reviews to 5 key insights in 45 minutes!}}
\end{center}
\end{columns}

\bottomnote{Hands-on experience: The same techniques used by Netflix, Amazon, and Spotify}
\end{frame}