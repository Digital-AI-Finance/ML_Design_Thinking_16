% Template Showcase Presentation
% Demonstrates all features of the Optimal Readability Master Template
% Generated: 2025-01-20 08:50

\input{NLP_slides/common/master_template.tex}

% Document metadata
\title[Template Showcase]{Master Template Showcase}
\subtitle{Optimal Readability Design System}
\author[NLP Course]{Natural Language Processing Course 2025}
\institute{Department of Computer Science}
\date{\today}

\begin{document}

% ====================================
% TITLE SLIDE
% ====================================
\begin{frame}
\titlepage
\end{frame}

% ====================================
% TABLE OF CONTENTS
% ====================================
\begin{frame}{Presentation Overview}
\tableofcontents
\end{frame}

% ====================================
% SECTION 1: COLOR SYSTEM
% ====================================
\section{Optimal Readability Color System}

\begin{frame}{Color Palette - WCAG AAA Compliant}
\begin{columns}[T]
\column{0.5\textwidth}
\textbf{Primary Colors:}
\begin{itemize}
\item \textcolor{PureBlack}{Pure Black (21:1 contrast)}
\item \textcolor{DeepBlue}{Deep Blue (12.6:1 contrast)}
\item \textcolor{DarkGray}{Dark Gray (9.7:1 contrast)}
\end{itemize}

\vspace{1em}
\textbf{Chart Colors (Colorblind-Safe):}
\begin{itemize}
\item \textcolor{ChartBlue}{Chart Blue - Primary data}
\item \textcolor{ChartOrange}{Chart Orange - Secondary data}
\item \textcolor{ChartTeal}{Chart Teal - Tertiary data}
\item \textcolor{ChartPurple}{Chart Purple - Quaternary data}
\end{itemize}

\column{0.5\textwidth}
\textbf{Semantic Colors:}
\begin{itemize}
\item \textcolor{DarkGreen}{Dark Green - Success/Positive}
\item \textcolor{DarkRed}{Dark Red - Warning/Negative}
\item \textcolor{LightGray}{Light Gray - Borders/Grids}
\end{itemize}

\vspace{1em}
\keypoint{All colors meet WCAG AAA standards for accessibility}
\end{columns}
\end{frame}

% ====================================
% SECTION 2: SLIDE LAYOUTS
% ====================================
\section{Standard Slide Layouts}

\twocolslide{Two-Column Layout}{
\textbf{Left Column Content:}
\begin{itemize}
\item Primary information
\item Key concepts
\item Main arguments
\item Core definitions
\end{itemize}

\vspace{1em}
This layout is ideal for:
\begin{enumerate}
\item Comparisons
\item Before/after scenarios
\item Theory vs practice
\item Problem/solution pairs
\end{enumerate}
}{
\textbf{Right Column Content:}
\begin{itemize}
\item Supporting details
\item Examples
\item Visualizations
\item Code snippets
\end{itemize}

\vspace{1em}
\formula{P(A|B) = \frac{P(B|A) \cdot P(A)}{P(B)}}

\secondary{Secondary text appears in gray for visual hierarchy}
}

\threecolslide{Three-Column Layout}{
\textbf{Column 1}
\begin{itemize}
\item Concept A
\item Detail 1
\item Detail 2
\end{itemize}

\vspace{0.5em}
\success{Advantages:}
\begin{itemize}
\item Pro 1
\item Pro 2
\end{itemize}
}{
\textbf{Column 2}
\begin{itemize}
\item Concept B
\item Detail 1
\item Detail 2
\end{itemize}

\vspace{0.5em}
\warning{Limitations:}
\begin{itemize}
\item Con 1
\item Con 2
\end{itemize}
}{
\textbf{Column 3}
\begin{itemize}
\item Concept C
\item Detail 1
\item Detail 2
\end{itemize}

\vspace{0.5em}
\data{Results:}
\begin{itemize}
\item Finding 1
\item Finding 2
\end{itemize}
}

% ====================================
% SECTION 3: VISUALIZATION INTEGRATION
% ====================================
\section{Chart and Visualization Integration}

\fullchartslide{Full Chart Display}{figures/model_comparison.pdf}

\chartslide{Scaled Chart with Caption}{0.75}{figures/attention_heatmap.pdf}

\conceptslide{Concept Slide with Visualization}{
\textbf{Encoder-Decoder Architecture}

Key components:
\begin{itemize}
\item \highlight{Encoder}: Processes input sequence
\item \highlight{Context Vector}: Fixed-size representation
\item \highlight{Decoder}: Generates output sequence
\end{itemize}

\vspace{1em}
Mathematical formulation:
\formula{h_t = f(x_t, h_{t-1})}
\formula{c = q(h_1, h_2, ..., h_T)}
\formula{s_t = g(y_{t-1}, s_{t-1}, c)}

\vspace{1em}
\secondary{Note: Variable length input/output capability}
}{figures/encoder_decoder.pdf}

\begin{frame}{Multi-Panel Dashboard}
\begin{center}
\includegraphics[width=0.78\textwidth]{figures/training_dashboard.pdf}
\end{center}
\secondary{Comprehensive training monitoring with multiple metrics}
\end{frame}

% ====================================
% SECTION 4: CODE INTEGRATION
% ====================================
\section{Code and Algorithm Display}

\begin{frame}[fragile]{Code Display with Syntax Highlighting}
\begin{lstlisting}[language=Python]
import torch
import torch.nn as nn

class Seq2Seq(nn.Module):
    def __init__(self, encoder, decoder, device):
        super().__init__()
        self.encoder = encoder
        self.decoder = decoder
        self.device = device

    def forward(self, src, trg, teacher_forcing_ratio=0.5):
        # Encoder processes entire source sequence
        encoder_outputs, hidden = self.encoder(src)

        # Decoder generates target sequence
        outputs = []
        input = trg[0]  # <SOS> token

        for t in range(1, trg.shape[0]):
            output, hidden = self.decoder(input, hidden, encoder_outputs)
            outputs.append(output)

            # Teacher forcing decision
            use_teacher_force = random.random() < teacher_forcing_ratio
            input = trg[t] if use_teacher_force else output.argmax(1)

        return torch.stack(outputs)
\end{lstlisting}
\end{frame}

% ====================================
% SECTION 5: TABLES AND DATA
% ====================================
\section{Tables and Data Presentation}

\tableslide{Performance Comparison Table}{
\begin{tabular}{lccc}
\toprule
\textbf{Model} & \textbf{BLEU} & \textbf{Speed} & \textbf{Memory} \\
\midrule
RNN & 23.5 & Fast & Low \\
LSTM & 31.2 & Medium & Medium \\
GRU & 30.8 & Medium & Medium \\
Transformer & \highlight{41.3} & Slow & High \\
\bottomrule
\end{tabular}

\vspace{1em}
\keypoint{Transformer achieves best quality at computational cost}
}

\begin{frame}{Complex Table with Multiple Sections}
\begin{center}
\begin{tabular}{l|ccc|ccc}
\toprule
& \multicolumn{3}{c|}{\textbf{Training}} & \multicolumn{3}{c}{\textbf{Inference}} \\
\textbf{Architecture} & Time & GPU & Batch & Speed & Memory & Latency \\
\midrule
\textbf{Sequential Models} & & & & & & \\
\quad RNN & 2h & 4GB & 32 & 1000/s & 1GB & 5ms \\
\quad LSTM & 3h & 8GB & 32 & 800/s & 2GB & 6ms \\
\quad GRU & 2.5h & 6GB & 32 & 850/s & 1.5GB & 5.5ms \\
\midrule
\textbf{Attention Models} & & & & & & \\
\quad Seq2Seq+Attn & 4h & 12GB & 16 & 400/s & 3GB & 12ms \\
\quad Transformer & 8h & 16GB & 8 & 200/s & 4GB & 20ms \\
\bottomrule
\end{tabular}
\end{center}

\vspace{1em}
\secondary{Performance metrics across different model architectures and operational phases}
\end{frame}

% ====================================
% SECTION 6: MATHEMATICAL NOTATION
% ====================================
\section{Mathematical Notation and Formulas}

\begin{frame}{Mathematical Expressions}
\begin{columns}[T]
\column{0.5\textwidth}
\textbf{Probability Notation:}
\begin{itemize}
\item Joint: $\prob{A, B}$
\item Conditional: $\prob{A \given B}$
\item Marginal: $\prob{A} = \sum_b \prob{A, B=b}$
\end{itemize}

\vspace{1em}
\textbf{Optimization:}
\formula{\theta^* = \argmax_\theta \mathcal{L}(\theta)}
\formula{\hat{y} = \argmax_y P(y \given x)}

\column{0.5\textwidth}
\textbf{Neural Network Operations:}

Attention mechanism:
\formula{\alpha_{ij} = \frac{\exp(e_{ij})}{\sum_{k=1}^{T_x} \exp(e_{ik})}}

Context vector:
\formula{c_i = \sum_{j=1}^{T_x} \alpha_{ij} h_j}

Output distribution:
\formula{P(y_t) = \softmax(W_o \cdot s_t)}
\end{columns}

\vspace{1em}
\keypoint{All mathematical notation follows standard ML conventions}
\end{frame}

% ====================================
% SECTION 7: PEDAGOGICAL ELEMENTS
% ====================================
\section{Educational Components}

\begin{frame}{Text Highlighting and Emphasis}
\begin{columns}[T]
\column{0.5\textwidth}
\textbf{Highlighting Commands:}
\begin{itemize}
\item \highlight{Primary highlight} - key concepts
\item \secondary{Secondary text} - supporting info
\item \success{Success message} - positive outcomes
\item \warning{Warning text} - important cautions
\item \data{Data reference} - metrics/values
\item \dataalt{Alternative data} - comparisons
\end{itemize}

\column{0.5\textwidth}
\textbf{Usage Examples:}

The \highlight{attention mechanism} solved the \warning{bottleneck problem} in sequence models.

Performance improved from \data{23.5 BLEU} to \dataalt{41.3 BLEU}.

\success{Key achievement}: Variable-length sequences

\secondary{Note: Results may vary with dataset size}
\end{columns}

\vspace{1em}
\keypoint{Consistent color coding enhances comprehension and retention}
\end{frame}

% ====================================
% SECTION 8: COMPARISON VISUALIZATIONS
% ====================================
\section{Comparison and Analysis}

\begin{frame}{Side-by-Side Comparison}
\begin{columns}[T]
\column{0.48\textwidth}
\begin{center}
\textbf{Traditional RNN}
\end{center}
\begin{itemize}
\item Sequential processing
\item Fixed context window
\item Gradient vanishing
\item Fast inference
\item Low memory usage
\end{itemize}

\vspace{0.5em}
\formula{h_t = \tanh(W_h h_{t-1} + W_x x_t)}

\vspace{0.5em}
\success{Strengths:}
\begin{itemize}
\item Simple implementation
\item Efficient for short sequences
\end{itemize}

\column{0.48\textwidth}
\begin{center}
\textbf{Transformer}
\end{center}
\begin{itemize}
\item Parallel processing
\item Global context access
\item Stable gradients
\item Slower inference
\item High memory usage
\end{itemize}

\vspace{0.5em}
\formula{\text{Attention}(Q,K,V) = \softmax(\frac{QK^T}{\sqrt{d_k}})V}

\vspace{0.5em}
\success{Strengths:}
\begin{itemize}
\item Superior for long sequences
\item Better context modeling
\end{itemize}
\end{columns}
\end{frame}

\begin{frame}{Performance Analysis Charts}
\begin{columns}[T]
\column{0.5\textwidth}
\begin{center}
\includegraphics[width=0.95\textwidth]{figures/sequence_analysis.pdf}
\end{center}

\column{0.5\textwidth}
\begin{center}
\includegraphics[width=0.95\textwidth]{figures/bleu_comparison.pdf}
\end{center}
\end{columns}

\vspace{0.5em}
\secondary{Left: Computational complexity | Right: Translation quality evolution}
\end{frame}

% ====================================
% SECTION 9: BEST PRACTICES
% ====================================
\section{Template Best Practices}

\begin{frame}{Design Principles}
\begin{columns}[T]
\column{0.5\textwidth}
\textbf{Typography Guidelines:}
\begin{itemize}
\item 8pt base font size for readability
\item Bold for emphasis, not color alone
\item Consistent heading hierarchy
\item Maximum 3 font sizes per slide
\end{itemize}

\vspace{1em}
\textbf{Color Usage:}
\begin{itemize}
\item Black text on white background
\item Accents for structure, not decoration
\item Semantic colors for meaning
\item Colorblind-safe chart palettes
\end{itemize}

\column{0.5\textwidth}
\textbf{Layout Principles:}
\begin{itemize}
\item Maximum 3 key points per slide
\item Consistent spacing and alignment
\item Visual hierarchy through size/weight
\item White space for clarity
\end{itemize}

\vspace{1em}
\textbf{Content Organization:}
\begin{itemize}
\item Clear section divisions
\item Logical flow and progression
\item Summary/takeaway boxes
\item Minimal text, maximum impact
\end{itemize}
\end{columns}

\vspace{1em}
\keypoint{Simplicity and consistency enhance learning outcomes}
\end{frame}

% ====================================
% SECTION 10: SUMMARY
% ====================================
\section{Summary and Resources}

\summaryslide{Template Features Summary}{
\begin{itemize}
\item \highlight{WCAG AAA compliant} color system
\item \highlight{Colorblind-safe} visualization palette
\item \highlight{Multiple layout} templates
\item \highlight{Integrated chart} generation system
\item \highlight{Mathematical notation} support
\item \highlight{Code highlighting} with Python/LaTeX
\item \highlight{Semantic color} coding
\item \highlight{Accessibility-first} design
\end{itemize}
}

\begin{frame}{Resources and Files}
\begin{columns}[T]
\column{0.5\textwidth}
\textbf{Core Template Files:}
\begin{itemize}
\item \texttt{master\_template.tex}
\item \texttt{chart\_utils.py}
\item \texttt{generate\_charts.py}
\end{itemize}

\vspace{1em}
\textbf{Documentation:}
\begin{itemize}
\item Layout command reference
\item Color palette guide
\item Chart generation examples
\item Best practices checklist
\end{itemize}

\column{0.5\textwidth}
\textbf{Example Usage:}
\begin{itemize}
\item Week 4: Seq2Seq models
\item Attention mechanisms
\item Training dashboards
\item Performance comparisons
\end{itemize}

\vspace{1em}
\textbf{File Organization:}
\begin{itemize}
\item \texttt{figures/} - Generated charts
\item \texttt{scripts/} - Python utilities
\item \texttt{previous/} - Version archive
\end{itemize}
\end{columns}

\vspace{1em}
\keypoint{This template ensures consistent, accessible presentations across the course}
\end{frame}

% ====================================
% END SLIDE
% ====================================
\begin{frame}
\begin{center}
\vspace{2cm}
{\Huge \textbf{Thank You}}

\vspace{1cm}
{\Large Questions?}

\vspace{2cm}
\secondary{Template Version 1.0 - Optimal Readability}

\vspace{0.5cm}
\secondary{Natural Language Processing Course 2025}
\end{center}
\end{frame}

\end{document}