\documentclass[8pt]{beamer}
\usetheme{Madrid}
\usepackage{graphicx}
\usepackage{booktabs}
\usepackage{adjustbox}
\usepackage{multicol}
\usepackage{amsmath}

% Color definitions
\definecolor{mlblue}{RGB}{31, 119, 180}
\definecolor{mlorange}{RGB}{255, 127, 14}
\definecolor{mlgreen}{RGB}{44, 160, 44}
\definecolor{mlred}{RGB}{214, 39, 40}
\definecolor{mlpurple}{RGB}{148, 103, 189}
\definecolor{mlgray}{RGB}{127, 127, 127}

% Remove navigation symbols
\setbeamertemplate{navigation symbols}{}

% Clean itemize/enumerate
\setbeamertemplate{itemize items}[circle]
\setbeamertemplate{enumerate items}[default]

% Reduce margins for more content space
\setbeamersize{text margin left=5mm,text margin right=5mm}

% Title information
\title{Week 3: NLP for Emotional Context}
\subtitle{Understanding Language as Window to User Experience}
\author{Prof. Dr. Joerg Osterrieder}
\institute{ML-Augmented Design Thinking - BSc Course}
\date{\today}

\begin{document}

% Title slide
\begin{frame}[t]
\titlepage
\end{frame}

% Table of contents
\begin{frame}[t]{Week 3 Overview}
\tableofcontents
\vfill
\footnotesize
\textbf{Learning Objectives:} Master sentiment analysis from basic to advanced. Understand BERT and transformers. Apply NLP to design thinking. Build production sentiment systems. Integrate emotional insights into UX design.
\end{frame}

% Section 1: Introduction to NLP and Sentiment Analysis
\section{Introduction to NLP and Sentiment Analysis}

\begin{frame}[t]
\vfill
\centering
\begin{beamercolorbox}[sep=8pt,center]{title}
\usebeamerfont{title}\Large Introduction to NLP and Sentiment Analysis\par
\end{beamercolorbox}
\vfill
\end{frame}

\begin{frame}[t]{The Power of Understanding Emotion}
\begin{columns}[T]
\begin{column}{0.38\textwidth}
\textbf{Traditional Analysis:}
\begin{itemize}
\item Keyword counting
\item Manual categorization
\item Surface-level insights
\item Limited scale
\item Misses context
\end{itemize}
\vspace{10pt}
\textbf{Challenge:}\\
How to understand emotion at scale?
\end{column}
\begin{column}{0.58\textwidth}
\includegraphics[width=\textwidth]{charts/emotion_spectrum_heatmap.pdf}
\end{column}
\end{columns}
\vfill
\footnotesize
\textbf{NLP Revolution:} Modern NLP transforms emotion understanding. BERT achieves 94\% accuracy. Real-time processing handles 10K+ reviews/minute. Contextual understanding detects sarcasm and nuanced emotions.
\end{frame}

\begin{frame}[t]{Language as Window to User Experience}
\centering
\includegraphics[width=0.95\textwidth]{charts/language_emotion_flow.pdf}
\vspace{5pt}
\small Every word reveals frustration points, delight moments, and hidden needs
\vfill
\footnotesize
\textbf{Design Applications:} Automated user research through review mining. Real-time sentiment monitoring. Persona development based on emotional patterns. Journey mapping with sentiment analysis. A/B testing with emotional impact measurement.
\end{frame}

\begin{frame}[t]{Context is Everything}
\begin{columns}[T]
\begin{column}{0.38\textwidth}
\textbf{Same Words, Different Meanings:}
\begin{itemize}
\item ``This is sick!''
\item ``It's fine''
\item ``Interesting choice''
\item ``Thanks for nothing''
\end{itemize}
\vspace{10pt}
\textbf{Context Sources:}
\begin{itemize}
\item Domain knowledge
\item User demographics
\item Historical patterns
\item Surrounding text
\end{itemize}
\end{column}
\begin{column}{0.58\textwidth}
\includegraphics[width=\textwidth]{charts/context_sentiment_examples.pdf}
\end{column}
\end{columns}
\vfill
\footnotesize
\textbf{Technical Implementation:} Contextual embeddings capture semantic meaning. BERT processes bidirectionally for complete context. Domain adaptation fine-tunes for specific use cases. Attention mechanisms highlight relevant context automatically.
\end{frame}

% Section 2: Technical Foundations
\section{Technical Foundations}

\begin{frame}[t]
\vfill
\centering
\begin{beamercolorbox}[sep=8pt,center]{title}
\usebeamerfont{title}\Large Technical Foundations\par
\end{beamercolorbox}
\vfill
\end{frame}

\begin{frame}[t]{Text Preprocessing Pipeline}
\begin{columns}[T]
\begin{column}{0.25\textwidth}
\textbf{Critical Steps:}
\begin{enumerate}
\item HTML/URL removal
\item Character handling
\item Encoding normalization
\item Tokenization
\item Lowercasing
\item Stopword removal
\end{enumerate}
\vspace{10pt}
\textbf{Quality Check:}\\
99.5\% clean text required
\end{column}
\begin{column}{0.75\textwidth}
\includegraphics[width=\textwidth]{charts/text_preprocessing_pipeline.pdf}
\end{column}
\end{columns}
\vfill
\footnotesize
\textbf{Implementation:} Regular expressions handle 80\% of cleaning. Unicode normalization prevents encoding issues. Custom tokenizers preserve domain terms. Quality metrics track effectiveness. Automated pipeline processes 1M+ documents/hour.
\end{frame}

\begin{frame}[t]{BERT: Bidirectional Understanding}
\centering
\includegraphics[width=0.85\textwidth]{charts/bert_bidirectional.pdf}
\vspace{5pt}
\small Revolutionary bidirectional context understanding changes everything
\vfill
\footnotesize
\textbf{BERT Innovation:} Unlike left-to-right models, BERT sees entire context simultaneously. Masked language modeling trains bidirectional representations. Pre-training on 3.3B words then fine-tuning. 110M parameters for Base, 340M for Large.
\end{frame}

% Section 3: Implementation Methods
\section{Implementation Methods}

\begin{frame}[t]
\vfill
\centering
\begin{beamercolorbox}[sep=8pt,center]{title}
\usebeamerfont{title}\Large Implementation Methods\par
\end{beamercolorbox}
\vfill
\end{frame}

\begin{frame}[t]{Model Performance Comparison}
\centering
\includegraphics[width=0.95\textwidth]{charts/model_comparison_enhanced.pdf}
\vspace{5pt}
\small Comprehensive evaluation across accuracy, speed, and resource requirements
\vfill
\footnotesize
\textbf{Evaluation Framework:} Cross-validation ensures robust estimates. Multiple datasets test generalization. Statistical significance validates improvements. Error analysis identifies failure modes. Human evaluation validates automated metrics.
\end{frame}

% Section 4: Real-World Applications
\section{Real-World Applications}

\begin{frame}[t]
\vfill
\centering
\begin{beamercolorbox}[sep=8pt,center]{title}
\usebeamerfont{title}\Large Real-World Applications\par
\end{beamercolorbox}
\vfill
\end{frame}

\begin{frame}[t]{Amazon Case Study Overview}
\centering
\includegraphics[width=\textwidth]{charts/amazon_case_overview.pdf}
\vfill
\footnotesize
\textbf{Scale and Scope:} Amazon processes 2M+ customer reviews daily. Multi-language support for global markets. Real-time sentiment analysis influences search rankings. Review quality scoring prevents manipulation. Automated moderation flags inappropriate content.
\end{frame}

% Section 5: Performance Assessment
\section{Performance Assessment}

\begin{frame}[t]
\vfill
\centering
\begin{beamercolorbox}[sep=8pt,center]{title}
\usebeamerfont{title}\Large Performance Assessment\par
\end{beamercolorbox}
\vfill
\end{frame}

\begin{frame}[t]{Performance Metrics Summary}
\small
\begin{table}
\centering
\begin{tabular}{lcccc}
\toprule
Method & Accuracy & Speed & Resources & Interpretability \\
\midrule
Rule-Based & 68\% & Very Fast & Low & High \\
Traditional ML & 78\% & Fast & Medium & Medium \\
Deep Learning & 88\% & Medium & High & Low \\
BERT/Transformers & 94\% & Slow & Very High & Medium \\
\bottomrule
\end{tabular}
\end{table}
\vspace{10pt}
\includegraphics[width=0.8\textwidth]{charts/method_comparison.pdf}
\vfill
\footnotesize
\textbf{Selection Guidelines:} Rule-based for interpretability and speed. Traditional ML for balanced performance. Deep learning for accuracy with sufficient data. Transformers for state-of-the-art results with computational resources.
\end{frame}

% Conclusions
\section{Conclusions and Next Steps}

\begin{frame}[t]{Key Takeaways}
\begin{columns}[T]
\begin{column}{0.4\textwidth}
\textbf{Technical Achievements:}
\begin{itemize}
\item 94\% sentiment accuracy
\item Real-time processing
\item Multilingual support
\item Aspect-level analysis
\item Production deployment
\end{itemize}
\vspace{10pt}
\textbf{Design Integration:}
\begin{itemize}
\item Journey mapping enhancement
\item Pain point identification
\item Delight moment discovery
\item Persona development
\item Impact measurement
\end{itemize}
\end{column}
\begin{column}{0.6\textwidth}
\includegraphics[width=\textwidth]{charts/nlp_impact_metrics.pdf}
\end{column}
\end{columns}
\vfill
\footnotesize
\textbf{Future Directions:} Multimodal sentiment analysis combining text, image, and audio. Real-time personalization based on emotional state. Ethical AI ensuring fair and unbiased analysis. Cross-cultural emotion understanding for global applications.
\end{frame}

\begin{frame}[t]{Thank You}
\centering
\Large Questions and Discussion\\
\vspace{20pt}
\normalsize
\textbf{Week 3 Summary:}\\
From basic sentiment to advanced NLP systems\\
\vspace{10pt}
\textbf{Next Week:}\\
Classification for Problem Definition\\
\vspace{10pt}
\textbf{Practical Exercise:}\\
Build your own sentiment analyzer with BERT
\end{frame}

\end{document}