% ==================== PART 5: HANDS-ON EXERCISES ====================
\section{Hands-on Practice}

% Exercise Overview
\begin{frame}[t]{Practice Exercises: From Theory to Application}
\Large\textbf{Three Levels of Hands-on Learning}
\normalsize

\begin{columns}[T]
\column{0.32\textwidth}
\textbf{Exercise 1: Basic}\\
\textbf{Sentiment Analysis}\\[0.3em]
\footnotesize
\textbf{Time:} 20 minutes\\
\textbf{Difficulty:} Beginner\\[0.3em]
\textbf{Task:}
\begin{itemize}
\footnotesize
\item Use pre-trained model
\item Analyze 100 reviews
\item Classify positive/negative
\item Calculate accuracy
\end{itemize}

\vspace{0.3em}
\textbf{Learning Goal:}\\
Understand basic NLP pipeline

\vspace{0.3em}
\textbf{Tools:}
\begin{itemize}
\footnotesize
\item TextBlob or
\item Hugging Face pipeline
\end{itemize}

\column{0.32\textwidth}
\textbf{Exercise 2: Intermediate}\\
\textbf{Emotion Detection}\\[0.3em]
\footnotesize
\textbf{Time:} 45 minutes\\
\textbf{Difficulty:} Medium\\[0.3em]
\textbf{Task:}
\begin{itemize}
\footnotesize
\item Fine-tune BERT
\item 6-class emotions
\item Analyze attention weights
\item Visualize results
\end{itemize}

\vspace{0.3em}
\textbf{Learning Goal:}\\
Work with transformers

\vspace{0.3em}
\textbf{Tools:}
\begin{itemize}
\footnotesize
\item Transformers library
\item Pre-labeled dataset
\end{itemize}

\column{0.32\textwidth}
\textbf{Exercise 3: Advanced}\\
\textbf{Full Pipeline}\\[0.3em]
\footnotesize
\textbf{Time:} 90 minutes\\
\textbf{Difficulty:} Challenging\\[0.3em]
\textbf{Task:}
\begin{itemize}
\footnotesize
\item Scrape real reviews
\item Preprocess text
\item Fine-tune model
\item Generate personas
\item Create dashboard
\end{itemize}

\vspace{0.3em}
\textbf{Learning Goal:}\\
End-to-end implementation

\vspace{0.3em}
\textbf{Deliverable:}\\
Emotion insights report
\end{columns}

\vspace{0.5em}
\begin{center}
\begin{tcolorbox}[colback=mlgreen!10, colframe=mlgreen, width=0.9\textwidth]
\centering
\textbf{Resources:} Notebooks at github.com/ml-design-course/week3-nlp
\end{tcolorbox}
\end{center}

\bottomnote{Each exercise builds on the previous - complete all three for mastery}
\end{frame}

% Key Takeaways
\begin{frame}[t]{Key Takeaways: What You've Learned}
\Large\textbf{From Words to Design Insights}
\normalsize

\begin{columns}[T]
\column{0.48\textwidth}
\textbf{Technical Understanding:}
\begin{itemize}
\item Why context matters in language
\item How Bag of Words loses information
\item What attention mechanisms do
\item How BERT reads bidirectionally
\item Why pre-training + fine-tuning works
\end{itemize}

\vspace{0.5em}
\textbf{Practical Skills:}
\begin{itemize}
\item Identify emotion in text at scale
\item Use pre-trained models effectively
\item Fine-tune for specific domains
\item Interpret attention visualizations
\item Convert ML outputs to insights
\end{itemize}

\column{0.48\textwidth}
\textbf{Design Applications:}
\begin{itemize}
\item Data-driven persona creation
\item Emotion-based user journeys
\item Priority matrices from ML analysis
\item Scalable empathy systems
\item Personalization strategies
\end{itemize}

\vspace{0.5em}
\textbf{Remember:}
\begin{tcolorbox}[colback=mllavender4, colframe=mlpurple]
\footnotesize
\textbf{The Goal:} Not to read faster, but to understand deeper\\[0.3em]
\textbf{The Method:} Selective attention to what matters\\[0.3em]
\textbf{The Result:} True user understanding at scale
\end{tcolorbox}
\end{columns}

\vspace{0.5em}
\begin{center}
\Large\textcolor{mlpurple}{\textbf{Next Week: Classification \& Problem Definition}}
\end{center}

\bottomnote{You now have the tools to understand what millions of users really feel}
\end{frame}