% Slide 1: Want to CREATE, not just classify
\begin{frame}
\frametitle{The Creation Challenge}
\framesubtitle{Moving Beyond Classification}

\begin{columns}[c]
\column{0.48\textwidth}
\Large
\textbf{Traditional ML:} \\
``What is this?''

\vspace{0.5cm}
\normalsize
\begin{itemize}
\item Email: spam or not?
\item Image: cat or dog?
\item Text: positive sentiment?
\item Patient: high risk?
\end{itemize}

\vspace{0.5cm}
\textcolor{mlred}{\textbf{Limitation:}} Analysis only

\column{0.48\textwidth}
\Large
\textbf{Generative AI:} \\
``Create something new''

\vspace{0.5cm}
\normalsize
\begin{itemize}
\item Generate: realistic images
\item Write: coherent articles
\item Compose: original music
\item Design: novel molecules
\end{itemize}

\vspace{0.5cm}
\textcolor{mlgreen}{\textbf{Power:}} Creation \& innovation
\end{columns}

\vspace{\fill}
\small \textcolor{gray}{
The fundamental shift: from understanding existing data to creating new possibilities
}
\end{frame}

% Slide 2: Generative vs discriminative models distinction
\begin{frame}
\frametitle{Mathematical Foundation}
\framesubtitle{Two Approaches to Learning}

\begin{columns}[c]
\column{0.48\textwidth}
\Large
\textbf{Discriminative Models}

\vspace{0.3cm}
\normalsize
Learn: $P(y|x)$ \\
``Given input x, what's the label y?''

\vspace{0.3cm}
\textbf{Examples:}
\begin{itemize}
\item Logistic regression
\item Random Forest
\item Neural networks (classification)
\item SVM
\end{itemize}

\vspace{0.3cm}
\textcolor{mlblue}{\textbf{Goal:}} Decision boundaries

\column{0.48\textwidth}
\Large
\textbf{Generative Models}

\vspace{0.3cm}
\normalsize
Learn: $P(x)$ or $P(x,y)$ \\
``What does the data distribution look like?''

\vspace{0.3cm}
\textbf{Examples:}
\begin{itemize}
\item Gaussian Mixture Models
\item Variational Autoencoders
\item GANs
\item Diffusion models
\end{itemize}

\vspace{0.3cm}
\textcolor{mlgreen}{\textbf{Goal:}} Data generation
\end{columns}

\vspace{\fill}
\small \textcolor{gray}{
Discriminative: ``Is this a cat?'' | Generative: ``Draw me a cat''
}
\end{frame}

% Slide 3: Capturing full data distribution (hard problem)
\begin{frame}
\frametitle{The Hard Problem}
\framesubtitle{Why Generation is Fundamentally Difficult}

\begin{center}
\includegraphics[width=0.8\textwidth]{charts/distribution_complexity.pdf}
\end{center}

\vspace{0.3cm}

\begin{columns}[c]
\column{0.48\textwidth}
\textbf{Challenges:}
\begin{itemize}
\item High-dimensional spaces
\item Complex dependencies
\item Multimodal distributions
\item Sparse meaningful regions
\end{itemize}

\column{0.48\textwidth}
\textbf{Requirements:}
\begin{itemize}
\item Capture ALL patterns
\item Generate diverse samples
\item Maintain realism
\item Avoid mode collapse
\end{itemize}
\end{columns}

\vspace{\fill}
\small \textcolor{gray}{
Real data lives on complex, high-dimensional manifolds - learning the full distribution is exponentially hard
}
\end{frame}

% Slide 4: Realistic vs diverse tradeoff
\begin{frame}
\frametitle{The Fundamental Tradeoff}
\framesubtitle{Quality vs Diversity Dilemma}

\begin{center}
\includegraphics[width=0.9\textwidth]{charts/quality_diversity_tradeoff.pdf}
\end{center}

\vspace{0.3cm}

\begin{columns}[c]
\column{0.32\textwidth}
\textcolor{mlred}{\textbf{High Quality,}} \\
\textcolor{mlred}{\textbf{Low Diversity}}
\begin{itemize}
\item Mode collapse
\item Repetitive outputs
\item Safe but boring
\end{itemize}

\column{0.32\textwidth}
\textcolor{mlorange}{\textbf{Balanced}} \\
\textcolor{mlorange}{\textbf{Generation}}
\begin{itemize}
\item Realistic variety
\item Novel combinations
\item Creative outputs
\end{itemize}

\column{0.32\textwidth}
\textcolor{mlblue}{\textbf{High Diversity,}} \\
\textcolor{mlblue}{\textbf{Low Quality}}
\begin{itemize}
\item Unrealistic samples
\item Incoherent outputs
\item Creative but unusable
\end{itemize}
\end{columns}

\vspace{\fill}
\small \textcolor{gray}{
The holy grail: generating samples that are both realistic AND diverse
}
\end{frame}

% Slide 5: Quantify: Inception Score, FID, perplexity
\begin{frame}
\frametitle{Measuring Generation Quality}
\framesubtitle{Metrics for Evaluating Generative Models}

\begin{center}
\includegraphics[width=0.9\textwidth]{charts/generation_metrics.pdf}
\end{center}

\vspace{0.3cm}

\begin{columns}[c]
\column{0.32\textwidth}
\textbf{Inception Score (IS)}
\begin{itemize}
\item Range: 1-1000+
\item Higher = better
\item Quality \& diversity
\item $IS = \exp(E[KL(p(y|x)||p(y))])$
\end{itemize}

\column{0.32\textwidth}
\textbf{FID Score}
\begin{itemize}
\item Range: 0-500+
\item Lower = better
\item Feature distance
\item Real vs generated
\end{itemize}

\column{0.32\textwidth}
\textbf{Perplexity (Text)}
\begin{itemize}
\item Range: 1-10,000+
\item Lower = better
\item Predictability
\item Language fluency
\end{itemize}
\end{columns}

\vspace{\fill}
\small \textcolor{gray}{
Quantitative evaluation: IS=300+ (excellent), FID<10 (photorealistic), Perplexity<20 (human-like text)
}
\end{frame}