% Part 1: Machine Learning Foundations
\section{Machine Learning Foundations}

% Section divider
\begin{frame}[plain]
\vfill
\centering
\begin{beamercolorbox}[sep=16pt,center]{title}
\usebeamerfont{title}\Large Part 1: Machine Learning Foundations\par
\vspace{0.5em}
\large Theory, Definitions, and Core Concepts\par
\end{beamercolorbox}
\vfill
\end{frame}

% What is Machine Learning
\begin{frame}{What is Machine Learning? Building from Scratch}
\twocolslide{
\Large\textbf{You Want a Program That Gets Better}
\normalsize
\vspace{0.5em}

Think about email spam detection:
\begin{itemize}
\item You \textbf{show it} 10,000 examples (spam and not spam)
\item It learns patterns in the data
\item It gets better at recognizing new spam
\end{itemize}

\vspace{0.5em}
\textbf{Tom Mitchell (1997) formalized this:}

A program learns from \textbf{Experience} $E$ at \textbf{Task} $T$ measured by \textbf{Performance} $P$ if its performance improves with experience.

\vspace{0.5em}
\textbf{Concrete Example:}
\begin{itemize}
\item[$E$:] 10,000 labeled emails
\item[$T$:] Classify spam vs non-spam
\item[$P$:] 85\% → 95\% accuracy after training
\end{itemize}
}{
\Large\textbf{The Mathematical Pattern}
\normalsize
\vspace{0.5em}

What the algorithm actually does:

\textbf{Step 1:} Given labeled examples
\formula{\mathcal{D} = \{(x_i, y_i)\}_{i=1}^n}

\textbf{Step 2:} Find function that maps inputs to outputs
\formula{f: \mathcal{X} \rightarrow \mathcal{Y}}

\textbf{Step 3:} Minimize errors on training data
\formula{\hat{f} = \argmin_{f \in \mathcal{F}} \sum_{i=1}^n L(y_i, f(x_i)) + \lambda R(f)}

where $L$ measures mistakes, $R$ prevents overfitting

\vspace{0.3em}
\keypoint{This optimization is what ``learning'' means mathematically}
}
\end{frame}

% Types of Learning
\begin{frame}{Three Paradigms of Machine Learning}
\begin{columns}[T]
\begin{column}{0.32\textwidth}
\centering
\textcolor{chartblue}{\Large\textbf{Supervised}}
\normalsize
\vspace{0.3em}

% Chart placeholder
\includegraphics[width=0.85\textwidth]{charts/supervised_paradigm.pdf}

\formula{\{(x_i, y_i)\}_{i=1}^n \rightarrow \hat{f}}

\textbf{Applications:}
\begin{itemize}
\item Email spam detection
\item Medical diagnosis
\item Stock price prediction
\item Image recognition
\end{itemize}

\textbf{Key Algorithms:}
\begin{itemize}
\item Linear Regression
\item Random Forest
\item Neural Networks
\end{itemize}
\end{column}

\begin{column}{0.32\textwidth}
\centering
\textcolor{chartteal}{\Large\textbf{Unsupervised}}
\normalsize
\vspace{0.3em}

% Chart placeholder
\includegraphics[width=0.85\textwidth]{charts/unsupervised_paradigm.pdf}

\formula{\{x_i\}_{i=1}^n \rightarrow \text{Structure}}

\textbf{Applications:}
\begin{itemize}
\item Customer segmentation
\item Anomaly detection
\item Data compression
\item Market basket analysis
\end{itemize}

\textbf{Key Algorithms:}
\begin{itemize}
\item K-means clustering
\item PCA
\item Autoencoders
\end{itemize}
\end{column}

\begin{column}{0.32\textwidth}
\centering
\textcolor{chartorange}{\Large\textbf{Reinforcement}}
\normalsize
\vspace{0.3em}

% Chart placeholder
\includegraphics[width=0.85\textwidth]{charts/reinforcement_paradigm.pdf}

\formula{(s_t, a_t, r_t, s_{t+1}) \rightarrow \pi^*}

\textbf{Applications:}
\begin{itemize}
\item Game playing (Chess, Go)
\item Autonomous vehicles
\item Robotics control
\item Resource allocation
\end{itemize}

\textbf{Key Algorithms:}
\begin{itemize}
\item Q-Learning
\item Policy Gradients
\item Actor-Critic
\end{itemize}
\end{column}
\end{columns}
\end{frame}

% The Learning Process
\begin{frame}{The Machine Learning Pipeline}
\centering
\includegraphics[width=0.85\textwidth]{charts/ml_pipeline.pdf}

\vspace{0.5em}
\textcolor{darkgray}{\small Data Collection $\rightarrow$ Preprocessing $\rightarrow$ Feature Engineering $\rightarrow$ Model Training $\rightarrow$ Validation $\rightarrow$ Deployment}
\end{frame}

% Bias-Variance Tradeoff
\begin{frame}{The Fundamental Tradeoff: Bias vs Variance}
\twocolslide{
\Large\textbf{Mathematical Framework}
\normalsize
\vspace{0.5em}

For any learning algorithm, the expected error can be decomposed as:

\formula{\text{Error} = \text{Bias}^2 + \text{Variance} + \text{Noise}}

\textbf{Bias:} Error from oversimplifying assumptions
\formula{\text{Bias}[\hat{f}(x)] = E[\hat{f}(x)] - f(x)}

\textbf{Variance:} Error from sensitivity to training data
\formula{\text{Var}[\hat{f}(x)] = E[(\hat{f}(x) - E[\hat{f}(x)])^2]}

\keypoint{Key Insight:} There's a fundamental tradeoff between bias and variance
}{
\centering
\includegraphics[width=0.95\textwidth]{charts/bias_variance_tradeoff.pdf}

\vspace{0.5em}
\textbf{Model Complexity Examples:}
\begin{itemize}
\item \textcolor{darkred}{High Bias:} Linear models on nonlinear data
\item \textcolor{chartorange}{Balanced:} Regularized models
\item \textcolor{darkred}{High Variance:} Deep trees, k-NN with small k
\end{itemize}
}
\end{frame}

% ML vs Traditional Programming
\begin{frame}{Machine Learning vs Traditional Programming}
\twocolslide{
\centering
\textcolor{darkgray}{\Large\textbf{Traditional Programming}}
\normalsize
\vspace{0.5em}

\textbf{Process:}
\begin{itemize}
\item Write explicit rules
\item Code logic step by step
\item Handle edge cases manually
\item Deterministic outputs
\end{itemize}

\textbf{Example: Email Classification}
\begin{itemize}
\item IF contains ``FREE'' AND ``LIMITED TIME''
\item THEN classify as spam
\item Requires manual rule updates
\end{itemize}

\textcolor{darkred}{\textbf{Limitations:}}
\begin{itemize}
\item Rules become complex
\item Hard to handle exceptions
\item Doesn't adapt to new patterns
\end{itemize}
}{
\centering
\textcolor{chartblue}{\Large\textbf{Machine Learning}}
\normalsize
\vspace{0.5em}

\textbf{Process:}
\begin{itemize}
\item Provide example data
\item Algorithm learns patterns
\item Generalizes to new cases
\item Probabilistic outputs
\end{itemize}

\textbf{Example: Email Classification}
\begin{itemize}
\item Train on 10,000 labeled emails
\item Learn complex word patterns
\item Automatically adapts to new spam
\end{itemize}

\textcolor{darkgreen}{\textbf{Advantages:}}
\begin{itemize}
\item Handles complex patterns
\item Adapts to new data
\item Discovers hidden relationships
\end{itemize}
}
\end{frame}

% Training, Validation, Test
\begin{frame}{Data Splitting: Training, Validation, and Test Sets}
\twocolslide{
\Large\textbf{Why Split Data?}
\normalsize
\vspace{0.5em}

\highlight{Training Set (60\%):}
\begin{itemize}
\item Used to fit model parameters
\item Algorithm learns from this data
\item Larger is generally better
\end{itemize}

\highlight{Validation Set (20\%):}
\begin{itemize}
\item Used for hyperparameter tuning
\item Model selection and comparison
\item Prevents overfitting to training data
\end{itemize}

\highlight{Test Set (20\%):}
\begin{itemize}
\item Final unbiased evaluation
\item Never seen during development
\item Estimates real-world performance
\end{itemize}
}{
\centering
\includegraphics[width=0.95\textwidth]{charts/data_splitting.pdf}

\vspace{0.5em}
\textbf{Cross-Validation:}
\formula{CV = \frac{1}{k}\sum_{i=1}^k \text{Error}_i}

\begin{itemize}
\item k-fold CV: Split data into k parts
\item Train on k-1, validate on 1
\item Repeat k times, average results
\item More robust than single split
\end{itemize}
}
\end{frame}

% Performance Metrics Overview
\begin{frame}{Evaluating Machine Learning Models}
\twocolslide{
\Large\textbf{Classification Metrics}
\normalsize
\vspace{0.5em}

\textbf{Accuracy:}
\formula{\text{Accuracy} = \frac{\text{Correct Predictions}}{\text{Total Predictions}}}

\textbf{Precision:}
\formula{\text{Precision} = \frac{\text{True Positives}}{\text{True Positives + False Positives}}}

\textbf{Recall (Sensitivity):}
\formula{\text{Recall} = \frac{\text{True Positives}}{\text{True Positives + False Negatives}}}

\textbf{F1-Score:}
\formula{F1 = 2 \cdot \frac{\text{Precision} \times \text{Recall}}{\text{Precision} + \text{Recall}}}
}{
\Large\textbf{Regression Metrics}
\normalsize
\vspace{0.5em}

\textbf{Mean Squared Error:}
\formula{MSE = \frac{1}{n}\sum_{i=1}^n (y_i - \hat{y}_i)^2}

\textbf{Root Mean Squared Error:}
\formula{RMSE = \sqrt{MSE}}

\textbf{Mean Absolute Error:}
\formula{MAE = \frac{1}{n}\sum_{i=1}^n |y_i - \hat{y}_i|}

\textbf{R-squared:}
\formula{R^2 = 1 - \frac{SS_{res}}{SS_{tot}}}
}
\end{frame}