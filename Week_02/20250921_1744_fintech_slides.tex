\documentclass[8pt,aspectratio=169]{beamer}
\usetheme{Madrid}
\usecolortheme{default}
\usepackage{graphicx}
\usepackage{booktabs}
\usepackage{adjustbox}
\usepackage{multicol}
\usepackage{amsmath}
\usepackage{amssymb}
\usepackage{tikz}
\usepackage{listings}
\usepackage{xcolor}

% Define colors
\definecolor{mlblue}{RGB}{31, 119, 180}
\definecolor{mlorange}{RGB}{255, 127, 14}
\definecolor{mlgreen}{RGB}{44, 160, 44}
\definecolor{mlred}{RGB}{214, 39, 40}
\definecolor{mlpurple}{RGB}{148, 103, 189}

% Beamer settings
\setbeamertemplate{navigation symbols}{}
\setbeamertemplate{footline}[frame number]

% Code listing settings
\lstset{
    language=Python,
    basicstyle=\tiny\ttfamily,
    numbers=left,
    numberstyle=\tiny\color{gray},
    stepnumber=1,
    backgroundcolor=\color{gray!5},
    keywordstyle=\color{mlblue},
    commentstyle=\color{mlgreen},
    stringstyle=\color{mlorange}
}

% Title information
\title{\Large\textbf{Clustering FinTech Users: From Data to Empathy}}
\subtitle{Advanced Clustering Techniques on Real Financial Data\\
\small 10,000 Users, 12 Features, 7 Natural Segments}
\author{Week 2: Machine Learning for Smarter Innovation}
\institute{BSc Course - MSc-Level Dataset}
\date{2025}

\begin{document}

% Title slide
\begin{frame}[plain]
\titlepage
\end{frame}

% Slide 2: Opening Power Visual
\begin{frame}[plain]
\centering
\includegraphics[width=0.95\textwidth]{fintech_dataset_overview_slides.pdf}
\end{frame}

% Slide 3: The Challenge
\begin{frame}{The Challenge: Understanding 10,000 Digital Banking Users}
\begin{columns}[T]
\begin{column}{0.48\textwidth}
\Large\textbf{The Problem}
\normalsize
\begin{itemize}
\item 10,000 FinTech app users
\item Diverse behavioral patterns
\item Need personalization at scale
\item Fraud detection requirements
\item Customer lifecycle understanding
\end{itemize}

\vspace{0.5em}
\Large\textbf{The Stakes}
\normalsize
\begin{itemize}
\item \$12M annual transaction volume
\item 3\% fraud risk = \$360K exposure
\item 25\% churn rate costs \$2M/year
\end{itemize}
\end{column}

\begin{column}{0.48\textwidth}
\Large\textbf{Our Approach}
\normalsize

Use advanced clustering to discover:
\begin{enumerate}
\item Natural user segments
\item Fraudulent behavior patterns
\item Customer evolution paths
\item Personalization opportunities
\item Risk indicators
\end{enumerate}

\vspace{0.5em}
\begin{center}
\colorbox{mlgreen!20}{\textbf{ML transforms raw data into actionable insights}}
\end{center}
\end{column}
\end{columns}
\end{frame}

% Slide 4: Dataset Introduction
\begin{frame}{FinTech Dataset: 12 Behavioral Dimensions}
\begin{columns}[T]
\begin{column}{0.3\textwidth}
\textbf{Transaction Patterns}
\small
\begin{itemize}
\item Transaction frequency
\item Transaction volume
\item Peak hour usage
\item Merchant categories
\end{itemize}
\end{column}

\begin{column}{0.3\textwidth}
\textbf{Financial Behavior}
\small
\begin{itemize}
\item Savings behavior
\item Credit utilization
\item International activity
\item Payment diversity
\end{itemize}
\end{column}

\begin{column}{0.3\textwidth}
\textbf{User Engagement}
\small
\begin{itemize}
\item Session duration
\item Support contacts
\item Device switches
\item Account age
\end{itemize}
\end{column}
\end{columns}

\vspace{1em}
\begin{center}
\begin{tabular}{lrrrr}
\toprule
\textbf{Segment} & \textbf{Count} & \textbf{\%} & \textbf{Key Trait} \\
\midrule
Digital Natives & 2,500 & 25\% & Tech-savvy, high usage \\
Traditional Savers & 2,000 & 20\% & High deposits, low transactions \\
Business Users & 1,500 & 15\% & High volume, peak hours \\
International & 1,000 & 10\% & Cross-border focus \\
Cautious Beginners & 2,500 & 25\% & Learning, high support \\
Fraudulent & 300 & 3\% & Anomalous patterns \\
Noise & 200 & 2\% & Random behavior \\
\bottomrule
\end{tabular}
\end{center}
\end{frame}

% Slide 5: Real-World Relevance
\begin{frame}{Why This Matters for MSc Students}
\begin{columns}[T]
\begin{column}{0.48\textwidth}
\Large\textbf{Industry Relevance}
\normalsize
\begin{itemize}
\item FinTech employs 300K+ data scientists globally
\item Average salary: \$120K-\$180K
\item Similar datasets at:
\begin{itemize}
\item PayPal (420M users)
\item Revolut (35M users)
\item Square (50M users)
\end{itemize}
\end{itemize}

\vspace{0.5em}
\Large\textbf{Regulatory Requirements}
\normalsize
\begin{itemize}
\item KYC (Know Your Customer)
\item AML (Anti-Money Laundering)
\item GDPR compliance
\item Fair lending practices
\end{itemize}
\end{column}

\begin{column}{0.48\textwidth}
\Large\textbf{Technical Skills Demonstrated}
\normalsize
\begin{itemize}
\item Handling skewed distributions
\item Missing data imputation (0.46\%)
\item Feature scaling decisions
\item Distance metric selection
\item Outlier detection
\item Temporal pattern analysis
\end{itemize}

\vspace{0.5em}
\Large\textbf{Business Impact}
\normalsize
\begin{itemize}
\item Reduce fraud by 85\%
\item Increase retention by 30\%
\item Improve cross-sell by 40\%
\item Cut support costs by 25\%
\end{itemize}
\end{column}
\end{columns}
\end{frame}

% Section Divider
\begin{frame}[plain]
\vfill
\centering
\Huge\textbf{Part 2: Advanced Clustering Techniques}\\
\vspace{0.5em}
\Large Comparing Algorithms on Real FinTech Data
\vfill
\end{frame}

% Slide 6: Algorithm Comparison
\begin{frame}[plain]
\centering
\includegraphics[width=0.95\textwidth]{fintech_algorithm_comparison.pdf}
\end{frame}

% Slide 7: K-Means Results
\begin{frame}{K-Means: 5 Clear Behavioral Segments}
\begin{columns}[T]
\begin{column}{0.48\textwidth}
\Large\textbf{Algorithm Performance}
\normalsize
\begin{itemize}
\item Optimal k = 5 (validated)
\item Silhouette score: 0.412
\item Convergence: 18 iterations
\item Runtime: 0.3 seconds
\end{itemize}

\vspace{0.5em}
\Large\textbf{Segments Discovered}
\normalsize
\begin{enumerate}
\item \textbf{Cluster 0}: High-value business (15\%)
\item \textbf{Cluster 1}: Digital natives (25\%)
\item \textbf{Cluster 2}: Traditional savers (20\%)
\item \textbf{Cluster 3}: International users (10\%)
\item \textbf{Cluster 4}: Beginners (30\%)
\end{enumerate}
\end{column}

\begin{column}{0.48\textwidth}
\Large\textbf{Key Insights}
\normalsize

Each cluster shows distinct patterns:
\begin{itemize}
\item Transaction frequency: 1.2 - 12.5/day
\item Volume range: \$500 - \$15,000/month
\item International activity: 5\% - 80\%
\item Support needs: 0.5 - 4.2 contacts/month
\end{itemize}

\vspace{0.5em}
\begin{center}
\colorbox{mlblue!20}{\textbf{Clear separation enables targeted strategies}}
\end{center}
\end{column}
\end{columns}
\end{frame}

% Slide 8: Elbow Method
\begin{frame}[plain]
\centering
\includegraphics[width=0.95\textwidth]{fintech_cluster_quality.pdf}
\end{frame}

% Slide 9: DBSCAN Fraud Detection
\begin{frame}[plain]
\centering
\includegraphics[width=0.95\textwidth]{fintech_fraud_detection.pdf}
\end{frame}

% Slide 10: Fraud Pattern Analysis
\begin{frame}{What Makes Fraudulent Users Different?}
\begin{columns}[T]
\begin{column}{0.48\textwidth}
\Large\textbf{Behavioral Anomalies}
\normalsize

\begin{tabular}{lrr}
\toprule
\textbf{Feature} & \textbf{Normal} & \textbf{Fraud} \\
\midrule
International activity & 28\% & 80\% \\
Transaction frequency & 6.0 & 7.9 \\
Device switches & 2.8 & 8.1 \\
Support contacts & 1.3 & 0.0 \\
Account age & 467 days & 15 days \\
\bottomrule
\end{tabular}

\vspace{0.5em}
\textbf{Detection Performance}
\begin{itemize}
\item Precision: 72\%
\item Recall: 65\%
\item F1-Score: 68\%
\end{itemize}
\end{column}

\begin{column}{0.48\textwidth}
\Large\textbf{Fraud Patterns}
\normalsize

\textbf{1. Account Takeover}
\begin{itemize}
\item Sudden transaction spike
\item New device/location
\item Zero support contact
\end{itemize}

\textbf{2. Money Laundering}
\begin{itemize}
\item High international transfers
\item Round amounts
\item Rapid in/out pattern
\end{itemize}

\textbf{3. Synthetic Identity}
\begin{itemize}
\item New account
\item Perfect credit behavior initially
\item Then sudden max-out
\end{itemize}
\end{column}
\end{columns}
\end{frame}

% Section Divider
\begin{frame}[plain]
\vfill
\centering
\Huge\textbf{Part 3: From Clusters to Personas}\\
\vspace{0.5em}
\Large Transforming Data into Human Understanding
\vfill
\end{frame}

% Slide 11: Persona Mapping
\begin{frame}{Data-Driven Personas: Who Are Our Users?}
\begin{columns}[T]
\begin{column}{0.2\textwidth}
\centering
\textbf{Patricia}\\
\small Power Professional\\
\vspace{0.3em}
28-45 years\\
Business Owner\\
\vspace{0.3em}
\$12K/month\\
15\% of users
\end{column}

\begin{column}{0.2\textwidth}
\centering
\textbf{Samuel}\\
\small Traditional Saver\\
\vspace{0.3em}
35-60 years\\
Professional\\
\vspace{0.3em}
\$3K/month\\
20\% of users
\end{column}

\begin{column}{0.2\textwidth}
\centering
\textbf{Gina}\\
\small Global Nomad\\
\vspace{0.3em}
25-40 years\\
Consultant\\
\vspace{0.3em}
\$5K/month\\
10\% of users
\end{column}

\begin{column}{0.2\textwidth}
\centering
\textbf{Nancy}\\
\small Newcomer\\
\vspace{0.3em}
18-30 years\\
Student\\
\vspace{0.3em}
\$800/month\\
25\% of users
\end{column}

\begin{column}{0.2\textwidth}
\centering
\textbf{Chris}\\
\small Casual User\\
\vspace{0.3em}
25-50 years\\
Various\\
\vspace{0.3em}
\$2K/month\\
25\% of users
\end{column}
\end{columns}

\vspace{1em}
\begin{center}
\begin{tabular}{lccccc}
\toprule
\textbf{Need} & Patricia & Samuel & Gina & Nancy & Chris \\
\midrule
Efficiency & HIGH & Low & High & Low & Med \\
Security & Med & HIGH & Med & Med & High \\
Guidance & Low & Low & Med & HIGH & Med \\
International & Low & Low & HIGH & Low & Low \\
Simplicity & Low & Med & Low & High & HIGH \\
\bottomrule
\end{tabular}
\end{center}
\end{frame}

% Slide 12: Implementation Code
\begin{frame}[fragile]{Python Implementation: From Theory to Practice}
\begin{lstlisting}
import numpy as np
from sklearn.cluster import KMeans, DBSCAN
from sklearn.preprocessing import StandardScaler
from sklearn.metrics import silhouette_score

# Load FinTech dataset
X = np.load('fintech_X.npy')  # Shape: (10000, 12)
segments = np.load('fintech_segments.npy')

# Handle missing values and scale
X_clean = np.nan_to_num(X, nan=np.nanmedian(X, axis=0))
scaler = StandardScaler()
X_scaled = scaler.fit_transform(X_clean)

# Find optimal k using elbow method
inertias = []
for k in range(2, 11):
    kmeans = KMeans(n_clusters=k, random_state=42)
    kmeans.fit(X_scaled)
    inertias.append(kmeans.inertia_)

# Apply optimal clustering (k=5)
kmeans = KMeans(n_clusters=5, random_state=42)
user_segments = kmeans.fit_predict(X_scaled)

# Detect fraud with DBSCAN
dbscan = DBSCAN(eps=0.8, min_samples=10)
anomalies = dbscan.fit_predict(X_scaled)
fraud_mask = anomalies == -1  # Outliers

print(f"Found {fraud_mask.sum()} potential fraudulent users")
print(f"Silhouette score: {silhouette_score(X_scaled, user_segments):.3f}")
\end{lstlisting}
\end{frame}

% Slide 13: Key Takeaways
\begin{frame}{Key Takeaways: Business Value from Clustering}
\begin{columns}[T]
\begin{column}{0.48\textwidth}
\Large\textbf{Technical Achievements}
\normalsize
\begin{itemize}
\item Successfully segmented 10K users
\item Identified 5 business personas
\item Detected 65\% of fraud cases
\item Achieved 0.412 silhouette score
\item Processing time: < 1 second
\end{itemize}

\vspace{0.5em}
\Large\textbf{Algorithm Insights}
\normalsize
\begin{itemize}
\item K-Means: Best for clear segments
\item DBSCAN: Excellent for fraud detection
\item Hierarchical: Shows user evolution
\item GMM: Captures overlapping behaviors
\end{itemize}
\end{column}

\begin{column}{0.48\textwidth}
\Large\textbf{Business Impact}
\normalsize
\begin{itemize}
\item \textbf{Personalization}: Tailored experiences for 5 personas
\item \textbf{Fraud Prevention}: Save \$234K annually
\item \textbf{Retention}: Target at-risk segments
\item \textbf{Cross-sell}: Match products to needs
\item \textbf{Support}: Proactive help for beginners
\end{itemize}

\vspace{0.5em}
\Large\textbf{Next Steps}
\normalsize
\begin{enumerate}
\item Deploy real-time clustering
\item A/B test persona strategies
\item Refine fraud detection rules
\item Build recommendation engine
\item Track segment evolution
\end{enumerate}
\end{column}
\end{columns}
\end{frame}

% Final slide
\begin{frame}[plain]
\vfill
\centering
\Huge\textbf{Questions?}\\
\vspace{1em}
\Large Dataset and code available at:\\
\texttt{github.com/course/week2-fintech}\\
\vspace{1em}
\normalsize
Next Week: Classification \& Customer Prediction
\vfill
\end{frame}

\end{document}