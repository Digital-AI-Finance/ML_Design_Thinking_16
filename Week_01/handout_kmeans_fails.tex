\documentclass[11pt,a4paper]{article}
\usepackage{graphicx}
\usepackage{amsmath}
\usepackage{amssymb}
\usepackage{multicol}
\usepackage{tcolorbox}
\usepackage{xcolor}
\usepackage{geometry}
\geometry{margin=1in}

% Define colors
\definecolor{mlblue}{RGB}{31, 119, 180}
\definecolor{mlorange}{RGB}{255, 127, 14}
\definecolor{mlgreen}{RGB}{44, 160, 44}
\definecolor{mlred}{RGB}{214, 39, 40}
\definecolor{mlpurple}{RGB}{148, 103, 189}

\title{\Large\textbf{Discovery Learning: When K-Means Fails}\\
\vspace{0.5em}
\large Hands-on Exploration of Clustering Limitations}
\author{Machine Learning for Smarter Innovation - Week 1}
\date{}

\begin{document}
\maketitle

\section*{Learning Objectives}
By completing this discovery exercise, you will:
\begin{itemize}
    \item Identify scenarios where K-means clustering fails
    \item Understand why these failures occur
    \item Discover which alternative algorithms to use
    \item Develop intuition for algorithm selection
\end{itemize}

\section*{Part 1: Observation Exercise}
Look at the chart showing 6 different data patterns. For each pattern, answer:

\begin{tcolorbox}[colback=mlblue!10, colframe=mlblue!50, title=Pattern Analysis]
\begin{enumerate}
    \item \textbf{Crescent Shapes (Technology Evolution Chains)}
    \begin{itemize}
        \item What shape do you see? \underline{\hspace{3cm}}
        \item Why might circles fail here? \underline{\hspace{5cm}}
        \item Real-world example: \underline{\hspace{6cm}}
    \end{itemize}
    
    \item \textbf{Nested Circles (Core vs Peripheral Innovation)}
    \begin{itemize}
        \item Describe the structure: \underline{\hspace{5cm}}
        \item What's the K-means assumption violated? \underline{\hspace{4cm}}
        \item Business analogy: \underline{\hspace{6cm}}
    \end{itemize}
    
    \item \textbf{Chain Patterns (Innovation Pipelines)}
    \begin{itemize}
        \item What's the data distribution? \underline{\hspace{5cm}}
        \item K-means draws what shape? \underline{\hspace{5cm}}
        \item Industry example: \underline{\hspace{6cm}}
    \end{itemize}
\end{enumerate}
\end{tcolorbox}

\section*{Part 2: Prediction Challenge}
Before looking at the solutions, predict how K-means would cluster these patterns:

\begin{multicols}{2}
\begin{tcolorbox}[colback=mlorange!10, colframe=mlorange!50, title=Your Predictions]
\textbf{Pattern 4: Different Densities}\\
Draw where you think K-means would split:\\
\vspace{3cm}

\textbf{Pattern 5: With Outliers}\\
Circle the outliers K-means would misassign:\\
\vspace{3cm}
\end{tcolorbox}

\begin{tcolorbox}[colback=mlgreen!10, colframe=mlgreen!50, title=Think About]
\begin{itemize}
    \item Does K-means handle density differences?
    \item Can K-means ignore outliers?
    \item What assumptions does K-means make?
    \item When should you NOT use K-means?
\end{itemize}
\end{tcolorbox}
\end{multicols}

\section*{Part 3: Algorithm Matching}
Match each problematic pattern with its best alternative algorithm:

\begin{center}
\begin{tabular}{|l|c|l|}
\hline
\textbf{Pattern Type} & \textbf{Match} & \textbf{Algorithm Options} \\
\hline
Non-spherical shapes & \_\_\_ & A. Hierarchical Clustering \\
Different densities & \_\_\_ & B. DBSCAN \\
Connected components & \_\_\_ & C. Gaussian Mixture Model \\
With outliers & \_\_\_ & D. DBSCAN \\
Nested structures & \_\_\_ & E. Spectral Clustering \\
Elongated clusters & \_\_\_ & F. GMM with full covariance \\
\hline
\end{tabular}
\end{center}

\section*{Part 4: Real-World Application}
Consider your own innovation data or business problem:

\begin{tcolorbox}[colback=mlpurple!10, colframe=mlpurple!50, title=Your Scenario]
\begin{enumerate}
    \item Describe your data: \underline{\hspace{8cm}}
    \item Expected cluster shapes: \underline{\hspace{7cm}}
    \item Potential outliers?: \underline{\hspace{8cm}}
    \item Density variations?: \underline{\hspace{7cm}}
    \item Your algorithm choice: \underline{\hspace{7cm}}
    \item Why this choice?: \underline{\hspace{8cm}}
\end{enumerate}
\end{tcolorbox}

\section*{Key Takeaways}
\begin{itemize}
    \item K-means assumes \textbf{spherical} clusters of \textbf{similar size}
    \item K-means is sensitive to \textbf{outliers} and \textbf{initialization}
    \item DBSCAN finds \textbf{arbitrary shapes} and handles \textbf{noise}
    \item Hierarchical clustering shows \textbf{relationships} at multiple scales
    \item GMM allows \textbf{overlapping} clusters with \textbf{soft assignments}
\end{itemize}

\section*{Challenge Question}
If you had customer behavior data with clear weekday vs. weekend patterns, seasonal variations, and some unusual one-time events, which clustering approach would you use and why?

\vspace{1cm}
\noindent\rule{\textwidth}{0.5pt}
\vspace{0.5cm}
\noindent\rule{\textwidth}{0.5pt}
\vspace{0.5cm}
\noindent\rule{\textwidth}{0.5pt}

\end{document}