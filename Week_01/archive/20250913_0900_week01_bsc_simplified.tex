% Machine Learning for Innovation
% Week 1: Finding Patterns with Clustering
% BSc-level course - No prerequisites required
% Version: 20250913_0900

\documentclass[8pt,aspectratio=169]{beamer}
\usetheme{Madrid}
\usecolortheme{default}

% Essential packages
\usepackage{graphicx}
\usepackage{booktabs}
\usepackage{amsmath}
\usepackage{amsfonts}
\usepackage{amssymb}
\usepackage{adjustbox}
\usepackage{multicol}
\usepackage{tcolorbox}

% Define colors
\definecolor{mlblue}{RGB}{31, 119, 180}
\definecolor{mlorange}{RGB}{255, 127, 14}
\definecolor{mlgreen}{RGB}{44, 160, 44}
\definecolor{mlred}{RGB}{214, 39, 40}
\definecolor{mlpurple}{RGB}{148, 103, 189}

% Beamer settings
\setbeamertemplate{navigation symbols}{}
\setbeamertemplate{footline}[frame number]
\setbeamertemplate{itemize items}[circle]

% Title information
\title{\Large Machine Learning for Innovation}
\subtitle{Week 1: Finding Patterns with Clustering}
\author{BSc Data Science \& AI Program}
\institute{No prerequisites required}
\date{\today}

\begin{document}

% Title slide
\begin{frame}
\titlepage
\end{frame}

% What We'll Learn Today
\begin{frame}
\frametitle{\Large What We'll Learn Today}
\framesubtitle{Finding Groups in Data}

\begin{columns}[T]
\begin{column}{0.48\textwidth}
\Large\textbf{By the end of today:}
\normalsize
\begin{itemize}
\item Understand what clustering does
\item Know when to use it
\item Apply 3 different methods
\item Interpret the results
\item Practice with real examples
\end{itemize}

\vspace{0.5em}
\Large\textbf{No math required!}
\normalsize

We'll use pictures and examples instead.
\end{column}

\begin{column}{0.48\textwidth}
\begin{center}
\includegraphics[width=\textwidth]{charts/simple_clustering_intro.pdf}
\end{center}
\end{column}
\end{columns}
\end{frame}

% Part 1: Understanding the Problem
\begin{frame}
\frametitle{}
\vfill
\begin{center}
{\Huge\textcolor{mlblue}{\textbf{Part 1}}}\\
\vspace{1em}
{\Large Understanding the Problem}\\
\vspace{0.5em}
\textit{Why we need to find patterns}
\end{center}
\vfill
\end{frame}

% The Innovation Challenge
\begin{frame}
\frametitle{\Large The Challenge: Too Much Information}
\framesubtitle{Why computers help us see patterns}

\begin{columns}[T]
\begin{column}{0.48\textwidth}
\Large\textbf{Imagine you have:}
\normalsize
\begin{itemize}
\item 1000 customer reviews
\item 500 product ideas
\item 10,000 survey responses
\end{itemize}

\vspace{0.5em}
\Large\textbf{The problem:}
\normalsize
\begin{itemize}
\item Too much to read manually
\item Hidden patterns we can't see
\item Takes too long to analyze
\end{itemize}

\vspace{0.5em}
\Large\textbf{The solution:}
\normalsize

Let computers find the groups for us!
\end{column}

\begin{column}{0.48\textwidth}
\begin{center}
\includegraphics[width=0.9\textwidth]{charts/chaos_to_clarity.pdf}
\end{center}
\textit{From chaos to organized groups}
\end{column}
\end{columns}
\end{frame}

% What is Clustering - Simple
\begin{frame}
\frametitle{\Large What is Clustering?}
\framesubtitle{A simple explanation}

\begin{center}
\includegraphics[width=0.95\textwidth]{charts/simple_clustering_intro.pdf}
\end{center}

\begin{center}
\Large\textcolor{mlblue}{Clustering = Finding natural groups in data}
\end{center}
\end{frame}

% Real Examples
\begin{frame}
\frametitle{\Large Where We Use Clustering}
\framesubtitle{Everyday examples}

\begin{columns}[T]
\begin{column}{0.48\textwidth}
\Large\textbf{Business:}
\normalsize
\begin{itemize}
\item Group similar customers
\item Organize products in store
\item Find spending patterns
\end{itemize}

\vspace{0.5em}
\Large\textbf{Science:}
\normalsize
\begin{itemize}
\item Group similar genes
\item Classify star types
\item Identify weather patterns
\end{itemize}
\end{column}

\begin{column}{0.48\textwidth}
\Large\textbf{Daily Life:}
\normalsize
\begin{itemize}
\item Music playlists (similar songs)
\item Friend suggestions (similar interests)
\item News categories (similar topics)
\end{itemize}

\vspace{0.5em}
\Large\textbf{Innovation:}
\normalsize
\begin{itemize}
\item Group similar ideas
\item Find user needs
\item Identify opportunities
\end{itemize}
\end{column}
\end{columns}

\vspace{1em}
\begin{center}
\textcolor{mlgreen}{\textbf{All these use clustering to find patterns!}}
\end{center}
\end{frame}

% Part 2: How It Works
\begin{frame}
\frametitle{}
\vfill
\begin{center}
{\Huge\textcolor{mlorange}{\textbf{Part 2}}}\\
\vspace{1em}
{\Large How Clustering Works}\\
\vspace{0.5em}
\textit{Three different approaches}
\end{center}
\vfill
\end{frame}

% Algorithm Comparison - Simple
\begin{frame}
\frametitle{\Large Three Ways to Find Groups}
\framesubtitle{Each has strengths and weaknesses}

\begin{center}
\includegraphics[width=0.95\textwidth]{charts/beginner_algorithm_comparison.pdf}
\end{center}
\end{frame}

% K-Means Detailed
\begin{frame}
\frametitle{\Large K-Means: The Most Common Method}
\framesubtitle{Like organizing by neighborhoods}

\begin{columns}[T]
\begin{column}{0.48\textwidth}
\Large\textbf{How it works:}
\normalsize
\begin{enumerate}
\item Decide how many groups (K)
\item Place K center points randomly
\item Assign each point to nearest center
\item Move centers to middle of their group
\item Repeat until stable
\end{enumerate}

\vspace{0.5em}
\Large\textbf{Good for:}
\normalsize
\begin{itemize}
\item Round, similar-sized groups
\item When you know how many groups
\item Need fast results
\end{itemize}
\end{column}

\begin{column}{0.48\textwidth}
\begin{center}
\includegraphics[width=\textwidth]{charts/kmeans_animation.pdf}
\end{center}
\textit{Watch the centers move to find groups}
\end{column}
\end{columns}
\end{frame}

% DBSCAN Detailed
\begin{frame}
\frametitle{\Large DBSCAN: Finding Any Shape}
\framesubtitle{Like finding crowds at a party}

\begin{columns}[T]
\begin{column}{0.48\textwidth}
\Large\textbf{How it works:}
\normalsize
\begin{enumerate}
\item Look at each point
\item Count neighbors within distance
\item If enough neighbors → core point
\item Connect core points → groups
\item Points with few neighbors → outliers
\end{enumerate}

\vspace{0.5em}
\Large\textbf{Good for:}
\normalsize
\begin{itemize}
\item Weird-shaped groups
\item Finding outliers
\item Don't know how many groups
\end{itemize}
\end{column}

\begin{column}{0.48\textwidth}
\begin{center}
\includegraphics[width=\textwidth]{charts/dbscan_shapes.pdf}
\end{center}
\textit{Can find any shape, even crescents!}
\end{column}
\end{columns}
\end{frame}

% How Good Are Our Groups?
\begin{frame}
\frametitle{\Large How Good Are Our Groups?}
\framesubtitle{Checking if clustering worked}

\begin{columns}[T]
\begin{column}{0.55\textwidth}
\Large\textbf{What makes groups good?}
\normalsize
\begin{itemize}
\item Points in same group are similar
\item Different groups are different
\item Groups make sense for your problem
\end{itemize}

\vspace{0.5em}
\Large\textbf{Simple checks:}
\normalsize
\begin{enumerate}
\item \textbf{Visual}: Do groups look separated?
\item \textbf{Size}: Are groups reasonable sizes?
\item \textbf{Meaning}: Can you explain each group?
\item \textbf{Stability}: Same result if run again?
\end{enumerate}
\end{column}

\begin{column}{0.43\textwidth}
\begin{center}
\includegraphics[width=\textwidth]{charts/cluster_quality.pdf}
\end{center}

\vspace{0.5em}
\textbf{Silhouette Score:}\\
-1 = Bad (overlapping)\\
0 = OK (touching)\\
+1 = Good (separated)
\end{column}
\end{columns}
\end{frame}

% Choosing the Right Method
\begin{frame}
\frametitle{\Large Which Method Should I Use?}
\framesubtitle{A simple decision guide}

\begin{center}
\Large\textbf{Quick Decision Tree}
\end{center}

\begin{columns}[T]
\begin{column}{0.33\textwidth}
\begin{tcolorbox}[colback=mlblue!10, colframe=mlblue!50, title=Use K-Means if:]
\begin{itemize}
\small
\item Need speed
\item Know number of groups
\item Groups are round
\item Similar sizes expected
\end{itemize}
\end{tcolorbox}
\end{column}

\begin{column}{0.33\textwidth}
\begin{tcolorbox}[colback=mlorange!10, colframe=mlorange!50, title=Use DBSCAN if:]
\begin{itemize}
\small
\item Groups have weird shapes
\item Want to find outliers
\item Don't know group count
\item Groups have different densities
\end{itemize}
\end{tcolorbox}
\end{column}

\begin{column}{0.33\textwidth}
\begin{tcolorbox}[colback=mlgreen!10, colframe=mlgreen!50, title=Use Hierarchical if:]
\begin{itemize}
\small
\item Small dataset (<500 points)
\item Want to see all groupings
\item Need tree structure
\item Can wait for results
\end{itemize}
\end{tcolorbox}
\end{column}
\end{columns}

\vspace{1em}
\begin{center}
\textcolor{mlred}{\textbf{Not sure? Try K-Means first - it's simplest!}}
\end{center}
\end{frame}

% Part 3: Practice
\begin{frame}
\frametitle{}
\vfill
\begin{center}
{\Huge\textcolor{mlgreen}{\textbf{Part 3}}}\\
\vspace{1em}
{\Large Let's Practice}\\
\vspace{0.5em}
\textit{A simple example you can follow}
\end{center}
\vfill
\end{frame}

% Practice Example
\begin{frame}
\frametitle{\Large Practice: Student Study Patterns}
\framesubtitle{Follow along with this example}

\begin{center}
\includegraphics[width=0.95\textwidth]{charts/simple_practice_example.pdf}
\end{center}
\end{frame}

% Your Exercise
\begin{frame}
\frametitle{\Large Your Turn: Exercise}
\framesubtitle{Try it yourself}

\begin{columns}[T]
\begin{column}{0.55\textwidth}
\Large\textbf{Dataset: Store Products}
\normalsize

You have data about 200 products:
\begin{itemize}
\item Price (0-100)
\item Customer rating (1-5 stars)
\item Sales per month
\end{itemize}

\vspace{0.5em}
\Large\textbf{Your tasks:}
\normalsize
\begin{enumerate}
\item Load the data
\item Apply K-Means with K=3
\item Plot the results
\item Describe each group
\item Try K=4, which is better?
\end{enumerate}
\end{column}

\begin{column}{0.43\textwidth}
\Large\textbf{Starter code:}
\normalsize
\begin{tcolorbox}[colback=gray!10]
\footnotesize\ttfamily
import pandas as pd\\
from sklearn.cluster import KMeans\\
import matplotlib.pyplot as plt\\
\\
\# Load data\\
data = pd.read\_csv('products.csv')\\
\\
\# Apply clustering\\
kmeans = KMeans(n\_clusters=3)\\
labels = kmeans.fit\_predict(data)\\
\\
\# Plot results\\
plt.scatter(data['price'],\\
\hspace{1cm}data['rating'],\\
\hspace{1cm}c=labels)\\
plt.show()
\end{tcolorbox}
\end{column}
\end{columns}
\end{frame}

% Common Mistakes
\begin{frame}
\frametitle{\Large Common Mistakes to Avoid}
\framesubtitle{Learn from these errors}

\begin{columns}[T]
\begin{column}{0.48\textwidth}
\Large\textbf{Mistake 1: Forgetting to scale}
\normalsize
\begin{itemize}
\item Problem: Price (0-100) vs Rating (1-5)
\item Price dominates because bigger numbers
\item Solution: Always scale your data first!
\end{itemize}

\vspace{0.5em}
\Large\textbf{Mistake 2: Wrong K}
\normalsize
\begin{itemize}
\item Too few groups: Miss important patterns
\item Too many groups: Overly complex
\item Solution: Try different K values
\end{itemize}
\end{column}

\begin{column}{0.48\textwidth}
\Large\textbf{Mistake 3: Ignoring outliers}
\normalsize
\begin{itemize}
\item One weird point can ruin groups
\item K-Means pulls centers toward outliers
\item Solution: Check for outliers first
\end{itemize}

\vspace{0.5em}
\Large\textbf{Mistake 4: Not checking results}
\normalsize
\begin{itemize}
\item Algorithm always gives an answer
\item Doesn't mean it's meaningful!
\item Solution: Always visualize and interpret
\end{itemize}
\end{column}
\end{columns}

\vspace{0.5em}
\begin{center}
\textcolor{mlred}{\textbf{Remember: The computer finds patterns, YOU decide if they make sense!}}
\end{center}
\end{frame}

% Summary
\begin{frame}
\frametitle{\Large Week 1 Summary}
\framesubtitle{What you learned today}

\begin{columns}[T]
\begin{column}{0.33\textwidth}
\begin{tcolorbox}[colback=mlblue!10, colframe=mlblue!50, title=Concepts]
\begin{itemize}
\small
\item Clustering finds groups
\item No labels needed
\item Distance = similarity
\item Multiple methods exist
\end{itemize}
\end{tcolorbox}
\end{column}

\begin{column}{0.33\textwidth}
\begin{tcolorbox}[colback=mlorange!10, colframe=mlorange!50, title=Methods]
\begin{itemize}
\small
\item K-Means (fast, simple)
\item DBSCAN (any shape)
\item Hierarchical (tree view)
\item Each has trade-offs
\end{itemize}
\end{tcolorbox}
\end{column}

\begin{column}{0.33\textwidth}
\begin{tcolorbox}[colback=mlgreen!10, colframe=mlgreen!50, title=Skills]
\begin{itemize}
\small
\item Choose algorithm
\item Apply clustering
\item Check quality
\item Interpret results
\end{itemize}
\end{tcolorbox}
\end{column}
\end{columns}

\vspace{1em}
\begin{center}
\Large\textcolor{mlpurple}{\textbf{You can now find patterns in data!}}
\end{center}

\vspace{0.5em}
\begin{center}
Next week: More advanced clustering techniques
\end{center}
\end{frame}

% Resources
\begin{frame}
\frametitle{\Large Resources for Practice}
\framesubtitle{Where to learn more}

\begin{columns}[T]
\begin{column}{0.48\textwidth}
\Large\textbf{Practice datasets:}
\normalsize
\begin{itemize}
\item Iris flowers (150 samples, 4 features)
\item Wine quality (178 samples, 13 features)
\item Mall customers (200 samples, 5 features)
\end{itemize}

\vspace{0.5em}
\Large\textbf{Python libraries:}
\normalsize
\begin{itemize}
\item scikit-learn (all algorithms)
\item pandas (data handling)
\item matplotlib (visualization)
\end{itemize}
\end{column}

\begin{column}{0.48\textwidth}
\Large\textbf{Online resources:}
\normalsize
\begin{itemize}
\item scikit-learn.org/stable/modules/clustering
\item Google Colab (free Python online)
\item Kaggle Learn (free courses)
\end{itemize}

\vspace{0.5em}
\Large\textbf{Help available:}
\normalsize
\begin{itemize}
\item Office hours: Wed 3-5pm
\item Course forum
\item Study groups
\end{itemize}
\end{column}
\end{columns}

\vspace{1em}
\begin{center}
\textcolor{mlblue}{\textbf{Questions? Just ask - no question is too simple!}}
\end{center}
\end{frame}

\end{document}