% Machine Learning for Smarter Innovation
% Week 1: Clustering for Innovation Discovery
% BSc-level course with comprehensive improvements
% Version: 20250913_0845

\documentclass[8pt,aspectratio=169]{beamer}
\usetheme{Madrid}
\usecolortheme{default}

% Essential packages
\usepackage{graphicx}
\usepackage{booktabs}
\usepackage{amsmath}
\usepackage{amsfonts}
\usepackage{amssymb}
\usepackage{adjustbox}
\usepackage{multicol}
\usepackage{tcolorbox}
\usepackage{tikz}
\usetikzlibrary{shapes.geometric, arrows, positioning}

% Define colors for consistency
\definecolor{mlblue}{RGB}{31, 119, 180}
\definecolor{mlorange}{RGB}{255, 127, 14}
\definecolor{mlgreen}{RGB}{44, 160, 44}
\definecolor{mlred}{RGB}{214, 39, 40}
\definecolor{mlpurple}{RGB}{148, 103, 189}

% Beamer settings
\setbeamertemplate{navigation symbols}{}
\setbeamertemplate{footline}[frame number]
\setbeamertemplate{itemize items}[circle]

% Title information
\title{\Large Machine Learning for Smarter Innovation}
\subtitle{Week 1: Clustering for Innovation Discovery}
\author{BSc Data Science \& AI Program}
\institute{Innovation \& Design Thinking Lab}
\date{\today}

\begin{document}

% Title slide
\begin{frame}
\titlepage
\end{frame}

% Week Overview: The Journey Ahead
\begin{frame}
\frametitle{\Large Week 1 Overview: Your Innovation Discovery Journey}
\framesubtitle{From Chaos to Clarity Through Clustering}

\begin{center}
\includegraphics[width=0.75\textwidth]{../ML_Design_Course/course_visuals/unified_pipeline.pdf}
\end{center}

\vspace{0.5em}
\begin{center}
\normalsize\textbf{Today's Mission:} Transform scattered innovation signals into actionable categories\\
\vspace{0.3em}
\textcolor{mlblue}{``In the noise of innovation, clustering reveals the symphony of patterns''}
\end{center}
\end{frame}

% PART 1 SECTION DIVIDER
\begin{frame}
\frametitle{}
\vfill
\begin{center}
{\Huge\textcolor{mlblue}{\textbf{Part 1}}}\\
\vspace{1em}
{\Large Foundation \& Context}\\
\vspace{0.5em}
\textit{Understanding the Innovation Discovery Challenge}
\end{center}
\vfill
\end{frame}

% Part 1 Learning Objectives
\begin{frame}
\frametitle{\Large Part 1: Learning Objectives}
\framesubtitle{What You'll Master in This Section}

\begin{columns}[T]
\begin{column}{0.48\textwidth}
\begin{tcolorbox}[colback=mlblue!10, colframe=mlblue!50, title={By the end of Part 1, you will:}]
\begin{enumerate}
\normalsize
\item \textbf{Understand} why clustering is essential for innovation discovery
\item \textbf{Identify} the challenges of pattern recognition in innovation data
\item \textbf{Recognize} how unsupervised learning differs from supervised approaches
\item \textbf{Connect} clustering to the empathize stage of design thinking
\end{enumerate}
\end{tcolorbox}
\end{column}

\begin{column}{0.48\textwidth}
\begin{tcolorbox}[colback=mlgreen!10, colframe=mlgreen!50, title={Key Concepts}]
\begin{itemize}
\normalsize
\item Innovation categories vs random ideas
\item Pattern discovery in unstructured data
\item The curse of dimensionality
\item From chaos to actionable insights
\end{itemize}
\end{tcolorbox}

\vspace{0.5em}
\begin{center}
\textcolor{mlorange}{\textbf{Time: 10 minutes}}
\end{center}
\end{column}
\end{columns}
\end{frame}

% PART 1: FOUNDATION & CONTEXT (8 slides)

% Slide 3: The Innovation Challenge
\begin{frame}
\frametitle{\Large The Innovation Discovery Challenge}
\framesubtitle{Why We Need Machine Learning}

\begin{columns}[T]
\begin{column}{0.48\textwidth}
\Large\textbf{The Problem:}
\normalsize
\begin{itemize}
\item 1000s of innovation ideas scattered
\item No clear categories or patterns
\item Hidden connections invisible
\item Manual analysis takes months
\item Biases cloud human judgment
\end{itemize}

\vspace{0.5em}
\Large\textbf{The Opportunity:}
\normalsize
\begin{itemize}
\item Discover natural groupings
\item Find innovation white spaces
\item Identify emerging themes
\item Accelerate decision making
\end{itemize}
\end{column}

\begin{column}{0.48\textwidth}
\begin{center}
\includegraphics[width=\textwidth]{../ML_Design_Course/course_visuals/dual_pipeline.pdf}
\end{center}
\end{column}
\end{columns}
\end{frame}

% Slide 4: What is Clustering?
\begin{frame}
\frametitle{\Large What is Clustering?}
\framesubtitle{Finding Order in Innovation Chaos}

\begin{columns}[T]
\begin{column}{0.55\textwidth}
\Large\textbf{Definition:}
\normalsize
Unsupervised learning that groups similar items without predefined labels

\vspace{0.5em}
\Large\textbf{Innovation Context:}
\normalsize
\begin{itemize}
\item \textbf{Input:} Raw innovation data (ideas, features, feedback)
\item \textbf{Process:} Algorithm finds natural groupings
\item \textbf{Output:} Innovation categories and patterns
\end{itemize}

\vspace{0.5em}
\Large\textbf{Key Difference:}
\normalsize
\begin{itemize}
\item \textcolor{mlred}{Supervised:} You know the categories
\item \textcolor{mlgreen}{Unsupervised:} You discover the categories
\end{itemize}
\end{column}

\begin{column}{0.43\textwidth}
\begin{center}
\includegraphics[width=\textwidth]{../ML_Design_Course/course_visuals/journey_roadmap.pdf}
\end{center}
\end{column}
\end{columns}
\end{frame}

% Knowledge Checkpoint 1: After Part 1
\begin{frame}
\frametitle{\Large Knowledge Check: Part 1}
\framesubtitle{Test Your Understanding of Foundation Concepts}

\begin{center}
\includegraphics[width=0.9\textwidth]{charts/checkpoint_slides.pdf}
\end{center}

\begin{columns}[T]
\begin{column}{0.48\textwidth}
\textbf{Quick Review:}
\begin{itemize}
\normalsize
\item Clustering discovers hidden patterns
\item Unsupervised = no predefined labels
\item Essential for innovation discovery
\item Connects to empathy in design thinking
\end{itemize}
\end{column}

\begin{column}{0.48\textwidth}
\textbf{Ready to Continue?}
\begin{itemize}
\normalsize
\item[\textcolor{mlgreen}{$\checkmark$}] Understood the problem
\item[\textcolor{mlgreen}{$\checkmark$}] Know what clustering does
\item[\textcolor{mlgreen}{$\checkmark$}] See the innovation connection
\item[\textcolor{mlorange}{$\rightarrow$}] Let's dive into algorithms!
\end{itemize}
\end{column}
\end{columns}
\end{frame}

% PART 2 SECTION DIVIDER
\begin{frame}
\frametitle{}
\vfill
\begin{center}
{\Huge\textcolor{mlorange}{\textbf{Part 2}}}\\
\vspace{1em}
{\Large Technical Deep Dive}\\
\vspace{0.5em}
\textit{Mastering Clustering Algorithms}
\end{center}
\vfill
\end{frame}

% Part 2 Learning Objectives
\begin{frame}
\frametitle{\Large Part 2: Learning Objectives}
\framesubtitle{Technical Skills You'll Develop}

\begin{columns}[T]
\begin{column}{0.48\textwidth}
\begin{tcolorbox}[colback=mlgreen!10, colframe=mlgreen!50, title={By the end of Part 2, you will:}]
\begin{enumerate}
\normalsize
\item \textbf{Master} four core clustering algorithms
\item \textbf{Understand} algorithm complexity and scalability
\item \textbf{Apply} evaluation metrics effectively
\item \textbf{Select} the right algorithm for your data
\item \textbf{Optimize} parameters for best results
\end{enumerate}
\end{tcolorbox}
\end{column}

\begin{column}{0.48\textwidth}
\begin{tcolorbox}[colback=mlorange!10, colframe=mlorange!50, title={Algorithms Covered}]
\begin{itemize}
\normalsize
\item \textbf{K-Means:} Fast and simple
\item \textbf{DBSCAN:} Density-based
\item \textbf{Hierarchical:} Tree structure
\item \textbf{GMM:} Soft clustering
\end{itemize}
\end{tcolorbox}

\vspace{0.5em}
\begin{center}
\textcolor{mlblue}{\textbf{Time: 20 minutes}}
\end{center}
\end{column}
\end{columns}
\end{frame}

% PART 2: TECHNICAL CORE (Enhanced with new content)

% Algorithm Complexity Analysis (NEW)
\begin{frame}
\frametitle{\Large Algorithm Complexity \& Performance}
\framesubtitle{Understanding Computational Requirements}

\begin{center}
\includegraphics[width=0.95\textwidth]{charts/algorithm_complexity.pdf}
\end{center}

\begin{columns}[T]
\begin{column}{0.33\textwidth}
\textbf{Time Complexity:}
\begin{itemize}
\small
\item K-means: O(n·k·i·d)
\item DBSCAN: O(n log n)*
\item Hierarchical: O(n²)
\item GMM: O(n·k²·i·d)
\end{itemize}
\end{column}

\begin{column}{0.33\textwidth}
\textbf{Memory Usage:}
\begin{itemize}
\small
\item K-means: Low
\item DBSCAN: Low
\item Hierarchical: High
\item GMM: Moderate
\end{itemize}
\end{column}

\begin{column}{0.33\textwidth}
\textbf{Best For:}
\begin{itemize}
\small
\item K-means: Large data
\item DBSCAN: Arbitrary shapes
\item Hierarchical: Small data
\item GMM: Overlapping clusters
\end{itemize}
\end{column}
\end{columns}
\end{frame}

% Real Datasets Clustering (NEW)
\begin{frame}
\frametitle{\Large Clustering on Real Datasets}
\framesubtitle{Iris, Wine, and Customer Segmentation Examples}

\begin{center}
\includegraphics[width=0.95\textwidth]{charts/real_datasets_clustering.pdf}
\end{center}

\begin{columns}[T]
\begin{column}{0.48\textwidth}
\textbf{Dataset Characteristics:}
\begin{itemize}
\small
\item \textbf{Iris:} 150 samples, 4 features, 3 species
\item \textbf{Wine:} 178 samples, 13 features, 3 cultivars
\item \textbf{Customers:} 200 samples, 3 features, 4 segments
\end{itemize}
\end{column}

\begin{column}{0.48\textwidth}
\textbf{Performance Results:}
\begin{itemize}
\small
\item Iris: K-Means excels (Silhouette=0.46)
\item Wine: GMM handles overlap (Silhouette=0.29)
\item Customers: Clear separation (Silhouette=0.56)
\end{itemize}
\end{column}
\end{columns}
\end{frame}

% Mini Case Study: Spotify (NEW)
\begin{frame}
\frametitle{\Large Case Study: Spotify's Music Discovery}
\framesubtitle{How Clustering Powers Personalized Playlists}

\begin{columns}[T]
\begin{column}{0.55\textwidth}
\Large\textbf{The Challenge:}
\normalsize
\begin{itemize}
\item 100+ million songs in catalog
\item 500+ million users globally
\item Diverse musical tastes
\item Need personalized discovery
\end{itemize}

\vspace{0.5em}
\Large\textbf{The Solution:}
\normalsize
\begin{itemize}
\item \textbf{Audio Features:} Extract 13 dimensions
\item \textbf{Clustering:} Group similar songs
\item \textbf{User Profiles:} Map listening to clusters
\item \textbf{Recommendations:} Adjacent clusters
\end{itemize}
\end{column}

\begin{column}{0.43\textwidth}
\Large\textbf{Results:}
\normalsize
\begin{tcolorbox}[colback=mlgreen!10, colframe=mlgreen!50]
\begin{itemize}
\item 40\% increase in discovery
\item 2.7B Discover Weekly streams
\item 30\% longer listening sessions
\item 75\% user retention
\end{itemize}
\end{tcolorbox}

\vspace{0.5em}
\textbf{Key Insight:}\\
\textit{``Clustering revealed micro-genres users didn't know they loved''}
\end{column}
\end{columns}
\end{frame}

% Knowledge Checkpoint 2: After Part 2
\begin{frame}
\frametitle{\Large Knowledge Check: Part 2}
\framesubtitle{Test Your Technical Understanding}

\begin{columns}[T]
\begin{column}{0.48\textwidth}
\textbf{Algorithm Selection Quiz:}
\begin{enumerate}
\normalsize
\item Large dataset (1M points)?\\
\textcolor{mlgreen}{→ K-Means or MiniBatch K-Means}
\item Non-spherical clusters?\\
\textcolor{mlgreen}{→ DBSCAN}
\item Need probability scores?\\
\textcolor{mlgreen}{→ GMM}
\item Want dendrogram?\\
\textcolor{mlgreen}{→ Hierarchical}
\end{enumerate}
\end{column}

\begin{column}{0.48\textwidth}
\textbf{Complexity Check:}
\begin{itemize}
\normalsize
\item[\textcolor{mlgreen}{$\checkmark$}] K-means: O(n·k·i·d)
\item[\textcolor{mlgreen}{$\checkmark$}] DBSCAN: O(n log n) with index
\item[\textcolor{mlgreen}{$\checkmark$}] Hierarchical: O(n²) memory
\item[\textcolor{mlgreen}{$\checkmark$}] GMM: Soft clustering
\end{itemize}

\vspace{0.5em}
\textbf{Ready for Design Integration?}\\
\textcolor{mlorange}{Let's apply this to innovation!}
\end{column}
\end{columns}
\end{frame}

% PART 3 SECTION DIVIDER
\begin{frame}
\frametitle{}
\vfill
\begin{center}
{\Huge\textcolor{mlpurple}{\textbf{Part 3}}}\\
\vspace{1em}
{\Large Design Integration}\\
\vspace{0.5em}
\textit{Applying Clustering to Innovation Discovery}
\end{center}
\vfill
\end{frame}

% PART 3: DESIGN INTEGRATION (Continue with existing content...)

% Ethical Considerations (NEW)
\begin{frame}
\frametitle{\Large Ethical Considerations in Clustering}
\framesubtitle{Responsible AI for Innovation Discovery}

\begin{columns}[T]
\begin{column}{0.48\textwidth}
\Large\textbf{Potential Biases:}
\normalsize
\begin{itemize}
\item \textbf{Selection Bias:} Who's included in data?
\item \textbf{Feature Bias:} What dimensions matter?
\item \textbf{Algorithmic Bias:} Distance metrics assumptions
\item \textbf{Interpretation Bias:} Label assignment
\end{itemize}

\vspace{0.5em}
\Large\textbf{Mitigation Strategies:}
\normalsize
\begin{itemize}
\item Diverse data collection
\item Multiple algorithm comparison
\item Expert validation
\item Transparent documentation
\end{itemize}
\end{column}

\begin{column}{0.48\textwidth}
\Large\textbf{Fair Clustering Checklist:}
\normalsize
\begin{tcolorbox}[colback=mlred!10, colframe=mlred!50]
\begin{itemize}
\item[$\square$] Representative sampling?
\item[$\square$] Protected attributes removed?
\item[$\square$] Cluster balance checked?
\item[$\square$] Minority groups visible?
\item[$\square$] Results interpretable?
\item[$\square$] Decisions reversible?
\end{itemize}
\end{tcolorbox}

\vspace{0.5em}
\textbf{Remember:}\\
\textit{``Clusters are hypotheses, not truth''}
\end{column}
\end{columns}
\end{frame}

% Cloud & Scalability Options (NEW)
\begin{frame}
\frametitle{\Large Scaling Your Clustering: Cloud \& Distributed Options}
\framesubtitle{From Prototype to Production}

\begin{columns}[T]
\begin{column}{0.48\textwidth}
\Large\textbf{Local Development:}
\normalsize
\begin{itemize}
\item Scikit-learn (< 100K points)
\item Jupyter notebooks
\item Single machine
\item Rapid prototyping
\end{itemize}

\vspace{0.5em}
\Large\textbf{Cloud Platforms:}
\normalsize
\begin{itemize}
\item \textbf{AWS SageMaker:} Built-in algorithms
\item \textbf{Google Cloud AI:} AutoML clustering
\item \textbf{Azure ML:} Drag-and-drop interface
\item \textbf{Databricks:} Spark MLlib integration
\end{itemize}
\end{column}

\begin{column}{0.48\textwidth}
\Large\textbf{Distributed Computing:}
\normalsize
\begin{tcolorbox}[colback=mlblue!10, colframe=mlblue!50]
\textbf{Apache Spark MLlib:}
\begin{itemize}
\small
\item Handles billions of points
\item Distributed K-means
\item Bisecting K-means
\item Gaussian Mixture
\end{itemize}
\end{tcolorbox}

\vspace{0.5em}
\textbf{Cost-Performance Trade-offs:}
\begin{itemize}
\small
\item Local: Free but limited
\item Cloud: Pay-per-use, scalable
\item On-premise: High initial, unlimited use
\end{itemize}
\end{column}
\end{columns}
\end{frame}

% Knowledge Checkpoint 3: After Part 3
\begin{frame}
\frametitle{\Large Knowledge Check: Part 3}
\framesubtitle{Design Integration Mastery}

\begin{columns}[T]
\begin{column}{0.48\textwidth}
\textbf{Application Quiz:}
\begin{enumerate}
\normalsize
\item Clusters reveal what?\\
\textcolor{mlgreen}{→ Innovation patterns}
\item Validation requires?\\
\textcolor{mlgreen}{→ Domain experts}
\item Ethical concerns include?\\
\textcolor{mlgreen}{→ Bias \& fairness}
\item Scale with?\\
\textcolor{mlgreen}{→ Cloud platforms}
\end{enumerate}
\end{column}

\begin{column}{0.48\textwidth}
\textbf{You Can Now:}
\begin{itemize}
\normalsize
\item[\textcolor{mlgreen}{$\checkmark$}] Choose algorithms wisely
\item[\textcolor{mlgreen}{$\checkmark$}] Apply to real data
\item[\textcolor{mlgreen}{$\checkmark$}] Consider ethics
\item[\textcolor{mlgreen}{$\checkmark$}] Scale solutions
\item[\textcolor{mlgreen}{$\checkmark$}] Extract insights
\end{itemize}

\vspace{0.5em}
\textbf{Next: Practice Time!}
\end{column}
\end{columns}
\end{frame}

% PART 4 SECTION DIVIDER
\begin{frame}
\frametitle{}
\vfill
\begin{center}
{\Huge\textcolor{mlgreen}{\textbf{Part 4}}}\\
\vspace{1em}
{\Large Summary \& Practice}\\
\vspace{0.5em}
\textit{Putting It All Together}
\end{center}
\vfill
\end{frame}

% Practice Exercise with Template (ENHANCED)
\begin{frame}
\frametitle{\Large Practice Exercise: Innovation Clustering}
\framesubtitle{Your Turn to Discover Patterns}

\begin{columns}[T]
\begin{column}{0.55\textwidth}
\Large\textbf{The Challenge:}
\normalsize
Analyze 500 innovation proposals from a hackathon

\vspace{0.5em}
\textbf{Starter Code Template:}
\begin{tcolorbox}[colback=gray!10, colframe=gray!50]
\footnotesize
\texttt{import pandas as pd}\\
\texttt{from sklearn.cluster import KMeans}\\
\texttt{from sklearn.preprocessing import StandardScaler}\\
\texttt{}\\
\texttt{\# Load data}\\
\texttt{data = pd.read\_csv('innovations.csv')}\\
\texttt{}\\
\texttt{\# Preprocess}\\
\texttt{scaler = StandardScaler()}\\
\texttt{X = scaler.fit\_transform(data)}\\
\texttt{}\\
\texttt{\# Cluster}\\
\texttt{kmeans = KMeans(n\_clusters=?)}\\
\texttt{labels = kmeans.fit\_predict(X)}
\end{tcolorbox}
\end{column}

\begin{column}{0.43\textwidth}
\Large\textbf{Your Tasks:}
\normalsize
\begin{enumerate}
\item Choose optimal K
\item Apply clustering
\item Evaluate results
\item Interpret patterns
\item Present findings
\end{enumerate}

\vspace{0.5em}
\textbf{Resources Provided:}
\begin{itemize}
\item Jupyter notebook template
\item Sample dataset
\item Evaluation rubric
\item Solution walkthrough
\end{itemize}

\vspace{0.5em}
\textcolor{mlgreen}{\textbf{Submit by: Next Week}}
\end{column}
\end{columns}
\end{frame}

% Key Takeaways with Visual Summary
\begin{frame}
\frametitle{\Large Week 1: Key Takeaways}
\framesubtitle{Your Innovation Discovery Toolkit}

\begin{columns}[T]
\begin{column}{0.33\textwidth}
\begin{tcolorbox}[colback=mlblue!10, colframe=mlblue!50, title=Foundation]
\begin{itemize}
\small
\item Clustering finds hidden patterns
\item Unsupervised learning
\item No labels needed
\item Connects to empathy stage
\end{itemize}
\end{tcolorbox}
\end{column}

\begin{column}{0.33\textwidth}
\begin{tcolorbox}[colback=mlorange!10, colframe=mlorange!50, title=Technical]
\begin{itemize}
\small
\item 4 algorithms mastered
\item Complexity understood
\item Metrics applied
\item Real data processed
\end{itemize}
\end{tcolorbox}
\end{column}

\begin{column}{0.33\textwidth}
\begin{tcolorbox}[colback=mlgreen!10, colframe=mlgreen!50, title=Application]
\begin{itemize}
\small
\item Innovation patterns found
\item Ethical considerations
\item Scalability options
\item Practice exercise ready
\end{itemize}
\end{tcolorbox}
\end{column}
\end{columns}

\vspace{1em}
\begin{center}
\Large\textcolor{mlpurple}{\textbf{You're Ready to Discover Innovation Patterns!}}
\end{center}
\end{frame}

% Additional Resources & Next Week Preview
\begin{frame}
\frametitle{\Large Resources \& Next Week}
\framesubtitle{Continue Your Learning Journey}

\begin{columns}[T]
\begin{column}{0.48\textwidth}
\Large\textbf{Resources:}
\normalsize
\begin{tcolorbox}[colback=gray!10, colframe=gray!50]
\textbf{Documentation:}
\begin{itemize}
\normalsize
\item Scikit-learn clustering guide
\item Course GitHub repository
\item Jupyter notebook templates
\end{itemize}

\textbf{Datasets:}
\begin{itemize}
\normalsize
\item UCI ML Repository
\item Kaggle competitions
\item Innovation dataset collection
\end{itemize}

\textbf{Community:}
\begin{itemize}
\normalsize
\item Course Slack channel
\item Office hours: Wed 3-5pm
\item Peer study groups
\end{itemize}
\end{tcolorbox}
\end{column}

\begin{column}{0.48\textwidth}
\Large\textbf{Next Week Preview:}
\normalsize
\begin{center}
\includegraphics[width=0.9\textwidth]{charts/week2_preview.pdf}
\end{center}

\textbf{Week 2: Advanced Clustering}
\begin{itemize}
\item Spectral clustering
\item Mean shift algorithm
\item Affinity propagation
\item Ensemble methods
\end{itemize}
\end{column}
\end{columns}
\end{frame}

% Glossary of Technical Terms
\begin{frame}
\frametitle{\Large Glossary of Technical Terms}
\framesubtitle{Key Concepts Quick Reference}

\begin{columns}[T]
\begin{column}{0.48\textwidth}
\textbf{Clustering Algorithms:}
\begin{itemize}
\normalsize
\item \textbf{K-Means}: Partitions data into K predefined clusters
\item \textbf{DBSCAN}: Density-based spatial clustering
\item \textbf{Hierarchical}: Builds cluster tree (dendrogram)
\item \textbf{GMM}: Gaussian Mixture Models, soft clustering
\end{itemize}

\vspace{0.5em}
\textbf{Key Parameters:}
\begin{itemize}
\normalsize
\item \textbf{K}: Number of clusters
\item \textbf{eps}: Neighborhood radius (DBSCAN)
\item \textbf{min\_samples}: Minimum points for density
\item \textbf{n\_init}: Number of random initializations
\end{itemize}
\end{column}

\begin{column}{0.48\textwidth}
\textbf{Evaluation Metrics:}
\begin{itemize}
\normalsize
\item \textbf{Silhouette}: Cluster cohesion vs separation [-1,1]
\item \textbf{Inertia}: Sum of squared distances to centroids
\item \textbf{Davies-Bouldin}: Ratio of within to between distances
\item \textbf{Calinski-Harabasz}: Ratio of dispersions
\end{itemize}

\vspace{0.5em}
\textbf{Innovation Terms:}
\begin{itemize}
\normalsize
\item \textbf{Empathy Mapping}: Understanding user perspectives
\item \textbf{Pain Points}: User problems/frustrations
\item \textbf{User Archetypes}: Representative user groups
\item \textbf{Innovation Ecosystem}: Connected stakeholders
\end{itemize}
\end{column}
\end{columns}
\end{frame}

% Implementation Checklist
\begin{frame}
\frametitle{\Large Implementation Checklist}
\framesubtitle{Your Step-by-Step Guide to Success}

\begin{columns}[T]
\begin{column}{0.48\textwidth}
\Large\textbf{Data Preparation:}
\normalsize
\begin{itemize}
\item[$\square$] Collect innovation feedback data
\item[$\square$] Clean and remove duplicates
\item[$\square$] Handle missing values
\item[$\square$] Normalize/standardize features
\item[$\square$] Create feature vectors
\end{itemize}

\vspace{0.5em}
\Large\textbf{Algorithm Selection:}
\normalsize
\begin{itemize}
\item[$\square$] Analyze data distribution
\item[$\square$] Choose appropriate algorithm
\item[$\square$] Set initial parameters
\item[$\square$] Prepare validation strategy
\end{itemize}
\end{column}

\begin{column}{0.48\textwidth}
\Large\textbf{Implementation:}
\normalsize
\begin{itemize}
\item[$\square$] Run clustering algorithm
\item[$\square$] Calculate evaluation metrics
\item[$\square$] Visualize results (PCA/t-SNE)
\item[$\square$] Validate with domain experts
\item[$\square$] Iterate and refine
\end{itemize}

\vspace{0.5em}
\Large\textbf{Innovation Application:}
\normalsize
\begin{itemize}
\item[$\square$] Map clusters to user personas
\item[$\square$] Identify innovation opportunities
\item[$\square$] Create targeted solutions
\item[$\square$] Design prototype features
\item[$\square$] Test with user groups
\end{itemize}
\end{column}
\end{columns}

\vspace{0.5em}
\begin{center}
\textcolor{mlgreen}{\textbf{Ready? Start with data preparation and work your way down!}}
\end{center}
\end{frame}

% APPENDIX: TECHNICAL DEEP DIVE
\appendix
\section{Technical Appendix}

% Appendix content continues as before...
% (Include all existing appendix slides from the original file)

\end{document}