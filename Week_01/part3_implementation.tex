% Part 3: Implementation - Turning Clusters into Innovation Insights
\section{Implementation: From Theory to Practice}

% PART 3 Section Divider
\begin{frame}[plain]
\begin{center}
\vspace{2em}
{\Huge\textcolor{mlorange}{\textbf{PART 3}}}\\
\vspace{0.5em}
{\Large\textbf{Design Integration}}\\
\vspace{1em}
\textit{Turning clusters into innovation insights}\\
\vspace{2em}
\Large
\textbf{What You'll Create:}\\
\vspace{0.5em}
\normalsize
\begin{itemize}
\item Innovation archetypes from clusters
\item Journey maps for each segment
\item Opportunity heat maps
\item Priority matrices
\item Action plans
\end{itemize}
\vspace{1em}
\Large\textcolor{mlpurple}{\textbf{From data to design decisions}}
\end{center}
\end{frame}

% Enhanced Innovation Archetypes from Clusters
\begin{frame}
\frametitle{\Large From Clusters to Innovation Archetypes}
\framesubtitle{Transforming Mathematical Groups into Actionable Personas}

\begin{columns}[T]
\begin{column}{0.48\textwidth}
\begin{center}
\includegraphics[width=0.85\textwidth]{charts/innovation_archetypes.pdf}
\end{center}
\end{column}

\begin{column}{0.48\textwidth}
\begin{tcolorbox}[colback=mlorange!10, colframe=mlorange!50, title=Creating Archetypes]
\small
\textbf{Step 1: Analyze cluster characteristics}
\begin{itemize}
\item Common features
\item Behavioral patterns
\item Pain points
\end{itemize}

\textbf{Step 2: Build personas}
\begin{itemize}
\item Name the archetype
\item Define key traits
\item Identify needs
\end{itemize}

\textbf{Step 3: Design strategies}
\begin{itemize}
\item Tailored solutions
\item Specific messaging
\item Custom journeys
\end{itemize}
\end{tcolorbox}
\end{column}
\end{columns}

\begin{center}
\begin{tcolorbox}[colback=mlpurple!20, colframe=mlpurple!60, width=0.9\textwidth]
\centering
\textbf{Example:} Cluster 3 → "Early Adopters" → Need bleeding-edge features and exclusivity
\end{tcolorbox}
\end{center}
\end{frame}

% Enhanced Opportunity Heat Map
\begin{frame}
\frametitle{\Large Innovation Opportunity Heat Map}
\framesubtitle{Where to Focus Your Innovation Efforts}

\begin{columns}[T]
\begin{column}{0.65\textwidth}
\begin{center}
\includegraphics[width=0.85\textwidth]{charts/opportunity_heatmap.pdf}
\end{center}
\end{column}

\begin{column}{0.33\textwidth}
\begin{tcolorbox}[colback=mlred!10, colframe=mlred!50, title=Reading the Map]
\small
\textbf{Color intensity:}
\begin{itemize}
\item Dark red: High opportunity
\item Orange: Medium potential
\item Yellow: Low priority
\end{itemize}

\textbf{Key findings:}
\begin{itemize}
\item Disruptive: Scalability gaps
\item Incremental: Integration needs
\item Platform: Network effects
\end{itemize}

\textbf{Action:}\\
Focus on red zones first for maximum impact
\end{tcolorbox}
\end{column}
\end{columns}

\begin{center}
\begin{tcolorbox}[colback=mlyellow!20, colframe=mlorange!60, width=0.9\textwidth]
\centering
\textbf{Insight:} 80% of innovation value often comes from 20% of opportunities (Pareto principle)
\end{tcolorbox}
\end{center}
\end{frame}

% Design Priority Matrix Enhanced
\begin{frame}
\frametitle{\Large Design Priority Matrix: Where to Start}
\framesubtitle{Balancing Impact and Effort for Smart Innovation}

\begin{columns}[T]
\begin{column}{0.55\textwidth}
\begin{center}
\includegraphics[width=0.85\textwidth]{charts/design_priority_matrix.pdf}
\end{center}
\end{column}

\begin{column}{0.43\textwidth}
\begin{tcolorbox}[colback=mlgreen!10, colframe=mlgreen!50, title=Action Guide]
\small
\textbf{Quadrant 1: Quick Wins}\\
\textcolor{mlgreen}{High Impact, Low Effort}
\begin{itemize}
\item Do these first!
\item Fast validation
\item Build momentum
\end{itemize}

\textbf{Quadrant 2: Strategic}\\
\textcolor{mlblue}{High Impact, High Effort}
\begin{itemize}
\item Plan carefully
\item Allocate resources
\item Long-term value
\end{itemize}

\textbf{Quadrant 3: Fill-ins}\\
\textcolor{mlorange}{Low Impact, Low Effort}
\begin{itemize}
\item Do when free
\item Nice to have
\end{itemize}

\textbf{Quadrant 4: Avoid}\\
\textcolor{mlred}{Low Impact, High Effort}
\begin{itemize}
\item Not worth it!
\end{itemize}
\end{tcolorbox}
\end{column}
\end{columns}
\end{frame}

% Journey Mapping by Cluster
\begin{frame}
\frametitle{\Large Cluster-Specific Innovation Journeys}
\framesubtitle{Different Paths for Different Innovation Types}

\begin{columns}[T]
\begin{column}{0.55\textwidth}
\begin{center}
\includegraphics[width=0.85\textwidth]{charts/journey_map_clusters.pdf}
\end{center}
\end{column}

\begin{column}{0.43\textwidth}
\begin{tcolorbox}[colback=mlpurple!10, colframe=mlpurple!50, title=Journey Insights]
\small
\textbf{Disruptive (Red):}
\begin{itemize}
\item Fast adoption curve
\item High initial resistance
\item Exponential growth
\end{itemize}

\textbf{Incremental (Blue):}
\begin{itemize}
\item Steady progression
\item Low resistance
\item Linear growth
\end{itemize}

\textbf{Platform (Green):}
\begin{itemize}
\item Network effects
\item Slow start, fast scale
\item Community-driven
\end{itemize}

\textbf{Design implication:}\\
Each needs different support!
\end{tcolorbox}
\end{column}
\end{columns}

\begin{center}
\begin{tcolorbox}[colback=mlyellow!20, colframe=mlorange!60, width=0.9\textwidth]
\centering
\textbf{Key Lesson:} One innovation process doesn't fit all - customize by cluster!
\end{tcolorbox}
\end{center}
\end{frame}