% PART 1 SECTION DIVIDER
\begin{frame}[plain]
\begin{center}
\vspace{2em}
\Large\textcolor{mlblue}{\textbf{PART 1}}\\
\vspace{0.5em}
\Large\textbf{Foundation \& Context}\\
\vspace{2em}
\Large
What we'll explore:\\
\vspace{1em}
\normalsize
\begin{itemize}
\item Why traditional design hits limits
\item How ML amplifies human insight
\item The dual pipeline approach
\item Your learning journey ahead
\end{itemize}
\vspace{2em}
\Large\textcolor{mlpurple}{\textbf{Setting the stage for transformation}}
\end{center}
\end{frame}

% Learning Objectives for Part 1
\begin{frame}
\frametitle{\Large Part 1: Learning Objectives}
\framesubtitle{What You'll Learn in This Section}

\begin{columns}[T]
\begin{column}{0.48\textwidth}
\begin{tcolorbox}[colback=mlblue!10, colframe=mlblue!50, title={By the end of Part 1, you will be able to:}]
\normalsize
\begin{itemize}
\item \textbf{Understand} the limitations of traditional innovation approaches
\item \textbf{Recognize} how ML enhances human creativity
\item \textbf{Explain} the dual pipeline methodology
\item \textbf{Navigate} the 10-week learning journey
\item \textbf{Identify} Week 1's role in the overall course
\end{itemize}
\end{tcolorbox}
\end{column}

\begin{column}{0.48\textwidth}
\begin{tcolorbox}[colback=mlgreen!10, colframe=mlgreen!50, title=Success Criteria]
\normalsize
\begin{itemize}
\item Can articulate 3+ traditional design limitations
\item Can describe ML's value proposition
\item Can map ML pipeline to design pipeline
\item Understand clustering's role in innovation
\end{itemize}
\end{tcolorbox}
\end{column}
\end{columns}
\end{frame}

% PART 1: FOUNDATION & CONTEXT (8 slides)

% Section Divider: Part 1
\begin{frame}
\begin{center}
\vspace{2cm}
{\Huge\textbf{PART 1}}

\vspace{0.5cm}
{\LARGE Foundation \& Context}

\vspace{1cm}
{\large Understanding the Innovation Challenge}
\end{center}
\end{frame}

% Slide 3: Innovation Discovery - The Starting Point
\begin{frame}
\frametitle{\Large Innovation Discovery}
\framesubtitle{Finding Patterns in the Chaos}

\begin{center}
\includegraphics[width=0.85\textwidth]{charts/innovation_discovery.pdf}
\end{center}

\vspace{0.5em}
\begin{center}
\large \textcolor{mlpurple}{\textbf{Where do we even begin?}}
\end{center}
\end{frame}

% NEW: Innovation Feature Complexity
\begin{frame}
\frametitle{\Large The Hidden Complexity}
\framesubtitle{Each Innovation Depends on Hundreds of Features}

\begin{center}
\includegraphics[width=0.9\textwidth]{charts/innovation_feature_complexity.pdf}
\end{center}
\end{frame}

% Slide 4: The Innovation Challenge
\begin{frame}
\frametitle{\Large The Innovation Challenge}
\framesubtitle{Why Traditional Design Needs AI Enhancement}

\begin{columns}[T]
\begin{column}{0.48\textwidth}
\begin{tcolorbox}[colback=mlred!10, colframe=mlred!50, title=Traditional Design Limits]
\begin{itemize}
\item \textbf{Scale}: Can analyze 50 ideas, not 50,000
\item \textbf{Speed}: Months for insights
\item \textbf{Bias}: Designer's perspective dominates
\item \textbf{Patterns}: Miss hidden connections
\item \textbf{Iteration}: Slow feedback loops
\end{itemize}
\end{tcolorbox}
\end{column}

\begin{column}{0.48\textwidth}
\begin{tcolorbox}[colback=mlgreen!10, colframe=mlgreen!50, title=AI-Enhanced Innovation]
\begin{itemize}
\item \textbf{Scale}: Analyze millions of data points
\item \textbf{Speed}: Real-time insights
\item \textbf{Objectivity}: Data-driven discovery
\item \textbf{Patterns}: Find non-obvious relationships
\item \textbf{Iteration}: Continuous learning
\end{itemize}
\end{tcolorbox}
\end{column}
\end{columns}

\vspace{0.5em}
\begin{center}
\Large\textcolor{mlpurple}{\textbf{The Promise: 100x more insights, 10x faster innovation}}
\end{center}
\end{frame}

% Slide 4: The Dual Pipeline Chart
\begin{frame}
\frametitle{\Large The Dual Pipeline}
\framesubtitle{Where ML Meets Design Thinking}

\begin{center}
\includegraphics[width=0.85\textwidth]{../ML_Design_Course/course_visuals/dual_pipeline.pdf}
\end{center}

\vspace{0.5em}
\begin{center}
\Large\textcolor{mlpurple}{\textbf{Two Powerful Methodologies Converge}}
\end{center}
\end{frame}

% Slide 5: The Dual Pipeline Comparison
\begin{frame}
\frametitle{\Large The Dual Pipeline (Continued)}
\framesubtitle{Understanding Both Worlds}

\begin{columns}[T]
\begin{column}{0.48\textwidth}
\begin{tcolorbox}[colback=mlblue!10, colframe=mlblue!50, title=ML Pipeline]
\normalsize
\textbf{Data} $\rightarrow$ \textbf{Preprocess} $\rightarrow$ \textbf{Model} $\rightarrow$ \textbf{Evaluate} $\rightarrow$ \textbf{Deploy}
\vspace{0.3em}
\begin{itemize}
\item Collect innovation data
\item Clean and transform
\item Train algorithms
\item Validate accuracy
\item Scale to production
\end{itemize}
\end{tcolorbox}
\end{column}

\begin{column}{0.48\textwidth}
\begin{tcolorbox}[colback=mlorange!10, colframe=mlorange!50, title=Design Pipeline]
\normalsize
\textbf{Empathize} $\rightarrow$ \textbf{Define} $\rightarrow$ \textbf{Ideate} $\rightarrow$ \textbf{Prototype} $\rightarrow$ \textbf{Test}
\vspace{0.3em}
\begin{itemize}
\item Understand innovation needs
\item Frame problems
\item Generate solutions
\item Build concepts
\item Validate innovation impact
\end{itemize}
\end{tcolorbox}
\end{column}
\end{columns}

\vspace{0.5em}
\begin{center}
\Large\textcolor{mlpurple}{\textbf{Integration = Innovation at Scale}}
\end{center}
\end{frame}

% Slide 6: Your Innovation Journey Chart
\begin{frame}
\frametitle{\Large Your Innovation Journey}
\framesubtitle{10 Weeks to Understanding AI-Powered Design}

\begin{center}
\includegraphics[width=0.85\textwidth]{../ML_Design_Course/course_visuals/journey_roadmap.pdf}
\end{center}
\end{frame}

% Slide 7: Your Innovation Journey Details
\begin{frame}
\frametitle{\Large Your Innovation Journey (Continued)}
\framesubtitle{What You'll Learn in Each Stage}

\begin{columns}[T]
\begin{column}{0.48\textwidth}
\textbf{\large Innovation Stages}
\vspace{0.5em}

\textcolor{mlpurple}{\textbf{Discover (Weeks 1-2)}}\\
Find hidden innovation opportunities

\vspace{0.5em}
\textcolor{mlblue}{\textbf{Define (Weeks 3-4)}}\\
Identify the right problems to solve

\vspace{0.5em}
\textcolor{mlgreen}{\textbf{Ideate (Weeks 5-6)}}\\
Generate novel solutions with AI
\end{column}

\begin{column}{0.48\textwidth}
\textbf{\large Building Innovation Skills}
\vspace{0.5em}

\textcolor{mlorange}{\textbf{Prototype (Weeks 7-8)}}\\
Build smart, adaptive concepts

\vspace{0.5em}
\textcolor{mlred}{\textbf{Test (Weeks 9-10)}}\\
Evolve through continuous learning

\vspace{0.5em}
\textcolor{mlpurple}{\textbf{This Week:}}\\
Clustering for Innovation Pattern Discovery
\end{column}
\end{columns}
\end{frame}

% Slide 8: Week 1 Focus
\begin{frame}
\frametitle{\Large Week 1: Clustering for Innovation}
\framesubtitle{From Scattered Ideas to Innovation Patterns}

\begin{columns}[T]
\begin{column}{0.48\textwidth}
\Large\textbf{What We'll Learn:}
\normalsize
\begin{itemize}
\item How clustering reveals innovation categories
\item K-means algorithm fundamentals
\item Finding the optimal number of clusters
\item Quality metrics for validation
\item Advanced clustering techniques
\end{itemize}
\end{column}

\begin{column}{0.48\textwidth}
\Large\textbf{Design Applications:}
\normalsize
\begin{itemize}
\item Create innovation archetypes
\item Map innovation evolution paths
\item Identify opportunities systematically
\item Prioritize design efforts
\item Scale analysis to thousands of ideas
\end{itemize}
\end{column}
\end{columns}

\vspace{0.5em}
\begin{center}
\Large\textcolor{mlpurple}{\textbf{Goal: Transform scattered ideas into innovation patterns}}
\end{center}
\end{frame}

% The Convergence Flow Visualization
\begin{frame}[plain]
\begin{center}
\includegraphics[width=0.85\textwidth]{charts/convergence_flow.pdf}
\end{center}
\begin{center}
\Large\textcolor{mlpurple}{\textbf{The Convergence Flow: Order from Chaos}}\\
\normalsize\textit{Watch 5000 innovation ideas self-organize into meaningful patterns}
\end{center}
\end{frame}

% CHECKPOINT SLIDE 1
\begin{frame}
\frametitle{\Large Check Your Understanding - Part 1}
\framesubtitle{Quick Knowledge Check \hfill \textcolor{mlblue}{\textbf{Progress: 1/3}}}

\begin{columns}[T]
\begin{column}{0.48\textwidth}
\begin{tcolorbox}[colback=mlblue!10, colframe=mlblue!50, title=True or False?]
\normalsize
\begin{enumerate}
\item Clustering requires labeled data \textcolor{mlred}{(F)}
\item ML can process more data than humans \textcolor{mlgreen}{(T)}
\item Design thinking has 5 stages \textcolor{mlgreen}{(T)}
\item Clustering finds hidden patterns \textcolor{mlgreen}{(T)}
\end{enumerate}
\end{tcolorbox}
\end{column}

\begin{column}{0.48\textwidth}
\begin{tcolorbox}[colback=mlgreen!10, colframe=mlgreen!50, title=Can You Explain?]
\normalsize
\begin{itemize}
\item What is the dual pipeline approach?
\item Why combine ML with design thinking?
\item What problem does clustering solve?
\end{itemize}
\end{tcolorbox}
\end{column}
\end{columns}

\vspace{1em}
\begin{center}
\Large\textcolor{mlpurple}{\textbf{Ready for Part 2? Let's dive into the technical details!}}\\
\vspace{0.3em}
\normalsize\textit{Next: Clustering algorithms, evaluation metrics, and implementation}
\end{center}
\end{frame}

% TRANSITION TO PART 2
\begin{frame}
\frametitle{\Large Now Let's Get Technical}
\framesubtitle{From Understanding the Problem to Finding Solutions}

\begin{center}
\Large\textbf{We've seen the challenge:}\\
\normalsize
Thousands of innovation ideas with hidden connections\\
\vspace{1em}
\Large\textbf{Traditional approach:}\\
\normalsize
Manual segmentation based on demographics\\
\vspace{1em}
\Large\textcolor{mlgreen}{\textbf{The ML solution:}}\\
\normalsize
Let the data reveal its own natural groups\\
\vspace{2em}
\Large\textcolor{mlpurple}{\textbf{Enter: Clustering Algorithms}}
\end{center}
\end{frame}