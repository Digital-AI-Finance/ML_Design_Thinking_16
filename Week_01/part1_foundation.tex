% PART 1 SECTION DIVIDER
\begin{frame}[plain]
\begin{center}
\vspace{2em}
\Large\textcolor{mlblue}{\textbf{PART 1}}\\
\vspace{0.5em}
\Large\textbf{Foundation \& Context}\\
\vspace{2em}
\Large
What we'll explore:\\
\vspace{1em}
\normalsize
\begin{itemize}
\item Why traditional design hits limits
\item How ML amplifies human insight
\item The dual pipeline approach
\end{itemize}
\vspace{2em}
\Large\textcolor{mlpurple}{\textbf{Setting the stage for transformation}}
\end{center}
\end{frame}

% Learning Objectives for Part 1
\begin{frame}
\frametitle{\Large Part 1: Learning Objectives}
\framesubtitle{What You'll Learn in This Section}

\begin{columns}[T]
\begin{column}{0.48\textwidth}
\begin{tcolorbox}[colback=mlblue!10, colframe=mlblue!50, title={By the end of Part 1, you will be able to:}]
\normalsize
\begin{itemize}
\item \textbf{Understand} the limitations of traditional innovation approaches
\item \textbf{Recognize} how ML enhances human creativity
\item \textbf{Explain} the dual pipeline methodology
\item \textbf{Navigate} the 10-week learning journey
\item \textbf{Identify} Week 1's role in the overall course
\end{itemize}
\end{tcolorbox}
\end{column}

\begin{column}{0.48\textwidth}
\begin{tcolorbox}[colback=mlgreen!10, colframe=mlgreen!50, title=Success Criteria]
\normalsize
\begin{itemize}
\item Can articulate 3+ traditional design limitations
\item Can describe ML's value proposition
\item Can map ML pipeline to design pipeline
\item Understand clustering's role in innovation
\end{itemize}
\end{tcolorbox}
\end{column}
\end{columns}
\end{frame}

% PART 1: FOUNDATION & CONTEXT (8 slides)

% Section Divider: Part 1
\begin{frame}
\begin{center}
\vspace{2cm}
{\Huge\textbf{PART 1}}

\vspace{0.5cm}
{\LARGE Foundation \& Context}

\vspace{1cm}
{\large Understanding the Innovation Challenge}
\end{center}
\end{frame}

% Slide 3: Innovation Discovery - The Starting Point
\begin{frame}
\frametitle{\Large Innovation Discovery}
\framesubtitle{Finding Patterns in the Chaos}

\begin{center}
\includegraphics[width=0.65\textwidth]{charts/innovation_discovery.pdf}
\end{center}

\vspace{0.5em}
\begin{center}
\large \textcolor{mlpurple}{\textbf{Where do we even begin?}}
\end{center}
\end{frame}

% NEW: Innovation Feature Complexity
\begin{frame}
\frametitle{\Large The Hidden Complexity}
\framesubtitle{Each Innovation Depends on Hundreds of Features}

\begin{center}
\includegraphics[width=0.65\textwidth]{charts/innovation_feature_complexity.pdf}
\end{center}
\end{frame}

% Slide 4: The Innovation Challenge
\begin{frame}
\frametitle{\Large The Innovation Challenge}
\framesubtitle{Why Traditional Design Needs AI Enhancement}

\begin{columns}[T]
\begin{column}{0.48\textwidth}
\begin{tcolorbox}[colback=mlred!10, colframe=mlred!50, title=Traditional Design Limits]
\begin{itemize}
\item \textbf{Scale}: Can analyze 50 ideas, not 50,000
\item \textbf{Speed}: Months for insights
\item \textbf{Bias}: Designer's perspective dominates
\item \textbf{Patterns}: Miss hidden connections
\item \textbf{Iteration}: Slow feedback loops
\end{itemize}
\end{tcolorbox}
\end{column}

\begin{column}{0.48\textwidth}
\begin{tcolorbox}[colback=mlgreen!10, colframe=mlgreen!50, title=AI-Enhanced Innovation]
\begin{itemize}
\item \textbf{Scale}: Analyze millions of data points
\item \textbf{Speed}: Real-time insights
\item \textbf{Objectivity}: Data-driven discovery
\item \textbf{Patterns}: Find non-obvious relationships
\item \textbf{Iteration}: Continuous learning
\end{itemize}
\end{tcolorbox}
\end{column}
\end{columns}

\vspace{0.5em}
\begin{center}
\Large\textcolor{mlpurple}{\textbf{The Promise: 100x more insights, 10x faster innovation}}
\end{center}
\end{frame}

% Design Thinking Recap (moved here before Dual Pipeline)
\begin{frame}
\frametitle{\Large Quick Recap: The Design Thinking Process}
\framesubtitle{You've Seen This Before - Let's Connect It to ML}

\begin{center}
\begin{tikzpicture}[scale=0.9, transform shape]
    % Define styles
    \tikzstyle{stage} = [rectangle, rounded corners, minimum width=2.2cm, minimum height=1cm, text centered, draw=black, line width=1pt]
    \tikzstyle{arrow} = [thick,->,>=stealth]
    
    % Draw stages
    \node[stage, fill=mlblue!30] (empathize) at (0,0) {\shortstack{\textbf{Empathize}\\Understand users}};
    \node[stage, fill=mlorange!30] (define) at (3,0) {\shortstack{\textbf{Define}\\Frame problems}};
    \node[stage, fill=mlgreen!30] (ideate) at (6,0) {\shortstack{\textbf{Ideate}\\Generate ideas}};
    \node[stage, fill=mlred!30] (prototype) at (9,0) {\shortstack{\textbf{Prototype}\\Build solutions}};
    \node[stage, fill=mlpurple!30] (test) at (12,0) {\shortstack{\textbf{Test}\\Validate impact}};
    
    % Draw arrows
    \draw[arrow] (empathize) -- (define);
    \draw[arrow] (define) -- (ideate);
    \draw[arrow] (ideate) -- (prototype);
    \draw[arrow] (prototype) -- (test);
    
    % Feedback loop
    \draw[arrow, dashed, gray] (test.south) .. controls (6,-1.5) .. (empathize.south);
    
    % Label
    \node at (6,-2) {\large\textcolor{mlpurple}{\textbf{Iteration is key}}};
\end{tikzpicture}
\end{center}

\vspace{0.5em}
\begin{columns}[T]
\begin{column}{0.48\textwidth}
\begin{tcolorbox}[colback=mlblue!10, colframe=mlblue!50, title=Traditional Approach]
\small
\begin{itemize}
\item Manual interviews
\item Limited sample size
\item Qualitative insights
\item Slow iteration
\end{itemize}
\end{tcolorbox}
\end{column}

\begin{column}{0.48\textwidth}
\begin{tcolorbox}[colback=mlgreen!10, colframe=mlgreen!50, title=ML-Enhanced Approach]
\small
\begin{itemize}
\item Data-driven discovery
\item Massive scale analysis
\item Quantitative patterns
\item Real-time adaptation
\end{itemize}
\end{tcolorbox}
\end{column}
\end{columns}
\end{frame}

% The Dual Pipeline Chart
\begin{frame}
\frametitle{\Large The Dual Pipeline}
\framesubtitle{Where ML Meets Design Thinking}

\begin{center}
\includegraphics[width=0.85\textwidth]{../ML_Design_Course/course_visuals/dual_pipeline.pdf}
\end{center}

\vspace{0.5em}
\begin{center}
\Large\textcolor{mlpurple}{\textbf{Two Powerful Methodologies Converge}}
\end{center}
\end{frame}


% Slide 6: Your Innovation Journey Chart
\begin{frame}
\frametitle{\Large Your Innovation Journey}
\framesubtitle{10 Weeks to Understanding AI-Powered Design}

\begin{center}
\includegraphics[width=0.95\textwidth]{../ML_Design_Course/course_visuals/journey_roadmap.pdf}
\end{center}
\end{frame}

% The Convergence Flow Visualization
\begin{frame}[plain]
\begin{center}
\includegraphics[width=0.85\textwidth]{charts/convergence_flow.pdf}
\end{center}
\begin{center}
\Large\textcolor{mlpurple}{\textbf{The Convergence Flow: Order from Chaos}}\\
\normalsize\textit{Watch 5000 innovation ideas self-organize into meaningful patterns}
\end{center}
\end{frame}

