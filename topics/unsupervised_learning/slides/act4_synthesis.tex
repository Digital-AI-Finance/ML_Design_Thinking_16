% Act 4: Synthesis (4 slides)

\section{\color{synthesispurple}Act 4: Synthesis}

\begin{frame}{Slide 21: Implementation: sklearn.cluster}
\begin{columns}[T]
\column{0.48\textwidth}
\textbf{\color{synthesispurple}Scikit-learn Implementation}

\textbf{K-means Implementation:}
\begin{itemize}
\item \texttt{from sklearn.cluster import KMeans}
\item \texttt{kmeans = KMeans(n\_clusters=3)}
\item \texttt{labels = kmeans.fit\_predict(X)}
\item \texttt{centroids = kmeans.cluster\_centers\_}
\end{itemize}

\textbf{DBSCAN Implementation:}
\begin{itemize}
\item \texttt{from sklearn.cluster import DBSCAN}
\item \texttt{dbscan = DBSCAN(eps=0.5, min\_samples=5)}
\item \texttt{labels = dbscan.fit\_predict(X)}
\end{itemize}

\textbf{Hierarchical Implementation:}
\begin{itemize}
\item \texttt{from sklearn.cluster import}
\item \texttt{AgglomerativeClustering}
\item \texttt{labels = hierarchical.fit\_predict(X)}
\end{itemize}

\column{0.48\textwidth}
\textbf{Production Pipeline:}
\begin{center}
\includegraphics[width=\textwidth]{charts/sklearn_pipeline.pdf}
\end{center}

\textbf{Evaluation Tools:}
\begin{itemize}
\item \texttt{from sklearn.metrics import}
\item \texttt{silhouette\_score, adjusted\_rand\_score}
\item \texttt{sil\_score = silhouette\_score(X, labels)}
\item \texttt{ari\_score = adjusted\_rand\_score(}
\item \texttt{true\_labels, predicted\_labels)}
\end{itemize}
\end{columns}

\vspace{\fill}
{\footnotesize \color{textgray}Implementation: Production-ready clustering with scikit-learn}
\end{frame}

\begin{frame}{Slide 22: Clustering Method Taxonomy}
\begin{columns}[T]
\column{0.48\textwidth}
\textbf{\color{synthesispurple}Clustering Algorithm Families}

\textbf{Centroid-Based:}
\begin{itemize}
\item K-means, K-medoids
\item Assumes spherical clusters
\item Fast, scalable
\item Requires k specification
\end{itemize}

\textbf{Density-Based:}
\begin{itemize}
\item DBSCAN, OPTICS, Mean-shift
\item Handles arbitrary shapes
\item Automatic noise detection
\item Parameter sensitive
\end{itemize}

\textbf{Hierarchical:}
\begin{itemize}
\item Agglomerative, Divisive
\item Creates cluster tree
\item No k pre-specification
\item Computationally expensive
\end{itemize}

\column{0.48\textwidth}
\textbf{Algorithm Taxonomy Tree:}
\begin{center}
\includegraphics[width=\textwidth]{charts/clustering_taxonomy.pdf}
\end{center}

\textbf{Modern Extensions:}
\begin{itemize}
\item Spectral clustering (graph-based)
\item Gaussian mixture models (probabilistic)
\item Deep clustering (neural networks)
\item Fuzzy clustering (soft assignments)
\end{itemize}

\textbf{Selection Criteria:} Data shape, size, interpretability needs.
\end{columns}

\vspace{\fill}
{\footnotesize \color{textgray}Taxonomy: Understanding the clustering algorithm landscape}
\end{frame}

\begin{frame}{Slide 23: Algorithm Selection Guide}
\begin{columns}[T]
\column{0.48\textwidth}
\textbf{\color{synthesispurple}Decision Framework}

\textbf{Ask These Questions:}

\textbf{1. What shapes do you expect?}
\begin{itemize}
\item Spherical \textrightarrow K-means
\item Arbitrary \textrightarrow DBSCAN
\item Unknown \textrightarrow Hierarchical
\end{itemize}

\textbf{2. Do you know k?}
\begin{itemize}
\item Yes \textrightarrow K-means/Hierarchical
\item No \textrightarrow DBSCAN
\end{itemize}

\textbf{3. How much noise?}
\begin{itemize}
\item Clean data \textrightarrow K-means
\item Noisy data \textrightarrow DBSCAN
\end{itemize}

\textbf{4. What's your dataset size?}
\begin{itemize}
\item Large (>10K) \textrightarrow K-means
\item Medium \textrightarrow Any method
\item Small (<1K) \textrightarrow Hierarchical
\end{itemize}

\column{0.48\textwidth}
\textbf{Selection Decision Tree:}
\begin{center}
\includegraphics[width=\textwidth]{charts/algorithm_selection.pdf}
\end{center}

\textbf{Business Considerations:}
\begin{itemize}
\item Interpretability: Hierarchical wins
\item Speed: K-means fastest
\item Robustness: DBSCAN handles outliers
\item Exploration: Try multiple methods
\end{itemize}

\textbf{Hybrid Approach:} Start with K-means, validate with DBSCAN.
\end{columns}

\vspace{\fill}
{\footnotesize \color{textgray}Practical guide: Choosing the right clustering algorithm for your problem}
\end{frame}

\begin{frame}{Slide 24: Modern Applications \& Neural Network Preview}
\begin{columns}[T]
\column{0.48\textwidth}
\textbf{\color{synthesispurple}Real-World Applications}

\textbf{Anomaly Detection:}
\begin{itemize}
\item Fraud detection in banking
\item Network intrusion detection
\item Quality control in manufacturing
\item Medical diagnosis outliers
\end{itemize}

\textbf{Recommendation Systems:}
\begin{itemize}
\item User behavior clustering
\item Product category discovery
\item Content similarity grouping
\item Market basket analysis
\end{itemize}

\textbf{Business Intelligence:}
\begin{itemize}
\item Customer segmentation
\item Market research
\item Operational optimization
\item Risk assessment
\end{itemize}

\column{0.48\textwidth}
\textbf{Modern Clustering Evolution:}
\begin{center}
\includegraphics[width=\textwidth]{charts/modern_applications.pdf}
\end{center}

\textbf{Neural Network Preview:}
\begin{itemize}
\item Autoencoders for dimensionality reduction
\item Self-organizing maps (SOMs)
\item Deep embedded clustering
\item Variational autoencoders
\end{itemize}

\textbf{Next Steps:} Deep learning approaches that learn representations and cluster simultaneously.
\end{columns}

\vspace{\fill}
{\footnotesize \color{textgray}Future directions: From classical clustering to neural network approaches}
\end{frame}