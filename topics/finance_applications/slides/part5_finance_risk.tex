% Part 5: Risk & Portfolio Management
\section{Risk \& Portfolio Management}

\begin{frame}[plain]
\vfill
\centering
\begin{beamercolorbox}[sep=16pt,center]{title}
\usebeamerfont{title}\Large Part 5: Risk \& Portfolio Management\par
\vspace{0.5em}
\large ML-Enhanced Risk Analytics\par
\end{beamercolorbox}
\vfill
\end{frame}

\begin{frame}{Value at Risk (VaR): Three Approaches}
\begin{columns}[T]
\begin{column}{0.55\textwidth}
\Large\textbf{Risk Measurement Fundamentals}
\normalsize
\vspace{0.5em}

\textbf{Definition:}
$$\Prob(L > \text{VaR}_\alpha) = \alpha$$

$\alpha$-quantile of loss distribution (typically $\alpha = 0.05$ or 0.01)

\vspace{0.3em}
\textbf{1. Historical VaR:}
\begin{itemize}
\item Use empirical quantile from historical data
\item $\text{VaR}_{0.05}$ = 5th percentile of returns
\item Non-parametric, distribution-free
\item Assumes past predicts future
\end{itemize}

\textbf{2. Parametric VaR (Gaussian):}
$$\text{VaR}_\alpha = \mu + \sigma \Phi^{-1}(\alpha)$$
where $\Phi^{-1}$ is inverse normal CDF

\textbf{3. Monte Carlo VaR:}
\begin{itemize}
\item Simulate 10,000+ scenarios
\item Estimate empirical quantile
\item Flexible for complex portfolios
\end{itemize}
\end{column}

\begin{column}{0.43\textwidth}
\textbf{ML-Enhanced VaR:}

\textcolor{finblue}{\textbf{GARCH-based VaR:}}
$$\sigma_t^2 = \omega + \alpha \epsilon_{t-1}^2 + \beta \sigma_{t-1}^2$$
Time-varying volatility improves accuracy

\textcolor{fingreen}{\textbf{Quantile Regression:}}
$$\min_\beta \sum_i \rho_\alpha(r_i - x_i^T\beta)$$
where $\rho_\alpha(u) = u(\alpha - \mathbb{1}_{u<0})$

Directly estimates quantiles without distributional assumptions

\textbf{Comparative Performance:}
\begin{center}
\footnotesize
\begin{tabular}{lcc}
\toprule
\textbf{Method} & \textbf{Coverage} & \textbf{Violations} \\
\midrule
Historical & 95.2\% & 48/1000 \\
Gaussian & 92.8\% & 72/1000 \\
GARCH & 94.8\% & 52/1000 \\
Quantile RF & 95.1\% & 49/1000 \\
\bottomrule
\end{tabular}
\end{center}
\end{column}
\end{columns}
\end{frame}

\begin{frame}{Conditional VaR (Expected Shortfall)}
\begin{columns}[T]
\begin{column}{0.55\textwidth}
\Large\textbf{CVaR: Coherent Risk Measure}
\normalsize
\vspace{0.5em}

\textbf{Definition:}
$$\text{CVaR}_\alpha = \E[L | L > \text{VaR}_\alpha]$$

Average loss beyond VaR threshold

\vspace{0.3em}
\textbf{Mathematical Properties:}
\begin{itemize}
\item \textbf{Sub-additivity:} $\text{CVaR}(X+Y) \leq \text{CVaR}(X) + \text{CVaR}(Y)$
\item \textbf{Monotonicity:} $X \leq Y \Rightarrow \text{CVaR}(X) \leq \text{CVaR}(Y)$
\item \textbf{Positive homogeneity:} $\text{CVaR}(\lambda X) = \lambda \text{CVaR}(X)$
\item \textbf{Translation invariance:} $\text{CVaR}(X + c) = \text{CVaR}(X) + c$
\end{itemize}

\textbf{Optimization Form:}
$$\text{CVaR}_\alpha = \min_t \left\{t + \frac{1}{\alpha}\E[\max(L-t, 0)]\right\}$$

Enables portfolio optimization with CVaR constraints
\end{column}

\begin{column}{0.43\textwidth}
\textbf{Why CVaR > VaR:}

\textcolor{fingreen}{\textbf{Advantages:}}
\begin{itemize}
\item Captures tail risk severity
\item Coherent risk measure
\item Sub-additive (diversification)
\item Optimization-friendly
\end{itemize}

\textcolor{finred}{\textbf{VaR Limitations:}}
\begin{itemize}
\item Not sub-additive
\item Ignores tail losses
\item Multiple local minima
\end{itemize}

\vspace{0.3em}
\textbf{Example: Portfolio Loss}
\begin{itemize}
\item VaR$_{0.05}$: -\$1M (95th percentile)
\item CVaR$_{0.05}$: -\$2.5M (avg beyond VaR)
\item Tail risk: CVaR captures 2.5x worse scenarios
\end{itemize}

\textbf{Regulatory Adoption:}
\begin{itemize}
\item Basel III prefers ES over VaR
\item Better capital adequacy
\item Captures tail dependencies
\end{itemize}
\end{column}
\end{columns}
\end{frame}

\begin{frame}{ML-Based Stress Testing}
\begin{columns}[T]
\begin{column}{0.55\textwidth}
\Large\textbf{Scenario Generation and Analysis}
\normalsize
\vspace{0.5em}

\textbf{Traditional Stress Testing:}
\begin{itemize}
\item Historical scenarios (2008 crisis, COVID-19)
\item Hypothetical shocks (+3 std vol, -20\% market)
\item Reverse stress testing (find breaking point)
\end{itemize}

\textbf{ML-Enhanced Approaches:}

\textcolor{finblue}{\textbf{1. VAE Scenario Generation:}}
$$z \sim \mathcal{N}(0, I), \quad x = \text{Decoder}(z)$$
Generate plausible but unseen market scenarios

\textcolor{fingreen}{\textbf{2. GAN Stress Scenarios:}}
\begin{itemize}
\item Train on crisis periods
\item Generate extreme but realistic scenarios
\item Capture tail dependencies
\end{itemize}

\textbf{3. Random Forest Sensitivity:}
\begin{itemize}
\item Feature importance for risk drivers
\item Partial dependence plots
\item Interaction effects
\end{itemize}
\end{column}

\begin{column}{0.43\textwidth}
\textbf{Stress Testing Framework:}

\textbf{Steps:}
\begin{enumerate}
\item Define risk factors (rates, spreads, FX)
\item Generate stressed scenarios
\item Revalue portfolio under stress
\item Measure impact on P\&L, VaR, capital
\end{enumerate}

\textbf{Example Scenarios:}
\begin{center}
\footnotesize
\begin{tabular}{lc}
\toprule
\textbf{Scenario} & \textbf{Portfolio Loss} \\
\midrule
Rate +200bp & -\$15M \\
Equity -30\% & -\$45M \\
Credit spread +150bp & -\$22M \\
Combined (ML) & -\$68M \\
\bottomrule
\end{tabular}
\end{center}

\textbf{ML Advantages:}
\begin{itemize}
\item Discovers non-obvious correlations
\item Generates tail scenarios
\item Faster than Monte Carlo
\item Learns from recent crises
\end{itemize}
\end{column}
\end{columns}
\end{frame}

\begin{frame}{GARCH Models for Volatility Forecasting}
\begin{columns}[T]
\begin{column}{0.55\textwidth}
\Large\textbf{Time-Varying Volatility Models}
\normalsize
\vspace{0.5em}

\textbf{GARCH(1,1) Specification:}
$$r_t = \mu + \epsilon_t, \quad \epsilon_t = \sigma_t z_t, \quad z_t \sim \mathcal{N}(0,1)$$
$$\sigma_t^2 = \omega + \alpha \epsilon_{t-1}^2 + \beta \sigma_{t-1}^2$$

\textbf{Parameters:}
\begin{itemize}
\item $\omega$: Baseline variance (long-run vol)
\item $\alpha$: ARCH effect (shock impact)
\item $\beta$: GARCH effect (persistence)
\item Stationarity: $\alpha + \beta < 1$
\end{itemize}

\textbf{Extensions:}
\begin{itemize}
\item \textbf{EGARCH:} Exponential (leverage effects)
\item \textbf{GJR-GARCH:} Asymmetric (bad news > good news)
\item \textbf{DCC-GARCH:} Dynamic conditional correlation
\end{itemize}

\textbf{Maximum Likelihood Estimation:}
$$\ell(\theta) = \sum_{t=1}^T \left[-\frac{1}{2}\log(2\pi) - \frac{1}{2}\log(\sigma_t^2) - \frac{\epsilon_t^2}{2\sigma_t^2}\right]$$
\end{column}

\begin{column}{0.43\textwidth}
\textbf{ML Hybrid Approaches:}

\textcolor{finblue}{\textbf{Neural Network GARCH:}}
$$\sigma_t^2 = NN(\epsilon_{t-1}, \ldots, \epsilon_{t-p}, \sigma_{t-1}, \ldots, \sigma_{t-q})$$

Learns nonlinear volatility dynamics

\textbf{LSTM-GARCH:}
\begin{itemize}
\item Captures long-memory effects
\item Handles regime switches
\item Outperforms standard GARCH
\end{itemize}

\textbf{Forecast Performance (S\&P 500):}
\begin{center}
\footnotesize
\begin{tabular}{lcc}
\toprule
\textbf{Model} & \textbf{RMSE} & \textbf{MAE} \\
\midrule
GARCH(1,1) & 2.8\% & 1.9\% \\
EGARCH & 2.6\% & 1.8\% \\
LSTM-GARCH & 2.2\% & 1.5\% \\
\bottomrule
\end{tabular}
\end{center}

\textbf{Applications:}
\begin{itemize}
\item Option pricing (implied vol)
\item Risk management (VaR forecasting)
\item Portfolio allocation (volatility timing)
\end{itemize}
\end{column}
\end{columns}
\end{frame}