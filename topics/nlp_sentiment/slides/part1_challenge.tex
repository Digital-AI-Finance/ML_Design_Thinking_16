% ==================== PART 1: THE CHALLENGE ====================
\section{Part 1: The Emotional Understanding Challenge}

% Slide 1: Human Hook - Motivate from Experience
\begin{frame}[t]{Your Challenge: 50,000 Reviews, 10 Critical Insights}
\Large\textbf{Which Reviews Reveal Why Users Really Quit?}
\normalsize

\vspace{0.5em}

\begin{columns}[T]
\column{0.55\textwidth}
\textbf{The Scenario You Face:}
\begin{itemize}
\item Your product has 50,000 user reviews
\item Average review: 100 words
\item Total: 5 million words to process
\item Hidden inside: The 10 reviews that explain 80\% of churn
\end{itemize}

\vspace{0.5em}
\textbf{The Human Limit:}
\begin{itemize}
\item Reading speed: 200 words/minute
\item Time to read all: 417 hours (10 weeks!)
\item Reviews arrive: 500 new ones daily
\item \textcolor{mlred}{You're already behind before you start}
\end{itemize}

\column{0.43\textwidth}
\begin{tcolorbox}[colback=mllavender4, colframe=mlred]
\textbf{Real Review Examples:}\\[0.3em]
\footnotesize
``Love it but...'' (quit in 2 weeks)\\[0.2em]
``Not bad for the price'' (renewed 3 years)\\[0.2em]
``Just what I expected!'' (1-star rating)\\[0.2em]
``Finally someone gets it'' (5-star champion)
\end{tcolorbox}

\vspace{0.5em}
\begin{center}
\includegraphics[width=0.9\textwidth]{charts/review_volume_growth.pdf}
\end{center}
\end{columns}

\bottomnote{Data velocity exceeds human processing capacity - automated analysis becomes necessity rather than optimization}
\end{frame}

% Slide 2: First Foundational Concept - Words to Meaning
\begin{frame}[t]{Building Block 1: Words Don't Have Fixed Meanings}
\Large\textbf{Context Changes Everything}
\normalsize

\begin{columns}[T]
\column{0.48\textwidth}
\textbf{Same Word, Different Meanings:}

\vspace{0.5em}
\textbf{``Cold''}
\begin{itemize}
\item Restaurant: ``The soup was cold'' \textcolor{mlred}{→ Negative}
\item Service: ``Support was cold'' \textcolor{mlred}{→ Negative}
\item Beer: ``Nice and cold'' \textcolor{mlgreen}{→ Positive}
\item Logic: ``Cold hard facts'' \textcolor{mlblue}{→ Neutral}
\end{itemize}

\vspace{0.5em}
\textbf{``Fast''}
\begin{itemize}
\item Delivery: ``Super fast!'' \textcolor{mlgreen}{→ Positive}
\item Battery: ``Dies too fast'' \textcolor{mlred}{→ Negative}
\item Customer service: ``Too fast, felt rushed'' \textcolor{mlred}{→ Negative}
\end{itemize}

\column{0.48\textwidth}
\textbf{The Computer's Problem:}

\begin{center}
\includegraphics[width=0.95\textwidth]{charts/context_dependency.pdf}
\end{center}

\begin{tcolorbox}[colback=mlyellow!20, colframe=mlorange]
\footnotesize
\textbf{Key Insight:} A word's emotional meaning depends on ALL surrounding words, not just the word itself.\\[0.3em]
This is why simple keyword counting fails spectacularly.
\end{tcolorbox}
\end{columns}

\bottomnote{Context determines meaning - word-level analysis fails when semantic interpretation requires surrounding information}
\end{frame}

% Slide 3: Second Foundational Concept - Emotion Layers
\begin{frame}[t]{Building Block 2: Emotion Has Multiple Layers}
\Large\textbf{What People Say ≠ What They Feel}
\normalsize

\begin{columns}[T]
\column{0.31\textwidth}
\textbf{Layer 1: Literal}
\begin{itemize}
\item ``Great product''
\item Clear positive
\item Easy to detect
\item 15\% of reviews
\end{itemize}

\vspace{0.5em}
\textbf{Example:}\\
\footnotesize
``This is excellent. I love every feature. Will buy again.''\\
\textcolor{mlgreen}{→ Genuinely positive}

\column{0.31\textwidth}
\textbf{Layer 2: Sarcastic}
\begin{itemize}
\item ``Just wonderful''
\item Opposite meaning
\item Context reveals truth
\item 25\% of reviews
\end{itemize}

\vspace{0.5em}
\textbf{Example:}\\
\footnotesize
``Oh great, another update that breaks everything''\\
\textcolor{mlred}{→ Actually negative}

\column{0.31\textwidth}
\textbf{Layer 3: Cultural}
\begin{itemize}
\item ``It's okay''
\item Varies by culture
\item UK: Negative
\item US: Neutral
\end{itemize}

\vspace{0.5em}
\textbf{Example:}\\
\footnotesize
``It does what it says''\\
UK: \textcolor{mlorange}{Disappointed}\\
US: \textcolor{mlgreen}{Satisfied}
\end{columns}

\vspace{0.5em}

\begin{center}
\begin{tcolorbox}[colback=mllavender4, colframe=mlpurple, width=0.9\textwidth]
\centering
\textbf{The Complexity:} 60\% of emotional meaning is NOT in the words themselves but in how they're used together
\end{tcolorbox}
\end{center}

\bottomnote{Interpretive complexity limits consensus - emotional content carries inherent ambiguity even among expert human evaluators}
\end{frame}

% Slide 4: Combine Concepts to Show Challenge
\begin{frame}[t]{The Impossible Task: Context × Emotion × Scale}
\Large\textbf{Why This Problem Explodes in Complexity}
\normalsize

\begin{center}
\includegraphics[width=0.75\textwidth]{charts/complexity_explosion.pdf}
\end{center}

\begin{columns}[T]
\column{0.48\textwidth}
\textbf{The Multiplication Effect:}
\begin{itemize}
\item 10,000 unique words in reviews
\item × 5 possible emotional states
\item × 3 context types (product/service/price)
\item × 4 cultural backgrounds
\item = \textcolor{mlred}{600,000 combinations}
\end{itemize}

\column{0.48\textwidth}
\textbf{Information Overload:}
\begin{itemize}
\item Each review: 100 words
\item Each word: Needs context of ±5 words
\item Actual analysis: 100 × 11 = 1,100 decisions
\item For 50,000 reviews: \textcolor{mlred}{55 million decisions}
\end{itemize}
\end{columns}

\vspace{0.5em}
\begin{tcolorbox}[colback=mlred!10, colframe=mlred]
\centering
\textbf{The Core Challenge:} How do we compress 55 million contextual decisions into actionable insights?
\end{tcolorbox}

\bottomnote{Scale forces automation - volume growth outpaces linear increases in human analytical capacity regardless of investment}
\end{frame}

% Slide 5: Quantify the Challenge with Information Theory
\begin{frame}[t]{Information Theory: Why Compression Loses Emotion}
\Large\textbf{The Mathematical Impossibility}
\normalsize

\begin{columns}[T]
\column{0.55\textwidth}
\textbf{Information Content Analysis:}

\begin{tabular}{lrr}
\toprule
\textbf{Component} & \textbf{Bits} & \textbf{Information} \\
\midrule
One word & 14 bits & Which of 10,000 words \\
+ Position & 7 bits & Where in sentence \\
+ Emotion & 3 bits & 8 emotion types \\
+ Context & 10 bits & Surrounding words \\
+ Sarcasm & 1 bit & Yes/No \\
\midrule
\textbf{Total per word} & 35 bits & \\
\textbf{Per review (100 words)} & 3,500 bits & \\
\textbf{50,000 reviews} & \textcolor{mlred}{175M bits} & \textcolor{mlred}{22 MB} \\
\bottomrule
\end{tabular}

\vspace{0.5em}
\textbf{Traditional Compression:}
\begin{itemize}
\item Keyword counts: 10,000 → 100 words
\item Compression ratio: 100:1
\item Information lost: \textcolor{mlred}{99\%}
\end{itemize}

\column{0.43\textwidth}
\begin{center}
\includegraphics[width=0.95\textwidth]{charts/information_loss.pdf}
\end{center}

\begin{tcolorbox}[colback=mlyellow!20, colframe=mlorange]
\textbf{The Dilemma:}\\[0.3em]
Compress too much → Lose emotional nuance\\
Keep everything → Impossible to process\\[0.3em]
\textbf{Need:} Selective preservation of emotional signal
\end{tcolorbox}
\end{columns}

\vspace{0.5em}
\begin{center}
\Large\textcolor{mlpurple}{\textbf{Can we preserve emotion while compressing 100:1?}}
\end{center}

\bottomnote{Information-theoretic limits constrain compression - high-entropy signals require proportionally complex representations to preserve fidelity}
\end{frame}