% Part 2: Algorithms - The Mathematics of Discovery
\section{Algorithms: Topic Modeling Techniques}

% LDA Overview
\begin{frame}{Latent Dirichlet Allocation (LDA)}
\begin{columns}[T]
\column{0.55\textwidth}
\includegraphics[width=0.85\textwidth]{lda_plate_notation.pdf}

\vspace{0.3cm}
\small
\textcolor{mlblue}{\textbf{Generative Process:}}
\begin{enumerate}
\item Choose topic distribution for document
\item For each word position:
   \begin{itemize}
   \item Choose a topic
   \item Choose a word from that topic
   \end{itemize}
\end{enumerate}

\column{0.43\textwidth}
\textcolor{mlorange}{\textbf{Key Parameters}}

\normalsize
\begin{itemize}
\item \textbf{K}: Number of topics
\item \textbf{$\alpha$}: Document-topic density
\item \textbf{$\beta$}: Topic-word density
\end{itemize}

\vspace{0.5cm}
\textcolor{mlgreen}{\textbf{Strengths:}}
\small
\begin{itemize}
\item Probabilistic interpretation
\item Handles uncertainty
\item Industry standard
\end{itemize}

\textcolor{mlred}{\textbf{Limitations:}}
\small
\begin{itemize}
\item Fixed number of topics
\item Assumes bag-of-words
\end{itemize}
\end{columns}
\end{frame}

% LDA Intuition
\begin{frame}{LDA Intuition: Restaurant Reviews Example}
\begin{columns}[T]
\column{0.48\textwidth}
\textcolor{mlblue}{\textbf{Input Reviews:}}

\footnotesize
``The pasta was delicious and service excellent''\\
``Great atmosphere, loved the wine selection''\\
``Fast delivery, food arrived hot''\\
``Cozy ambiance, romantic lighting''

\vspace{0.3cm}
\textcolor{mlorange}{\textbf{LDA Discovers:}}

\small
\textbf{Topic 1: Food Quality}\\
\footnotesize pasta, delicious, hot, fresh, taste

\small
\textbf{Topic 2: Service}\\
\footnotesize service, delivery, fast, staff, friendly

\small
\textbf{Topic 3: Atmosphere}\\
\footnotesize atmosphere, ambiance, cozy, romantic

\column{0.48\textwidth}
\includegraphics[width=0.85\textwidth]{lda_document_topics.pdf}

\vspace{0.3cm}
\small\textcolor{mlgray}{Each review contains multiple topics in different proportions}
\end{columns}
\end{frame}

% NMF Overview
\begin{frame}{Non-negative Matrix Factorization (NMF)}
\begin{columns}[T]
\column{0.48\textwidth}
\textcolor{mlblue}{\textbf{Core Concept}}

\normalsize
Decompose document-term matrix:\\
\textbf{V = W × H}

\small
\begin{itemize}
\item \textbf{V}: Original documents
\item \textbf{W}: Document-topic weights
\item \textbf{H}: Topic-term weights
\end{itemize}

\vspace{0.3cm}
\textcolor{mlorange}{\textbf{Key Difference from LDA:}}
\begin{itemize}
\item Deterministic
\item Parts-based representation
\item No probabilistic assumptions
\end{itemize}

\column{0.48\textwidth}
\includegraphics[width=0.85\textwidth]{nmf_decomposition.pdf}

\vspace{0.3cm}
\textcolor{mlgreen}{\textbf{When to Use:}}
\small
\begin{itemize}
\item Short texts (tweets, titles)
\item Need interpretable parts
\item Speed is critical
\item Sparse data
\end{itemize}
\end{columns}
\end{frame}

% LSA Overview
\begin{frame}{Latent Semantic Analysis (LSA)}
\begin{columns}[T]
\column{0.48\textwidth}
\textcolor{mlblue}{\textbf{Singular Value Decomposition}}

\normalsize
\textbf{A = U$\Sigma$V$^T$}

\small
\begin{itemize}
\item Reduces dimensionality
\item Captures semantic relationships
\item Handles synonyms naturally
\end{itemize}

\vspace{0.3cm}
\textcolor{mlorange}{\textbf{Process:}}
\begin{enumerate}
\item Build term-document matrix
\item Apply TF-IDF weighting
\item Perform SVD
\item Keep top k dimensions
\end{enumerate}

\column{0.48\textwidth}
\includegraphics[width=0.85\textwidth]{lsa_semantic_space.pdf}

\vspace{0.3cm}
\textcolor{mlgreen}{\textbf{Applications:}}
\small
\begin{itemize}
\item Document similarity
\item Information retrieval
\item Concept extraction
\item Dimensionality reduction
\end{itemize}
\end{columns}
\end{frame}

% Algorithm Comparison
\begin{frame}{Algorithm Comparison}
\begin{center}
\includegraphics[width=0.85\textwidth]{algorithm_comparison_matrix.pdf}
\end{center}

\vspace{-0.2cm}
\begin{columns}[T]
\column{0.32\textwidth}
\centering\small
\textcolor{mlblue}{\textbf{LDA}}\\
\footnotesize Best for interpretability

\column{0.32\textwidth}
\centering\small
\textcolor{mlorange}{\textbf{NMF}}\\
\footnotesize Best for speed

\column{0.32\textwidth}
\centering\small
\textcolor{mlgreen}{\textbf{BERTopic}}\\
\footnotesize Best for accuracy
\end{columns}
\end{frame}

% Choosing Number of Topics
\begin{frame}{Choosing the Number of Topics}
\begin{columns}[T]
\column{0.55\textwidth}
\includegraphics[width=0.85\textwidth]{topic_coherence_plot.pdf}

\vspace{0.3cm}
\small
\textcolor{mlblue}{\textbf{Metrics to Consider:}}
\begin{itemize}
\item \textbf{Coherence}: Semantic similarity
\item \textbf{Perplexity}: Model fit
\item \textbf{Distinctiveness}: Topic separation
\end{itemize}

\column{0.43\textwidth}
\textcolor{mlorange}{\textbf{Practical Guidelines}}

\normalsize
\textbf{Rule of Thumb:}\\
\small $K = \sqrt{N/2}$ for N documents

\vspace{0.3cm}
\textbf{Domain-Specific:}
\begin{itemize}
\item News: 20-50 topics
\item Reviews: 5-15 topics
\item Research: 30-100 topics
\item Social media: 10-30 topics
\end{itemize}

\vspace{0.3cm}
\textcolor{mlgreen}{\textbf{Best Practice:}}\\
\small Test multiple values, validate with domain experts
\end{columns}
\end{frame}

% Topic Quality Metrics
\begin{frame}{Evaluating Topic Quality}
\begin{columns}[T]
\column{0.48\textwidth}
\textcolor{mlblue}{\textbf{Coherence Measures}}

\small
\textbf{UMass Coherence:}\\
\footnotesize Based on document co-occurrence

\small
\textbf{C\_V Coherence:}\\
\footnotesize Sliding window + word embeddings

\small
\textbf{C\_NPMI:}\\
\footnotesize Normalized pointwise mutual information

\vspace{0.5cm}
\textcolor{mlorange}{\textbf{Human Evaluation}}
\begin{itemize}
\item Topic interpretability
\item Word intrusion test
\item Topic intrusion test
\end{itemize}

\column{0.48\textwidth}
\includegraphics[width=0.85\textwidth]{topic_quality_dashboard.pdf}

\vspace{0.3cm}
\small\textcolor{mlgreen}{\textbf{Quality Thresholds:}}
\begin{itemize}
\item Coherence $>$ 0.5: Good
\item Distinctiveness $>$ 0.7: Good
\item Coverage $>$ 80\%: Good
\end{itemize}
\end{columns}
\end{frame}

% Modern Approaches - BERTopic
\begin{frame}{Modern Approach: BERTopic}
\begin{columns}[T]
\column{0.48\textwidth}
\textcolor{mlblue}{\textbf{Pipeline:}}
\begin{enumerate}
\item Embed documents (BERT)
\item Cluster embeddings (HDBSCAN)
\item Create topics (c-TF-IDF)
\item Fine-tune representations
\end{enumerate}

\vspace{0.3cm}
\textcolor{mlorange}{\textbf{Advantages:}}
\begin{itemize}
\item Contextual understanding
\item Dynamic number of topics
\item Outlier detection
\item Hierarchical topics
\end{itemize}

\column{0.48\textwidth}
\includegraphics[width=0.85\textwidth]{bertopic_pipeline.pdf}

\vspace{0.3cm}
\textcolor{mlgreen}{\textbf{Use Cases:}}
\small
\begin{itemize}
\item Dynamic content streams
\item Multi-lingual datasets
\item Short text (tweets)
\item Evolving topics over time
\end{itemize}
\end{columns}
\end{frame}

% Preprocessing for Topics
\begin{frame}{Preprocessing for Topic Models}
\begin{columns}[T]
\column{0.48\textwidth}
\textcolor{mlblue}{\textbf{Essential Steps:}}
\begin{enumerate}
\item \textbf{Tokenization}\\
   \footnotesize Split into meaningful units
\item \textbf{Lowercasing}\\
   \footnotesize Normalize text
\item \textbf{Remove stopwords}\\
   \footnotesize Filter common words
\item \textbf{Lemmatization}\\
   \footnotesize Reduce to base forms
\item \textbf{Filter extremes}\\
   \footnotesize Remove rare/common terms
\end{enumerate}

\column{0.48\textwidth}
\textcolor{mlorange}{\textbf{Advanced Techniques:}}
\begin{itemize}
\item \textbf{Bigrams/Trigrams}\\
   \footnotesize ``machine learning'' as one token
\item \textbf{Named Entity Recognition}\\
   \footnotesize Preserve ``New York''
\item \textbf{Part-of-speech filtering}\\
   \footnotesize Keep nouns and verbs
\item \textbf{Domain stopwords}\\
   \footnotesize Remove domain-specific noise
\end{itemize}

\vspace{0.3cm}
\small\textcolor{mlgreen}{Quality preprocessing = Better topics}
\end{columns}
\end{frame}

% Algorithm Summary
\begin{frame}{Algorithm Summary}
\begin{columns}[T]
\column{0.48\textwidth}
\textcolor{mlblue}{\textbf{Classical Methods}}
\begin{itemize}
\item \textbf{LDA}: Probabilistic gold standard
\item \textbf{NMF}: Fast and interpretable
\item \textbf{LSA}: Semantic relationships
\end{itemize}

\vspace{0.3cm}
\textcolor{mlorange}{\textbf{Modern Methods}}
\begin{itemize}
\item \textbf{BERTopic}: Contextual understanding
\item \textbf{Top2Vec}: Document embeddings
\item \textbf{CTM}: Correlated topics
\end{itemize}

\column{0.48\textwidth}
\textcolor{mlgreen}{\textbf{Selection Criteria}}
\begin{itemize}
\item Data size and type
\item Interpretability needs
\item Computational resources
\item Real-time requirements
\item Language complexity
\end{itemize}

\vspace{0.3cm}
\textcolor{mlred}{\textbf{Remember:}}\\
\small No single best algorithm - choose based on your specific use case
\end{columns}

\vspace{\fill}
\begin{center}
\normalsize\textcolor{mlpurple}{\textbf{Next:} Implementation in practice}
\end{center}
\end{frame}