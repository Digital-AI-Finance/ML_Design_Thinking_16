% ==================== PART 1: THE HIDDEN PATTERN PROBLEM ====================
\section{Part 1: The Hidden Pattern Problem}

% Slide 1: The Opening Hook
\begin{frame}[t]{Your Challenge: 1 Million Feedback Comments}
\Large\textbf{What Are People Really Saying?}
\normalsize

\vspace{0.5em}

\begin{columns}[T]
\column{0.48\textwidth}
\textbf{The Scenario:}
\begin{itemize}
\item Online retailer with 1M reviews
\item Need insights by tomorrow
\item Competitors analyzing manually
\item Missing patterns = lost opportunities
\end{itemize}

\vspace{0.5em}
\textbf{Manual Approach:}
\begin{itemize}
\item Read 100 reviews/day
\item 10,000 days to finish (27 years!)
\item Cost: 50 analysts × \$50K = \$2.5M/year
\item Still miss cross-cutting themes
\end{itemize}

\column{0.48\textwidth}
\textbf{What You're Missing:}
\begin{center}
\begin{tikzpicture}[scale=0.7]
% Iceberg
\draw[fill=mllavender2] (0,0) -- (-1.5,-0.5) -- (-1.2,-3) -- (1.2,-3) -- (1.5,-0.5) -- cycle;
\draw[fill=mllavender3] (0,0) -- (-1.5,-0.5) -- (1.5,-0.5) -- (0.8,0.3) -- (-0.8,0.3) -- cycle;
\draw[thick] (-2,-0.5) -- (2,-0.5);
\node at (0,0.8) {\small \textbf{Visible: 5\%}};
\node at (0,-1.5) {\small \textbf{Hidden: 95\%}};
% Labels
\node[align=left] at (2.5,0.3) {\footnotesize Obvious\\complaints};
\node[align=left] at (2.5,-1) {\footnotesize Emerging\\issues};
\node[align=left] at (2.5,-2) {\footnotesize Latent\\needs};
\end{tikzpicture}
\end{center}

\begin{tcolorbox}[colback=mlred!10, colframe=mlred]
\centering
\textbf{Result:} You see complaints,\\
miss opportunities
\end{tcolorbox}
\end{columns}

\bottomnote{Volume necessitates automation - pattern discovery scales beyond manual capacity when data growth exceeds analyst availability}
\end{frame}

% Slide 2: The Cost of Missing Patterns
\begin{frame}[t]{The Real Cost of Missing Hidden Patterns}
\Large\textbf{When You Can't See the Forest for the Trees}
\normalsize

\vspace{0.5em}

\begin{columns}[T]
\column{0.48\textwidth}
\textbf{Blockbuster (2000-2010):}
\begin{itemize}
\item Had millions of rental records
\item Categorized by genre (Action, Drama)
\item Missed micro-preferences
\item Couldn't see "Films with strong female leads from the 80s"
\item Result: Bankruptcy in 2010
\end{itemize}

\vspace{0.5em}
\textbf{Netflix (Same Period):}
\begin{itemize}
\item Applied topic modeling to viewing data
\item Discovered 76,897 micro-genres
\item "Critically-acclaimed emotional dramas"
\item "Witty foreign thrillers"
\item Result: \$240B market cap
\end{itemize}

\column{0.48\textwidth}
\textbf{The Pattern Discovery Gap:}
\begin{center}
% \includegraphics[width=0.9\textwidth]{charts/hidden_patterns_revealed.pdf}
\textcolor{mllavender}{[Chart: Pattern Discovery Comparison]}
\end{center}

\begin{tcolorbox}[colback=mlgreen!10, colframe=mlgreen]
\footnotesize
\textbf{Netflix found:}\\
• Micro-genres humans never named\\
• Cross-category preferences\\
• Time-based viewing patterns\\
• Mood-driven selections
\end{tcolorbox}
\end{columns}

\bottomnote{Algorithmic pattern detection reveals latent structure - computational approaches expose relationships human intuition overlooks}
\end{frame}

% Slide 3: Human Categorization Fails at Scale
\begin{frame}[t]{Why Human Categorization Breaks Down}
\Large\textbf{Our Brains Aren't Built for Big Data}
\normalsize

\vspace{0.5em}

\begin{columns}[T]
\column{0.48\textwidth}
\textbf{Human Limits:}

\textbf{1. Cognitive Capacity}
\begin{itemize}
\footnotesize
\item Can track ~7 categories at once
\item After 50 items: accuracy drops 40\%
\item After 500 items: random guessing
\end{itemize}

\textbf{2. Consistency Problem}
\begin{itemize}
\footnotesize
\item Same text, different day = different category
\item Two analysts = 60\% agreement max
\item Fatigue changes decisions
\end{itemize}

\textbf{3. Bias Blindness}
\begin{itemize}
\footnotesize
\item See what we expect to see
\item Miss emerging trends
\item Overlook weak signals
\end{itemize}

\column{0.48\textwidth}
\textbf{Scale Comparison:}
\begin{center}
\begin{tikzpicture}[scale=0.7]
% Human capacity
\draw[fill=mlred!30] (0,0) rectangle (1,0.5);
\node[right] at (1.2,0.25) {\footnotesize Human: 100/day};

% Daily data
\draw[fill=mlorange!30] (0,1) rectangle (5,1.5);
\node[right] at (5.2,1.25) {\footnotesize Daily: 10,000};

% Monthly data
\draw[fill=mlyellow!30] (0,2) rectangle (7,2.5);
\node[right] at (7.2,2.25) {\footnotesize Monthly: 300,000};

% Yearly data
\draw[fill=mlgreen!30] (0,3) rectangle (8.5,3.5);
\node[right] at (8.7,3.25) {\footnotesize Yearly: 3.6M};

\draw[->] (0,0) -- (0,4);
\node[rotate=90] at (-0.5,2) {\footnotesize Volume};
\end{tikzpicture}
\end{center}

\vspace{0.5em}
\begin{tcolorbox}[colback=mllavender4, colframe=mlpurple]
\centering
\textbf{The Gap:} Human capacity is linear,\\
data growth is exponential
\end{tcolorbox}
\end{columns}

\bottomnote{Real-time analysis demands computational methods - latency requirements eliminate manual processing as viable option}
\end{frame}

% Slide 4: The Cross-Cutting Problem
\begin{frame}[t]{The Cross-Cutting Theme Challenge}
\Large\textbf{When Topics Don't Fit in Boxes}
\normalsize

\vspace{0.5em}

\begin{columns}[T]
\column{0.48\textwidth}
\textbf{Traditional Categories:}
\begin{center}
\begin{tikzpicture}[scale=0.6]
% Boxes
\draw[fill=mlred!30] (0,0) rectangle (2,1.5);
\draw[fill=mlblue!30] (2.5,0) rectangle (4.5,1.5);
\draw[fill=mlgreen!30] (5,0) rectangle (7,1.5);
\node at (1,0.75) {Price};
\node at (3.5,0.75) {Quality};
\node at (6,0.75) {Service};

% Problem documents
\node[draw,circle,fill=mlyellow] (d1) at (1.25,2.5) {Doc};
\draw[->] (d1) -- (1,1.5);
\draw[->] (d1) -- (3.5,1.5);
\node[right] at (2,2.5) {\footnotesize Where does this go?};
\end{tikzpicture}
\end{center}

\textbf{Real Review:}
\footnotesize
"Great value for money, though shipping was slow. Product quality exceeded expectations given the price point."

\normalsize
\textbf{Problem:} Mentions price, quality, AND service - which box?

\column{0.48\textwidth}
\textbf{Topic Modeling Solution:}
\begin{center}
\begin{tikzpicture}[scale=0.6]
% Overlapping circles
\draw[fill=mlred!30,opacity=0.5] (1,1) circle (1.2);
\draw[fill=mlblue!30,opacity=0.5] (2.5,1) circle (1.2);
\draw[fill=mlgreen!30,opacity=0.5] (1.75,2.2) circle (1.2);
\node at (0.5,0.5) {\footnotesize Price};
\node at (3,0.5) {\footnotesize Quality};
\node at (1.75,3) {\footnotesize Service};

% Document in overlap
\node[draw,circle,fill=mlyellow] at (1.75,1.3) {Doc};
\end{tikzpicture}
\end{center}

\textbf{Document Mixture:}
\begin{itemize}
\footnotesize
\item 40\% about price/value
\item 35\% about quality
\item 25\% about service
\end{itemize}

\vspace{0.3em}
\textbf{Benefit:} Captures full meaning, not forced choice
\end{columns}

\bottomnote{Every document is a unique mixture of topics - forcing single categories loses information}
\end{frame}

% Slide 5: Enter Topic Modeling
\begin{frame}[t]{Enter Topic Modeling: Pattern Discovery at Scale}
\Large\textbf{From Human Limits to Machine Intelligence}
\normalsize

\vspace{0.5em}

\begin{columns}[T]
\column{0.48\textwidth}
\textbf{What Topic Modeling Does:}

\vspace{0.3em}
\textbf{1. Discovers Hidden Themes}
\begin{itemize}
\footnotesize
\item No predefined categories
\item Themes emerge from data
\item Finds unexpected connections
\end{itemize}

\textbf{2. Handles Scale}
\begin{itemize}
\footnotesize
\item 1M documents in hours
\item Consistent analysis
\item Never gets tired
\end{itemize}

\textbf{3. Captures Nuance}
\begin{itemize}
\footnotesize
\item Documents as topic mixtures
\item Probabilistic understanding
\item Cross-cutting themes
\end{itemize}

\textbf{4. Evolves with Data}
\begin{itemize}
\footnotesize
\item Detects emerging trends
\item Tracks topic evolution
\item Adapts to new patterns
\end{itemize}

\column{0.48\textwidth}
\textbf{The Transformation:}
\begin{center}
\includegraphics[width=0.9\textwidth]{charts/topic_discovery_landscape.pdf}
\end{center}

\begin{tcolorbox}[colback=mlgreen!10, colframe=mlgreen]
\centering
\textbf{Real Impact:}\\[0.3em]
\footnotesize
• 10,000 documents → 20 themes\\
• Processing time: 5 minutes\\
• Human equivalent: 3 months\\
• Patterns found: 15 unexpected
\end{tcolorbox}
\end{columns}

\vspace{0.5em}
\begin{center}
\textcolor{mlpurple}{\textbf{Next: How do machines find these hidden patterns?}}
\end{center}

\bottomnote{Structure extraction from unstructured data enables systematic analysis - organizing text by latent themes transforms exploration into strategy}
\end{frame}